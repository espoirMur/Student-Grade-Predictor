
% Default to the notebook output style

    


% Inherit from the specified cell style.




    
\documentclass[11pt]{article}

    
    
    \usepackage[T1]{fontenc}
    % Nicer default font (+ math font) than Computer Modern for most use cases
    \usepackage{mathpazo}

    % Basic figure setup, for now with no caption control since it's done
    % automatically by Pandoc (which extracts ![](path) syntax from Markdown).
    \usepackage{graphicx}
    % We will generate all images so they have a width \maxwidth. This means
    % that they will get their normal width if they fit onto the page, but
    % are scaled down if they would overflow the margins.
    \makeatletter
    \def\maxwidth{\ifdim\Gin@nat@width>\linewidth\linewidth
    \else\Gin@nat@width\fi}
    \makeatother
    \let\Oldincludegraphics\includegraphics
    % Set max figure width to be 80% of text width, for now hardcoded.
    \renewcommand{\includegraphics}[1]{\Oldincludegraphics[width=.8\maxwidth]{#1}}
    % Ensure that by default, figures have no caption (until we provide a
    % proper Figure object with a Caption API and a way to capture that
    % in the conversion process - todo).
    \usepackage{caption}
    \DeclareCaptionLabelFormat{nolabel}{}
    \captionsetup{labelformat=nolabel}

    \usepackage{adjustbox} % Used to constrain images to a maximum size 
    \usepackage{xcolor} % Allow colors to be defined
    \usepackage{enumerate} % Needed for markdown enumerations to work
    \usepackage{geometry} % Used to adjust the document margins
    \usepackage{amsmath} % Equations
    \usepackage{amssymb} % Equations
    \usepackage{textcomp} % defines textquotesingle
    % Hack from http://tex.stackexchange.com/a/47451/13684:
    \AtBeginDocument{%
        \def\PYZsq{\textquotesingle}% Upright quotes in Pygmentized code
    }
    \usepackage{upquote} % Upright quotes for verbatim code
    \usepackage{eurosym} % defines \euro
    \usepackage[mathletters]{ucs} % Extended unicode (utf-8) support
    \usepackage[utf8x]{inputenc} % Allow utf-8 characters in the tex document
    \usepackage{fancyvrb} % verbatim replacement that allows latex
    \usepackage{grffile} % extends the file name processing of package graphics 
                         % to support a larger range 
    % The hyperref package gives us a pdf with properly built
    % internal navigation ('pdf bookmarks' for the table of contents,
    % internal cross-reference links, web links for URLs, etc.)
    \usepackage{hyperref}
    \usepackage{longtable} % longtable support required by pandoc >1.10
    \usepackage{booktabs}  % table support for pandoc > 1.12.2
    \usepackage[inline]{enumitem} % IRkernel/repr support (it uses the enumerate* environment)
    \usepackage[normalem]{ulem} % ulem is needed to support strikethroughs (\sout)
                                % normalem makes italics be italics, not underlines
    

    
    
    % Colors for the hyperref package
    \definecolor{urlcolor}{rgb}{0,.145,.698}
    \definecolor{linkcolor}{rgb}{.71,0.21,0.01}
    \definecolor{citecolor}{rgb}{.12,.54,.11}

    % ANSI colors
    \definecolor{ansi-black}{HTML}{3E424D}
    \definecolor{ansi-black-intense}{HTML}{282C36}
    \definecolor{ansi-red}{HTML}{E75C58}
    \definecolor{ansi-red-intense}{HTML}{B22B31}
    \definecolor{ansi-green}{HTML}{00A250}
    \definecolor{ansi-green-intense}{HTML}{007427}
    \definecolor{ansi-yellow}{HTML}{DDB62B}
    \definecolor{ansi-yellow-intense}{HTML}{B27D12}
    \definecolor{ansi-blue}{HTML}{208FFB}
    \definecolor{ansi-blue-intense}{HTML}{0065CA}
    \definecolor{ansi-magenta}{HTML}{D160C4}
    \definecolor{ansi-magenta-intense}{HTML}{A03196}
    \definecolor{ansi-cyan}{HTML}{60C6C8}
    \definecolor{ansi-cyan-intense}{HTML}{258F8F}
    \definecolor{ansi-white}{HTML}{C5C1B4}
    \definecolor{ansi-white-intense}{HTML}{A1A6B2}

    % commands and environments needed by pandoc snippets
    % extracted from the output of `pandoc -s`
    \providecommand{\tightlist}{%
      \setlength{\itemsep}{0pt}\setlength{\parskip}{0pt}}
    \DefineVerbatimEnvironment{Highlighting}{Verbatim}{commandchars=\\\{\}}
    % Add ',fontsize=\small' for more characters per line
    \newenvironment{Shaded}{}{}
    \newcommand{\KeywordTok}[1]{\textcolor[rgb]{0.00,0.44,0.13}{\textbf{{#1}}}}
    \newcommand{\DataTypeTok}[1]{\textcolor[rgb]{0.56,0.13,0.00}{{#1}}}
    \newcommand{\DecValTok}[1]{\textcolor[rgb]{0.25,0.63,0.44}{{#1}}}
    \newcommand{\BaseNTok}[1]{\textcolor[rgb]{0.25,0.63,0.44}{{#1}}}
    \newcommand{\FloatTok}[1]{\textcolor[rgb]{0.25,0.63,0.44}{{#1}}}
    \newcommand{\CharTok}[1]{\textcolor[rgb]{0.25,0.44,0.63}{{#1}}}
    \newcommand{\StringTok}[1]{\textcolor[rgb]{0.25,0.44,0.63}{{#1}}}
    \newcommand{\CommentTok}[1]{\textcolor[rgb]{0.38,0.63,0.69}{\textit{{#1}}}}
    \newcommand{\OtherTok}[1]{\textcolor[rgb]{0.00,0.44,0.13}{{#1}}}
    \newcommand{\AlertTok}[1]{\textcolor[rgb]{1.00,0.00,0.00}{\textbf{{#1}}}}
    \newcommand{\FunctionTok}[1]{\textcolor[rgb]{0.02,0.16,0.49}{{#1}}}
    \newcommand{\RegionMarkerTok}[1]{{#1}}
    \newcommand{\ErrorTok}[1]{\textcolor[rgb]{1.00,0.00,0.00}{\textbf{{#1}}}}
    \newcommand{\NormalTok}[1]{{#1}}
    
    % Additional commands for more recent versions of Pandoc
    \newcommand{\ConstantTok}[1]{\textcolor[rgb]{0.53,0.00,0.00}{{#1}}}
    \newcommand{\SpecialCharTok}[1]{\textcolor[rgb]{0.25,0.44,0.63}{{#1}}}
    \newcommand{\VerbatimStringTok}[1]{\textcolor[rgb]{0.25,0.44,0.63}{{#1}}}
    \newcommand{\SpecialStringTok}[1]{\textcolor[rgb]{0.73,0.40,0.53}{{#1}}}
    \newcommand{\ImportTok}[1]{{#1}}
    \newcommand{\DocumentationTok}[1]{\textcolor[rgb]{0.73,0.13,0.13}{\textit{{#1}}}}
    \newcommand{\AnnotationTok}[1]{\textcolor[rgb]{0.38,0.63,0.69}{\textbf{\textit{{#1}}}}}
    \newcommand{\CommentVarTok}[1]{\textcolor[rgb]{0.38,0.63,0.69}{\textbf{\textit{{#1}}}}}
    \newcommand{\VariableTok}[1]{\textcolor[rgb]{0.10,0.09,0.49}{{#1}}}
    \newcommand{\ControlFlowTok}[1]{\textcolor[rgb]{0.00,0.44,0.13}{\textbf{{#1}}}}
    \newcommand{\OperatorTok}[1]{\textcolor[rgb]{0.40,0.40,0.40}{{#1}}}
    \newcommand{\BuiltInTok}[1]{{#1}}
    \newcommand{\ExtensionTok}[1]{{#1}}
    \newcommand{\PreprocessorTok}[1]{\textcolor[rgb]{0.74,0.48,0.00}{{#1}}}
    \newcommand{\AttributeTok}[1]{\textcolor[rgb]{0.49,0.56,0.16}{{#1}}}
    \newcommand{\InformationTok}[1]{\textcolor[rgb]{0.38,0.63,0.69}{\textbf{\textit{{#1}}}}}
    \newcommand{\WarningTok}[1]{\textcolor[rgb]{0.38,0.63,0.69}{\textbf{\textit{{#1}}}}}
    
    
    % Define a nice break command that doesn't care if a line doesn't already
    % exist.
    \def\br{\hspace*{\fill} \\* }
    % Math Jax compatability definitions
    \def\gt{>}
    \def\lt{<}
    % Document parameters
    \title{OuputPreparationDraf1}
    
    
    

    % Pygments definitions
    
\makeatletter
\def\PY@reset{\let\PY@it=\relax \let\PY@bf=\relax%
    \let\PY@ul=\relax \let\PY@tc=\relax%
    \let\PY@bc=\relax \let\PY@ff=\relax}
\def\PY@tok#1{\csname PY@tok@#1\endcsname}
\def\PY@toks#1+{\ifx\relax#1\empty\else%
    \PY@tok{#1}\expandafter\PY@toks\fi}
\def\PY@do#1{\PY@bc{\PY@tc{\PY@ul{%
    \PY@it{\PY@bf{\PY@ff{#1}}}}}}}
\def\PY#1#2{\PY@reset\PY@toks#1+\relax+\PY@do{#2}}

\expandafter\def\csname PY@tok@gd\endcsname{\def\PY@tc##1{\textcolor[rgb]{0.63,0.00,0.00}{##1}}}
\expandafter\def\csname PY@tok@gu\endcsname{\let\PY@bf=\textbf\def\PY@tc##1{\textcolor[rgb]{0.50,0.00,0.50}{##1}}}
\expandafter\def\csname PY@tok@gt\endcsname{\def\PY@tc##1{\textcolor[rgb]{0.00,0.27,0.87}{##1}}}
\expandafter\def\csname PY@tok@gs\endcsname{\let\PY@bf=\textbf}
\expandafter\def\csname PY@tok@gr\endcsname{\def\PY@tc##1{\textcolor[rgb]{1.00,0.00,0.00}{##1}}}
\expandafter\def\csname PY@tok@cm\endcsname{\let\PY@it=\textit\def\PY@tc##1{\textcolor[rgb]{0.25,0.50,0.50}{##1}}}
\expandafter\def\csname PY@tok@vg\endcsname{\def\PY@tc##1{\textcolor[rgb]{0.10,0.09,0.49}{##1}}}
\expandafter\def\csname PY@tok@vi\endcsname{\def\PY@tc##1{\textcolor[rgb]{0.10,0.09,0.49}{##1}}}
\expandafter\def\csname PY@tok@vm\endcsname{\def\PY@tc##1{\textcolor[rgb]{0.10,0.09,0.49}{##1}}}
\expandafter\def\csname PY@tok@mh\endcsname{\def\PY@tc##1{\textcolor[rgb]{0.40,0.40,0.40}{##1}}}
\expandafter\def\csname PY@tok@cs\endcsname{\let\PY@it=\textit\def\PY@tc##1{\textcolor[rgb]{0.25,0.50,0.50}{##1}}}
\expandafter\def\csname PY@tok@ge\endcsname{\let\PY@it=\textit}
\expandafter\def\csname PY@tok@vc\endcsname{\def\PY@tc##1{\textcolor[rgb]{0.10,0.09,0.49}{##1}}}
\expandafter\def\csname PY@tok@il\endcsname{\def\PY@tc##1{\textcolor[rgb]{0.40,0.40,0.40}{##1}}}
\expandafter\def\csname PY@tok@go\endcsname{\def\PY@tc##1{\textcolor[rgb]{0.53,0.53,0.53}{##1}}}
\expandafter\def\csname PY@tok@cp\endcsname{\def\PY@tc##1{\textcolor[rgb]{0.74,0.48,0.00}{##1}}}
\expandafter\def\csname PY@tok@gi\endcsname{\def\PY@tc##1{\textcolor[rgb]{0.00,0.63,0.00}{##1}}}
\expandafter\def\csname PY@tok@gh\endcsname{\let\PY@bf=\textbf\def\PY@tc##1{\textcolor[rgb]{0.00,0.00,0.50}{##1}}}
\expandafter\def\csname PY@tok@ni\endcsname{\let\PY@bf=\textbf\def\PY@tc##1{\textcolor[rgb]{0.60,0.60,0.60}{##1}}}
\expandafter\def\csname PY@tok@nl\endcsname{\def\PY@tc##1{\textcolor[rgb]{0.63,0.63,0.00}{##1}}}
\expandafter\def\csname PY@tok@nn\endcsname{\let\PY@bf=\textbf\def\PY@tc##1{\textcolor[rgb]{0.00,0.00,1.00}{##1}}}
\expandafter\def\csname PY@tok@no\endcsname{\def\PY@tc##1{\textcolor[rgb]{0.53,0.00,0.00}{##1}}}
\expandafter\def\csname PY@tok@na\endcsname{\def\PY@tc##1{\textcolor[rgb]{0.49,0.56,0.16}{##1}}}
\expandafter\def\csname PY@tok@nb\endcsname{\def\PY@tc##1{\textcolor[rgb]{0.00,0.50,0.00}{##1}}}
\expandafter\def\csname PY@tok@nc\endcsname{\let\PY@bf=\textbf\def\PY@tc##1{\textcolor[rgb]{0.00,0.00,1.00}{##1}}}
\expandafter\def\csname PY@tok@nd\endcsname{\def\PY@tc##1{\textcolor[rgb]{0.67,0.13,1.00}{##1}}}
\expandafter\def\csname PY@tok@ne\endcsname{\let\PY@bf=\textbf\def\PY@tc##1{\textcolor[rgb]{0.82,0.25,0.23}{##1}}}
\expandafter\def\csname PY@tok@nf\endcsname{\def\PY@tc##1{\textcolor[rgb]{0.00,0.00,1.00}{##1}}}
\expandafter\def\csname PY@tok@si\endcsname{\let\PY@bf=\textbf\def\PY@tc##1{\textcolor[rgb]{0.73,0.40,0.53}{##1}}}
\expandafter\def\csname PY@tok@s2\endcsname{\def\PY@tc##1{\textcolor[rgb]{0.73,0.13,0.13}{##1}}}
\expandafter\def\csname PY@tok@nt\endcsname{\let\PY@bf=\textbf\def\PY@tc##1{\textcolor[rgb]{0.00,0.50,0.00}{##1}}}
\expandafter\def\csname PY@tok@nv\endcsname{\def\PY@tc##1{\textcolor[rgb]{0.10,0.09,0.49}{##1}}}
\expandafter\def\csname PY@tok@s1\endcsname{\def\PY@tc##1{\textcolor[rgb]{0.73,0.13,0.13}{##1}}}
\expandafter\def\csname PY@tok@dl\endcsname{\def\PY@tc##1{\textcolor[rgb]{0.73,0.13,0.13}{##1}}}
\expandafter\def\csname PY@tok@ch\endcsname{\let\PY@it=\textit\def\PY@tc##1{\textcolor[rgb]{0.25,0.50,0.50}{##1}}}
\expandafter\def\csname PY@tok@m\endcsname{\def\PY@tc##1{\textcolor[rgb]{0.40,0.40,0.40}{##1}}}
\expandafter\def\csname PY@tok@gp\endcsname{\let\PY@bf=\textbf\def\PY@tc##1{\textcolor[rgb]{0.00,0.00,0.50}{##1}}}
\expandafter\def\csname PY@tok@sh\endcsname{\def\PY@tc##1{\textcolor[rgb]{0.73,0.13,0.13}{##1}}}
\expandafter\def\csname PY@tok@ow\endcsname{\let\PY@bf=\textbf\def\PY@tc##1{\textcolor[rgb]{0.67,0.13,1.00}{##1}}}
\expandafter\def\csname PY@tok@sx\endcsname{\def\PY@tc##1{\textcolor[rgb]{0.00,0.50,0.00}{##1}}}
\expandafter\def\csname PY@tok@bp\endcsname{\def\PY@tc##1{\textcolor[rgb]{0.00,0.50,0.00}{##1}}}
\expandafter\def\csname PY@tok@c1\endcsname{\let\PY@it=\textit\def\PY@tc##1{\textcolor[rgb]{0.25,0.50,0.50}{##1}}}
\expandafter\def\csname PY@tok@fm\endcsname{\def\PY@tc##1{\textcolor[rgb]{0.00,0.00,1.00}{##1}}}
\expandafter\def\csname PY@tok@o\endcsname{\def\PY@tc##1{\textcolor[rgb]{0.40,0.40,0.40}{##1}}}
\expandafter\def\csname PY@tok@kc\endcsname{\let\PY@bf=\textbf\def\PY@tc##1{\textcolor[rgb]{0.00,0.50,0.00}{##1}}}
\expandafter\def\csname PY@tok@c\endcsname{\let\PY@it=\textit\def\PY@tc##1{\textcolor[rgb]{0.25,0.50,0.50}{##1}}}
\expandafter\def\csname PY@tok@mf\endcsname{\def\PY@tc##1{\textcolor[rgb]{0.40,0.40,0.40}{##1}}}
\expandafter\def\csname PY@tok@err\endcsname{\def\PY@bc##1{\setlength{\fboxsep}{0pt}\fcolorbox[rgb]{1.00,0.00,0.00}{1,1,1}{\strut ##1}}}
\expandafter\def\csname PY@tok@mb\endcsname{\def\PY@tc##1{\textcolor[rgb]{0.40,0.40,0.40}{##1}}}
\expandafter\def\csname PY@tok@ss\endcsname{\def\PY@tc##1{\textcolor[rgb]{0.10,0.09,0.49}{##1}}}
\expandafter\def\csname PY@tok@sr\endcsname{\def\PY@tc##1{\textcolor[rgb]{0.73,0.40,0.53}{##1}}}
\expandafter\def\csname PY@tok@mo\endcsname{\def\PY@tc##1{\textcolor[rgb]{0.40,0.40,0.40}{##1}}}
\expandafter\def\csname PY@tok@kd\endcsname{\let\PY@bf=\textbf\def\PY@tc##1{\textcolor[rgb]{0.00,0.50,0.00}{##1}}}
\expandafter\def\csname PY@tok@mi\endcsname{\def\PY@tc##1{\textcolor[rgb]{0.40,0.40,0.40}{##1}}}
\expandafter\def\csname PY@tok@kn\endcsname{\let\PY@bf=\textbf\def\PY@tc##1{\textcolor[rgb]{0.00,0.50,0.00}{##1}}}
\expandafter\def\csname PY@tok@cpf\endcsname{\let\PY@it=\textit\def\PY@tc##1{\textcolor[rgb]{0.25,0.50,0.50}{##1}}}
\expandafter\def\csname PY@tok@kr\endcsname{\let\PY@bf=\textbf\def\PY@tc##1{\textcolor[rgb]{0.00,0.50,0.00}{##1}}}
\expandafter\def\csname PY@tok@s\endcsname{\def\PY@tc##1{\textcolor[rgb]{0.73,0.13,0.13}{##1}}}
\expandafter\def\csname PY@tok@kp\endcsname{\def\PY@tc##1{\textcolor[rgb]{0.00,0.50,0.00}{##1}}}
\expandafter\def\csname PY@tok@w\endcsname{\def\PY@tc##1{\textcolor[rgb]{0.73,0.73,0.73}{##1}}}
\expandafter\def\csname PY@tok@kt\endcsname{\def\PY@tc##1{\textcolor[rgb]{0.69,0.00,0.25}{##1}}}
\expandafter\def\csname PY@tok@sc\endcsname{\def\PY@tc##1{\textcolor[rgb]{0.73,0.13,0.13}{##1}}}
\expandafter\def\csname PY@tok@sb\endcsname{\def\PY@tc##1{\textcolor[rgb]{0.73,0.13,0.13}{##1}}}
\expandafter\def\csname PY@tok@sa\endcsname{\def\PY@tc##1{\textcolor[rgb]{0.73,0.13,0.13}{##1}}}
\expandafter\def\csname PY@tok@k\endcsname{\let\PY@bf=\textbf\def\PY@tc##1{\textcolor[rgb]{0.00,0.50,0.00}{##1}}}
\expandafter\def\csname PY@tok@se\endcsname{\let\PY@bf=\textbf\def\PY@tc##1{\textcolor[rgb]{0.73,0.40,0.13}{##1}}}
\expandafter\def\csname PY@tok@sd\endcsname{\let\PY@it=\textit\def\PY@tc##1{\textcolor[rgb]{0.73,0.13,0.13}{##1}}}

\def\PYZbs{\char`\\}
\def\PYZus{\char`\_}
\def\PYZob{\char`\{}
\def\PYZcb{\char`\}}
\def\PYZca{\char`\^}
\def\PYZam{\char`\&}
\def\PYZlt{\char`\<}
\def\PYZgt{\char`\>}
\def\PYZsh{\char`\#}
\def\PYZpc{\char`\%}
\def\PYZdl{\char`\$}
\def\PYZhy{\char`\-}
\def\PYZsq{\char`\'}
\def\PYZdq{\char`\"}
\def\PYZti{\char`\~}
% for compatibility with earlier versions
\def\PYZat{@}
\def\PYZlb{[}
\def\PYZrb{]}
\makeatother


    % Exact colors from NB
    \definecolor{incolor}{rgb}{0.0, 0.0, 0.5}
    \definecolor{outcolor}{rgb}{0.545, 0.0, 0.0}



    
    % Prevent overflowing lines due to hard-to-break entities
    \sloppy 
    % Setup hyperref package
    \hypersetup{
      breaklinks=true,  % so long urls are correctly broken across lines
      colorlinks=true,
      urlcolor=urlcolor,
      linkcolor=linkcolor,
      citecolor=citecolor,
      }
    % Slightly bigger margins than the latex defaults
    
    \geometry{verbose,tmargin=1in,bmargin=1in,lmargin=1in,rmargin=1in}
    
    

    \begin{document}
    
    
    \maketitle
    
    

    
    \section{Preparation des nos variables de
sortie}\label{preparation-des-nos-variables-de-sortie}

    Dans cette partie nous allons preparer notre ensemble de sortie en
construisant nos varibles de sortie , celle ci seront constiruer à
l'aide des divers fonction qui nous aiderons à determiner pour une année
si sur base des résultat que nous aons si un étudiant est passé en
premiere seion, s'il a distingué ou bien s'il a échoue, nous
determinerons enfin le pourcentage obtebnu.

    Notons que nous nous sommes butés à 2 grands problème concernant les
données obtenus et qui risquent de biaiser nos calculs: - Nous n'avons
pas pu obtenir le pourcentage final de certains étudiant ayant echouer
ou abandonner leurs études avant la fin de l'année académique ces
étudiants qui après la seconde session sont AA (Assimilé aux Ajournées )
n'ont pas de pourcentage final pour palier àce probleme nous avons
remplacer cette valeur manquante par une valeur aleatoiree choisie entre
40 et 50 pourcent ce qui represente l'echec! - Pour nous étudiant le
nombre des années académiques pour lequel nous avons le résultat n'est
pas le meme pour certains étudiant nous avons des informations pour 1
annéé académique, d'autres pour 2 , etc. Pour palier à ce problème nous
avons utliser des ration pour nous valeur de sortie qui consiter à
diviser le nombre sur le no,bre des année académique que l'étudiant a
passée à la faculté

    Comencons par charger nos libraries

    \begin{Verbatim}[commandchars=\\\{\}]
{\color{incolor}In [{\color{incolor}7}]:} \PY{k+kn}{import} \PY{n+nn}{pandas} \PY{k+kn}{as} \PY{n+nn}{pd}
        \PY{k+kn}{import} \PY{n+nn}{numpy} \PY{k+kn}{as} \PY{n+nn}{np}
        \PY{k+kn}{import} \PY{n+nn}{matplotlib.pyplot} \PY{k+kn}{as} \PY{n+nn}{plt}
        \PY{k+kn}{import} \PY{n+nn}{seaborn} \PY{k+kn}{as} \PY{n+nn}{sns} \PY{c+c1}{\PYZsh{}for beatiful visualizations}
        \PY{o}{\PYZpc{}}\PY{k}{matplotlib} inline 
        \PY{k+kn}{import} \PY{n+nn}{scipy.stats} \PY{k+kn}{as} \PY{n+nn}{scs} \PY{c+c1}{\PYZsh{}for statistics}
        \PY{k+kn}{import} \PY{n+nn}{operator}
        \PY{k+kn}{from} \PY{n+nn}{scipy.stats} \PY{k+kn}{import} \PY{n}{chi2\PYZus{}contingency}
        \PY{k+kn}{import} \PY{n+nn}{matplotlib.ticker} \PY{k+kn}{as} \PY{n+nn}{ticker}
        \PY{k+kn}{import} \PY{n+nn}{statsmodels.api} \PY{k+kn}{as} \PY{n+nn}{sm}
        \PY{k+kn}{from} \PY{n+nn}{statsmodels.formula.api} \PY{k+kn}{import} \PY{n}{ols}
\end{Verbatim}

    Notons que dans cette partie nous allons Aussi faire une analyse
statistique univarié et bi-varié entre nos données d"entree et celle de
sortie

    \begin{Verbatim}[commandchars=\\\{\}]
{\color{incolor}In [{\color{incolor}26}]:} \PY{n}{dataset}\PY{o}{=}\PY{n}{pd}\PY{o}{.}\PY{n}{read\PYZus{}csv}\PY{p}{(}\PY{l+s+s2}{\PYZdq{}}\PY{l+s+s2}{dataset/DatasetOut.csv}\PY{l+s+s2}{\PYZdq{}}\PY{p}{,}\PY{n}{index\PYZus{}col}\PY{o}{=}\PY{l+s+s2}{\PYZdq{}}\PY{l+s+s2}{Unnamed: 0}\PY{l+s+s2}{\PYZdq{}}\PY{p}{)}
\end{Verbatim}

    \begin{Verbatim}[commandchars=\\\{\}]
{\color{incolor}In [{\color{incolor}27}]:} \PY{n}{dataset}\PY{o}{.}\PY{n}{rename}\PY{p}{(}\PY{n}{columns}\PY{o}{=}\PY{p}{\PYZob{}}\PY{l+s+s1}{\PYZsq{}}\PY{l+s+s1}{IDENTIFICATION}\PY{l+s+s1}{\PYZsq{}}\PY{p}{:}\PY{l+s+s1}{\PYZsq{}}\PY{l+s+s1}{ID}\PY{l+s+s1}{\PYZsq{}}\PY{p}{\PYZcb{}}\PY{p}{,}\PY{n}{inplace}\PY{o}{=}\PY{n+nb+bp}{True}\PY{p}{)}
\end{Verbatim}

    \begin{Verbatim}[commandchars=\\\{\}]
{\color{incolor}In [{\color{incolor}4}]:} \PY{n}{SuccefullMent}\PY{o}{=}\PY{p}{[}\PY{l+s+s1}{\PYZsq{}}\PY{l+s+s1}{S}\PY{l+s+s1}{\PYZsq{}}\PY{p}{]}
        \PY{n}{PartialSuccess}\PY{o}{=}\PY{p}{[}\PY{l+s+s1}{\PYZsq{}}\PY{l+s+s1}{ADM}\PY{l+s+s1}{\PYZsq{}}\PY{p}{,} \PY{l+s+s1}{\PYZsq{}}\PY{l+s+s1}{ADTFC}\PY{l+s+s1}{\PYZsq{}}\PY{p}{,} \PY{l+s+s1}{\PYZsq{}}\PY{l+s+s1}{ADSTM}\PY{l+s+s1}{\PYZsq{}}\PY{p}{,} \PY{l+s+s1}{\PYZsq{}}\PY{l+s+s1}{ADSTFC}\PY{l+s+s1}{\PYZsq{}}\PY{p}{,} \PY{l+s+s1}{\PYZsq{}}\PY{l+s+s1}{ADST}\PY{l+s+s1}{\PYZsq{}}\PY{p}{]}
        \PY{n}{Distinction}\PY{o}{=}\PY{p}{[}\PY{l+s+s1}{\PYZsq{}}\PY{l+s+s1}{D}\PY{l+s+s1}{\PYZsq{}}\PY{p}{,}\PY{l+s+s1}{\PYZsq{}}\PY{l+s+s1}{GD}\PY{l+s+s1}{\PYZsq{}}\PY{p}{]}
        \PY{n}{FailSession}\PY{o}{=}\PY{p}{[}\PY{l+s+s1}{\PYZsq{}}\PY{l+s+s1}{AA}\PY{l+s+s1}{\PYZsq{}}\PY{p}{,}\PY{l+s+s1}{\PYZsq{}}\PY{l+s+s1}{AR}\PY{l+s+s1}{\PYZsq{}}\PY{p}{,}\PY{l+s+s1}{\PYZsq{}}\PY{l+s+s1}{nan}\PY{l+s+s1}{\PYZsq{}}\PY{p}{]}
        \PY{n}{FailureMention}\PY{o}{=}\PY{p}{[}\PY{l+s+s1}{\PYZsq{}}\PY{l+s+s1}{A}\PY{l+s+s1}{\PYZsq{}}\PY{p}{,}\PY{l+s+s1}{\PYZsq{}}\PY{l+s+s1}{NAF}\PY{l+s+s1}{\PYZsq{}}\PY{p}{,}\PY{l+s+s1}{\PYZsq{}}\PY{l+s+s1}{ANAF}\PY{l+s+s1}{\PYZsq{}}\PY{p}{]}
\end{Verbatim}

    \begin{Verbatim}[commandchars=\\\{\}]
{\color{incolor}In [{\color{incolor}5}]:} \PY{n}{dataset}\PY{o}{.}\PY{n}{MENT1}\PY{o}{.}\PY{n}{value\PYZus{}counts}\PY{p}{(}\PY{p}{)}
\end{Verbatim}

            \begin{Verbatim}[commandchars=\\\{\}]
{\color{outcolor}Out[{\color{outcolor}5}]:} AA        4569
        A         2351
        S         2037
        D          268
        ADM        101
        ADTFC       81
        ADSTM       74
        ADSTFC      55
        ADST        45
        NAF         17
        GD           7
        ANAF         1
        Name: MENT1, dtype: int64
\end{Verbatim}
        
    \begin{Verbatim}[commandchars=\\\{\}]
{\color{incolor}In [{\color{incolor}6}]:} \PY{n}{dataset}\PY{o}{.}\PY{n}{columns}
\end{Verbatim}

            \begin{Verbatim}[commandchars=\\\{\}]
{\color{outcolor}Out[{\color{outcolor}6}]:} Index([u'ID', u'ACADYEAR', u'PERC1', u'MENT1', u'PERC2', u'MENT2', u'FAC',
               u'OPT', u'PROM'],
              dtype='object')
\end{Verbatim}
        
    Premierement Voici la fonction que nous avons utiliser pour determiner
si un étudiant a reussi ou a echouer pour une année academique

    \begin{Verbatim}[commandchars=\\\{\}]
{\color{incolor}In [{\color{incolor}8}]:} \PY{k}{def} \PY{n+nf}{calculateSucess}\PY{p}{(}\PY{n}{x}\PY{p}{)}\PY{p}{:}
            \PY{l+s+sd}{\PYZdq{}\PYZdq{}\PYZdq{}this function will check if a student pass of fail in an aacademic year\PYZdq{}\PYZdq{}\PYZdq{}}
            \PY{k}{if} \PY{n+nb}{str}\PY{p}{(}\PY{n}{x}\PY{p}{[}\PY{l+s+s1}{\PYZsq{}}\PY{l+s+s1}{MENT2}\PY{l+s+s1}{\PYZsq{}}\PY{p}{]}\PY{p}{)} \PY{o+ow}{in} \PY{n}{FailSession}\PY{o}{+}\PY{n}{FailureMention} \PY{o+ow}{and} \PY{p}{(}\PY{n+nb}{str}\PY{p}{(}\PY{n}{x}\PY{p}{[}\PY{l+s+s1}{\PYZsq{}}\PY{l+s+s1}{MENT1}\PY{l+s+s1}{\PYZsq{}}\PY{p}{]}\PY{p}{)} \PY{o+ow}{not} \PY{o+ow}{in} \PY{n}{SuccefullMent}\PY{o}{+}\PY{n}{PartialSuccess}\PY{o}{+}\PY{n}{Distinction}\PY{p}{)}\PY{p}{:}
                \PY{k}{return} \PY{n}{pd}\PY{o}{.}\PY{n}{Series}\PY{p}{(}\PY{n+nb+bp}{True}\PY{p}{)}
            \PY{k}{else} \PY{p}{:}
                \PY{k}{return} \PY{n}{pd}\PY{o}{.}\PY{n}{Series}\PY{p}{(}\PY{n+nb+bp}{False}\PY{p}{)}
\end{Verbatim}

    \begin{Verbatim}[commandchars=\\\{\}]
{\color{incolor}In [{\color{incolor}10}]:} \PY{n}{Echec}\PY{o}{=}\PY{n}{dataset}\PY{o}{.}\PY{n}{apply}\PY{p}{(}\PY{n}{calculateSucess}\PY{p}{,}\PY{n}{axis}\PY{o}{=}\PY{l+m+mi}{1}\PY{p}{)}
\end{Verbatim}

    \begin{Verbatim}[commandchars=\\\{\}]
{\color{incolor}In [{\color{incolor}12}]:} \PY{n}{Echec}\PY{p}{[}\PY{p}{:}\PY{l+m+mi}{6}\PY{p}{]}
\end{Verbatim}

            \begin{Verbatim}[commandchars=\\\{\}]
{\color{outcolor}Out[{\color{outcolor}12}]:}        0
         0   True
         1  False
         2   True
         3  False
         4  False
         5   True
\end{Verbatim}
        
    \begin{Verbatim}[commandchars=\\\{\}]
{\color{incolor}In [{\color{incolor}22}]:} \PY{n}{dataset}\PY{p}{[}\PY{p}{:}\PY{l+m+mi}{6}\PY{p}{]}
\end{Verbatim}

            \begin{Verbatim}[commandchars=\\\{\}]
{\color{outcolor}Out[{\color{outcolor}22}]:}     ID   ACADYEAR      PERC1  MENT1      PERC2 MENT2   FAC  \textbackslash{}
         0   45  2013-2014        NaN     AA        NaN   NaN  FPSE   
         1  215  2012-2013        NaN    ADM  63.099998     S    FD   
         2  343  2015-2016        NaN     AA  52.200001     A  FSEG   
         3  356  2015-2016        NaN  ADSTM  59.900002     S  FSEG   
         4  398  2012-2013  65.099998      S        NaN   NaN  FSDC   
         5  429  2013-2014        NaN     AA        NaN     A    FD   
         
                                              OPT PROM  Echec  
         0                Sciences de l'Éducation   L2   True  
         1              Droit Privé et Judiciaire   L2  False  
         2                           Tronc commun   G2   True  
         3                Gestion des Entreprises   L2  False  
         4  Santé et Développement Communautaires   L2  False  
         5                           Tronc commun   G1   True  
\end{Verbatim}
        
    Avec cette unction nous avons pu determier si un étudiant a reussi ou
echoué dans une année académique

    \begin{Verbatim}[commandchars=\\\{\}]
{\color{incolor}In [{\color{incolor}21}]:} \PY{n}{dataset}\PY{p}{[}\PY{l+s+s1}{\PYZsq{}}\PY{l+s+s1}{Echec}\PY{l+s+s1}{\PYZsq{}}\PY{p}{]}\PY{o}{=}\PY{n}{Echec}
\end{Verbatim}

    Ensuite On cherche à déterminer si l'étudiant est passé en premiere
session dans une anné academique

    Voici la fonction qui nous permet d'effectuer ses calculs:

    \begin{Verbatim}[commandchars=\\\{\}]
{\color{incolor}In [{\color{incolor}27}]:} \PY{k}{def} \PY{n+nf}{calculatePass1estSes}\PY{p}{(}\PY{n}{x}\PY{p}{)}\PY{p}{:}
             \PY{l+s+sd}{\PYZdq{}\PYZdq{}\PYZdq{}this function will check if a student pass in the 1st session in an academic year\PYZdq{}\PYZdq{}\PYZdq{}}
             \PY{k}{if} \PY{n}{x}\PY{p}{[}\PY{l+s+s1}{\PYZsq{}}\PY{l+s+s1}{MENT1}\PY{l+s+s1}{\PYZsq{}}\PY{p}{]} \PY{o+ow}{in} \PY{n}{SuccefullMent}\PY{o}{+}\PY{n}{Distinction}\PY{p}{:}
                 \PY{k}{return} \PY{n}{pd}\PY{o}{.}\PY{n}{Series}\PY{p}{(}\PY{n+nb+bp}{True}\PY{p}{)}
             \PY{k}{else} \PY{p}{:}
                 \PY{k}{return} \PY{n}{pd}\PY{o}{.}\PY{n}{Series}\PY{p}{(}\PY{n+nb+bp}{False}\PY{p}{)}
\end{Verbatim}

    \begin{Verbatim}[commandchars=\\\{\}]
{\color{incolor}In [{\color{incolor}28}]:} \PY{n}{Pass1erSession}\PY{o}{=}\PY{n}{dataset}\PY{o}{.}\PY{n}{apply}\PY{p}{(}\PY{n}{calculatePass1estSes}\PY{p}{,}\PY{n}{axis}\PY{o}{=}\PY{l+m+mi}{1}\PY{p}{)}
\end{Verbatim}

    \begin{Verbatim}[commandchars=\\\{\}]
{\color{incolor}In [{\color{incolor}29}]:} \PY{n}{dataset}\PY{o}{.}\PY{n}{loc}\PY{p}{[}\PY{p}{:}\PY{p}{,}\PY{l+s+s1}{\PYZsq{}}\PY{l+s+s1}{Pass1erSession}\PY{l+s+s1}{\PYZsq{}}\PY{p}{]}\PY{o}{=}\PY{n}{Pass1erSession}
         \PY{n}{dataset}\PY{p}{[}\PY{l+s+s1}{\PYZsq{}}\PY{l+s+s1}{Pass1erSession}\PY{l+s+s1}{\PYZsq{}}\PY{p}{]}\PY{o}{=}\PY{n}{Pass1erSession}
\end{Verbatim}

    \begin{Verbatim}[commandchars=\\\{\}]
{\color{incolor}In [{\color{incolor}30}]:} \PY{n}{dataset}\PY{o}{.}\PY{n}{head}\PY{p}{(}\PY{l+m+mi}{8}\PY{p}{)}
\end{Verbatim}

            \begin{Verbatim}[commandchars=\\\{\}]
{\color{outcolor}Out[{\color{outcolor}30}]:}     ID   ACADYEAR      PERC1  MENT1      PERC2 MENT2   FAC  \textbackslash{}
         0   45  2013-2014        NaN     AA        NaN   NaN  FPSE   
         1  215  2012-2013        NaN    ADM  63.099998     S    FD   
         2  343  2015-2016        NaN     AA  52.200001     A  FSEG   
         3  356  2015-2016        NaN  ADSTM  59.900002     S  FSEG   
         4  398  2012-2013  65.099998      S        NaN   NaN  FSDC   
         5  429  2013-2014        NaN     AA        NaN     A    FD   
         6  474  2014-2015        NaN     AA        NaN   NaN    FD   
         7  474  2015-2016  62.500000      S        NaN   NaN    FD   
         
                                              OPT PROM  Echec Pass1erSession  
         0                Sciences de l'Éducation   L2   True          False  
         1              Droit Privé et Judiciaire   L2  False          False  
         2                           Tronc commun   G2   True          False  
         3                Gestion des Entreprises   L2  False          False  
         4  Santé et Développement Communautaires   L2  False           True  
         5                           Tronc commun   G1   True          False  
         6              Droit Privé et Judiciaire   G3   True          False  
         7              Droit Privé et Judiciaire   G3  False           True  
\end{Verbatim}
        
    Ensuite on cherche à determiner si un étudiant a distinguer au cours
d'une académique , voicila function:

    \begin{Verbatim}[commandchars=\\\{\}]
{\color{incolor}In [{\color{incolor}34}]:} \PY{k}{def} \PY{n+nf}{calculateDistinCtion}\PY{p}{(}\PY{n}{x}\PY{p}{)}\PY{p}{:}
             \PY{l+s+sd}{\PYZdq{}\PYZdq{}\PYZdq{}this function will check if a student get a distinction mention in aacademic year\PYZdq{}\PYZdq{}\PYZdq{}}
             \PY{k}{if} \PY{n}{x}\PY{p}{[}\PY{l+s+s1}{\PYZsq{}}\PY{l+s+s1}{MENT1}\PY{l+s+s1}{\PYZsq{}}\PY{p}{]} \PY{o+ow}{in} \PY{n}{Distinction} \PY{o+ow}{or} \PY{n}{x}\PY{p}{[}\PY{l+s+s1}{\PYZsq{}}\PY{l+s+s1}{MENT2}\PY{l+s+s1}{\PYZsq{}}\PY{p}{]}\PY{o+ow}{in} \PY{n}{Distinction}\PY{p}{:}
                 \PY{k}{return} \PY{n}{pd}\PY{o}{.}\PY{n}{Series}\PY{p}{(}\PY{n+nb+bp}{True}\PY{p}{)}
             \PY{k}{else} \PY{p}{:}
                 \PY{k}{return} \PY{n}{pd}\PY{o}{.}\PY{n}{Series}\PY{p}{(}\PY{n+nb+bp}{False}\PY{p}{)}
\end{Verbatim}

    \begin{Verbatim}[commandchars=\\\{\}]
{\color{incolor}In [{\color{incolor}35}]:} \PY{n}{Distinction}\PY{o}{=}\PY{n}{dataset}\PY{o}{.}\PY{n}{apply}\PY{p}{(}\PY{n}{calculateDistinCtion}\PY{p}{,}\PY{n}{axis}\PY{o}{=}\PY{l+m+mi}{1}\PY{p}{)}
\end{Verbatim}

    \begin{Verbatim}[commandchars=\\\{\}]
{\color{incolor}In [{\color{incolor}36}]:} \PY{n}{dataset}\PY{o}{.}\PY{n}{loc}\PY{p}{[}\PY{p}{:}\PY{p}{,}\PY{l+s+s1}{\PYZsq{}}\PY{l+s+s1}{Distinction}\PY{l+s+s1}{\PYZsq{}}\PY{p}{]}\PY{o}{=}\PY{n}{Distinction}
         \PY{n}{dataset}\PY{p}{[}\PY{l+s+s1}{\PYZsq{}}\PY{l+s+s1}{Distinction}\PY{l+s+s1}{\PYZsq{}}\PY{p}{]}\PY{o}{=}\PY{n}{Distinction}
\end{Verbatim}

    \begin{Verbatim}[commandchars=\\\{\}]
{\color{incolor}In [{\color{incolor}39}]:} \PY{n}{dataset}\PY{o}{.}\PY{n}{loc}\PY{p}{[}\PY{n}{dataset}\PY{o}{.}\PY{n}{Distinction}\PY{p}{]}\PY{o}{.}\PY{n}{head}\PY{p}{(}\PY{l+m+mi}{8}\PY{p}{)}
\end{Verbatim}

            \begin{Verbatim}[commandchars=\\\{\}]
{\color{outcolor}Out[{\color{outcolor}39}]:}        ID   ACADYEAR      PERC1  MENT1      PERC2 MENT2   FAC   PROM  Echec  \textbackslash{}
         47   2221  2012-2013  70.199997      D        NaN   NaN    FD     L2  False   
         57   2457  2012-2013  76.400002      D        NaN   NaN  FSTA  TECH2  False   
         96   3298  2015-2016  70.699997      D        NaN   NaN  FPSE     L2  False   
         106  3578  2012-2013  70.099998      D        NaN   NaN    FD     L2  False   
         110  3588  2013-2014        NaN     AA  73.500000     D  FSTA  Tech1  False   
         114  3591  2013-2014        NaN  ADSTM  73.900002     D  FSTA  Tech2  False   
         123  3626  2014-2015  71.900002      D        NaN   NaN  FSTA  Tech1  False   
         140  3646  2013-2014        NaN     AA  73.300003     D  FSTA  Tech1  False   
         
             Pass1erSession Distinction  
         47            True        True  
         57            True        True  
         96            True        True  
         106           True        True  
         110          False        True  
         114          False        True  
         123           True        True  
         140          False        True  
\end{Verbatim}
        
    Enfin Calculons le pourcentage final qu'a obtenu un étudiant au cours
d'une anné academique

    \begin{Verbatim}[commandchars=\\\{\}]
{\color{incolor}In [{\color{incolor}155}]:} \PY{k}{def} \PY{n+nf}{calculatePourcenatge}\PY{p}{(}\PY{n}{x}\PY{p}{)}\PY{p}{:}
              \PY{l+s+sd}{\PYZdq{}\PYZdq{}\PYZdq{}this function will check if a student pass of fail in an aacademic year\PYZdq{}\PYZdq{}\PYZdq{}}
              \PY{k}{if} \PY{n}{x}\PY{p}{[}\PY{l+s+s1}{\PYZsq{}}\PY{l+s+s1}{Pass1erSession}\PY{l+s+s1}{\PYZsq{}}\PY{p}{]} \PY{p}{:}
                  \PY{k}{return} \PY{n}{pd}\PY{o}{.}\PY{n}{Series}\PY{p}{(}\PY{n}{x}\PY{p}{[}\PY{l+s+s1}{\PYZsq{}}\PY{l+s+s1}{PERC1}\PY{l+s+s1}{\PYZsq{}}\PY{p}{]}\PY{p}{)}
              \PY{k}{elif} \PY{n}{x}\PY{p}{[}\PY{l+s+s1}{\PYZsq{}}\PY{l+s+s1}{MENT2}\PY{l+s+s1}{\PYZsq{}}\PY{p}{]} \PY{o+ow}{in} \PY{n}{SuccefullMent} \PY{o+ow}{and}  \PY{n}{x}\PY{p}{[}\PY{l+s+s1}{\PYZsq{}}\PY{l+s+s1}{MENT1}\PY{l+s+s1}{\PYZsq{}}\PY{p}{]} \PY{o+ow}{in} \PY{n}{FailSession}\PY{o}{+}\PY{n}{FailureMention}\PY{o}{+}\PY{n}{PartialSuccess}\PY{p}{:}
                  \PY{k}{return} \PY{n}{pd}\PY{o}{.}\PY{n}{Series}\PY{p}{(}\PY{n}{x}\PY{p}{[}\PY{l+s+s1}{\PYZsq{}}\PY{l+s+s1}{PERC2}\PY{l+s+s1}{\PYZsq{}}\PY{p}{]}\PY{p}{)}
              \PY{k}{elif} \PY{n+nb}{str}\PY{p}{(}\PY{n}{x}\PY{p}{[}\PY{l+s+s1}{\PYZsq{}}\PY{l+s+s1}{MENT2}\PY{l+s+s1}{\PYZsq{}}\PY{p}{]}\PY{p}{)} \PY{o+ow}{in} \PY{n}{Distinction} \PY{o+ow}{and}  \PY{n+nb}{str}\PY{p}{(}\PY{n}{x}\PY{p}{[}\PY{l+s+s1}{\PYZsq{}}\PY{l+s+s1}{MENT1}\PY{l+s+s1}{\PYZsq{}}\PY{p}{]}\PY{p}{)} \PY{o+ow}{in} \PY{n}{FailSession}\PY{o}{+}\PY{n}{FailureMention}\PY{o}{+}\PY{n}{PartialSuccess} \PY{p}{:}
                  \PY{k}{return} \PY{n}{pd}\PY{o}{.}\PY{n}{Series}\PY{p}{(}\PY{n}{x}\PY{p}{[}\PY{l+s+s1}{\PYZsq{}}\PY{l+s+s1}{PERC2}\PY{l+s+s1}{\PYZsq{}}\PY{p}{]}\PY{p}{)}
              \PY{k}{else}\PY{p}{:}
                  \PY{k}{return} \PY{n}{pd}\PY{o}{.}\PY{n}{Series}\PY{p}{(}\PY{n}{x}\PY{p}{[}\PY{l+s+s1}{\PYZsq{}}\PY{l+s+s1}{PERC2}\PY{l+s+s1}{\PYZsq{}}\PY{p}{]}\PY{p}{)}
\end{Verbatim}

    \begin{Verbatim}[commandchars=\\\{\}]
{\color{incolor}In [{\color{incolor}116}]:} \PY{k}{def} \PY{n+nf}{calculateDistinCtion2emeS}\PY{p}{(}\PY{n}{x}\PY{p}{)}\PY{p}{:}
              \PY{k}{if} \PY{n+nb}{str}\PY{p}{(}\PY{n}{x}\PY{p}{[}\PY{l+s+s1}{\PYZsq{}}\PY{l+s+s1}{MENT2}\PY{l+s+s1}{\PYZsq{}}\PY{p}{]}\PY{p}{)} \PY{o+ow}{in} \PY{n}{Distinction} \PY{p}{:}
                  \PY{k}{return} \PY{n}{pd}\PY{o}{.}\PY{n}{Series}\PY{p}{(}\PY{n}{x}\PY{p}{[}\PY{l+s+s1}{\PYZsq{}}\PY{l+s+s1}{MENT2}\PY{l+s+s1}{\PYZsq{}}\PY{p}{]}\PY{p}{)}
\end{Verbatim}

    \begin{Verbatim}[commandchars=\\\{\}]
{\color{incolor}In [{\color{incolor}156}]:} \PY{n}{PercFinal}\PY{o}{=}\PY{n}{dataset}\PY{o}{.}\PY{n}{apply}\PY{p}{(}\PY{n}{calculatePourcenatge}\PY{p}{,}\PY{n}{axis}\PY{o}{=}\PY{l+m+mi}{1}\PY{p}{)}
\end{Verbatim}

    \begin{Verbatim}[commandchars=\\\{\}]
{\color{incolor}In [{\color{incolor}157}]:} \PY{n}{PercFinal}\PY{o}{.}\PY{n}{loc}\PY{p}{[}\PY{n}{dataset}\PY{o}{.}\PY{n}{Pass1erSession}\PY{p}{]}
\end{Verbatim}

            \begin{Verbatim}[commandchars=\\\{\}]
{\color{outcolor}Out[{\color{outcolor}157}]:}               0
          4     65.099998
          7     62.500000
          8     65.000000
          9     60.400002
          10    68.000000
          17    62.900002
          18    68.800003
          23    62.400002
          28    66.500000
          34    65.099998
          36    65.800003
          37    63.799999
          39    60.599998
          40    64.199997
          42    66.500000
          46    66.000000
          47    70.199997
          48    61.799999
          57    76.400002
          60    60.200001
          61    64.099998
          62    67.500000
          63    64.099998
          67    59.700001
          71    61.599998
          74    62.799999
          78    62.000000
          80    65.199997
          82    61.599998
          89    64.800003
          {\ldots}         {\ldots}
          9381  66.599998
          9382  57.700001
          9383  58.400002
          9384  60.599998
          9387  63.799999
          9388  58.500000
          9399  61.799999
          9402  64.599998
          9403  63.700001
          9409  59.799999
          9423  74.900002
          9426  64.400002
          9438  70.000000
          9461  65.900002
          9480  57.400002
          9487  67.599998
          9489  64.699997
          9491  63.400002
          9500  67.800003
          9502  63.599998
          9508  63.599998
          9509  63.799999
          9513  61.500000
          9517  57.099998
          9525  57.900002
          9545  61.000000
          9578  60.799999
          9594  56.900002
          9595  59.799999
          9603  60.200001
          
          [2312 rows x 1 columns]
\end{Verbatim}
        
    \begin{Verbatim}[commandchars=\\\{\}]
{\color{incolor}In [{\color{incolor}168}]:} \PY{n}{dataset}\PY{o}{.}\PY{n}{loc}\PY{p}{[}\PY{p}{:}\PY{p}{,}\PY{l+s+s1}{\PYZsq{}}\PY{l+s+s1}{PercFinal}\PY{l+s+s1}{\PYZsq{}}\PY{p}{]}\PY{o}{=}\PY{n}{PercFinal}
          \PY{n}{dataset}\PY{p}{[}\PY{l+s+s1}{\PYZsq{}}\PY{l+s+s1}{PercFinal}\PY{l+s+s1}{\PYZsq{}}\PY{p}{]}\PY{o}{=}\PY{n}{PercFinal}
\end{Verbatim}

    il ya des étudiant qui on reussi mais qui n'ont pas de pourcentage final

    \begin{Verbatim}[commandchars=\\\{\}]
{\color{incolor}In [{\color{incolor}169}]:} \PY{n}{dataset}\PY{o}{.}\PY{n}{loc}\PY{p}{[}\PY{n}{operator}\PY{o}{.}\PY{n}{and\PYZus{}}\PY{p}{(}\PY{n}{np}\PY{o}{.}\PY{n}{isnan}\PY{p}{(}\PY{n}{PercFinal}\PY{o}{.}\PY{n}{get}\PY{p}{(}\PY{l+m+mi}{0}\PY{p}{)}\PY{p}{)} \PY{p}{,}\PY{n}{operator}\PY{o}{.}\PY{n}{or\PYZus{}}\PY{p}{(}\PY{n}{dataset}\PY{o}{.}\PY{n}{MENT2}\PY{o}{==}\PY{l+s+s1}{\PYZsq{}}\PY{l+s+s1}{S}\PY{l+s+s1}{\PYZsq{}}\PY{p}{,}\PY{n}{dataset}\PY{o}{.}\PY{n}{MENT1}\PY{o}{==}\PY{l+s+s1}{\PYZsq{}}\PY{l+s+s1}{S}\PY{l+s+s1}{\PYZsq{}}\PY{p}{)}\PY{p}{)}\PY{p}{]}
\end{Verbatim}

            \begin{Verbatim}[commandchars=\\\{\}]
{\color{outcolor}Out[{\color{outcolor}169}]:}         ID   ACADYEAR      PERC1   MENT1  PERC2 MENT2   FAC PROM  Echec  \textbackslash{}
          248   3978  2013-2014        NaN      AA    NaN     S  FSDC   L2  False   
          2176  7150  2012-2013  50.400002       A    NaN     S  FSEG   G2  False   
          2700  7530  2012-2013        NaN       S    NaN   NaN  FSTA   G1  False   
          2739  7569  2012-2013        NaN       S    NaN   NaN  FSTA   G1  False   
          2837  7630  2012-2013        NaN       S    NaN   NaN  FSTA   G1  False   
          3022  7750  2013-2014        NaN  ADSTFC    NaN     S    FD   G3  False   
          3100  7804  2013-2014        NaN      AA    NaN     S    FD   G3  False   
          3701  8300  2012-2013        NaN       S    NaN   NaN  FSTA   G0  False   
          4089  8477  2012-2013        NaN       S    NaN   NaN  FSTA   G0  False   
          6648  9688  2012-2013        NaN       S    NaN   NaN  FSTA   G0  False   
          
               Pass1erSession Distinction  PercFinal  
          248           False       False        NaN  
          2176          False       False        NaN  
          2700           True       False        NaN  
          2739           True       False        NaN  
          2837           True       False        NaN  
          3022          False       False        NaN  
          3100          False       False        NaN  
          3701           True       False        NaN  
          4089           True       False        NaN  
          6648           True       False        NaN  
\end{Verbatim}
        
    \begin{Verbatim}[commandchars=\\\{\}]
{\color{incolor}In [{\color{incolor}170}]:} \PY{n}{dataset}\PY{o}{.}\PY{n}{loc}\PY{p}{[}\PY{n}{operator}\PY{o}{.}\PY{n}{and\PYZus{}}\PY{p}{(}\PY{n}{np}\PY{o}{.}\PY{n}{isnan}\PY{p}{(}\PY{n}{PercFinal}\PY{o}{.}\PY{n}{get}\PY{p}{(}\PY{l+m+mi}{0}\PY{p}{)}\PY{p}{)} \PY{p}{,}\PY{n}{operator}\PY{o}{.}\PY{n}{or\PYZus{}}\PY{p}{(}\PY{n}{dataset}\PY{o}{.}\PY{n}{MENT2}\PY{o}{==}\PY{l+s+s1}{\PYZsq{}}\PY{l+s+s1}{S}\PY{l+s+s1}{\PYZsq{}}\PY{p}{,}\PY{n}{dataset}\PY{o}{.}\PY{n}{MENT1}\PY{o}{==}\PY{l+s+s1}{\PYZsq{}}\PY{l+s+s1}{S}\PY{l+s+s1}{\PYZsq{}}\PY{p}{)}\PY{p}{)}\PY{p}{,}\PY{l+s+s1}{\PYZsq{}}\PY{l+s+s1}{PercFinal}\PY{l+s+s1}{\PYZsq{}}\PY{p}{]}\PY{o}{=}\PY{l+m+mi}{55}
\end{Verbatim}

    \begin{Verbatim}[commandchars=\\\{\}]
{\color{incolor}In [{\color{incolor}192}]:} \PY{n}{dataset}\PY{o}{.}\PY{n}{loc}\PY{p}{[}\PY{n}{dataset}\PY{o}{.}\PY{n}{MENT2}\PY{o}{.}\PY{n}{isin}\PY{p}{(}\PY{n}{PartialSuccess}\PY{p}{)}\PY{p}{,}\PY{l+s+s1}{\PYZsq{}}\PY{l+s+s1}{PercFinal}\PY{l+s+s1}{\PYZsq{}}\PY{p}{]}\PY{o}{=}\PY{l+m+mi}{55}
\end{Verbatim}

    \begin{Verbatim}[commandchars=\\\{\}]
{\color{incolor}In [{\color{incolor}153}]:} \PY{n}{dataset}\PY{o}{.}\PY{n}{head}\PY{p}{(}\PY{l+m+mi}{8}\PY{p}{)}
\end{Verbatim}

            \begin{Verbatim}[commandchars=\\\{\}]
{\color{outcolor}Out[{\color{outcolor}153}]:}     ID   ACADYEAR      PERC1  MENT1      PERC2 MENT2   FAC PROM  Echec  \textbackslash{}
          0   45  2013-2014        NaN     AA        NaN   NaN  FPSE   L2   True   
          1  215  2012-2013        NaN    ADM  63.099998     S    FD   L2  False   
          2  343  2015-2016        NaN     AA  52.200001     A  FSEG   G2   True   
          3  356  2015-2016        NaN  ADSTM  59.900002     S  FSEG   L2  False   
          4  398  2012-2013  65.099998      S        NaN   NaN  FSDC   L2  False   
          5  429  2013-2014        NaN     AA        NaN     A    FD   G1   True   
          6  474  2014-2015        NaN     AA        NaN   NaN    FD   G3   True   
          7  474  2015-2016  62.500000      S        NaN   NaN    FD   G3  False   
          
            Pass1erSession Distinction  PercFinal  
          0          False       False        NaN  
          1          False       False  63.099998  
          2          False       False  52.200001  
          3          False       False  59.900002  
          4           True       False        NaN  
          5          False       False        NaN  
          6          False       False        NaN  
          7           True       False        NaN  
\end{Verbatim}
        
    Nous allons remplacer les valeurs manquantes des pourcentage final par
un nombre aleatoire symbolisant l'echec

    \begin{Verbatim}[commandchars=\\\{\}]
{\color{incolor}In [{\color{incolor}122}]:} \PY{n}{dataset}\PY{o}{.}\PY{n}{loc}\PY{p}{[}\PY{n}{dataset}\PY{o}{.}\PY{n}{Distinction}\PY{p}{]}\PY{p}{[}\PY{p}{[}\PY{l+s+s1}{\PYZsq{}}\PY{l+s+s1}{PERC1}\PY{l+s+s1}{\PYZsq{}}\PY{p}{,}\PY{l+s+s1}{\PYZsq{}}\PY{l+s+s1}{PERC2}\PY{l+s+s1}{\PYZsq{}}\PY{p}{,}\PY{l+s+s1}{\PYZsq{}}\PY{l+s+s1}{PercFinal}\PY{l+s+s1}{\PYZsq{}}\PY{p}{,}\PY{l+s+s1}{\PYZsq{}}\PY{l+s+s1}{MENT1}\PY{l+s+s1}{\PYZsq{}}\PY{p}{,}\PY{l+s+s1}{\PYZsq{}}\PY{l+s+s1}{MENT2}\PY{l+s+s1}{\PYZsq{}}\PY{p}{]}\PY{p}{]}\PY{o}{.}\PY{n}{MENT1}\PY{o}{.}\PY{n}{value\PYZus{}counts}\PY{p}{(}\PY{p}{)}
\end{Verbatim}

            \begin{Verbatim}[commandchars=\\\{\}]
{\color{outcolor}Out[{\color{outcolor}122}]:} D         268
          AA         74
          ADSTM       7
          GD          7
          A           5
          ADM         4
          ADTFC       4
          ADSTFC      2
          ADST        1
          Name: MENT1, dtype: int64
\end{Verbatim}
        
    \begin{Verbatim}[commandchars=\\\{\}]
{\color{incolor}In [{\color{incolor} }]:} \PY{n}{Verifions} \PY{n}{encore}
\end{Verbatim}

    \begin{Verbatim}[commandchars=\\\{\}]
{\color{incolor}In [{\color{incolor}172}]:} \PY{n}{dataset}\PY{o}{.}\PY{n}{loc}\PY{p}{[}\PY{n}{np}\PY{o}{.}\PY{n}{isnan}\PY{p}{(}\PY{n}{dataset}\PY{o}{.}\PY{n}{PercFinal}\PY{p}{)}\PY{p}{]}\PY{p}{[}\PY{l+s+s1}{\PYZsq{}}\PY{l+s+s1}{MENT2}\PY{l+s+s1}{\PYZsq{}}\PY{p}{]}\PY{o}{.}\PY{n}{value\PYZus{}counts}\PY{p}{(}\PY{p}{)}
\end{Verbatim}

            \begin{Verbatim}[commandchars=\\\{\}]
{\color{outcolor}Out[{\color{outcolor}172}]:} AA        515
          A         485
          NAF        46
          ADM        23
          ADSTM      19
          ADTFC       5
          ADST        5
          ANAF        3
          AR          2
          ADSTFC      2
          Name: MENT2, dtype: int64
\end{Verbatim}
        
    \begin{Verbatim}[commandchars=\\\{\}]
{\color{incolor}In [{\color{incolor}173}]:} \PY{n}{dataset}\PY{o}{.}\PY{n}{PercFinal}\PY{o}{=}\PY{n}{dataset}\PY{o}{.}\PY{n}{PercFinal}\PY{o}{.}\PY{n}{apply}\PY{p}{(}\PY{k}{lambda} \PY{n}{x}\PY{p}{:} \PY{n}{np}\PY{o}{.}\PY{n}{random}\PY{o}{.}\PY{n}{choice}\PY{p}{(}\PY{p}{[}\PY{n}{x} \PY{k}{for} \PY{n}{x} \PY{o+ow}{in} \PY{n+nb}{range}\PY{p}{(}\PY{l+m+mi}{40}\PY{p}{,}\PY{l+m+mi}{50}\PY{p}{)}\PY{p}{]}\PY{p}{)} \PY{k}{if} \PY{p}{(}\PY{n}{np}\PY{o}{.}\PY{n}{isnan}\PY{p}{(}\PY{n}{x}\PY{p}{)}\PY{p}{)} \PY{k}{else} \PY{n}{x}\PY{p}{)}
\end{Verbatim}

    \begin{Verbatim}[commandchars=\\\{\}]
{\color{incolor}In [{\color{incolor}174}]:} \PY{n}{dataset}\PY{o}{.}\PY{n}{head}\PY{p}{(}\PY{l+m+mi}{8}\PY{p}{)}
\end{Verbatim}

            \begin{Verbatim}[commandchars=\\\{\}]
{\color{outcolor}Out[{\color{outcolor}174}]:}     ID   ACADYEAR      PERC1  MENT1      PERC2 MENT2   FAC PROM  Echec  \textbackslash{}
          0   45  2013-2014        NaN     AA        NaN   NaN  FPSE   L2   True   
          1  215  2012-2013        NaN    ADM  63.099998     S    FD   L2  False   
          2  343  2015-2016        NaN     AA  52.200001     A  FSEG   G2   True   
          3  356  2015-2016        NaN  ADSTM  59.900002     S  FSEG   L2  False   
          4  398  2012-2013  65.099998      S        NaN   NaN  FSDC   L2  False   
          5  429  2013-2014        NaN     AA        NaN     A    FD   G1   True   
          6  474  2014-2015        NaN     AA        NaN   NaN    FD   G3   True   
          7  474  2015-2016  62.500000      S        NaN   NaN    FD   G3  False   
          
            Pass1erSession Distinction  PercFinal  
          0          False       False  43.000000  
          1          False       False  63.099998  
          2          False       False  52.200001  
          3          False       False  59.900002  
          4           True       False  65.099998  
          5          False       False  41.000000  
          6          False       False  46.000000  
          7           True       False  62.500000  
\end{Verbatim}
        
    Nous venons de finir le calcul de nos metriques d'evaluations groupons
enfin les metriques pour l'étudiant m

    \begin{Verbatim}[commandchars=\\\{\}]
{\color{incolor}In [{\color{incolor}175}]:}  \PY{k}{def} \PY{n+nf}{f}\PY{p}{(}\PY{n}{x}\PY{p}{)}\PY{p}{:}
              \PY{n}{A}\PY{o}{=}\PY{n+nb}{list}\PY{p}{(}\PY{n}{x}\PY{p}{[}\PY{l+s+s1}{\PYZsq{}}\PY{l+s+s1}{Echec}\PY{l+s+s1}{\PYZsq{}}\PY{p}{]}\PY{p}{)}
              \PY{n}{D}\PY{o}{=}\PY{n+nb}{list}\PY{p}{(}\PY{n}{x}\PY{p}{[}\PY{l+s+s1}{\PYZsq{}}\PY{l+s+s1}{Distinction}\PY{l+s+s1}{\PYZsq{}}\PY{p}{]}\PY{p}{)}
              \PY{n}{S1}\PY{o}{=}\PY{n+nb}{list}\PY{p}{(}\PY{n}{x}\PY{p}{[}\PY{l+s+s1}{\PYZsq{}}\PY{l+s+s1}{Pass1erSession}\PY{l+s+s1}{\PYZsq{}}\PY{p}{]}\PY{p}{)}
              \PY{n}{Nbre}\PY{o}{=}\PY{n+nb}{float}\PY{p}{(}\PY{n+nb}{len}\PY{p}{(}\PY{n}{x}\PY{p}{[}\PY{l+s+s1}{\PYZsq{}}\PY{l+s+s1}{ACADYEAR}\PY{l+s+s1}{\PYZsq{}}\PY{p}{]}\PY{p}{)}\PY{p}{)}
              \PY{k}{return} \PY{n}{pd}\PY{o}{.}\PY{n}{Series}\PY{p}{(}\PY{n+nb}{dict}\PY{p}{(}
               \PY{n}{EchecRatio}\PY{o}{=}\PY{n+nb}{int}\PY{p}{(}\PY{n}{A}\PY{o}{.}\PY{n}{count}\PY{p}{(}\PY{n+nb+bp}{True}\PY{p}{)}\PY{p}{)}\PY{o}{/}\PY{n}{Nbre}\PY{p}{,}
               \PY{n}{DistinctionRatio}\PY{o}{=}\PY{n+nb}{int}\PY{p}{(}\PY{n}{D}\PY{o}{.}\PY{n}{count}\PY{p}{(}\PY{n+nb+bp}{True}\PY{p}{)}\PY{p}{)}\PY{o}{/}\PY{n}{Nbre}\PY{p}{,}
               \PY{n}{Pass1stSessionRatio}\PY{o}{=}\PY{n+nb}{int}\PY{p}{(}\PY{n}{S1}\PY{o}{.}\PY{n}{count}\PY{p}{(}\PY{n+nb+bp}{True}\PY{p}{)}\PY{p}{)}\PY{o}{/}\PY{n}{Nbre}\PY{p}{,}
               \PY{n}{CGPA}\PY{o}{=}\PY{n}{np}\PY{o}{.}\PY{n}{mean}\PY{p}{(}\PY{n}{x}\PY{p}{[}\PY{l+s+s1}{\PYZsq{}}\PY{l+s+s1}{PercFinal}\PY{l+s+s1}{\PYZsq{}}\PY{p}{]}\PY{p}{)}\PY{p}{,}
               \PY{n}{NACADYEAR}\PY{o}{=}\PY{n+nb}{int}\PY{p}{(}\PY{n}{Nbre}\PY{p}{)}
              \PY{p}{)}\PY{p}{)}
\end{Verbatim}

    \begin{Verbatim}[commandchars=\\\{\}]
{\color{incolor}In [{\color{incolor}230}]:} \PY{n}{datasetRatio}\PY{o}{=}\PY{n}{dataset}\PY{o}{.}\PY{n}{groupby}\PY{p}{(}\PY{l+s+s1}{\PYZsq{}}\PY{l+s+s1}{ID}\PY{l+s+s1}{\PYZsq{}}\PY{p}{)}\PY{o}{.}\PY{n}{apply}\PY{p}{(}\PY{n}{f}\PY{p}{)}\PY{o}{.}\PY{n}{reset\PYZus{}index}\PY{p}{(}\PY{p}{)}
\end{Verbatim}

    \begin{Verbatim}[commandchars=\\\{\}]
{\color{incolor}In [{\color{incolor}231}]:} \PY{n}{datasetRatio}\PY{o}{.}\PY{n}{head}\PY{p}{(}\PY{l+m+mi}{8}\PY{p}{)}
\end{Verbatim}

            \begin{Verbatim}[commandchars=\\\{\}]
{\color{outcolor}Out[{\color{outcolor}231}]:}     ID       CGPA  DistinctionRatio  EchecRatio  NACADYEAR  \textbackslash{}
          0   45  43.000000               0.0         1.0        1.0   
          1  215  63.099998               0.0         0.0        1.0   
          2  343  52.200001               0.0         1.0        1.0   
          3  356  59.900002               0.0         0.0        1.0   
          4  398  65.099998               0.0         0.0        1.0   
          5  429  41.000000               0.0         1.0        1.0   
          6  474  54.250000               0.0         0.5        2.0   
          7  526  65.000000               0.0         0.0        1.0   
          
             Pass1stSessionRatio  
          0                  0.0  
          1                  0.0  
          2                  0.0  
          3                  0.0  
          4                  1.0  
          5                  0.0  
          6                  0.5  
          7                  1.0  
\end{Verbatim}
        
    \begin{Verbatim}[commandchars=\\\{\}]
{\color{incolor}In [{\color{incolor}212}]:} \PY{n}{datasetRatio}\PY{o}{.}\PY{n}{loc}\PY{p}{[}\PY{n}{operator}\PY{o}{.}\PY{n}{and\PYZus{}}\PY{p}{(}\PY{n}{datasetRatio}\PY{o}{.}\PY{n}{DistinctionRatio} \PY{o}{==} \PY{l+m+mi}{1} \PY{p}{,} \PY{n}{datasetRatio}\PY{o}{.}\PY{n}{NACADYEAR}\PY{o}{==}\PY{l+m+mi}{3}\PY{p}{)}\PY{p}{]}
\end{Verbatim}

            \begin{Verbatim}[commandchars=\\\{\}]
{\color{outcolor}Out[{\color{outcolor}212}]:}         ID       CGPA  DistinctionRatio  EchecRatio  NACADYEAR  \textbackslash{}
          211   4091  72.199999               1.0         0.0        3.0   
          671   5178  71.533333               1.0         0.0        3.0   
          1095  6914  74.900002               1.0         0.0        3.0   
          1928  8498  72.433334               1.0         0.0        3.0   
          2364  9048  71.733332               1.0         0.0        3.0   
          2393  9078  80.766665               1.0         0.0        3.0   
          2830  9527  75.866666               1.0         0.0        3.0   
          
                Pass1stSessionRatio  
          211              0.000000  
          671              0.666667  
          1095             1.000000  
          1928             1.000000  
          2364             0.333333  
          2393             1.000000  
          2830             1.000000  
\end{Verbatim}
        
    \begin{Verbatim}[commandchars=\\\{\}]
{\color{incolor}In [{\color{incolor}213}]:} \PY{n}{dataset}\PY{o}{.}\PY{n}{loc}\PY{p}{[}\PY{n}{dataset}\PY{o}{.}\PY{n}{ID}\PY{o}{==}\PY{l+m+mi}{9078}\PY{p}{]}
\end{Verbatim}

            \begin{Verbatim}[commandchars=\\\{\}]
{\color{outcolor}Out[{\color{outcolor}213}]:}         ID   ACADYEAR      PERC1 MENT1  PERC2 MENT2   FAC PROM  Echec  \textbackslash{}
          5311  9078  2013-2014  80.199997    GD    NaN   NaN  FSTA   G0  False   
          5312  9078  2014-2015  82.699997    GD    NaN   NaN  FSTA   G1  False   
          5313  9078  2015-2016  79.400002     D    NaN   NaN  FSTA   G2  False   
          
               Pass1erSession Distinction  PercFinal  
          5311           True        True  80.199997  
          5312           True        True  82.699997  
          5313           True        True  79.400002  
\end{Verbatim}
        
    Le modèle est parfait Nous pouvons maintenant passer à l'analyse des
données de sortie Pour nous 2 ensembles d'apprentissage

    \begin{Verbatim}[commandchars=\\\{\}]
{\color{incolor}In [{\color{incolor}233}]:} \PY{n}{dataset}\PY{o}{.}\PY{n}{to\PYZus{}csv}\PY{p}{(}\PY{l+s+s1}{\PYZsq{}}\PY{l+s+s1}{DatasetOutV2.csv}\PY{l+s+s1}{\PYZsq{}}\PY{p}{)}
          \PY{n}{datasetRatio}\PY{o}{.}\PY{n}{to\PYZus{}csv}\PY{p}{(}\PY{l+s+s1}{\PYZsq{}}\PY{l+s+s1}{DatasetOutRatioV1.csv}\PY{l+s+s1}{\PYZsq{}}\PY{p}{)}
\end{Verbatim}

    \begin{Verbatim}[commandchars=\\\{\}]
{\color{incolor}In [{\color{incolor}247}]:} \PY{n}{datasetRatio}\PY{o}{=}\PY{n}{pd}\PY{o}{.}\PY{n}{read\PYZus{}csv}\PY{p}{(}\PY{l+s+s1}{\PYZsq{}}\PY{l+s+s1}{DatasetOutRatioV1.csv}\PY{l+s+s1}{\PYZsq{}}\PY{p}{,}\PY{n}{index\PYZus{}col}\PY{o}{=}\PY{l+s+s1}{\PYZsq{}}\PY{l+s+s1}{Unnamed: 0}\PY{l+s+s1}{\PYZsq{}}\PY{p}{)}
\end{Verbatim}

    \begin{Verbatim}[commandchars=\\\{\}]
{\color{incolor}In [{\color{incolor}237}]:} \PY{n}{datasetRatio}\PY{o}{.}\PY{n}{head}\PY{p}{(}\PY{l+m+mi}{1}\PY{p}{)}
\end{Verbatim}

            \begin{Verbatim}[commandchars=\\\{\}]
{\color{outcolor}Out[{\color{outcolor}237}]:}    ID  CGPA  DistinctionRatio  EchecRatio  NACADYEAR  Pass1stSessionRatio
          0  45  43.0               0.0         1.0        1.0                  0.0
\end{Verbatim}
        
    \subsection{Analyse des données en entré Vs Donnnées en
Sortie}\label{analyse-des-donnuxe9es-en-entruxe9-vs-donnnuxe9es-en-sortie}

    Nous allons effectuer maintenat une analyse bi varié des données en
entrée VS les donnes en sortie et expliquer nos données de sortie

    Nous venons d'avoir notre ensemble d'apprentissage de sortie et disons
que notre sortie comprend que notre ensemble comprend non pas un seul
variable de sortie mais 4 variable de sortie qui sont :

\begin{itemize}
\tightlist
\item
  Le CGPA : La moyenne des points obtenus au cours d'une années
  académique
\item
  DistinctionRatio : Le nombre de fois que l'etudiant a distinctigué
  divisé par le nombre des années academique
\item
  EchecRatio : Le nombre de fois que l'etudiant a échoué divisé par le
  nombre des années academique
\item
  PassAstSessionRatio : Le nombre de fois que l'etudiant est passée en
  prèmiere session divisé par le nombre des années academique
\end{itemize}

    Nos 3 dernier ration sont des nombres variant qui peuvent prendre ses
differents valeurs : - 0 :signifie que l'etudiant a un ration de 0 - 1
:signifie ration: 1/1,2/2,3/3,4/4 -0.25 : 1/4 -0.3333: 1/3 -0.5:2/4 ou
1/2 -0.66666 :2/3 -0.75:3/4 Nous allons apprenhender ce probleme en
terme de classification donc nous allons affecter des lettre à ces
ration: - 0: G - 1:A - 0.25 : F - 0.333:E - 0.5:D - 0.666:C - 0.75:B qui
correspond au classe respectif!

    \begin{Verbatim}[commandchars=\\\{\}]
{\color{incolor}In [{\color{incolor}215}]:} \PY{n}{datasetRatio}\PY{o}{.}\PY{n}{columns}
\end{Verbatim}

            \begin{Verbatim}[commandchars=\\\{\}]
{\color{outcolor}Out[{\color{outcolor}215}]:} Index([u'ID', u'CGPA', u'DistinctionRatio', u'EchecRatio', u'NACADYEAR',
                 u'Pass1stSessionRatio'],
                dtype='object')
\end{Verbatim}
        
    \begin{Verbatim}[commandchars=\\\{\}]
{\color{incolor}In [{\color{incolor}257}]:} \PY{n}{datasetRatio}\PY{o}{.}\PY{n}{loc}\PY{p}{[}\PY{n}{datasetRatio}\PY{o}{.}\PY{n}{EchecRatio}\PY{o}{==}\PY{l+m+mf}{0.0}\PY{p}{,}\PY{l+s+s1}{\PYZsq{}}\PY{l+s+s1}{EchecRatio}\PY{l+s+s1}{\PYZsq{}}\PY{p}{]}\PY{o}{=}\PY{l+s+s1}{\PYZsq{}}\PY{l+s+s1}{G}\PY{l+s+s1}{\PYZsq{}}
          \PY{n}{datasetRatio}\PY{o}{.}\PY{n}{loc}\PY{p}{[}\PY{n}{datasetRatio}\PY{o}{.}\PY{n}{EchecRatio}\PY{o}{==}\PY{l+m+mf}{1.0}\PY{p}{,}\PY{l+s+s1}{\PYZsq{}}\PY{l+s+s1}{EchecRatio}\PY{l+s+s1}{\PYZsq{}}\PY{p}{]}\PY{o}{=}\PY{l+s+s1}{\PYZsq{}}\PY{l+s+s1}{A}\PY{l+s+s1}{\PYZsq{}}
          \PY{n}{datasetRatio}\PY{o}{.}\PY{n}{loc}\PY{p}{[}\PY{n}{datasetRatio}\PY{o}{.}\PY{n}{EchecRatio}\PY{o}{==}\PY{l+m+mf}{1.0}\PY{o}{/}\PY{l+m+mf}{4.0}\PY{p}{,}\PY{l+s+s1}{\PYZsq{}}\PY{l+s+s1}{EchecRatio}\PY{l+s+s1}{\PYZsq{}}\PY{p}{]}\PY{o}{=}\PY{l+s+s1}{\PYZsq{}}\PY{l+s+s1}{F}\PY{l+s+s1}{\PYZsq{}}
          \PY{n}{datasetRatio}\PY{o}{.}\PY{n}{loc}\PY{p}{[}\PY{n}{datasetRatio}\PY{o}{.}\PY{n}{EchecRatio}\PY{o}{==}\PY{l+m+mf}{0.333333333333}\PY{p}{,}\PY{l+s+s1}{\PYZsq{}}\PY{l+s+s1}{EchecRatio}\PY{l+s+s1}{\PYZsq{}}\PY{p}{]}\PY{o}{=}\PY{l+s+s1}{\PYZsq{}}\PY{l+s+s1}{E}\PY{l+s+s1}{\PYZsq{}}
          \PY{n}{datasetRatio}\PY{o}{.}\PY{n}{loc}\PY{p}{[}\PY{n}{datasetRatio}\PY{o}{.}\PY{n}{EchecRatio}\PY{o}{==}\PY{l+m+mf}{2.0}\PY{o}{/}\PY{l+m+mf}{4.0}\PY{p}{,}\PY{l+s+s1}{\PYZsq{}}\PY{l+s+s1}{EchecRatio}\PY{l+s+s1}{\PYZsq{}}\PY{p}{]}\PY{o}{=}\PY{l+s+s1}{\PYZsq{}}\PY{l+s+s1}{D}\PY{l+s+s1}{\PYZsq{}}
          \PY{n}{datasetRatio}\PY{o}{.}\PY{n}{loc}\PY{p}{[}\PY{n}{datasetRatio}\PY{o}{.}\PY{n}{EchecRatio}\PY{o}{==}\PY{l+m+mf}{0.666666666667}\PY{p}{,}\PY{l+s+s1}{\PYZsq{}}\PY{l+s+s1}{EchecRatio}\PY{l+s+s1}{\PYZsq{}}\PY{p}{]}\PY{o}{=}\PY{l+s+s1}{\PYZsq{}}\PY{l+s+s1}{C}\PY{l+s+s1}{\PYZsq{}}
          \PY{n}{datasetRatio}\PY{o}{.}\PY{n}{loc}\PY{p}{[}\PY{n}{datasetRatio}\PY{o}{.}\PY{n}{EchecRatio}\PY{o}{==}\PY{l+m+mf}{3.0}\PY{o}{/}\PY{l+m+mi}{4}\PY{p}{,}\PY{l+s+s1}{\PYZsq{}}\PY{l+s+s1}{EchecRatio}\PY{l+s+s1}{\PYZsq{}}\PY{p}{]}\PY{o}{=}\PY{l+s+s1}{\PYZsq{}}\PY{l+s+s1}{B}\PY{l+s+s1}{\PYZsq{}}
\end{Verbatim}

    \begin{Verbatim}[commandchars=\\\{\}]
{\color{incolor}In [{\color{incolor}258}]:} \PY{n}{datasetRatio}\PY{o}{.}\PY{n}{loc}\PY{p}{[}\PY{n}{datasetRatio}\PY{o}{.}\PY{n}{DistinctionRatio}\PY{o}{==}\PY{l+m+mf}{0.0}\PY{p}{,}\PY{l+s+s1}{\PYZsq{}}\PY{l+s+s1}{DistinctionRatio}\PY{l+s+s1}{\PYZsq{}}\PY{p}{]}\PY{o}{=}\PY{l+s+s1}{\PYZsq{}}\PY{l+s+s1}{G}\PY{l+s+s1}{\PYZsq{}}
          \PY{n}{datasetRatio}\PY{o}{.}\PY{n}{loc}\PY{p}{[}\PY{n}{datasetRatio}\PY{o}{.}\PY{n}{DistinctionRatio}\PY{o}{==}\PY{l+m+mf}{1.0}\PY{p}{,}\PY{l+s+s1}{\PYZsq{}}\PY{l+s+s1}{DistinctionRatio}\PY{l+s+s1}{\PYZsq{}}\PY{p}{]}\PY{o}{=}\PY{l+s+s1}{\PYZsq{}}\PY{l+s+s1}{A}\PY{l+s+s1}{\PYZsq{}}
          \PY{n}{datasetRatio}\PY{o}{.}\PY{n}{loc}\PY{p}{[}\PY{n}{datasetRatio}\PY{o}{.}\PY{n}{DistinctionRatio}\PY{o}{==}\PY{l+m+mf}{1.0}\PY{o}{/}\PY{l+m+mf}{4.0}\PY{p}{,}\PY{l+s+s1}{\PYZsq{}}\PY{l+s+s1}{DistinctionRatio}\PY{l+s+s1}{\PYZsq{}}\PY{p}{]}\PY{o}{=}\PY{l+s+s1}{\PYZsq{}}\PY{l+s+s1}{F}\PY{l+s+s1}{\PYZsq{}}
          \PY{n}{datasetRatio}\PY{o}{.}\PY{n}{loc}\PY{p}{[}\PY{n}{datasetRatio}\PY{o}{.}\PY{n}{DistinctionRatio}\PY{o}{==}\PY{l+m+mf}{0.333333333333}\PY{p}{,}\PY{l+s+s1}{\PYZsq{}}\PY{l+s+s1}{DistinctionRatio}\PY{l+s+s1}{\PYZsq{}}\PY{p}{]}\PY{o}{=}\PY{l+s+s1}{\PYZsq{}}\PY{l+s+s1}{E}\PY{l+s+s1}{\PYZsq{}}
          \PY{n}{datasetRatio}\PY{o}{.}\PY{n}{loc}\PY{p}{[}\PY{n}{datasetRatio}\PY{o}{.}\PY{n}{DistinctionRatio}\PY{o}{==}\PY{l+m+mf}{2.0}\PY{o}{/}\PY{l+m+mf}{4.0}\PY{p}{,}\PY{l+s+s1}{\PYZsq{}}\PY{l+s+s1}{DistinctionRatio}\PY{l+s+s1}{\PYZsq{}}\PY{p}{]}\PY{o}{=}\PY{l+s+s1}{\PYZsq{}}\PY{l+s+s1}{D}\PY{l+s+s1}{\PYZsq{}}
          \PY{n}{datasetRatio}\PY{o}{.}\PY{n}{loc}\PY{p}{[}\PY{n}{datasetRatio}\PY{o}{.}\PY{n}{DistinctionRatio}\PY{o}{==}\PY{l+m+mf}{0.666666666667}\PY{p}{,}\PY{l+s+s1}{\PYZsq{}}\PY{l+s+s1}{DistinctionRatio}\PY{l+s+s1}{\PYZsq{}}\PY{p}{]}\PY{o}{=}\PY{l+s+s1}{\PYZsq{}}\PY{l+s+s1}{C}\PY{l+s+s1}{\PYZsq{}}
          \PY{n}{datasetRatio}\PY{o}{.}\PY{n}{loc}\PY{p}{[}\PY{n}{datasetRatio}\PY{o}{.}\PY{n}{DistinctionRatio}\PY{o}{==}\PY{l+m+mf}{3.0}\PY{o}{/}\PY{l+m+mi}{4}\PY{p}{,}\PY{l+s+s1}{\PYZsq{}}\PY{l+s+s1}{DistinctionRatio}\PY{l+s+s1}{\PYZsq{}}\PY{p}{]}\PY{o}{=}\PY{l+s+s1}{\PYZsq{}}\PY{l+s+s1}{B}\PY{l+s+s1}{\PYZsq{}}
\end{Verbatim}

    \begin{Verbatim}[commandchars=\\\{\}]
{\color{incolor}In [{\color{incolor}264}]:} \PY{n}{datasetRatio}\PY{o}{.}\PY{n}{loc}\PY{p}{[}\PY{n}{datasetRatio}\PY{o}{.}\PY{n}{Pass1stSessionRatio}\PY{o}{==}\PY{l+m+mf}{0.0}\PY{p}{,}\PY{l+s+s1}{\PYZsq{}}\PY{l+s+s1}{Pass1stSessionRatio}\PY{l+s+s1}{\PYZsq{}}\PY{p}{]}\PY{o}{=}\PY{l+s+s1}{\PYZsq{}}\PY{l+s+s1}{G}\PY{l+s+s1}{\PYZsq{}}
          \PY{n}{datasetRatio}\PY{o}{.}\PY{n}{loc}\PY{p}{[}\PY{n}{datasetRatio}\PY{o}{.}\PY{n}{Pass1stSessionRatio}\PY{o}{==}\PY{l+m+mf}{1.0}\PY{p}{,}\PY{l+s+s1}{\PYZsq{}}\PY{l+s+s1}{Pass1stSessionRatio}\PY{l+s+s1}{\PYZsq{}}\PY{p}{]}\PY{o}{=}\PY{l+s+s1}{\PYZsq{}}\PY{l+s+s1}{A}\PY{l+s+s1}{\PYZsq{}}
          \PY{n}{datasetRatio}\PY{o}{.}\PY{n}{loc}\PY{p}{[}\PY{n}{datasetRatio}\PY{o}{.}\PY{n}{Pass1stSessionRatio}\PY{o}{==}\PY{l+m+mf}{1.0}\PY{o}{/}\PY{l+m+mf}{4.0}\PY{p}{,}\PY{l+s+s1}{\PYZsq{}}\PY{l+s+s1}{Pass1stSessionRatio}\PY{l+s+s1}{\PYZsq{}}\PY{p}{]}\PY{o}{=}\PY{l+s+s1}{\PYZsq{}}\PY{l+s+s1}{F}\PY{l+s+s1}{\PYZsq{}}
          \PY{n}{datasetRatio}\PY{o}{.}\PY{n}{loc}\PY{p}{[}\PY{n}{datasetRatio}\PY{o}{.}\PY{n}{Pass1stSessionRatio}\PY{o}{==}\PY{l+m+mf}{0.333333333333}\PY{p}{,}\PY{l+s+s1}{\PYZsq{}}\PY{l+s+s1}{Pass1stSessionRatio}\PY{l+s+s1}{\PYZsq{}}\PY{p}{]}\PY{o}{=}\PY{l+s+s1}{\PYZsq{}}\PY{l+s+s1}{E}\PY{l+s+s1}{\PYZsq{}}
          \PY{n}{datasetRatio}\PY{o}{.}\PY{n}{loc}\PY{p}{[}\PY{n}{datasetRatio}\PY{o}{.}\PY{n}{Pass1stSessionRatio}\PY{o}{==}\PY{l+m+mf}{2.0}\PY{o}{/}\PY{l+m+mf}{4.0}\PY{p}{,}\PY{l+s+s1}{\PYZsq{}}\PY{l+s+s1}{Pass1stSessionRatio}\PY{l+s+s1}{\PYZsq{}}\PY{p}{]}\PY{o}{=}\PY{l+s+s1}{\PYZsq{}}\PY{l+s+s1}{D}\PY{l+s+s1}{\PYZsq{}}
          \PY{n}{datasetRatio}\PY{o}{.}\PY{n}{loc}\PY{p}{[}\PY{n}{datasetRatio}\PY{o}{.}\PY{n}{Pass1stSessionRatio}\PY{o}{==}\PY{l+m+mf}{0.666666666667}\PY{p}{,}\PY{l+s+s1}{\PYZsq{}}\PY{l+s+s1}{Pass1stSessionRatio}\PY{l+s+s1}{\PYZsq{}}\PY{p}{]}\PY{o}{=}\PY{l+s+s1}{\PYZsq{}}\PY{l+s+s1}{C}\PY{l+s+s1}{\PYZsq{}}
          \PY{n}{datasetRatio}\PY{o}{.}\PY{n}{loc}\PY{p}{[}\PY{n}{datasetRatio}\PY{o}{.}\PY{n}{Pass1stSessionRatio}\PY{o}{==}\PY{l+m+mf}{3.0}\PY{o}{/}\PY{l+m+mi}{4}\PY{p}{,}\PY{l+s+s1}{\PYZsq{}}\PY{l+s+s1}{Pass1stSessionRatio}\PY{l+s+s1}{\PYZsq{}}\PY{p}{]}\PY{o}{=}\PY{l+s+s1}{\PYZsq{}}\PY{l+s+s1}{B}\PY{l+s+s1}{\PYZsq{}}
\end{Verbatim}

    \begin{Verbatim}[commandchars=\\\{\}]
{\color{incolor}In [{\color{incolor}263}]:} \PY{n}{datasetRatio}\PY{o}{.}\PY{n}{head}\PY{p}{(}\PY{l+m+mi}{8}\PY{p}{)}
\end{Verbatim}

            \begin{Verbatim}[commandchars=\\\{\}]
{\color{outcolor}Out[{\color{outcolor}263}]:}     ID       CGPA DistinctionRatio EchecRatio  NACADYEAR Pass1stSessionRatio
          0   45  43.000000                G          A        1.0                   G
          1  215  63.099998                G          G        1.0                   G
          2  343  52.200001                G          A        1.0                   G
          3  356  59.900002                G          G        1.0                   G
          4  398  65.099998                G          G        1.0                   A
          5  429  41.000000                G          A        1.0                   G
          6  474  54.250000                G          D        2.0                   D
          7  526  65.000000                G          G        1.0                   A
\end{Verbatim}
        
    \begin{Verbatim}[commandchars=\\\{\}]
{\color{incolor}In [{\color{incolor} }]:} \PY{n}{datasetRatio}\PY{o}{.}\PY{n}{loc}\PY{p}{[}\PY{n}{datasetRatio}\PY{o}{.}\PY{n}{EchecRatio}\PY{o}{==}\PY{l+m+mf}{0.0}\PY{p}{,}\PY{l+s+s1}{\PYZsq{}}\PY{l+s+s1}{EchecRatio}\PY{l+s+s1}{\PYZsq{}}\PY{p}{]}\PY{o}{=}\PY{l+s+s1}{\PYZsq{}}\PY{l+s+s1}{G}\PY{l+s+s1}{\PYZsq{}}
        \PY{n}{datasetRatio}\PY{o}{.}\PY{n}{loc}\PY{p}{[}\PY{n}{datasetRatio}\PY{o}{.}\PY{n}{EchecRatio}\PY{o}{==}\PY{l+m+mf}{1.0}\PY{p}{,}\PY{l+s+s1}{\PYZsq{}}\PY{l+s+s1}{EchecRatio}\PY{l+s+s1}{\PYZsq{}}\PY{p}{]}\PY{o}{=}\PY{l+s+s1}{\PYZsq{}}\PY{l+s+s1}{A}\PY{l+s+s1}{\PYZsq{}}
        \PY{n}{datasetRatio}\PY{o}{.}\PY{n}{loc}\PY{p}{[}\PY{n}{datasetRatio}\PY{o}{.}\PY{n}{EchecRatio}\PY{o}{==}\PY{l+m+mf}{1.0}\PY{o}{/}\PY{l+m+mf}{4.0}\PY{p}{,}\PY{l+s+s1}{\PYZsq{}}\PY{l+s+s1}{EchecRatio}\PY{l+s+s1}{\PYZsq{}}\PY{p}{]}\PY{o}{=}\PY{l+s+s1}{\PYZsq{}}\PY{l+s+s1}{F}\PY{l+s+s1}{\PYZsq{}}
        \PY{n}{datasetRatio}\PY{o}{.}\PY{n}{loc}\PY{p}{[}\PY{n}{datasetRatio}\PY{o}{.}\PY{n}{EchecRatio}\PY{o}{==}\PY{l+m+mf}{1.0}\PY{o}{/}\PY{l+m+mf}{3.0}\PY{p}{,}\PY{l+s+s1}{\PYZsq{}}\PY{l+s+s1}{EchecRatio}\PY{l+s+s1}{\PYZsq{}}\PY{p}{]}\PY{o}{=}\PY{l+s+s1}{\PYZsq{}}\PY{l+s+s1}{E}\PY{l+s+s1}{\PYZsq{}}
        \PY{n}{datasetRatio}\PY{o}{.}\PY{n}{loc}\PY{p}{[}\PY{n}{datasetRatio}\PY{o}{.}\PY{n}{EchecRatio}\PY{o}{==}\PY{l+m+mf}{2.0}\PY{o}{/}\PY{l+m+mf}{4.0}\PY{p}{,}\PY{l+s+s1}{\PYZsq{}}\PY{l+s+s1}{EchecRatio}\PY{l+s+s1}{\PYZsq{}}\PY{p}{]}\PY{o}{=}\PY{l+s+s1}{\PYZsq{}}\PY{l+s+s1}{D}\PY{l+s+s1}{\PYZsq{}}
        \PY{n}{datasetRatio}\PY{o}{.}\PY{n}{loc}\PY{p}{[}\PY{n}{datasetRatio}\PY{o}{.}\PY{n}{EchecRatio}\PY{o}{==}\PY{l+m+mf}{2.0}\PY{o}{/}\PY{l+m+mf}{3.0}\PY{p}{,}\PY{l+s+s1}{\PYZsq{}}\PY{l+s+s1}{EchecRatio}\PY{l+s+s1}{\PYZsq{}}\PY{p}{]}\PY{o}{=}\PY{l+s+s1}{\PYZsq{}}\PY{l+s+s1}{C}\PY{l+s+s1}{\PYZsq{}}
        \PY{n}{datasetRatio}\PY{o}{.}\PY{n}{loc}\PY{p}{[}\PY{n}{datasetRatio}\PY{o}{.}\PY{n}{EchecRatio}\PY{o}{==}\PY{l+m+mf}{3.0}\PY{o}{/}\PY{l+m+mi}{4}\PY{p}{,}\PY{l+s+s1}{\PYZsq{}}\PY{l+s+s1}{EchecRatio}\PY{l+s+s1}{\PYZsq{}}\PY{p}{]}\PY{o}{=}\PY{l+s+s1}{\PYZsq{}}\PY{l+s+s1}{B}\PY{l+s+s1}{\PYZsq{}}
\end{Verbatim}

    Combinons maitenant notre ensemble d'entre avec notre ensemble de sortie

    \begin{Verbatim}[commandchars=\\\{\}]
{\color{incolor}In [{\color{incolor}284}]:} \PY{n}{datasetRatio}\PY{o}{.}\PY{n}{groupby}\PY{p}{(}\PY{l+s+s1}{\PYZsq{}}\PY{l+s+s1}{DistinctionRatio}\PY{l+s+s1}{\PYZsq{}}\PY{p}{)}\PY{p}{[}\PY{l+s+s1}{\PYZsq{}}\PY{l+s+s1}{CGPA}\PY{l+s+s1}{\PYZsq{}}\PY{p}{]}\PY{o}{.}\PY{n}{mean}\PY{p}{(}\PY{p}{)}
\end{Verbatim}

            \begin{Verbatim}[commandchars=\\\{\}]
{\color{outcolor}Out[{\color{outcolor}284}]:} DistinctionRatio
          A    73.105980
          B    71.425000
          C    69.649275
          D    68.470455
          E    66.661364
          F    65.686905
          G    56.424207
          Name: CGPA, dtype: float64
\end{Verbatim}
        
    \begin{Verbatim}[commandchars=\\\{\}]
{\color{incolor}In [{\color{incolor}283}]:} \PY{n}{datasetRatio}\PY{o}{.}\PY{n}{loc}\PY{p}{[}\PY{n}{datasetRatio}\PY{o}{.}\PY{n}{DistinctionRatio}\PY{o}{\PYZlt{}}\PY{o}{=}\PY{l+m+mf}{0.333333333333}\PY{p}{,}\PY{l+s+s1}{\PYZsq{}}\PY{l+s+s1}{DistinctionRatio}\PY{l+s+s1}{\PYZsq{}}\PY{p}{]}\PY{o}{=}\PY{l+s+s1}{\PYZsq{}}\PY{l+s+s1}{E}\PY{l+s+s1}{\PYZsq{}}
\end{Verbatim}

    \begin{Verbatim}[commandchars=\\\{\}]
{\color{incolor}In [{\color{incolor}285}]:} \PY{n}{datasetRatio}\PY{o}{.}\PY{n}{loc}\PY{p}{[}\PY{n}{datasetRatio}\PY{o}{.}\PY{n}{Pass1stSessionRatio}\PY{o}{\PYZlt{}}\PY{o}{=}\PY{l+m+mf}{0.333333333333}\PY{p}{,}\PY{l+s+s1}{\PYZsq{}}\PY{l+s+s1}{Pass1stSessionRatio}\PY{l+s+s1}{\PYZsq{}}\PY{p}{]}\PY{o}{=}\PY{l+s+s1}{\PYZsq{}}\PY{l+s+s1}{E}\PY{l+s+s1}{\PYZsq{}}
\end{Verbatim}

    \begin{Verbatim}[commandchars=\\\{\}]
{\color{incolor}In [{\color{incolor}289}]:} \PY{n}{datasetRatio}\PY{o}{.}\PY{n}{loc}\PY{p}{[}\PY{n}{datasetRatio}\PY{o}{.}\PY{n}{EchecRatio}\PY{o}{\PYZlt{}}\PY{o}{=}\PY{l+m+mf}{0.333333333333}\PY{p}{,}\PY{l+s+s1}{\PYZsq{}}\PY{l+s+s1}{EchecRatio}\PY{l+s+s1}{\PYZsq{}}\PY{p}{]}\PY{o}{=}\PY{l+s+s1}{\PYZsq{}}\PY{l+s+s1}{E}\PY{l+s+s1}{\PYZsq{}}
\end{Verbatim}

    \begin{Verbatim}[commandchars=\\\{\}]
{\color{incolor}In [{\color{incolor}287}]:} \PY{n}{datasetRatio}\PY{o}{.}\PY{n}{groupby}\PY{p}{(}\PY{l+s+s1}{\PYZsq{}}\PY{l+s+s1}{Pass1stSessionRatio}\PY{l+s+s1}{\PYZsq{}}\PY{p}{)}\PY{p}{[}\PY{l+s+s1}{\PYZsq{}}\PY{l+s+s1}{CGPA}\PY{l+s+s1}{\PYZsq{}}\PY{p}{]}\PY{o}{.}\PY{n}{mean}\PY{p}{(}\PY{p}{)}
\end{Verbatim}

            \begin{Verbatim}[commandchars=\\\{\}]
{\color{outcolor}Out[{\color{outcolor}287}]:} Pass1stSessionRatio
          A    64.332685
          B    64.482927
          C    62.619497
          D    62.164337
          E    60.775089
          F    60.152778
          G    54.513650
          Name: CGPA, dtype: float64
\end{Verbatim}
        
    \begin{Verbatim}[commandchars=\\\{\}]
{\color{incolor}In [{\color{incolor}290}]:} \PY{n}{datasetRatio}\PY{o}{.}\PY{n}{groupby}\PY{p}{(}\PY{l+s+s1}{\PYZsq{}}\PY{l+s+s1}{EchecRatio}\PY{l+s+s1}{\PYZsq{}}\PY{p}{)}\PY{p}{[}\PY{l+s+s1}{\PYZsq{}}\PY{l+s+s1}{CGPA}\PY{l+s+s1}{\PYZsq{}}\PY{p}{]}\PY{o}{.}\PY{n}{mean}\PY{p}{(}\PY{p}{)}
\end{Verbatim}

            \begin{Verbatim}[commandchars=\\\{\}]
{\color{outcolor}Out[{\color{outcolor}290}]:} EchecRatio
          A    45.826616
          C    51.176271
          D    52.975986
          E    55.368557
          F    58.668333
          G    61.003406
          Name: CGPA, dtype: float64
\end{Verbatim}
        
    Nous venos de voir la repartion des moyenne des GPA dans les les
categoriesn de nos differentes valeurs des differentes mentions

    Passons maitenant a l'analyse des nos valeurs d'entrée face au GPA

    \begin{Verbatim}[commandchars=\\\{\}]
{\color{incolor}In [{\color{incolor}319}]:} \PY{n}{datasetIn}\PY{o}{=}\PY{n}{pd}\PY{o}{.}\PY{n}{read\PYZus{}csv}\PY{p}{(}\PY{l+s+s1}{\PYZsq{}}\PY{l+s+s1}{DatasetInClean.csv}\PY{l+s+s1}{\PYZsq{}}\PY{p}{,}\PY{n}{index\PYZus{}col}\PY{o}{=}\PY{l+s+s1}{\PYZsq{}}\PY{l+s+s1}{Unnamed: 0}\PY{l+s+s1}{\PYZsq{}}\PY{p}{)}
\end{Verbatim}

    \begin{Verbatim}[commandchars=\\\{\}]
{\color{incolor}In [{\color{incolor}320}]:} \PY{n}{datasetIn}\PY{o}{.}\PY{n}{head}\PY{p}{(}\PY{l+m+mi}{8}\PY{p}{)}
\end{Verbatim}

            \begin{Verbatim}[commandchars=\\\{\}]
{\color{outcolor}Out[{\color{outcolor}320}]:}    IDENTIFICATION SCHOOLSTATUS SCHOOL\_RIGHT         OPTION\_RIGHT   FAC  \textbackslash{}
          0            3895   protestant       zanner  commmerciale et adm  FSEG   
          1            4048   protestant       zanner  commmerciale et adm  FSEG   
          2            4217   protestant       zanner  commmerciale et adm  FSEG   
          3            4347   protestant       zanner  commmerciale et adm  FSEG   
          4            4409   protestant       zanner  commmerciale et adm  FSEG   
          5            4473   protestant       zanner  commmerciale et adm  FSEG   
          6            4627   protestant       zanner          latin philo    FD   
          7            4649   protestant       zanner          latin philo    FD   
          
            SCHOOLPROVINCE    BIRTHDAY GENDER  DIPLOMPERCENTAGE  AGE  
          0      NORD-KIVU  1985-07-16      H              52.0   31  
          1      NORD-KIVU  1987-04-20      H              53.0   30  
          2      NORD-KIVU  1989-01-22      H              54.0   28  
          3      NORD-KIVU  1987-07-11      H              53.0   29  
          4      NORD-KIVU  1989-11-19      H              58.0   27  
          5      NORD-KIVU  1986-03-18      H              52.0   31  
          6      NORD-KIVU  1988-03-16      F              52.0   29  
          7      NORD-KIVU  1990-10-08      F              53.0   26  
\end{Verbatim}
        
    \begin{Verbatim}[commandchars=\\\{\}]
{\color{incolor}In [{\color{incolor}291}]:} \PY{k+kn}{import} \PY{n+nn}{sys}
          \PY{n+nb}{reload}\PY{p}{(}\PY{n}{sys}\PY{p}{)}
          \PY{n}{sys}\PY{o}{.}\PY{n}{setdefaultencoding}\PY{p}{(}\PY{l+s+s1}{\PYZsq{}}\PY{l+s+s1}{utf\PYZhy{}8}\PY{l+s+s1}{\PYZsq{}}\PY{p}{)}
\end{Verbatim}

    \begin{Verbatim}[commandchars=\\\{\}]
{\color{incolor}In [{\color{incolor}321}]:} \PY{n}{datasetIn}\PY{o}{.}\PY{n}{rename}\PY{p}{(}\PY{n}{columns}\PY{o}{=}\PY{p}{\PYZob{}}\PY{l+s+s1}{\PYZsq{}}\PY{l+s+s1}{IDENTIFICATION}\PY{l+s+s1}{\PYZsq{}}\PY{p}{:}\PY{l+s+s1}{\PYZsq{}}\PY{l+s+s1}{ID}\PY{l+s+s1}{\PYZsq{}}\PY{p}{\PYZcb{}}\PY{p}{,}\PY{n}{inplace}\PY{o}{=}\PY{n+nb+bp}{True}\PY{p}{)}
\end{Verbatim}

    \begin{Verbatim}[commandchars=\\\{\}]
{\color{incolor}In [{\color{incolor}322}]:} \PY{n}{datasetIn}\PY{o}{.}\PY{n}{drop}\PY{p}{(}\PY{n}{labels}\PY{o}{=}\PY{p}{[}\PY{l+s+s1}{\PYZsq{}}\PY{l+s+s1}{BIRTHDAY}\PY{l+s+s1}{\PYZsq{}}\PY{p}{]}\PY{p}{,}\PY{n}{axis}\PY{o}{=}\PY{l+m+mi}{1}\PY{p}{,}\PY{n}{inplace}\PY{o}{=}\PY{n+nb+bp}{True}\PY{p}{)}
\end{Verbatim}

    \begin{Verbatim}[commandchars=\\\{\}]
{\color{incolor}In [{\color{incolor}323}]:} \PY{n}{datasetIn}\PY{o}{.}\PY{n}{rename}\PY{p}{(}\PY{n}{columns}\PY{o}{=}\PY{p}{\PYZob{}}\PY{l+s+s1}{\PYZsq{}}\PY{l+s+s1}{DIPLOMPERCENTAGE}\PY{l+s+s1}{\PYZsq{}}\PY{p}{:}\PY{l+s+s1}{\PYZsq{}}\PY{l+s+s1}{DIPPERC}\PY{l+s+s1}{\PYZsq{}}\PY{p}{\PYZcb{}}\PY{p}{,}\PY{n}{inplace}\PY{o}{=}\PY{n+nb+bp}{True}\PY{p}{)}
\end{Verbatim}

    \begin{Verbatim}[commandchars=\\\{\}]
{\color{incolor}In [{\color{incolor}324}]:} \PY{n}{datasetFin}\PY{o}{=}\PY{n}{pd}\PY{o}{.}\PY{n}{merge}\PY{p}{(}\PY{n}{datasetIn}\PY{p}{,}\PY{n}{datasetRatio}\PY{p}{,}\PY{n}{on}\PY{o}{=}\PY{l+s+s1}{\PYZsq{}}\PY{l+s+s1}{ID}\PY{l+s+s1}{\PYZsq{}}\PY{p}{)}
\end{Verbatim}

    \begin{Verbatim}[commandchars=\\\{\}]
{\color{incolor}In [{\color{incolor}325}]:} \PY{n}{datasetFin}\PY{o}{.}\PY{n}{to\PYZus{}csv}\PY{p}{(}\PY{l+s+s1}{\PYZsq{}}\PY{l+s+s1}{DatasetFinalV1}\PY{l+s+s1}{\PYZsq{}}\PY{p}{)}
\end{Verbatim}

    \begin{Verbatim}[commandchars=\\\{\}]
{\color{incolor}In [{\color{incolor}326}]:} \PY{n}{datasetFin}\PY{o}{.}\PY{n}{describe}\PY{p}{(}\PY{p}{)}
\end{Verbatim}

            \begin{Verbatim}[commandchars=\\\{\}]
{\color{outcolor}Out[{\color{outcolor}326}]:}                  ID      DIPPERC          AGE         CGPA    NACADYEAR
          count   4715.000000  4715.000000  4715.000000  4715.000000  4715.000000
          mean    8720.744221    56.878914    24.732768    57.324192     1.913892
          std     2451.528838     5.756663     4.621602     7.250374     0.905487
          min       45.000000    50.000000    18.000000    35.700001     1.000000
          25\%     7149.000000    52.000000    22.000000    53.599998     1.000000
          50\%     9172.000000    56.000000    24.000000    58.599998     2.000000
          75\%    10579.500000    60.000000    27.000000    62.099998     3.000000
          max    12360.000000    86.000000    59.000000    83.549999     4.000000
\end{Verbatim}
        
    Analyse dansla partie precdeante nous allons faire l'analyse Bi-varié à
8 niveau: 1. CGPA-Diplome Province : Pour voir la relation entre le CGPA
et la province d'origine de l'etudiant 2. CGPA-DIPPOURCENTAGE: Pour voir
la relation entre le CGPA et la pourcentage obtenu à l'exetat 3.
CGPA-AGE: Pour voir la relation entre l'age de l'etudiant et le CGPA 4.
CGPA-FAC-DIPLOMEOPTION: Pour voir la relation entre le CGPA et l'option
du diplome mais par rapport aux faculté choisie 5. CGPA-FAC-SEXE: Pour
la relation avec le sexe des étudiants et le CGPA Optenu 6.
CGPA-FAC-SCHOOL: pour la relation entre l'ecole de provenance la fac 7.
GPA-FAC-SCOOLSTATUS : pour la relation entre le status de l'ecole de la
FAC 8. FAC-CGPA

    \subsubsection{CGPA-DIPLOME POURCENTAGE}\label{cgpa-diplome-pourcentage}

    Pour evaluer ce genre de relation nous allons utliser le coeficient de
correlation de pearson:

    \begin{Verbatim}[commandchars=\\\{\}]
{\color{incolor}In [{\color{incolor}327}]:} \PY{k+kn}{from} \PY{n+nn}{pydoc} \PY{k+kn}{import} \PY{n}{help}
          \PY{k+kn}{from} \PY{n+nn}{scipy.stats.stats} \PY{k+kn}{import} \PY{n}{pearsonr}
          \PY{n}{help}\PY{p}{(}\PY{n}{pearsonr}\PY{p}{)}
\end{Verbatim}

    \begin{Verbatim}[commandchars=\\\{\}]
{\color{incolor}In [{\color{incolor}328}]:} \PY{n}{pearsonr}\PY{p}{(}\PY{n}{datasetFin}\PY{o}{.}\PY{n}{CGPA}\PY{p}{,}\PY{n}{datasetFin}\PY{o}{.}\PY{n}{DIPPERC}\PY{p}{)}
\end{Verbatim}

            \begin{Verbatim}[commandchars=\\\{\}]
{\color{outcolor}Out[{\color{outcolor}328}]:} (0.23255606339266433, 6.2451652148277288e-59)
\end{Verbatim}
        
    \begin{Verbatim}[commandchars=\\\{\}]
{\color{incolor}In [{\color{incolor}329}]:} \PY{n}{plt}\PY{o}{.}\PY{n}{figure}\PY{p}{(}\PY{n}{figsize}\PY{o}{=}\PY{p}{(}\PY{l+m+mi}{10}\PY{p}{,}\PY{l+m+mi}{10}\PY{p}{)}\PY{p}{)}
          \PY{n}{ax} \PY{o}{=} \PY{n}{sns}\PY{o}{.}\PY{n}{regplot}\PY{p}{(}\PY{n}{x}\PY{o}{=}\PY{l+s+s2}{\PYZdq{}}\PY{l+s+s2}{CGPA}\PY{l+s+s2}{\PYZdq{}}\PY{p}{,} \PY{n}{y}\PY{o}{=}\PY{l+s+s2}{\PYZdq{}}\PY{l+s+s2}{DIPPERC}\PY{l+s+s2}{\PYZdq{}}\PY{p}{,} \PY{n}{data}\PY{o}{=}\PY{n}{datasetFin}\PY{p}{,}\PY{n}{fit\PYZus{}reg}\PY{o}{=}\PY{n+nb+bp}{True}\PY{p}{)}
\end{Verbatim}

    \begin{center}
    \adjustimage{max size={0.9\linewidth}{0.9\paperheight}}{output_89_0.png}
    \end{center}
    { \hspace*{\fill} \\}
    
    Dans une premiere approche on remarque qu'il n'existe pas de correlation
dans entre le diplome obtenus à l'exetat et la moyenned u poucentage
final à l'université

    Regardons de plus pret au sein de chaque faculté s'il peut y avoir une
correlation.

    \begin{Verbatim}[commandchars=\\\{\}]
{\color{incolor}In [{\color{incolor}334}]:} \PY{n}{Medecine}\PY{o}{=}\PY{n}{datasetFin}\PY{o}{.}\PY{n}{loc}\PY{p}{[}\PY{n}{datasetFin}\PY{o}{.}\PY{n}{FAC}\PY{o}{==}\PY{l+s+s1}{\PYZsq{}}\PY{l+s+s1}{FM}\PY{l+s+s1}{\PYZsq{}}\PY{p}{]}
\end{Verbatim}

    \paragraph{Distribution DU GPA}\label{distribution-du-gpa}

    Essayons de voir la distribution et la variation du CGPA Globalement
avnt de l'analyser de plus pret au sein de chaque faculté

    \begin{Verbatim}[commandchars=\\\{\}]
{\color{incolor}In [{\color{incolor}532}]:} \PY{n}{plt}\PY{o}{.}\PY{n}{figure}\PY{p}{(}\PY{p}{)}
          \PY{n}{ax} \PY{o}{=} \PY{n}{sns}\PY{o}{.}\PY{n}{boxplot}\PY{p}{(}\PY{n}{x}\PY{o}{=}\PY{n}{datasetFin}\PY{p}{[}\PY{l+s+s1}{\PYZsq{}}\PY{l+s+s1}{CGPA}\PY{l+s+s1}{\PYZsq{}}\PY{p}{]}\PY{p}{)}
          \PY{n}{plt}\PY{o}{.}\PY{n}{savefig}\PY{p}{(}\PY{l+s+s1}{\PYZsq{}}\PY{l+s+s1}{CGPA.png}\PY{l+s+s1}{\PYZsq{}}\PY{p}{,}\PY{n}{dpi}\PY{o}{=}\PY{l+m+mi}{100}\PY{p}{)}
\end{Verbatim}

    \begin{center}
    \adjustimage{max size={0.9\linewidth}{0.9\paperheight}}{output_95_0.png}
    \end{center}
    { \hspace*{\fill} \\}
    
    \begin{Verbatim}[commandchars=\\\{\}]
{\color{incolor}In [{\color{incolor}533}]:} \PY{n}{plt}\PY{o}{.}\PY{n}{figure}\PY{p}{(}\PY{p}{)}
          \PY{n}{sns}\PY{o}{.}\PY{n}{distplot}\PY{p}{(}\PY{n}{datasetFin}\PY{p}{[}\PY{l+s+s1}{\PYZsq{}}\PY{l+s+s1}{CGPA}\PY{l+s+s1}{\PYZsq{}}\PY{p}{]}\PY{p}{,}\PY{n}{bins}\PY{o}{=}\PY{l+m+mi}{20}\PY{p}{,}\PY{n}{axlabel}\PY{o}{=}\PY{l+s+s1}{\PYZsq{}}\PY{l+s+s1}{CGPA}\PY{l+s+s1}{\PYZsq{}}\PY{p}{,}\PY{n}{kde}\PY{o}{=}\PY{l+m+mi}{1}\PY{p}{,}\PY{n}{norm\PYZus{}hist}\PY{o}{=}\PY{l+m+mi}{0}\PY{p}{)}
          \PY{n}{plt}\PY{o}{.}\PY{n}{savefig}\PY{p}{(}\PY{l+s+s1}{\PYZsq{}}\PY{l+s+s1}{CGPADist.png}\PY{l+s+s1}{\PYZsq{}}\PY{p}{,}\PY{n}{dpi}\PY{o}{=}\PY{l+m+mi}{100}\PY{p}{)}
\end{Verbatim}

    \begin{center}
    \adjustimage{max size={0.9\linewidth}{0.9\paperheight}}{output_96_0.png}
    \end{center}
    { \hspace*{\fill} \\}
    
    Nous Pouvons aisement remarquer que notre variable suit une distribution
presque normale si on prend compte uniqement des étudiant ayant un
pourcentage superieru à 50

    \paragraph{A. Faculté de Médecine}\label{a.-facultuxe9-de-muxe9decine}

    A.0 Distribution du CGPA

    \begin{Verbatim}[commandchars=\\\{\}]
{\color{incolor}In [{\color{incolor}535}]:} \PY{n}{plt}\PY{o}{.}\PY{n}{figure}\PY{p}{(}\PY{p}{)}
          \PY{n}{ax} \PY{o}{=} \PY{n}{sns}\PY{o}{.}\PY{n}{boxplot}\PY{p}{(}\PY{n}{x}\PY{o}{=}\PY{n}{Medecine}\PY{p}{[}\PY{l+s+s1}{\PYZsq{}}\PY{l+s+s1}{CGPA}\PY{l+s+s1}{\PYZsq{}}\PY{p}{]}\PY{p}{)}
          \PY{n}{plt}\PY{o}{.}\PY{n}{savefig}\PY{p}{(}\PY{l+s+s1}{\PYZsq{}}\PY{l+s+s1}{CGPA.png}\PY{l+s+s1}{\PYZsq{}}\PY{p}{,}\PY{n}{dpi}\PY{o}{=}\PY{l+m+mi}{100}\PY{p}{)}
\end{Verbatim}

    \begin{center}
    \adjustimage{max size={0.9\linewidth}{0.9\paperheight}}{output_100_0.png}
    \end{center}
    { \hspace*{\fill} \\}
    
    \begin{Verbatim}[commandchars=\\\{\}]
{\color{incolor}In [{\color{incolor}542}]:} \PY{n}{plt}\PY{o}{.}\PY{n}{figure}\PY{p}{(}\PY{p}{)}
          \PY{n}{sns}\PY{o}{.}\PY{n}{distplot}\PY{p}{(}\PY{n}{Medecine}\PY{p}{[}\PY{l+s+s1}{\PYZsq{}}\PY{l+s+s1}{CGPA}\PY{l+s+s1}{\PYZsq{}}\PY{p}{]}\PY{p}{,}\PY{n}{bins}\PY{o}{=}\PY{l+m+mi}{20}\PY{p}{,}\PY{n}{axlabel}\PY{o}{=}\PY{l+s+s1}{\PYZsq{}}\PY{l+s+s1}{CGPA}\PY{l+s+s1}{\PYZsq{}}\PY{p}{,}\PY{n}{kde}\PY{o}{=}\PY{l+m+mi}{1}\PY{p}{,}\PY{n}{norm\PYZus{}hist}\PY{o}{=}\PY{l+m+mi}{0}\PY{p}{)}
          \PY{n}{plt}\PY{o}{.}\PY{n}{savefig}\PY{p}{(}\PY{l+s+s1}{\PYZsq{}}\PY{l+s+s1}{CGPADistMed.png}\PY{l+s+s1}{\PYZsq{}}\PY{p}{,}\PY{n}{dpi}\PY{o}{=}\PY{l+m+mi}{100}\PY{p}{)}
\end{Verbatim}

    \begin{center}
    \adjustimage{max size={0.9\linewidth}{0.9\paperheight}}{output_101_0.png}
    \end{center}
    { \hspace*{\fill} \\}
    
    On remarque que cette distribution est normale avec un écart type de : 7

    \begin{Verbatim}[commandchars=\\\{\}]
{\color{incolor}In [{\color{incolor}543}]:} \PY{n}{np}\PY{o}{.}\PY{n}{std}\PY{p}{(}\PY{n}{Medecine}\PY{o}{.}\PY{n}{CGPA}\PY{p}{)}
\end{Verbatim}

            \begin{Verbatim}[commandchars=\\\{\}]
{\color{outcolor}Out[{\color{outcolor}543}]:} 6.948726408145192
\end{Verbatim}
        
    A.1 CGPA-POURCENTAGE

    Cherchons les differentes correlations pouvan exsiter au sein de la
faculté de medecine

    \begin{Verbatim}[commandchars=\\\{\}]
{\color{incolor}In [{\color{incolor}448}]:} \PY{n}{CorreMatrix}\PY{o}{=}\PY{n}{Medecine}\PY{o}{.}\PY{n}{corr}\PY{p}{(}\PY{n}{method}\PY{o}{=}\PY{l+s+s1}{\PYZsq{}}\PY{l+s+s1}{pearson}\PY{l+s+s1}{\PYZsq{}}\PY{p}{)}
\end{Verbatim}

    \begin{Verbatim}[commandchars=\\\{\}]
{\color{incolor}In [{\color{incolor}450}]:} \PY{n}{CorreMatrix}\PY{o}{.}\PY{n}{CGPA}\PY{o}{.}\PY{n}{sort\PYZus{}values}\PY{p}{(}\PY{n}{ascending}\PY{o}{=}\PY{n+nb+bp}{False}\PY{p}{)}
\end{Verbatim}

            \begin{Verbatim}[commandchars=\\\{\}]
{\color{outcolor}Out[{\color{outcolor}450}]:} CGPA         1.000000
          DIPPERC      0.447654
          NACADYEAR    0.158242
          ID           0.096283
          AGE         -0.256688
          Name: CGPA, dtype: float64
\end{Verbatim}
        
    \begin{Verbatim}[commandchars=\\\{\}]
{\color{incolor}In [{\color{incolor}337}]:} \PY{n}{pearsonr}\PY{p}{(}\PY{n}{Medecine}\PY{o}{.}\PY{n}{CGPA}\PY{p}{,}\PY{n}{Medecine}\PY{o}{.}\PY{n}{DIPPERC}\PY{p}{)}
\end{Verbatim}

            \begin{Verbatim}[commandchars=\\\{\}]
{\color{outcolor}Out[{\color{outcolor}337}]:} (0.44765396714867312, 2.5042090286791821e-13)
\end{Verbatim}
        
    Au sein de la faculté de medécine nous remarquons que la correlation
entre le pourcenatge à l'exetat et le CGPA est de 0.45 ce qui est pas
mal

    \begin{Verbatim}[commandchars=\\\{\}]
{\color{incolor}In [{\color{incolor}340}]:} \PY{n}{plt}\PY{o}{.}\PY{n}{figure}\PY{p}{(}\PY{n}{figsize}\PY{o}{=}\PY{p}{(}\PY{l+m+mi}{10}\PY{p}{,}\PY{l+m+mi}{10}\PY{p}{)}\PY{p}{)}
          \PY{n}{ax} \PY{o}{=} \PY{n}{sns}\PY{o}{.}\PY{n}{regplot}\PY{p}{(}\PY{n}{x}\PY{o}{=}\PY{l+s+s2}{\PYZdq{}}\PY{l+s+s2}{CGPA}\PY{l+s+s2}{\PYZdq{}}\PY{p}{,} \PY{n}{y}\PY{o}{=}\PY{l+s+s2}{\PYZdq{}}\PY{l+s+s2}{DIPPERC}\PY{l+s+s2}{\PYZdq{}}\PY{p}{,} \PY{n}{data}\PY{o}{=}\PY{n}{Medecine}\PY{p}{,}\PY{n}{fit\PYZus{}reg}\PY{o}{=}\PY{n+nb+bp}{True}\PY{p}{)}
\end{Verbatim}

    \begin{center}
    \adjustimage{max size={0.9\linewidth}{0.9\paperheight}}{output_110_0.png}
    \end{center}
    { \hspace*{\fill} \\}
    
    A.2 Age et CPGA

    \begin{Verbatim}[commandchars=\\\{\}]
{\color{incolor}In [{\color{incolor}341}]:} \PY{n}{pearsonr}\PY{p}{(}\PY{n}{Medecine}\PY{o}{.}\PY{n}{CGPA}\PY{p}{,}\PY{n}{Medecine}\PY{o}{.}\PY{n}{AGE}\PY{p}{)}
\end{Verbatim}

            \begin{Verbatim}[commandchars=\\\{\}]
{\color{outcolor}Out[{\color{outcolor}341}]:} (-0.2566879460545976, 5.3370210238152509e-05)
\end{Verbatim}
        
    \begin{Verbatim}[commandchars=\\\{\}]
{\color{incolor}In [{\color{incolor} }]:} \PY{n}{aucune} \PY{n}{relation} \PY{n}{entre} \PY{n}{l}\PY{l+s+s1}{\PYZsq{}}\PY{l+s+s1}{age  et le diplome pourcenatge }
\end{Verbatim}

    A.2 SchoolProvinece

    \begin{Verbatim}[commandchars=\\\{\}]
{\color{incolor}In [{\color{incolor}344}]:} \PY{n}{datasetFin}\PY{o}{.}\PY{n}{columns}
\end{Verbatim}

            \begin{Verbatim}[commandchars=\\\{\}]
{\color{outcolor}Out[{\color{outcolor}344}]:} Index([u'ID', u'SCHOOLSTATUS', u'SCHOOL\_RIGHT', u'OPTION\_RIGHT', u'FAC',
                 u'SCHOOLPROVINCE', u'GENDER', u'DIPPERC', u'AGE', u'CGPA',
                 u'DistinctionRatio', u'EchecRatio', u'NACADYEAR',
                 u'Pass1stSessionRatio'],
                dtype='object')
\end{Verbatim}
        
    \begin{Verbatim}[commandchars=\\\{\}]
{\color{incolor}In [{\color{incolor}348}]:} \PY{n}{moore\PYZus{}lm} \PY{o}{=} \PY{n}{ols}\PY{p}{(}\PY{l+s+s1}{\PYZsq{}}\PY{l+s+s1}{CGPA \PYZti{} C(SCHOOLSTATUS)}\PY{l+s+s1}{\PYZsq{}}\PY{p}{,}\PY{n}{data}\PY{o}{=}\PY{n}{Medecine}\PY{p}{)}\PY{o}{.}\PY{n}{fit}\PY{p}{(}\PY{p}{)}
          \PY{n}{aov\PYZus{}table} \PY{o}{=} \PY{n}{sm}\PY{o}{.}\PY{n}{stats}\PY{o}{.}\PY{n}{anova\PYZus{}lm}\PY{p}{(}\PY{n}{moore\PYZus{}lm}\PY{p}{,} \PY{n}{typ}\PY{o}{=}\PY{l+m+mi}{1}\PY{p}{)}
\end{Verbatim}

    \begin{Verbatim}[commandchars=\\\{\}]
{\color{incolor}In [{\color{incolor}349}]:} \PY{n}{aov\PYZus{}table}
\end{Verbatim}

            \begin{Verbatim}[commandchars=\\\{\}]
{\color{outcolor}Out[{\color{outcolor}349}]:}                     df        sum\_sq     mean\_sq         F    PR(>F)
          C(SCHOOLSTATUS)    5.0   1560.231935  312.046387  7.273601  0.000002
          Residual         236.0  10124.689350   42.901226       NaN       NaN
\end{Verbatim}
        
    \begin{Verbatim}[commandchars=\\\{\}]
{\color{incolor}In [{\color{incolor} }]:} 
\end{Verbatim}

    Il n'yas des correlation

    \begin{Verbatim}[commandchars=\\\{\}]
{\color{incolor}In [{\color{incolor}347}]:} \PY{n}{Medecine}\PY{o}{.}\PY{n}{groupby}\PY{p}{(}\PY{l+s+s1}{\PYZsq{}}\PY{l+s+s1}{SCHOOLSTATUS}\PY{l+s+s1}{\PYZsq{}}\PY{p}{)}\PY{o}{.}\PY{n}{mean}\PY{p}{(}\PY{p}{)}
\end{Verbatim}

            \begin{Verbatim}[commandchars=\\\{\}]
{\color{outcolor}Out[{\color{outcolor}347}]:}                         ID    DIPPERC        AGE       CGPA  NACADYEAR
          SCHOOLSTATUS                                                          
          catholique    10867.909091  62.610830  20.704545  60.312500   1.545455
          inconnu        9952.368421  55.052632  23.842105  55.044737   1.736842
          musulman      11037.500000  56.000000  22.000000  56.900000   1.000000
          privé         11150.379310  55.413793  21.758621  53.662069   1.344828
          protestant    10991.243243  59.424007  21.391892  58.155405   1.418919
          publique      11018.966667  57.033333  22.233333  54.351667   1.366667
\end{Verbatim}
        
    Nous pouvons aisement remarqué que les étudiants venant des écoles
catholiques reussisent mieux en faculté de medecine avec une moyennde de
60 ensuitent viennentceux des écoles protestatntes

    \begin{Verbatim}[commandchars=\\\{\}]
{\color{incolor}In [{\color{incolor}351}]:} \PY{n}{plt}\PY{o}{.}\PY{n}{figure}\PY{p}{(}\PY{n}{figsize}\PY{o}{=}\PY{p}{(}\PY{l+m+mi}{12}\PY{p}{,}\PY{l+m+mi}{6}\PY{p}{)}\PY{p}{)}
          \PY{n}{ax}\PY{o}{=}\PY{n}{sns}\PY{o}{.}\PY{n}{boxplot}\PY{p}{(}\PY{n}{x}\PY{o}{=}\PY{l+s+s2}{\PYZdq{}}\PY{l+s+s2}{SCHOOLSTATUS}\PY{l+s+s2}{\PYZdq{}}\PY{p}{,} \PY{n}{y}\PY{o}{=}\PY{l+s+s2}{\PYZdq{}}\PY{l+s+s2}{CGPA}\PY{l+s+s2}{\PYZdq{}}\PY{p}{,} \PY{n}{data}\PY{o}{=}\PY{n}{Medecine}\PY{p}{)}
          \PY{n}{plt}\PY{o}{.}\PY{n}{savefig}\PY{p}{(}\PY{l+s+s1}{\PYZsq{}}\PY{l+s+s1}{STATUS\PYZhy{}CGPA.png}\PY{l+s+s1}{\PYZsq{}}\PY{p}{,}\PY{n}{dpi}\PY{o}{=}\PY{l+m+mi}{100}\PY{p}{)}
\end{Verbatim}

    \begin{center}
    \adjustimage{max size={0.9\linewidth}{0.9\paperheight}}{output_122_0.png}
    \end{center}
    { \hspace*{\fill} \\}
    
    A.4 Sexe :

    \begin{Verbatim}[commandchars=\\\{\}]
{\color{incolor}In [{\color{incolor} }]:} \PY{n}{Index}\PY{p}{(}\PY{p}{[}\PY{l+s+sa}{u}\PY{l+s+s1}{\PYZsq{}}\PY{l+s+s1}{ID}\PY{l+s+s1}{\PYZsq{}}\PY{p}{,} \PY{l+s+sa}{u}\PY{l+s+s1}{\PYZsq{}}\PY{l+s+s1}{SCHOOLSTATUS}\PY{l+s+s1}{\PYZsq{}}\PY{p}{,} \PY{l+s+sa}{u}\PY{l+s+s1}{\PYZsq{}}\PY{l+s+s1}{SCHOOL\PYZus{}RIGHT}\PY{l+s+s1}{\PYZsq{}}\PY{p}{,} \PY{l+s+sa}{u}\PY{l+s+s1}{\PYZsq{}}\PY{l+s+s1}{OPTION\PYZus{}RIGHT}\PY{l+s+s1}{\PYZsq{}}\PY{p}{,} \PY{l+s+sa}{u}\PY{l+s+s1}{\PYZsq{}}\PY{l+s+s1}{FAC}\PY{l+s+s1}{\PYZsq{}}\PY{p}{,}
               \PY{l+s+sa}{u}\PY{l+s+s1}{\PYZsq{}}\PY{l+s+s1}{SCHOOLPROVINCE}\PY{l+s+s1}{\PYZsq{}}\PY{p}{,} \PY{l+s+sa}{u}\PY{l+s+s1}{\PYZsq{}}\PY{l+s+s1}{GENDER}\PY{l+s+s1}{\PYZsq{}}\PY{p}{,} \PY{l+s+sa}{u}\PY{l+s+s1}{\PYZsq{}}\PY{l+s+s1}{DIPPERC}\PY{l+s+s1}{\PYZsq{}}\PY{p}{,} \PY{l+s+sa}{u}\PY{l+s+s1}{\PYZsq{}}\PY{l+s+s1}{AGE}\PY{l+s+s1}{\PYZsq{}}\PY{p}{,} \PY{l+s+sa}{u}\PY{l+s+s1}{\PYZsq{}}\PY{l+s+s1}{CGPA}\PY{l+s+s1}{\PYZsq{}}\PY{p}{,}
               \PY{l+s+sa}{u}\PY{l+s+s1}{\PYZsq{}}\PY{l+s+s1}{DistinctionRatio}\PY{l+s+s1}{\PYZsq{}}\PY{p}{,} \PY{l+s+sa}{u}\PY{l+s+s1}{\PYZsq{}}\PY{l+s+s1}{EchecRatio}\PY{l+s+s1}{\PYZsq{}}\PY{p}{,} \PY{l+s+sa}{u}\PY{l+s+s1}{\PYZsq{}}\PY{l+s+s1}{NACADYEAR}\PY{l+s+s1}{\PYZsq{}}\PY{p}{,}
               \PY{l+s+sa}{u}\PY{l+s+s1}{\PYZsq{}}\PY{l+s+s1}{Pass1stSessionRatio}\PY{l+s+s1}{\PYZsq{}}\PY{p}{]}\PY{p}{,}
              \PY{n}{dtype}\PY{o}{=}\PY{l+s+s1}{\PYZsq{}}\PY{l+s+s1}{object}\PY{l+s+s1}{\PYZsq{}}\PY{p}{)}
\end{Verbatim}

    \begin{Verbatim}[commandchars=\\\{\}]
{\color{incolor}In [{\color{incolor}350}]:} \PY{n}{moore\PYZus{}lm} \PY{o}{=} \PY{n}{ols}\PY{p}{(}\PY{l+s+s1}{\PYZsq{}}\PY{l+s+s1}{CGPA \PYZti{} C(GENDER)}\PY{l+s+s1}{\PYZsq{}}\PY{p}{,}\PY{n}{data}\PY{o}{=}\PY{n}{Medecine}\PY{p}{)}\PY{o}{.}\PY{n}{fit}\PY{p}{(}\PY{p}{)}
          \PY{n}{aov\PYZus{}table} \PY{o}{=} \PY{n}{sm}\PY{o}{.}\PY{n}{stats}\PY{o}{.}\PY{n}{anova\PYZus{}lm}\PY{p}{(}\PY{n}{moore\PYZus{}lm}\PY{p}{,} \PY{n}{typ}\PY{o}{=}\PY{l+m+mi}{1}\PY{p}{)}
          \PY{n}{aov\PYZus{}table}
\end{Verbatim}

            \begin{Verbatim}[commandchars=\\\{\}]
{\color{outcolor}Out[{\color{outcolor}350}]:}               df        sum\_sq     mean\_sq         F    PR(>F)
          C(GENDER)    1.0    117.513228  117.513228  2.438159  0.119733
          Residual   240.0  11567.408056   48.197534       NaN       NaN
\end{Verbatim}
        
    On conclus sans hesiter que la moyenne est la pour les filles que pour
les garcons

    \begin{Verbatim}[commandchars=\\\{\}]
{\color{incolor}In [{\color{incolor}352}]:} 
\end{Verbatim}

            \begin{Verbatim}[commandchars=\\\{\}]
{\color{outcolor}Out[{\color{outcolor}352}]:}                   ID    DIPPERC        AGE       CGPA  NACADYEAR
          GENDER                                                          
          F       11029.880734  60.066542  20.990826  58.444954   1.440367
          H       10771.210526  58.916365  21.894737  57.044361   1.496241
\end{Verbatim}
        
    \begin{Verbatim}[commandchars=\\\{\}]
{\color{incolor}In [{\color{incolor}353}]:} \PY{n}{Medecine}\PY{o}{.}\PY{n}{groupby}\PY{p}{(}\PY{l+s+s1}{\PYZsq{}}\PY{l+s+s1}{GENDER}\PY{l+s+s1}{\PYZsq{}}\PY{p}{)}\PY{o}{.}\PY{n}{mean}\PY{p}{(}\PY{p}{)}
          \PY{n}{plt}\PY{o}{.}\PY{n}{figure}\PY{p}{(}\PY{n}{figsize}\PY{o}{=}\PY{p}{(}\PY{l+m+mi}{12}\PY{p}{,}\PY{l+m+mi}{6}\PY{p}{)}\PY{p}{)}
          \PY{n}{ax}\PY{o}{=}\PY{n}{sns}\PY{o}{.}\PY{n}{boxplot}\PY{p}{(}\PY{n}{x}\PY{o}{=}\PY{l+s+s2}{\PYZdq{}}\PY{l+s+s2}{GENDER}\PY{l+s+s2}{\PYZdq{}}\PY{p}{,} \PY{n}{y}\PY{o}{=}\PY{l+s+s2}{\PYZdq{}}\PY{l+s+s2}{CGPA}\PY{l+s+s2}{\PYZdq{}}\PY{p}{,} \PY{n}{data}\PY{o}{=}\PY{n}{Medecine}\PY{p}{)}
          \PY{n}{plt}\PY{o}{.}\PY{n}{savefig}\PY{p}{(}\PY{l+s+s1}{\PYZsq{}}\PY{l+s+s1}{GENDER\PYZhy{}CGPA.png}\PY{l+s+s1}{\PYZsq{}}\PY{p}{,}\PY{n}{dpi}\PY{o}{=}\PY{l+m+mi}{100}\PY{p}{)}
\end{Verbatim}

    \begin{center}
    \adjustimage{max size={0.9\linewidth}{0.9\paperheight}}{output_128_0.png}
    \end{center}
    { \hspace*{\fill} \\}
    
    On Peut Remarquer que les filles on des meilleurs resulat que les
garcons en Faculté de medecine surtout en terme des disticntions

    A.6 GPA - SCHOOL

    \begin{Verbatim}[commandchars=\\\{\}]
{\color{incolor}In [{\color{incolor}354}]:} \PY{n}{moore\PYZus{}lm} \PY{o}{=} \PY{n}{ols}\PY{p}{(}\PY{l+s+s1}{\PYZsq{}}\PY{l+s+s1}{CGPA \PYZti{} C(SCHOOL\PYZus{}RIGHT)}\PY{l+s+s1}{\PYZsq{}}\PY{p}{,}\PY{n}{data}\PY{o}{=}\PY{n}{Medecine}\PY{p}{)}\PY{o}{.}\PY{n}{fit}\PY{p}{(}\PY{p}{)}
          \PY{n}{aov\PYZus{}table} \PY{o}{=} \PY{n}{sm}\PY{o}{.}\PY{n}{stats}\PY{o}{.}\PY{n}{anova\PYZus{}lm}\PY{p}{(}\PY{n}{moore\PYZus{}lm}\PY{p}{,} \PY{n}{typ}\PY{o}{=}\PY{l+m+mi}{1}\PY{p}{)}
          \PY{n}{aov\PYZus{}table}
\end{Verbatim}

            \begin{Verbatim}[commandchars=\\\{\}]
{\color{outcolor}Out[{\color{outcolor}354}]:}                     df       sum\_sq    mean\_sq         F    PR(>F)
          C(SCHOOL\_RIGHT)   93.0  5751.643494  61.845629  1.542681  0.009244
          Residual         148.0  5933.277790  40.089715       NaN       NaN
\end{Verbatim}
        
    Nous allons verifier comment se presente cette moyenne dans les 10
écoles les plus representée

    \begin{Verbatim}[commandchars=\\\{\}]
{\color{incolor}In [{\color{incolor}359}]:} \PY{n}{SchoolGroup}\PY{o}{=}\PY{n}{Medecine}\PY{o}{.}\PY{n}{groupby}\PY{p}{(}\PY{l+s+s1}{\PYZsq{}}\PY{l+s+s1}{SCHOOL\PYZus{}RIGHT}\PY{l+s+s1}{\PYZsq{}}\PY{p}{)}\PY{o}{.}\PY{n}{mean}\PY{p}{(}\PY{p}{)}
\end{Verbatim}

    \begin{Verbatim}[commandchars=\\\{\}]
{\color{incolor}In [{\color{incolor}368}]:} \PY{n}{SchoolGroup}\PY{o}{.}\PY{n}{columns}
\end{Verbatim}

            \begin{Verbatim}[commandchars=\\\{\}]
{\color{outcolor}Out[{\color{outcolor}368}]:} Index([u'ID', u'DIPPERC', u'AGE', u'CGPA', u'NACADYEAR'], dtype='object')
\end{Verbatim}
        
    \begin{Verbatim}[commandchars=\\\{\}]
{\color{incolor}In [{\color{incolor}370}]:} \PY{n}{SchoolGroup}\PY{o}{.}\PY{n}{shape}
\end{Verbatim}

            \begin{Verbatim}[commandchars=\\\{\}]
{\color{outcolor}Out[{\color{outcolor}370}]:} (94, 5)
\end{Verbatim}
        
    \begin{Verbatim}[commandchars=\\\{\}]
{\color{incolor}In [{\color{incolor}373}]:} \PY{n}{SchoolGroup}\PY{o}{.}\PY{n}{sort}\PY{p}{(}\PY{n}{axis}\PY{o}{=}\PY{l+m+mi}{0}\PY{p}{,}\PY{n}{columns}\PY{o}{=}\PY{l+s+s1}{\PYZsq{}}\PY{l+s+s1}{CGPA}\PY{l+s+s1}{\PYZsq{}}\PY{p}{,}\PY{n}{ascending}\PY{o}{=}\PY{n+nb+bp}{False}\PY{p}{,}\PY{n}{inplace}\PY{o}{=}\PY{n+nb+bp}{True}\PY{p}{)}
\end{Verbatim}

    \begin{Verbatim}[commandchars=\\\{\}]
{\color{incolor}In [{\color{incolor}374}]:} \PY{n}{SchoolGroup}\PY{o}{.}\PY{n}{reset\PYZus{}index}\PY{p}{(}\PY{n}{inplace}\PY{o}{=}\PY{n+nb+bp}{True}\PY{p}{)}
\end{Verbatim}

    \begin{Verbatim}[commandchars=\\\{\}]
{\color{incolor}In [{\color{incolor}427}]:} \PY{n}{SchoolGroup}\PY{o}{.}\PY{n}{loc}\PY{p}{[} \PY{n+nb}{range}\PY{p}{(}\PY{l+m+mi}{0}\PY{p}{,}\PY{l+m+mi}{6}\PY{p}{)} \PY{o}{+} \PY{n+nb}{range}\PY{p}{(}\PY{l+m+mi}{89}\PY{p}{,}\PY{l+m+mi}{94}\PY{p}{)}\PY{p}{]}\PY{o}{.}\PY{n}{SCHOOL\PYZus{}RIGHT}
\end{Verbatim}

            \begin{Verbatim}[commandchars=\\\{\}]
{\color{outcolor}Out[{\color{outcolor}427}]:} 0              alfajiri
          1       Institut NJANJA
          2     idap isp rutshuru
          3           mama mulezi
          4                 bwito
          5              bsangani
          89          neema kwetu
          90            de bukavu
          91               masisi
          92                 amen
          93               hekima
          Name: SCHOOL\_RIGHT, dtype: object
\end{Verbatim}
        
    \begin{Verbatim}[commandchars=\\\{\}]
{\color{incolor}In [{\color{incolor}432}]:} \PY{n}{Medecine}\PY{o}{.}\PY{n}{shape}
\end{Verbatim}

            \begin{Verbatim}[commandchars=\\\{\}]
{\color{outcolor}Out[{\color{outcolor}432}]:} (242, 14)
\end{Verbatim}
        
    \begin{Verbatim}[commandchars=\\\{\}]
{\color{incolor}In [{\color{incolor}437}]:} \PY{n}{Medecine}\PY{o}{.}\PY{n}{groupby}\PY{p}{(}\PY{l+s+s1}{\PYZsq{}}\PY{l+s+s1}{SCHOOL\PYZus{}RIGHT}\PY{l+s+s1}{\PYZsq{}}\PY{p}{)}\PY{o}{.}\PY{n}{mean}\PY{p}{(}\PY{p}{)}
          \PY{n}{plt}\PY{o}{.}\PY{n}{figure}\PY{p}{(}\PY{n}{figsize}\PY{o}{=}\PY{p}{(}\PY{l+m+mi}{24}\PY{p}{,}\PY{l+m+mi}{6}\PY{p}{)}\PY{p}{)}
          \PY{n}{ax}\PY{o}{=}\PY{n}{sns}\PY{o}{.}\PY{n}{boxplot}\PY{p}{(}\PY{n}{x}\PY{o}{=}\PY{l+s+s2}{\PYZdq{}}\PY{l+s+s2}{SCHOOL\PYZus{}RIGHT}\PY{l+s+s2}{\PYZdq{}}\PY{p}{,} \PY{n}{y}\PY{o}{=}\PY{l+s+s2}{\PYZdq{}}\PY{l+s+s2}{CGPA}\PY{l+s+s2}{\PYZdq{}}\PY{p}{,} \PY{n}{data}\PY{o}{=}\PY{n}{Medecine}\PY{o}{.}\PY{n}{loc}\PY{p}{[}\PY{n}{operator}\PY{o}{.}\PY{n}{or\PYZus{}}\PY{p}{(}\PY{n}{Medecine}\PY{o}{.}\PY{n}{SCHOOL\PYZus{}RIGHT}\PY{o}{.}\PY{n}{isin}\PY{p}{(}\PY{n}{SchoolGroup}\PY{o}{.}\PY{n}{loc}\PY{p}{[} \PY{n+nb}{range}\PY{p}{(}\PY{l+m+mi}{0}\PY{p}{,}\PY{l+m+mi}{6}\PY{p}{)} \PY{o}{+} \PY{n+nb}{range}\PY{p}{(}\PY{l+m+mi}{89}\PY{p}{,}\PY{l+m+mi}{94}\PY{p}{)}\PY{p}{]}\PY{o}{.}\PY{n}{SCHOOL\PYZus{}RIGHT}\PY{p}{)} 
                       \PY{p}{,} \PY{n}{Medecine}\PY{o}{.}\PY{n}{SCHOOL\PYZus{}RIGHT}\PY{o}{.}\PY{n}{isin}\PY{p}{(}\PY{n}{Medecine}\PY{o}{.}\PY{n}{SCHOOL\PYZus{}RIGHT}\PY{o}{.}\PY{n}{value\PYZus{}counts}\PY{p}{(}\PY{p}{)}\PY{p}{[}\PY{p}{:}\PY{l+m+mi}{10}\PY{p}{]}\PY{o}{.}\PY{n}{index}\PY{p}{)}\PY{p}{)}\PY{p}{]}\PY{p}{)}
          \PY{n}{plt}\PY{o}{.}\PY{n}{savefig}\PY{p}{(}\PY{l+s+s1}{\PYZsq{}}\PY{l+s+s1}{SCHOOL\PYZus{}RIGHT\PYZhy{}CGPA.png}\PY{l+s+s1}{\PYZsq{}}\PY{p}{,}\PY{n}{dpi}\PY{o}{=}\PY{l+m+mi}{100}\PY{p}{)}
\end{Verbatim}

    \begin{center}
    \adjustimage{max size={0.9\linewidth}{0.9\paperheight}}{output_140_0.png}
    \end{center}
    { \hspace*{\fill} \\}
    
    Sur cette figure comment se repartissent le CGPA dans Les 20 ecoles dont
10 sont les plus representes et 5 sont ceux qui on une grande moyenne et
5 ceux qui ont une faible moyyenne on remarque que les étudiant
provenant des écoles comme lycée sainte ursule, mwanga, bakanja
,alfagiri, faraja,amani, n'echouent presque pas en faculté de medecine

    \begin{Verbatim}[commandchars=\\\{\}]
{\color{incolor}In [{\color{incolor}436}]:} \PY{k+kn}{import} \PY{n+nn}{matplotlib}
          \PY{n}{matplotlib}\PY{o}{.}\PY{n}{rcParams}\PY{o}{.}\PY{n}{update}\PY{p}{(}\PY{p}{\PYZob{}}\PY{l+s+s1}{\PYZsq{}}\PY{l+s+s1}{font.size}\PY{l+s+s1}{\PYZsq{}}\PY{p}{:} \PY{l+m+mi}{22}\PY{p}{\PYZcb{}}\PY{p}{)}
\end{Verbatim}

    \begin{Verbatim}[commandchars=\\\{\}]
{\color{incolor}In [{\color{incolor}438}]:} \PY{n}{OptionGroup}\PY{o}{=}\PY{n}{Medecine}\PY{o}{.}\PY{n}{groupby}\PY{p}{(}\PY{l+s+s1}{\PYZsq{}}\PY{l+s+s1}{OPTION\PYZus{}RIGHT}\PY{l+s+s1}{\PYZsq{}}\PY{p}{)}\PY{o}{.}\PY{n}{mean}\PY{p}{(}\PY{p}{)}
\end{Verbatim}

    \begin{Verbatim}[commandchars=\\\{\}]
{\color{incolor}In [{\color{incolor}442}]:} \PY{n}{OptionGroup}\PY{o}{.}\PY{n}{sort}\PY{p}{(}\PY{n}{axis}\PY{o}{=}\PY{l+m+mi}{0}\PY{p}{,}\PY{n}{columns}\PY{o}{=}\PY{l+s+s1}{\PYZsq{}}\PY{l+s+s1}{CGPA}\PY{l+s+s1}{\PYZsq{}}\PY{p}{,}\PY{n}{ascending}\PY{o}{=}\PY{n+nb+bp}{False}\PY{p}{,}\PY{n}{inplace}\PY{o}{=}\PY{n+nb+bp}{True}\PY{p}{)}
\end{Verbatim}

    \begin{Verbatim}[commandchars=\\\{\}]
{\color{incolor}In [{\color{incolor}443}]:} \PY{n}{OptionGroup}
\end{Verbatim}

            \begin{Verbatim}[commandchars=\\\{\}]
{\color{outcolor}Out[{\color{outcolor}443}]:}                                     ID    DIPPERC        AGE       CGPA  \textbackslash{}
          OPTION\_RIGHT                                                              
          secretariat               11686.000000  54.000000  20.000000  62.500000   
          commerciale informatique  10982.750000  55.000000  21.500000  60.200001   
          vétérinaire               10726.500000  57.469131  20.750000  59.000000   
          bio-chimie                10827.522523  61.590090  21.000000  58.831532   
          agronomie                 11523.000000  65.000000  20.000000  58.500000   
          nutr                      11319.857143  60.634037  20.428571  58.375000   
          commmerciale et adm        9272.600000  51.200000  25.400000  57.450000   
          latin philo               10803.666667  58.166667  21.083333  57.329167   
          pedagogie                 11160.875000  58.664094  22.416667  56.628125   
          math-physique             10560.076923  57.076923  23.615385  56.442308   
          sociale                   10837.666667  55.740741  20.888889  55.018519   
          elec indust               10631.000000  52.000000  22.000000  51.450001   
          agrecole                  12191.000000  60.000000  25.000000  48.000000   
          
                                    NACADYEAR  
          OPTION\_RIGHT                         
          secretariat                1.000000  
          commerciale informatique   1.500000  
          vétérinaire                1.750000  
          bio-chimie                 1.531532  
          agronomie                  1.000000  
          nutr                       1.428571  
          commmerciale et adm        1.600000  
          latin philo                1.583333  
          pedagogie                  1.312500  
          math-physique              1.461538  
          sociale                    1.444444  
          elec indust                2.000000  
          agrecole                   1.000000  
\end{Verbatim}
        
    \begin{Verbatim}[commandchars=\\\{\}]
{\color{incolor}In [{\color{incolor}444}]:} \PY{n}{pearsonr}\PY{p}{(}\PY{n}{OptionGroup}\PY{o}{.}\PY{n}{DIPPERC}\PY{p}{,}\PY{n}{OptionGroup}\PY{o}{.}\PY{n}{CGPA}\PY{p}{)}
\end{Verbatim}

            \begin{Verbatim}[commandchars=\\\{\}]
{\color{outcolor}Out[{\color{outcolor}444}]:} (0.024778204523123036, 0.93596027075298815)
\end{Verbatim}
        
    Dans cette table nous pouvons voir la réparttion des moyennes des
poucentage selon les option! voyons en image à quoi ça ressemble

    \begin{Verbatim}[commandchars=\\\{\}]
{\color{incolor}In [{\color{incolor}441}]:} \PY{n}{plt}\PY{o}{.}\PY{n}{figure}\PY{p}{(}\PY{n}{figsize}\PY{o}{=}\PY{p}{(}\PY{l+m+mi}{24}\PY{p}{,}\PY{l+m+mi}{6}\PY{p}{)}\PY{p}{)}
          \PY{n}{ax}\PY{o}{=}\PY{n}{sns}\PY{o}{.}\PY{n}{boxplot}\PY{p}{(}\PY{n}{x}\PY{o}{=}\PY{l+s+s2}{\PYZdq{}}\PY{l+s+s2}{OPTION\PYZus{}RIGHT}\PY{l+s+s2}{\PYZdq{}}\PY{p}{,} \PY{n}{y}\PY{o}{=}\PY{l+s+s2}{\PYZdq{}}\PY{l+s+s2}{CGPA}\PY{l+s+s2}{\PYZdq{}}\PY{p}{,} \PY{n}{data}\PY{o}{=}\PY{n}{Medecine}\PY{p}{)}
          \PY{n}{plt}\PY{o}{.}\PY{n}{savefig}\PY{p}{(}\PY{l+s+s1}{\PYZsq{}}\PY{l+s+s1}{OPTION\PYZus{}RIGHT\PYZhy{}CGPA.png}\PY{l+s+s1}{\PYZsq{}}\PY{p}{,}\PY{n}{dpi}\PY{o}{=}\PY{l+m+mi}{100}\PY{p}{)}
\end{Verbatim}

    \begin{center}
    \adjustimage{max size={0.9\linewidth}{0.9\paperheight}}{output_148_0.png}
    \end{center}
    { \hspace*{\fill} \\}
    
    Sur la table on remarque que l'option n'a aucune influance sur notre
variable CGPA mais notons que la moyenne la plus élevé vient de l'option
sécretariat

    \begin{Verbatim}[commandchars=\\\{\}]
{\color{incolor}In [{\color{incolor}451}]:} \PY{n}{moore\PYZus{}lm} \PY{o}{=} \PY{n}{ols}\PY{p}{(}\PY{l+s+s1}{\PYZsq{}}\PY{l+s+s1}{CGPA \PYZti{} C(OPTION\PYZus{}RIGHT)}\PY{l+s+s1}{\PYZsq{}}\PY{p}{,}\PY{n}{data}\PY{o}{=}\PY{n}{Medecine}\PY{p}{)}\PY{o}{.}\PY{n}{fit}\PY{p}{(}\PY{p}{)}
          \PY{n}{aov\PYZus{}table} \PY{o}{=} \PY{n}{sm}\PY{o}{.}\PY{n}{stats}\PY{o}{.}\PY{n}{anova\PYZus{}lm}\PY{p}{(}\PY{n}{moore\PYZus{}lm}\PY{p}{,} \PY{n}{typ}\PY{o}{=}\PY{l+m+mi}{1}\PY{p}{)}
          \PY{n}{aov\PYZus{}table}
\end{Verbatim}

            \begin{Verbatim}[commandchars=\\\{\}]
{\color{outcolor}Out[{\color{outcolor}451}]:}                     df        sum\_sq    mean\_sq         F    PR(>F)
          C(OPTION\_RIGHT)   12.0    608.756098  50.729675  1.048837  0.405075
          Residual         229.0  11076.165186  48.367534       NaN       NaN
\end{Verbatim}
        
    \subparagraph{Faculté de TechNologies}\label{facultuxe9-de-technologies}

    \begin{Verbatim}[commandchars=\\\{\}]
{\color{incolor}In [{\color{incolor}501}]:} \PY{n}{Techno}\PY{o}{=}\PY{n}{datasetFin}\PY{o}{.}\PY{n}{loc}\PY{p}{[}\PY{n}{datasetFin}\PY{o}{.}\PY{n}{FAC}\PY{o}{==}\PY{l+s+s1}{\PYZsq{}}\PY{l+s+s1}{FSTA}\PY{l+s+s1}{\PYZsq{}}\PY{p}{]}
\end{Verbatim}

    A.0 Distribution du CGPA

    \begin{Verbatim}[commandchars=\\\{\}]
{\color{incolor}In [{\color{incolor}540}]:} \PY{n}{plt}\PY{o}{.}\PY{n}{figure}\PY{p}{(}\PY{p}{)}
          \PY{n}{ax} \PY{o}{=} \PY{n}{sns}\PY{o}{.}\PY{n}{boxplot}\PY{p}{(}\PY{n}{x}\PY{o}{=}\PY{n}{Techno}\PY{p}{[}\PY{l+s+s1}{\PYZsq{}}\PY{l+s+s1}{CGPA}\PY{l+s+s1}{\PYZsq{}}\PY{p}{]}\PY{p}{)}
          \PY{n}{plt}\PY{o}{.}\PY{n}{savefig}\PY{p}{(}\PY{l+s+s1}{\PYZsq{}}\PY{l+s+s1}{CGPATech.png}\PY{l+s+s1}{\PYZsq{}}\PY{p}{,}\PY{n}{dpi}\PY{o}{=}\PY{l+m+mi}{100}\PY{p}{)}
\end{Verbatim}

    \begin{center}
    \adjustimage{max size={0.9\linewidth}{0.9\paperheight}}{output_154_0.png}
    \end{center}
    { \hspace*{\fill} \\}
    
    \begin{Verbatim}[commandchars=\\\{\}]
{\color{incolor}In [{\color{incolor}541}]:} \PY{n}{plt}\PY{o}{.}\PY{n}{figure}\PY{p}{(}\PY{p}{)}
          \PY{n}{sns}\PY{o}{.}\PY{n}{distplot}\PY{p}{(}\PY{n}{Techno}\PY{p}{[}\PY{l+s+s1}{\PYZsq{}}\PY{l+s+s1}{CGPA}\PY{l+s+s1}{\PYZsq{}}\PY{p}{]}\PY{p}{,}\PY{n}{bins}\PY{o}{=}\PY{l+m+mi}{20}\PY{p}{,}\PY{n}{axlabel}\PY{o}{=}\PY{l+s+s1}{\PYZsq{}}\PY{l+s+s1}{CGPA}\PY{l+s+s1}{\PYZsq{}}\PY{p}{,}\PY{n}{kde}\PY{o}{=}\PY{l+m+mi}{1}\PY{p}{,}\PY{n}{norm\PYZus{}hist}\PY{o}{=}\PY{l+m+mi}{0}\PY{p}{)}
          \PY{n}{plt}\PY{o}{.}\PY{n}{savefig}\PY{p}{(}\PY{l+s+s1}{\PYZsq{}}\PY{l+s+s1}{CGPADistMed.png}\PY{l+s+s1}{\PYZsq{}}\PY{p}{,}\PY{n}{dpi}\PY{o}{=}\PY{l+m+mi}{100}\PY{p}{)}
\end{Verbatim}

    \begin{center}
    \adjustimage{max size={0.9\linewidth}{0.9\paperheight}}{output_155_0.png}
    \end{center}
    { \hspace*{\fill} \\}
    
    

    Commencons par les attribues numeriques et analysons la correlation
chercher comme sur le point précedant la correlation avec le CGPA

    \paragraph{B.1 Attribue Age, Diplome Percentage Vs
CGPA}\label{b.1-attribue-age-diplome-percentage-vs-cgpa}

    \begin{Verbatim}[commandchars=\\\{\}]
{\color{incolor}In [{\color{incolor}461}]:} \PY{n}{Techno}\PY{o}{.}\PY{n}{corr}\PY{p}{(}\PY{p}{)}\PY{p}{[}\PY{l+s+s1}{\PYZsq{}}\PY{l+s+s1}{CGPA}\PY{l+s+s1}{\PYZsq{}}\PY{p}{]}\PY{o}{.}\PY{n}{sort\PYZus{}values}\PY{p}{(}\PY{p}{)}
\end{Verbatim}

            \begin{Verbatim}[commandchars=\\\{\}]
{\color{outcolor}Out[{\color{outcolor}461}]:} ID          -0.239869
          AGE         -0.032067
          DIPPERC      0.346685
          NACADYEAR    0.463141
          CGPA         1.000000
          Name: CGPA, dtype: float64
\end{Verbatim}
        
    \begin{itemize}
\tightlist
\item
  Diplome PErcentage : Le coefiicient de correlation de pearson est de
  0.34 donc il n'yas pas un lien considerable entre le pourcentage du
  diplome d'etat et le CGPA
\item
  Meme conclusion Pour l'age
\end{itemize}

    \paragraph{B.2 Gender}\label{b.2-gender}

    \begin{Verbatim}[commandchars=\\\{\}]
{\color{incolor}In [{\color{incolor}462}]:} \PY{n}{moore\PYZus{}lm} \PY{o}{=} \PY{n}{ols}\PY{p}{(}\PY{l+s+s1}{\PYZsq{}}\PY{l+s+s1}{CGPA \PYZti{} C(GENDER)}\PY{l+s+s1}{\PYZsq{}}\PY{p}{,}\PY{n}{data}\PY{o}{=}\PY{n}{Techno}\PY{p}{)}\PY{o}{.}\PY{n}{fit}\PY{p}{(}\PY{p}{)}
          \PY{n}{aov\PYZus{}table} \PY{o}{=} \PY{n}{sm}\PY{o}{.}\PY{n}{stats}\PY{o}{.}\PY{n}{anova\PYZus{}lm}\PY{p}{(}\PY{n}{moore\PYZus{}lm}\PY{p}{,} \PY{n}{typ}\PY{o}{=}\PY{l+m+mi}{1}\PY{p}{)}
          \PY{n}{aov\PYZus{}table}
\end{Verbatim}

            \begin{Verbatim}[commandchars=\\\{\}]
{\color{outcolor}Out[{\color{outcolor}462}]:}               df        sum\_sq     mean\_sq        F    PR(>F)
          C(GENDER)    1.0    232.763899  232.763899  3.04747  0.081204
          Residual   901.0  68817.820379   76.379379      NaN       NaN
\end{Verbatim}
        
    Notre metrique nous montre que la moyenne de CGPA est la meme pour les
deux sexes ! jettons un coup d'oeil aux distribution avec des box-plot.

    \begin{Verbatim}[commandchars=\\\{\}]
{\color{incolor}In [{\color{incolor}463}]:} \PY{n}{plt}\PY{o}{.}\PY{n}{figure}\PY{p}{(}\PY{n}{figsize}\PY{o}{=}\PY{p}{(}\PY{l+m+mi}{12}\PY{p}{,}\PY{l+m+mi}{6}\PY{p}{)}\PY{p}{)}
          \PY{n}{ax}\PY{o}{=}\PY{n}{sns}\PY{o}{.}\PY{n}{boxplot}\PY{p}{(}\PY{n}{x}\PY{o}{=}\PY{l+s+s2}{\PYZdq{}}\PY{l+s+s2}{GENDER}\PY{l+s+s2}{\PYZdq{}}\PY{p}{,} \PY{n}{y}\PY{o}{=}\PY{l+s+s2}{\PYZdq{}}\PY{l+s+s2}{CGPA}\PY{l+s+s2}{\PYZdq{}}\PY{p}{,} \PY{n}{data}\PY{o}{=}\PY{n}{Techno}\PY{p}{)}
          \PY{n}{plt}\PY{o}{.}\PY{n}{savefig}\PY{p}{(}\PY{l+s+s1}{\PYZsq{}}\PY{l+s+s1}{GENDER\PYZhy{}CGPA\PYZhy{}Tech.png}\PY{l+s+s1}{\PYZsq{}}\PY{p}{,}\PY{n}{dpi}\PY{o}{=}\PY{l+m+mi}{100}\PY{p}{)}
\end{Verbatim}

    \begin{center}
    \adjustimage{max size={0.9\linewidth}{0.9\paperheight}}{output_164_0.png}
    \end{center}
    { \hspace*{\fill} \\}
    
    \begin{Verbatim}[commandchars=\\\{\}]
{\color{incolor}In [{\color{incolor}465}]:} \PY{n}{Techno}\PY{o}{.}\PY{n}{GENDER}\PY{o}{.}\PY{n}{value\PYZus{}counts}\PY{p}{(}\PY{p}{)}\PY{o}{/}\PY{n+nb}{len}\PY{p}{(}\PY{n}{Techno}\PY{p}{)}
\end{Verbatim}

            \begin{Verbatim}[commandchars=\\\{\}]
{\color{outcolor}Out[{\color{outcolor}465}]:} H    0.844961
          F    0.155039
          Name: GENDER, dtype: float64
\end{Verbatim}
        
    Nous remarquons que les 2 graphiques sont les memes , memes il ya un
desequilibre car la facultée est constitué par 85\% des hommes et 15\%
des femmes

    \begin{Verbatim}[commandchars=\\\{\}]
{\color{incolor}In [{\color{incolor}466}]:} \PY{n}{Techno}\PY{o}{.}\PY{n}{groupby}\PY{p}{(}\PY{l+s+s1}{\PYZsq{}}\PY{l+s+s1}{GENDER}\PY{l+s+s1}{\PYZsq{}}\PY{p}{)}\PY{o}{.}\PY{n}{mean}\PY{p}{(}\PY{p}{)}
\end{Verbatim}

            \begin{Verbatim}[commandchars=\\\{\}]
{\color{outcolor}Out[{\color{outcolor}466}]:}                  ID    DIPPERC        AGE       CGPA  NACADYEAR
          GENDER                                                         
          F       9361.742857  60.905379  22.271429  56.604940   2.035714
          H       9130.601573  58.581266  23.498034  55.202206   1.858453
\end{Verbatim}
        
    Mais on peut remarquer ques les femmes reussient bien legerement que les
homme!

    \paragraph{B.4 Attribue Province}\label{b.4-attribue-province}

    \begin{Verbatim}[commandchars=\\\{\}]
{\color{incolor}In [{\color{incolor}467}]:} \PY{n}{Techno}\PY{o}{.}\PY{n}{columns}
\end{Verbatim}

            \begin{Verbatim}[commandchars=\\\{\}]
{\color{outcolor}Out[{\color{outcolor}467}]:} Index([u'ID', u'SCHOOLSTATUS', u'SCHOOL\_RIGHT', u'OPTION\_RIGHT', u'FAC',
                 u'SCHOOLPROVINCE', u'GENDER', u'DIPPERC', u'AGE', u'CGPA',
                 u'DistinctionRatio', u'EchecRatio', u'NACADYEAR',
                 u'Pass1stSessionRatio'],
                dtype='object')
\end{Verbatim}
        
    \begin{Verbatim}[commandchars=\\\{\}]
{\color{incolor}In [{\color{incolor}476}]:} \PY{n}{moore\PYZus{}lm} \PY{o}{=} \PY{n}{ols}\PY{p}{(}\PY{l+s+s1}{\PYZsq{}}\PY{l+s+s1}{CGPA \PYZti{} C(SCHOOLPROVINCE)}\PY{l+s+s1}{\PYZsq{}}\PY{p}{,}\PY{n}{data}\PY{o}{=}\PY{n}{Techno}\PY{p}{)}\PY{o}{.}\PY{n}{fit}\PY{p}{(}\PY{p}{)}
          \PY{n}{aov\PYZus{}table} \PY{o}{=} \PY{n}{sm}\PY{o}{.}\PY{n}{stats}\PY{o}{.}\PY{n}{anova\PYZus{}lm}\PY{p}{(}\PY{n}{moore\PYZus{}lm}\PY{p}{,} \PY{n}{typ}\PY{o}{=}\PY{l+m+mi}{1}\PY{p}{)}
          \PY{n}{aov\PYZus{}table}
\end{Verbatim}

            \begin{Verbatim}[commandchars=\\\{\}]
{\color{outcolor}Out[{\color{outcolor}476}]:}                       df        sum\_sq     mean\_sq         F   PR(>F)
          C(SCHOOLPROVINCE)   11.0   1388.355204  126.214109  1.662032  0.07731
          Residual           891.0  67662.229073   75.939651       NaN      NaN
\end{Verbatim}
        
    PR =0.078 est superieur à 0.05 nous pouvons conclure que la moyenne est
la meme sur toutes les facultés

    \begin{Verbatim}[commandchars=\\\{\}]
{\color{incolor}In [{\color{incolor}472}]:} \PY{n}{Techno}\PY{o}{.}\PY{n}{groupby}\PY{p}{(}\PY{l+s+s1}{\PYZsq{}}\PY{l+s+s1}{SCHOOLPROVINCE}\PY{l+s+s1}{\PYZsq{}}\PY{p}{)}\PY{o}{.}\PY{n}{mean}\PY{p}{(}\PY{p}{)}\PY{o}{.}\PY{n}{sort}\PY{p}{(}\PY{n}{axis}\PY{o}{=}\PY{l+m+mi}{0}\PY{p}{,}\PY{n}{columns}\PY{o}{=}\PY{l+s+s1}{\PYZsq{}}\PY{l+s+s1}{CGPA}\PY{l+s+s1}{\PYZsq{}}\PY{p}{,}\PY{n}{ascending}\PY{o}{=}\PY{n+nb+bp}{False}\PY{p}{)}
\end{Verbatim}

            \begin{Verbatim}[commandchars=\\\{\}]
{\color{outcolor}Out[{\color{outcolor}472}]:}                             ID    DIPPERC        AGE       CGPA  NACADYEAR
          SCHOOLPROVINCE                                                            
          KASAI ORIENTAL     9162.000000  66.000000  22.000000  61.599998   3.000000
          inconnu            6095.800000  56.950609  26.200000  58.213333   2.600000
          SUD-KIVU           8610.916667  58.295153  23.800926  56.657716   2.064815
          KASAI OCCIDENTAL   7537.000000  56.000000  24.000000  55.433333   3.000000
          NORD-KIVU          9360.864697  59.220456  22.958009  55.199728   1.828927
          ORIENTALE          9563.125000  59.437500  24.437500  54.013541   1.875000
          KINSHASA           8379.400000  59.400000  30.800000  52.480000   2.000000
          KATANGA           10921.166667  56.666667  23.166667  50.861111   1.333333
          K OR               7683.000000  55.000000  33.000000  49.100000   2.000000
          MANIEMA            9028.285714  55.142857  28.857143  48.642857   1.428571
          BANDUNDU          11548.000000  62.000000  22.000000  42.000000   1.000000
          EQUATEUR           8258.000000  56.000000  26.000000  41.000000   1.000000
\end{Verbatim}
        
    Nous remarquons que la moyenne de GPA est superieur Pour la province
Kasai oriental , ensuite vienne la province du sud-Kivu , le kasai
occiental et la province du Nord Kivu

    \begin{Verbatim}[commandchars=\\\{\}]
{\color{incolor}In [{\color{incolor}474}]:} \PY{n}{plt}\PY{o}{.}\PY{n}{figure}\PY{p}{(}\PY{n}{figsize}\PY{o}{=}\PY{p}{(}\PY{l+m+mi}{15}\PY{p}{,}\PY{l+m+mi}{6}\PY{p}{)}\PY{p}{)}
          \PY{n}{ax}\PY{o}{=}\PY{n}{sns}\PY{o}{.}\PY{n}{boxplot}\PY{p}{(}\PY{n}{x}\PY{o}{=}\PY{l+s+s2}{\PYZdq{}}\PY{l+s+s2}{SCHOOLPROVINCE}\PY{l+s+s2}{\PYZdq{}}\PY{p}{,} \PY{n}{y}\PY{o}{=}\PY{l+s+s2}{\PYZdq{}}\PY{l+s+s2}{CGPA}\PY{l+s+s2}{\PYZdq{}}\PY{p}{,} \PY{n}{data}\PY{o}{=}\PY{n}{Techno}\PY{p}{)}
          \PY{n}{plt}\PY{o}{.}\PY{n}{savefig}\PY{p}{(}\PY{l+s+s1}{\PYZsq{}}\PY{l+s+s1}{GENDER\PYZhy{}SCHOOLPROVINCE\PYZhy{}Tech.png}\PY{l+s+s1}{\PYZsq{}}\PY{p}{,}\PY{n}{dpi}\PY{o}{=}\PY{l+m+mi}{100}\PY{p}{)}
\end{Verbatim}

    \begin{center}
    \adjustimage{max size={0.9\linewidth}{0.9\paperheight}}{output_175_0.png}
    \end{center}
    { \hspace*{\fill} \\}
    
    Sur le graphique on conclu Rapidement que les étudiant provenant de la
province du sud Kivu sont plus performant en faulté de Technologie

    \paragraph{B.5 Stattus de L'école}\label{b.5-stattus-de-luxe9cole}

    \begin{Verbatim}[commandchars=\\\{\}]
{\color{incolor}In [{\color{incolor}477}]:} \PY{n}{moore\PYZus{}lm} \PY{o}{=} \PY{n}{ols}\PY{p}{(}\PY{l+s+s1}{\PYZsq{}}\PY{l+s+s1}{CGPA \PYZti{} C(SCHOOLSTATUS)}\PY{l+s+s1}{\PYZsq{}}\PY{p}{,}\PY{n}{data}\PY{o}{=}\PY{n}{Techno}\PY{p}{)}\PY{o}{.}\PY{n}{fit}\PY{p}{(}\PY{p}{)}
          \PY{n}{aov\PYZus{}table} \PY{o}{=} \PY{n}{sm}\PY{o}{.}\PY{n}{stats}\PY{o}{.}\PY{n}{anova\PYZus{}lm}\PY{p}{(}\PY{n}{moore\PYZus{}lm}\PY{p}{,} \PY{n}{typ}\PY{o}{=}\PY{l+m+mi}{1}\PY{p}{)}
          \PY{n}{aov\PYZus{}table}
\end{Verbatim}

            \begin{Verbatim}[commandchars=\\\{\}]
{\color{outcolor}Out[{\color{outcolor}477}]:}                     df        sum\_sq     mean\_sq          F        PR(>F)
          C(SCHOOLSTATUS)    7.0   5417.817584  773.973941  10.886006  3.123579e-13
          Residual         895.0  63632.766694   71.098063        NaN           NaN
\end{Verbatim}
        
    Il n'ya aucun lien entre les moyennes de CGPA pour les differents status
de l"ecole

    \begin{Verbatim}[commandchars=\\\{\}]
{\color{incolor}In [{\color{incolor}478}]:} \PY{n}{Techno}\PY{o}{.}\PY{n}{groupby}\PY{p}{(}\PY{l+s+s1}{\PYZsq{}}\PY{l+s+s1}{SCHOOLSTATUS}\PY{l+s+s1}{\PYZsq{}}\PY{p}{)}\PY{o}{.}\PY{n}{mean}\PY{p}{(}\PY{p}{)}\PY{o}{.}\PY{n}{sort}\PY{p}{(}\PY{n}{axis}\PY{o}{=}\PY{l+m+mi}{0}\PY{p}{,}\PY{n}{columns}\PY{o}{=}\PY{l+s+s1}{\PYZsq{}}\PY{l+s+s1}{CGPA}\PY{l+s+s1}{\PYZsq{}}\PY{p}{,}\PY{n}{ascending}\PY{o}{=}\PY{n+nb+bp}{False}\PY{p}{)}
\end{Verbatim}

            \begin{Verbatim}[commandchars=\\\{\}]
{\color{outcolor}Out[{\color{outcolor}478}]:}                         ID    DIPPERC        AGE       CGPA  NACADYEAR
          SCHOOLSTATUS                                                          
          catholique     8943.467422  61.121813  23.116147  58.326393   2.011331
          autodidacte    9656.222222  59.888889  24.888889  54.787963   2.000000
          protestant     9337.902326  58.241286  22.995349  54.296008   1.739535
          inconnu        8644.352000  56.811061  24.800000  53.958533   2.120000
          publique       9611.635659  57.348837  22.976744  52.992959   1.728682
          privé          9754.985714  56.742857  22.985714  51.556667   1.571429
          musulman      10336.000000  61.000000  23.000000  51.150000   2.000000
          kimbanguiste  12064.000000  55.000000  23.000000  47.000000   1.000000
\end{Verbatim}
        
    Nous remarquons que les éetudiant venant des écoles catholiques
reussiseent mieux en faculté de techno , ensuite les autodidacte
,ensuite ceux des ecoles protestatntes

    Voyons cela de plus pret avec box-plot

    \begin{Verbatim}[commandchars=\\\{\}]
{\color{incolor}In [{\color{incolor}479}]:} \PY{n}{plt}\PY{o}{.}\PY{n}{figure}\PY{p}{(}\PY{n}{figsize}\PY{o}{=}\PY{p}{(}\PY{l+m+mi}{12}\PY{p}{,}\PY{l+m+mi}{6}\PY{p}{)}\PY{p}{)}
          \PY{n}{ax}\PY{o}{=}\PY{n}{sns}\PY{o}{.}\PY{n}{boxplot}\PY{p}{(}\PY{n}{x}\PY{o}{=}\PY{l+s+s2}{\PYZdq{}}\PY{l+s+s2}{SCHOOLSTATUS}\PY{l+s+s2}{\PYZdq{}}\PY{p}{,} \PY{n}{y}\PY{o}{=}\PY{l+s+s2}{\PYZdq{}}\PY{l+s+s2}{CGPA}\PY{l+s+s2}{\PYZdq{}}\PY{p}{,} \PY{n}{data}\PY{o}{=}\PY{n}{Techno}\PY{p}{)}
          \PY{n}{plt}\PY{o}{.}\PY{n}{savefig}\PY{p}{(}\PY{l+s+s1}{\PYZsq{}}\PY{l+s+s1}{GENDER\PYZhy{}SCHOOLSTATUS\PYZhy{}Tech.png}\PY{l+s+s1}{\PYZsq{}}\PY{p}{,}\PY{n}{dpi}\PY{o}{=}\PY{l+m+mi}{100}\PY{p}{)}
\end{Verbatim}

    \begin{center}
    \adjustimage{max size={0.9\linewidth}{0.9\paperheight}}{output_183_0.png}
    \end{center}
    { \hspace*{\fill} \\}
    
    \paragraph{B.Les Ecoles de
provenances}\label{b.les-ecoles-de-provenances}

    \begin{Verbatim}[commandchars=\\\{\}]
{\color{incolor}In [{\color{incolor}480}]:} \PY{n}{moore\PYZus{}lm} \PY{o}{=} \PY{n}{ols}\PY{p}{(}\PY{l+s+s1}{\PYZsq{}}\PY{l+s+s1}{CGPA \PYZti{} C(SCHOOL\PYZus{}RIGHT)}\PY{l+s+s1}{\PYZsq{}}\PY{p}{,}\PY{n}{data}\PY{o}{=}\PY{n}{Techno}\PY{p}{)}\PY{o}{.}\PY{n}{fit}\PY{p}{(}\PY{p}{)}
          \PY{n}{aov\PYZus{}table} \PY{o}{=} \PY{n}{sm}\PY{o}{.}\PY{n}{stats}\PY{o}{.}\PY{n}{anova\PYZus{}lm}\PY{p}{(}\PY{n}{moore\PYZus{}lm}\PY{p}{,} \PY{n}{typ}\PY{o}{=}\PY{l+m+mi}{1}\PY{p}{)}
          \PY{n}{aov\PYZus{}table}
\end{Verbatim}

            \begin{Verbatim}[commandchars=\\\{\}]
{\color{outcolor}Out[{\color{outcolor}480}]:}                     df        sum\_sq     mean\_sq         F        PR(>F)
          C(SCHOOL\_RIGHT)  215.0  26367.460860  122.639353  1.973924  3.939156e-11
          Residual         687.0  42683.123418   62.129728       NaN           NaN
\end{Verbatim}
        
    Aucunne correlation n'existe entre l"ecole et le CGPA au vu de la valeur
de PR'

    \begin{Verbatim}[commandchars=\\\{\}]
{\color{incolor}In [{\color{incolor}504}]:} \PY{n}{SchoolGroup}\PY{o}{=}\PY{n}{Techno}\PY{o}{.}\PY{n}{groupby}\PY{p}{(}\PY{l+s+s1}{\PYZsq{}}\PY{l+s+s1}{SCHOOL\PYZus{}RIGHT}\PY{l+s+s1}{\PYZsq{}}\PY{p}{)}\PY{o}{.}\PY{n}{mean}\PY{p}{(}\PY{p}{)}\PY{o}{.}\PY{n}{sort}\PY{p}{(}\PY{n}{axis}\PY{o}{=}\PY{l+m+mi}{0}\PY{p}{,}\PY{n}{columns}\PY{o}{=}\PY{l+s+s1}{\PYZsq{}}\PY{l+s+s1}{CGPA}\PY{l+s+s1}{\PYZsq{}}\PY{p}{,}\PY{n}{ascending}\PY{o}{=}\PY{n+nb+bp}{False}\PY{p}{)}
\end{Verbatim}

    \begin{Verbatim}[commandchars=\\\{\}]
{\color{incolor}In [{\color{incolor}505}]:} \PY{n}{SchoolGroup}\PY{o}{.}\PY{n}{reset\PYZus{}index}\PY{p}{(}\PY{n}{inplace}\PY{o}{=}\PY{n+nb+bp}{True}\PY{p}{)}
\end{Verbatim}

    \begin{Verbatim}[commandchars=\\\{\}]
{\color{incolor}In [{\color{incolor}506}]:} \PY{n}{SchoolGroup}
\end{Verbatim}

            \begin{Verbatim}[commandchars=\\\{\}]
{\color{outcolor}Out[{\color{outcolor}506}]:}                       SCHOOL\_RIGHT            ID    DIPPERC        AGE  \textbackslash{}
          0                           nikisi   4249.000000  58.000000  26.000000   
          1        technique ind. de mahamba  12228.000000  62.000000  24.000000   
          2                    INST DE KATWA   3992.000000  67.000000  27.000000   
          3                       mululusake  11662.000000  73.000000  19.000000   
          4                           cirezi   8139.500000  58.750000  22.000000   
          5                            vungi   8077.666667  55.000000  23.666667   
          6                      itfm/bukavu   7963.804878  58.777476  23.902439   
          7                     saint michel  10016.000000  60.000000  23.000000   
          8                           katana   9144.000000  57.000000  22.000000   
          9                        it bugabo  10308.000000  50.000000  22.000000   
          10                         kambali   9772.500000  63.500000  22.500000   
          11                            wima   8213.400000  69.400000  22.600000   
          12                    epsk/fomulac  10251.000000  60.500000  22.500000   
          13                 edap/isp bukavu  10678.250000  61.250000  22.000000   
          14                          du lac  10452.000000  52.000000  21.000000   
          15                          MWANDU   6742.000000  58.000000  25.000000   
          16                          Lwanga  10275.666667  56.666667  21.666667   
          17                         i katwa   8508.000000  65.000000  23.000000   
          18                          LWANGA   7650.000000  66.000000  27.000000   
          19                              61   8477.000000  56.876522  23.000000   
          20                    kanyabayonga   8439.500000  53.500000  25.500000   
          21                         de beni  10213.000000  53.000000  20.000000   
          22                           amani   9254.590909  68.863636  21.863636   
          23                            itig   7725.847458  59.542373  24.949153   
          24                       mgr guido   8385.000000  57.000000  23.666667   
          25                        metanoia   8642.814815  62.740741  22.888889   
          26                    edap isp bkv   9219.333333  59.333333  22.666667   
          27                          avenir   8828.714286  53.571429  24.071429   
          28                        edap/isp   6666.857143  58.285714  26.142857   
          29                         majengo   8428.282051  61.358974  23.666667   
          ..                             {\ldots}           {\ldots}        {\ldots}        {\ldots}   
          183                        lumumba   9807.000000  53.000000  42.000000   
          184                           alfa   9161.000000  54.000000  24.000000   
          185                         masisi  11624.000000  52.000000  22.000000   
          186                       ipp/beni  11808.000000  59.000000  19.000000   
          187                  kahya cibanda  10081.500000  51.000000  23.500000   
          188                         mayele   8742.000000  58.000000  30.000000   
          189                      kiribunye   9074.000000  50.000000  24.000000   
          190                    st augustin  10765.000000  56.000000  26.000000   
          191                     la pereaux  11823.000000  58.000000  21.000000   
          192                          alpha  11990.000000  57.000000  23.000000   
          193                     kitsombiro   9882.000000  52.000000  24.000000   
          194                         bwindi   7550.000000  59.000000  26.000000   
          195                      nengapeta   4246.000000  51.000000  30.000000   
          196                        wai wai  11754.000000  53.000000  20.000000   
          197                           tuzo  10717.000000  51.000000  27.000000   
          198                         asseco   9520.000000  55.000000  24.000000   
          199                           mehe  10855.000000  63.000000  28.000000   
          200           saint mariya goritti  11548.000000  62.000000  22.000000   
          201            technique mapendano  11985.000000  52.000000  23.000000   
          202                         mululu   9949.000000  57.000000  21.000000   
          203                        mungano  12181.000000  57.000000  22.000000   
          204                           ovoa  10868.000000  57.000000  25.000000   
          205  complexe scolaire nova stella  10314.000000  55.000000  21.000000   
          206                         maboso   8258.000000  56.000000  26.000000   
          207                        molière   9210.000000  63.000000  22.000000   
          208                       kimbilio  10669.000000  59.000000  23.000000   
          209                         kahumo  11892.000000  59.000000  26.000000   
          210                          kimua   9893.000000  57.000000  24.000000   
          211                           kaya  10914.000000  52.000000  23.000000   
          212                        buhimba   9529.000000  60.000000  22.000000   
          
                    CGPA  NACADYEAR  
          0    75.349998   2.000000  
          1    71.900002   1.000000  
          2    68.900002   2.000000  
          3    68.500000   1.000000  
          4    65.970834   3.250000  
          5    64.772222   2.666667  
          6    64.395529   2.048780  
          7    64.049999   2.000000  
          8    63.466667   3.000000  
          9    62.799999   2.000000  
          10   62.529166   2.000000  
          11   62.156667   3.000000  
          12   61.916665   2.000000  
          13   61.912500   2.000000  
          14   61.650000   2.000000  
          15   61.633334   3.000000  
          16   61.583333   1.333333  
          17   61.550000   4.000000  
          18   61.366667   3.000000  
          19   61.025002   4.000000  
          20   60.958333   2.500000  
          21   60.799999   2.000000  
          22   60.732575   2.318182  
          23   60.627684   2.254237  
          24   60.200001   2.000000  
          25   60.040741   2.074074  
          26   59.901389   2.666667  
          27   59.677381   2.142857  
          28   59.600000   2.142857  
          29   59.442735   1.871795  
          ..         {\ldots}        {\ldots}  
          183  44.000000   1.000000  
          184  44.000000   1.000000  
          185  44.000000   1.000000  
          186  44.000000   1.000000  
          187  43.500000   1.000000  
          188  43.500000   2.000000  
          189  43.000000   1.000000  
          190  43.000000   2.000000  
          191  43.000000   1.000000  
          192  43.000000   1.000000  
          193  43.000000   1.000000  
          194  42.500000   2.000000  
          195  42.400000   2.000000  
          196  42.000000   1.000000  
          197  42.000000   1.000000  
          198  42.000000   1.000000  
          199  42.000000   1.000000  
          200  42.000000   1.000000  
          201  42.000000   1.000000  
          202  42.000000   1.000000  
          203  42.000000   1.000000  
          204  42.000000   1.000000  
          205  41.599998   1.000000  
          206  41.000000   1.000000  
          207  41.000000   1.000000  
          208  41.000000   1.000000  
          209  40.000000   1.000000  
          210  40.000000   1.000000  
          211  40.000000   1.000000  
          212  40.000000   1.000000  
          
          [213 rows x 6 columns]
\end{Verbatim}
        
    Nous allons Voir de plus pret pour les 10 ecoles les plus representé, le
5 éecoles avec un pourcentage elevé et 5 dernier

    \begin{Verbatim}[commandchars=\\\{\}]
{\color{incolor}In [{\color{incolor}519}]:} \PY{n}{plt}\PY{o}{.}\PY{n}{figure}\PY{p}{(}\PY{n}{figsize}\PY{o}{=}\PY{p}{(}\PY{l+m+mi}{30}\PY{p}{,}\PY{l+m+mi}{10}\PY{p}{)}\PY{p}{)}
          \PY{n}{ax}\PY{o}{=}\PY{n}{sns}\PY{o}{.}\PY{n}{boxplot}\PY{p}{(}\PY{n}{x}\PY{o}{=}\PY{l+s+s2}{\PYZdq{}}\PY{l+s+s2}{SCHOOL\PYZus{}RIGHT}\PY{l+s+s2}{\PYZdq{}}\PY{p}{,} \PY{n}{y}\PY{o}{=}\PY{l+s+s2}{\PYZdq{}}\PY{l+s+s2}{CGPA}\PY{l+s+s2}{\PYZdq{}}\PY{p}{,} \PY{n}{data}\PY{o}{=}\PY{n}{Techno}\PY{o}{.}\PY{n}{loc}\PY{p}{[}\PY{n}{operator}\PY{o}{.}\PY{n}{or\PYZus{}}\PY{p}{(}\PY{n}{Techno}\PY{o}{.}\PY{n}{SCHOOL\PYZus{}RIGHT}\PY{o}{.}\PY{n}{isin}\PY{p}{(}\PY{n}{SchoolGroup}\PY{o}{.}\PY{n}{loc}\PY{p}{[} \PY{n+nb}{range}\PY{p}{(}\PY{l+m+mi}{0}\PY{p}{,}\PY{l+m+mi}{11}\PY{p}{)}\PY{p}{]}\PY{o}{.}\PY{n}{SCHOOL\PYZus{}RIGHT}\PY{p}{)} 
                       \PY{p}{,} \PY{n}{Techno}\PY{o}{.}\PY{n}{SCHOOL\PYZus{}RIGHT}\PY{o}{.}\PY{n}{isin}\PY{p}{(}\PY{n}{Techno}\PY{o}{.}\PY{n}{SCHOOL\PYZus{}RIGHT}\PY{o}{.}\PY{n}{value\PYZus{}counts}\PY{p}{(}\PY{p}{)}\PY{p}{[}\PY{p}{:}\PY{l+m+mi}{10}\PY{p}{]}\PY{o}{.}\PY{n}{index} \PY{p}{)}\PY{p}{)}\PY{p}{]}\PY{p}{)}
          \PY{n}{plt}\PY{o}{.}\PY{n}{savefig}\PY{p}{(}\PY{l+s+s1}{\PYZsq{}}\PY{l+s+s1}{SCHOOL\PYZus{}RIGHT\PYZhy{}CGPA.png}\PY{l+s+s1}{\PYZsq{}}\PY{p}{,}\PY{n}{dpi}\PY{o}{=}\PY{l+m+mi}{100}\PY{p}{)}
\end{Verbatim}

    \begin{center}
    \adjustimage{max size={0.9\linewidth}{0.9\paperheight}}{output_191_0.png}
    \end{center}
    { \hspace*{\fill} \\}
    
    On remarques sur la figure que les grandes écoles connus on une bonne
moyenne et un bon maximun de CGPA comme le lycée Amani, l'ITIG , l'ITFM!

    \paragraph{B.7 OPTION DU DIPLOME}\label{b.7-option-du-diplome}

    \begin{Verbatim}[commandchars=\\\{\}]
{\color{incolor}In [{\color{incolor}521}]:} \PY{n+nb}{len}\PY{p}{(}\PY{n}{Techno}\PY{o}{.}\PY{n}{OPTION\PYZus{}RIGHT}\PY{o}{.}\PY{n}{value\PYZus{}counts}\PY{p}{(}\PY{p}{)}\PY{p}{)}
\end{Verbatim}

            \begin{Verbatim}[commandchars=\\\{\}]
{\color{outcolor}Out[{\color{outcolor}521}]:} 23
\end{Verbatim}
        
    \begin{Verbatim}[commandchars=\\\{\}]
{\color{incolor}In [{\color{incolor} }]:} 
\end{Verbatim}

    \begin{Verbatim}[commandchars=\\\{\}]
{\color{incolor}In [{\color{incolor} }]:} 
\end{Verbatim}

    \begin{Verbatim}[commandchars=\\\{\}]
{\color{incolor}In [{\color{incolor} }]:} 
\end{Verbatim}

    \begin{Verbatim}[commandchars=\\\{\}]
{\color{incolor}In [{\color{incolor} }]:} 
\end{Verbatim}

    \begin{Verbatim}[commandchars=\\\{\}]
{\color{incolor}In [{\color{incolor} }]:} 
\end{Verbatim}

    \begin{Verbatim}[commandchars=\\\{\}]
{\color{incolor}In [{\color{incolor} }]:} 
\end{Verbatim}

    \begin{Verbatim}[commandchars=\\\{\}]
{\color{incolor}In [{\color{incolor}499}]:} \PY{n}{datasetFin}\PY{o}{.}\PY{n}{loc}\PY{p}{[}\PY{n}{datasetFin}\PY{o}{.}\PY{n}{SCHOOL\PYZus{}RIGHT}\PY{o}{==}\PY{l+s+s1}{\PYZsq{}}\PY{l+s+s1}{itfm maendeleo}\PY{l+s+s1}{\PYZsq{}}\PY{p}{,}\PY{l+s+s1}{\PYZsq{}}\PY{l+s+s1}{SCHOOL\PYZus{}RIGHT}\PY{l+s+s1}{\PYZsq{}}\PY{p}{]}\PY{o}{=}\PY{l+s+s1}{\PYZsq{}}\PY{l+s+s1}{itfm/bukavu}\PY{l+s+s1}{\PYZsq{}}
\end{Verbatim}

    \begin{Verbatim}[commandchars=\\\{\}]
{\color{incolor}In [{\color{incolor}503}]:} \PY{n}{Techno}\PY{o}{.}\PY{n}{groupby}\PY{p}{(}\PY{l+s+s1}{\PYZsq{}}\PY{l+s+s1}{SCHOOL\PYZus{}RIGHT}\PY{l+s+s1}{\PYZsq{}}\PY{p}{)}\PY{o}{.}\PY{n}{mean}\PY{p}{(}\PY{p}{)}\PY{o}{.}\PY{n}{sort}\PY{p}{(}\PY{n}{axis}\PY{o}{=}\PY{l+m+mi}{0}\PY{p}{,}\PY{n}{columns}\PY{o}{=}\PY{l+s+s1}{\PYZsq{}}\PY{l+s+s1}{CGPA}\PY{l+s+s1}{\PYZsq{}}\PY{p}{,}\PY{n}{ascending}\PY{o}{=}\PY{n+nb+bp}{False}\PY{p}{)}
\end{Verbatim}

            \begin{Verbatim}[commandchars=\\\{\}]
{\color{outcolor}Out[{\color{outcolor}503}]:}                         ID    DIPPERC        AGE       CGPA  NACADYEAR
          SCHOOLSTATUS                                                          
          catholique     8951.403955  61.124294  23.107345  58.297222   2.008475
          autodidacte    9656.222222  59.888889  24.888889  54.787963   2.000000
          protestant     9337.902326  58.241286  22.995349  54.296008   1.739535
          inconnu        8644.352000  56.811061  24.800000  53.958533   2.120000
          publique       9594.906250  57.312500  23.000000  53.031966   1.734375
          privé          9754.985714  56.742857  22.985714  51.556667   1.571429
          musulman      10336.000000  61.000000  23.000000  51.150000   2.000000
          kimbanguiste  12064.000000  55.000000  23.000000  47.000000   1.000000
\end{Verbatim}
        
    Nous pouvons constater que nous avons 23 options differents Voyons
comment evoluer la moyenne

    \begin{Verbatim}[commandchars=\\\{\}]
{\color{incolor}In [{\color{incolor}522}]:} \PY{n}{moore\PYZus{}lm} \PY{o}{=} \PY{n}{ols}\PY{p}{(}\PY{l+s+s1}{\PYZsq{}}\PY{l+s+s1}{CGPA \PYZti{} C(OPTION\PYZus{}RIGHT)}\PY{l+s+s1}{\PYZsq{}}\PY{p}{,}\PY{n}{data}\PY{o}{=}\PY{n}{Techno}\PY{p}{)}\PY{o}{.}\PY{n}{fit}\PY{p}{(}\PY{p}{)}
          \PY{n}{aov\PYZus{}table} \PY{o}{=} \PY{n}{sm}\PY{o}{.}\PY{n}{stats}\PY{o}{.}\PY{n}{anova\PYZus{}lm}\PY{p}{(}\PY{n}{moore\PYZus{}lm}\PY{p}{,} \PY{n}{typ}\PY{o}{=}\PY{l+m+mi}{1}\PY{p}{)}
          \PY{n}{aov\PYZus{}table}
\end{Verbatim}

            \begin{Verbatim}[commandchars=\\\{\}]
{\color{outcolor}Out[{\color{outcolor}522}]:}                     df        sum\_sq     mean\_sq         F        PR(>F)
          C(OPTION\_RIGHT)   22.0  11622.024556  528.273843  8.094944  1.055221e-23
          Residual         880.0  57428.559721   65.259727       NaN           NaN
\end{Verbatim}
        
    \begin{Verbatim}[commandchars=\\\{\}]
{\color{incolor}In [{\color{incolor}500}]:} \PY{n}{datasetFin}\PY{o}{.}\PY{n}{loc}\PY{p}{[}\PY{n}{datasetFin}\PY{o}{.}\PY{n}{SCHOOL\PYZus{}RIGHT}\PY{o}{==}\PY{l+s+s1}{\PYZsq{}}\PY{l+s+s1}{itfm maendeleo}\PY{l+s+s1}{\PYZsq{}}\PY{p}{,}\PY{l+s+s1}{\PYZsq{}}\PY{l+s+s1}{SCHOOLSTATUS}\PY{l+s+s1}{\PYZsq{}}\PY{p}{]}\PY{o}{=}\PY{l+s+s1}{\PYZsq{}}\PY{l+s+s1}{catholique}\PY{l+s+s1}{\PYZsq{}}
\end{Verbatim}

    La valeur est inferieur à 0.05 donc il n'yas pas de relation entre les
valeurs

    \begin{Verbatim}[commandchars=\\\{\}]
{\color{incolor}In [{\color{incolor}527}]:} \PY{n}{GroupOption}\PY{o}{=}\PY{n}{Techno}\PY{o}{.}\PY{n}{groupby}\PY{p}{(}\PY{l+s+s1}{\PYZsq{}}\PY{l+s+s1}{OPTION\PYZus{}RIGHT}\PY{l+s+s1}{\PYZsq{}}\PY{p}{)}\PY{o}{.}\PY{n}{mean}\PY{p}{(}\PY{p}{)}\PY{o}{.}\PY{n}{sort}\PY{p}{(}\PY{n}{axis}\PY{o}{=}\PY{l+m+mi}{0}\PY{p}{,}\PY{n}{columns}\PY{o}{=}\PY{l+s+s1}{\PYZsq{}}\PY{l+s+s1}{CGPA}\PY{l+s+s1}{\PYZsq{}}\PY{p}{,}\PY{n}{ascending}\PY{o}{=}\PY{n+nb+bp}{False}\PY{p}{)}\PY{o}{.}\PY{n}{reset\PYZus{}index}\PY{p}{(}\PY{p}{)}
\end{Verbatim}

    Voila comme on pourait s'y attendre les options techniques et
Scientifiques viennent en tete de liste et en bas de l'echel vient les
ption littereraires et commerciales

    \begin{Verbatim}[commandchars=\\\{\}]
{\color{incolor}In [{\color{incolor}529}]:} \PY{n}{plt}\PY{o}{.}\PY{n}{figure}\PY{p}{(}\PY{n}{figsize}\PY{o}{=}\PY{p}{(}\PY{l+m+mi}{30}\PY{p}{,}\PY{l+m+mi}{10}\PY{p}{)}\PY{p}{)}
          \PY{n}{ax}\PY{o}{=}\PY{n}{sns}\PY{o}{.}\PY{n}{boxplot}\PY{p}{(}\PY{n}{x}\PY{o}{=}\PY{l+s+s2}{\PYZdq{}}\PY{l+s+s2}{OPTION\PYZus{}RIGHT}\PY{l+s+s2}{\PYZdq{}}\PY{p}{,} \PY{n}{y}\PY{o}{=}\PY{l+s+s2}{\PYZdq{}}\PY{l+s+s2}{CGPA}\PY{l+s+s2}{\PYZdq{}}\PY{p}{,} \PY{n}{data}\PY{o}{=}\PY{n}{Techno}\PY{p}{,}\PY{n}{order}\PY{o}{=}\PY{n}{GroupOption}\PY{o}{.}\PY{n}{OPTION\PYZus{}RIGHT}\PY{p}{)}
          \PY{n}{plt}\PY{o}{.}\PY{n}{savefig}\PY{p}{(}\PY{l+s+s1}{\PYZsq{}}\PY{l+s+s1}{OPTION\PYZus{}RIGHT\PYZhy{}CGPA.png}\PY{l+s+s1}{\PYZsq{}}\PY{p}{,}\PY{n}{dpi}\PY{o}{=}\PY{l+m+mi}{100}\PY{p}{)}
\end{Verbatim}

    \begin{center}
    \adjustimage{max size={0.9\linewidth}{0.9\paperheight}}{output_209_0.png}
    \end{center}
    { \hspace*{\fill} \\}
    
    cela peut se remarquer aisement sur la figures que les sections
techniques sont ceux dont le étudiants reussisnet le mieux

    \begin{Verbatim}[commandchars=\\\{\}]
{\color{incolor}In [{\color{incolor}530}]:} \PY{n}{datasetFin}\PY{o}{.}\PY{n}{to\PYZus{}csv}\PY{p}{(}\PY{l+s+s1}{\PYZsq{}}\PY{l+s+s1}{DatasetFinalV1.csv}\PY{l+s+s1}{\PYZsq{}}\PY{p}{)}
          \PY{n}{Techno}\PY{o}{.}\PY{n}{to\PYZus{}csv}\PY{p}{(}\PY{l+s+s2}{\PYZdq{}}\PY{l+s+s2}{DatasetTechno.csv}\PY{l+s+s2}{\PYZdq{}}\PY{p}{)}
          \PY{n}{Medecine}\PY{o}{.}\PY{n}{to\PYZus{}csv}\PY{p}{(}\PY{l+s+s1}{\PYZsq{}}\PY{l+s+s1}{DatasetMedecine.csv}\PY{l+s+s1}{\PYZsq{}}\PY{p}{)}
\end{Verbatim}

    \subparagraph{Faculté D'Economie}\label{facultuxe9-deconomie}

    \begin{Verbatim}[commandchars=\\\{\}]
{\color{incolor}In [{\color{incolor}549}]:} \PY{n}{Economie}\PY{o}{=}\PY{n}{datasetFin}\PY{o}{.}\PY{n}{loc}\PY{p}{[}\PY{n}{datasetFin}\PY{o}{.}\PY{n}{FAC}\PY{o}{==}\PY{l+s+s1}{\PYZsq{}}\PY{l+s+s1}{FSEG}\PY{l+s+s1}{\PYZsq{}}\PY{p}{]}
\end{Verbatim}

    \begin{Verbatim}[commandchars=\\\{\}]
{\color{incolor}In [{\color{incolor}550}]:} \PY{n}{Economie}\PY{o}{.}\PY{n}{shape}
\end{Verbatim}

            \begin{Verbatim}[commandchars=\\\{\}]
{\color{outcolor}Out[{\color{outcolor}550}]:} (1549, 14)
\end{Verbatim}
        
    \begin{Verbatim}[commandchars=\\\{\}]
{\color{incolor}In [{\color{incolor}551}]:} \PY{n}{Techno}\PY{o}{.}\PY{n}{shape}
\end{Verbatim}

            \begin{Verbatim}[commandchars=\\\{\}]
{\color{outcolor}Out[{\color{outcolor}551}]:} (903, 14)
\end{Verbatim}
        
    Nous voici enfin au sein de la faculté d'économie ! Commencons par voir
la répartition des probabilitées de notre variable CGPA

    https://www.wellbeingatschool.org.nz/information-sheet/understanding-and-interpreting-box-plots

    A.0 Distribution du CGPA

    \begin{Verbatim}[commandchars=\\\{\}]
{\color{incolor}In [{\color{incolor}552}]:} \PY{n}{plt}\PY{o}{.}\PY{n}{figure}\PY{p}{(}\PY{p}{)}
          \PY{n}{ax} \PY{o}{=} \PY{n}{sns}\PY{o}{.}\PY{n}{boxplot}\PY{p}{(}\PY{n}{x}\PY{o}{=}\PY{n}{Economie}\PY{p}{[}\PY{l+s+s1}{\PYZsq{}}\PY{l+s+s1}{CGPA}\PY{l+s+s1}{\PYZsq{}}\PY{p}{]}\PY{p}{)}
          \PY{n}{plt}\PY{o}{.}\PY{n}{savefig}\PY{p}{(}\PY{l+s+s1}{\PYZsq{}}\PY{l+s+s1}{CGPA\PYZhy{}ECO.png}\PY{l+s+s1}{\PYZsq{}}\PY{p}{,}\PY{n}{dpi}\PY{o}{=}\PY{l+m+mi}{100}\PY{p}{)}
\end{Verbatim}

    \begin{center}
    \adjustimage{max size={0.9\linewidth}{0.9\paperheight}}{output_219_0.png}
    \end{center}
    { \hspace*{\fill} \\}
    
    \begin{Verbatim}[commandchars=\\\{\}]
{\color{incolor}In [{\color{incolor}553}]:} \PY{n}{plt}\PY{o}{.}\PY{n}{figure}\PY{p}{(}\PY{p}{)}
          \PY{n}{sns}\PY{o}{.}\PY{n}{distplot}\PY{p}{(}\PY{n}{Medecine}\PY{p}{[}\PY{l+s+s1}{\PYZsq{}}\PY{l+s+s1}{CGPA}\PY{l+s+s1}{\PYZsq{}}\PY{p}{]}\PY{p}{,}\PY{n}{bins}\PY{o}{=}\PY{l+m+mi}{20}\PY{p}{,}\PY{n}{axlabel}\PY{o}{=}\PY{l+s+s1}{\PYZsq{}}\PY{l+s+s1}{CGPA}\PY{l+s+s1}{\PYZsq{}}\PY{p}{,}\PY{n}{kde}\PY{o}{=}\PY{l+m+mi}{1}\PY{p}{,}\PY{n}{norm\PYZus{}hist}\PY{o}{=}\PY{l+m+mi}{0}\PY{p}{)}
          \PY{n}{plt}\PY{o}{.}\PY{n}{savefig}\PY{p}{(}\PY{l+s+s1}{\PYZsq{}}\PY{l+s+s1}{CGPADistEco.png}\PY{l+s+s1}{\PYZsq{}}\PY{p}{,}\PY{n}{dpi}\PY{o}{=}\PY{l+m+mi}{100}\PY{p}{)}
\end{Verbatim}

    \begin{center}
    \adjustimage{max size={0.9\linewidth}{0.9\paperheight}}{output_220_0.png}
    \end{center}
    { \hspace*{\fill} \\}
    
    Il suit une distribution presque normale d'écart type : 6.23

    \begin{Verbatim}[commandchars=\\\{\}]
{\color{incolor}In [{\color{incolor}555}]:} \PY{n}{np}\PY{o}{.}\PY{n}{std}\PY{p}{(}\PY{n}{Economie}\PY{o}{.}\PY{n}{CGPA}\PY{p}{)}
\end{Verbatim}

            \begin{Verbatim}[commandchars=\\\{\}]
{\color{outcolor}Out[{\color{outcolor}555}]:} 6.2672402214029903
\end{Verbatim}
        
    \begin{Verbatim}[commandchars=\\\{\}]
{\color{incolor}In [{\color{incolor}557}]:} \PY{n}{np}\PY{o}{.}\PY{n}{mean}\PY{p}{(}\PY{n}{Economie}\PY{o}{.}\PY{n}{CGPA}\PY{p}{)}
\end{Verbatim}

            \begin{Verbatim}[commandchars=\\\{\}]
{\color{outcolor}Out[{\color{outcolor}557}]:} 56.446099617754676
\end{Verbatim}
        
    Commencons par les attribues numeriques et analysons la correlation
chercher comme sur le point précedant la correlation avec le CGPA

    \begin{Verbatim}[commandchars=\\\{\}]
{\color{incolor}In [{\color{incolor}559}]:} \PY{n}{Economie}\PY{o}{.}\PY{n}{corr}\PY{p}{(}\PY{p}{)}
\end{Verbatim}

            \begin{Verbatim}[commandchars=\\\{\}]
{\color{outcolor}Out[{\color{outcolor}559}]:}                  ID   DIPPERC       AGE      CGPA  NACADYEAR
          ID         1.000000 -0.015651 -0.673470 -0.295749  -0.065568
          DIPPERC   -0.015651  1.000000 -0.131960  0.288451   0.171187
          AGE       -0.673470 -0.131960  1.000000  0.085321  -0.025793
          CGPA      -0.295749  0.288451  0.085321  1.000000   0.302593
          NACADYEAR -0.065568  0.171187 -0.025793  0.302593   1.000000
\end{Verbatim}
        
    Nous pouvons aisement qu'il n'yas aucune correlation entre le le CGPA et
le pourcentage à l'exetat ni meme l'age des étudiants

    \begin{Verbatim}[commandchars=\\\{\}]
{\color{incolor}In [{\color{incolor} }]:} 
\end{Verbatim}

    \paragraph{C.2 GENDER}\label{c.2-gender}

    \begin{Verbatim}[commandchars=\\\{\}]
{\color{incolor}In [{\color{incolor}560}]:} \PY{n}{moore\PYZus{}lm} \PY{o}{=} \PY{n}{ols}\PY{p}{(}\PY{l+s+s1}{\PYZsq{}}\PY{l+s+s1}{CGPA \PYZti{} C(GENDER)}\PY{l+s+s1}{\PYZsq{}}\PY{p}{,}\PY{n}{data}\PY{o}{=}\PY{n}{Economie}\PY{p}{)}\PY{o}{.}\PY{n}{fit}\PY{p}{(}\PY{p}{)}
          \PY{n}{aov\PYZus{}table} \PY{o}{=} \PY{n}{sm}\PY{o}{.}\PY{n}{stats}\PY{o}{.}\PY{n}{anova\PYZus{}lm}\PY{p}{(}\PY{n}{moore\PYZus{}lm}\PY{p}{,} \PY{n}{typ}\PY{o}{=}\PY{l+m+mi}{1}\PY{p}{)}
          \PY{n}{aov\PYZus{}table}
\end{Verbatim}

            \begin{Verbatim}[commandchars=\\\{\}]
{\color{outcolor}Out[{\color{outcolor}560}]:}                df        sum\_sq    mean\_sq         F    PR(>F)
          C(GENDER)     1.0     11.697922  11.697922  0.297494  0.585536
          Residual   1547.0  60830.388767  39.321518       NaN       NaN
\end{Verbatim}
        
    Notre metrique nous montre que la moyenne de CGPA est la meme pour les
deux sexes ! jettons un coup d'oeil aux distribution avec des box-plot.

    \begin{Verbatim}[commandchars=\\\{\}]
{\color{incolor}In [{\color{incolor}562}]:} \PY{n}{plt}\PY{o}{.}\PY{n}{figure}\PY{p}{(}\PY{n}{figsize}\PY{o}{=}\PY{p}{(}\PY{l+m+mi}{12}\PY{p}{,}\PY{l+m+mi}{6}\PY{p}{)}\PY{p}{)}
          \PY{n}{ax}\PY{o}{=}\PY{n}{sns}\PY{o}{.}\PY{n}{boxplot}\PY{p}{(}\PY{n}{x}\PY{o}{=}\PY{l+s+s2}{\PYZdq{}}\PY{l+s+s2}{GENDER}\PY{l+s+s2}{\PYZdq{}}\PY{p}{,} \PY{n}{y}\PY{o}{=}\PY{l+s+s2}{\PYZdq{}}\PY{l+s+s2}{CGPA}\PY{l+s+s2}{\PYZdq{}}\PY{p}{,} \PY{n}{data}\PY{o}{=}\PY{n}{Economie}\PY{p}{)}
          \PY{n}{plt}\PY{o}{.}\PY{n}{savefig}\PY{p}{(}\PY{l+s+s1}{\PYZsq{}}\PY{l+s+s1}{GENDER\PYZhy{}CGPA\PYZhy{}Eco.png}\PY{l+s+s1}{\PYZsq{}}\PY{p}{,}\PY{n}{dpi}\PY{o}{=}\PY{l+m+mi}{100}\PY{p}{)}
\end{Verbatim}

    \begin{center}
    \adjustimage{max size={0.9\linewidth}{0.9\paperheight}}{output_231_0.png}
    \end{center}
    { \hspace*{\fill} \\}
    
    \begin{Verbatim}[commandchars=\\\{\}]
{\color{incolor}In [{\color{incolor}564}]:} \PY{n}{Economie}\PY{o}{.}\PY{n}{GENDER}\PY{o}{.}\PY{n}{value\PYZus{}counts}\PY{p}{(}\PY{p}{)}\PY{o}{/}\PY{n+nb}{len}\PY{p}{(}\PY{n}{Economie}\PY{p}{)}
\end{Verbatim}

            \begin{Verbatim}[commandchars=\\\{\}]
{\color{outcolor}Out[{\color{outcolor}564}]:} H    0.531311
          F    0.468689
          Name: GENDER, dtype: float64
\end{Verbatim}
        
    Nous remarquons que les 2 graphiques sont les memes , equilibre car la
facultée est constitué par 53\% des hommes et 46\% des femmes

    \begin{Verbatim}[commandchars=\\\{\}]
{\color{incolor}In [{\color{incolor}565}]:} \PY{n}{Economie}\PY{o}{.}\PY{n}{groupby}\PY{p}{(}\PY{l+s+s1}{\PYZsq{}}\PY{l+s+s1}{GENDER}\PY{l+s+s1}{\PYZsq{}}\PY{p}{)}\PY{o}{.}\PY{n}{mean}\PY{p}{(}\PY{p}{)}
\end{Verbatim}

            \begin{Verbatim}[commandchars=\\\{\}]
{\color{outcolor}Out[{\color{outcolor}565}]:}                  ID    DIPPERC        AGE       CGPA  NACADYEAR
          GENDER                                                         
          F       8676.130854  56.644475  23.617080  56.538625   1.917355
          H       8176.019441  56.998485  24.946537  56.364480   1.916160
\end{Verbatim}
        
    Mais on peut remarquer que les moyenees restent la meme

    \paragraph{B.4 Attribue Province}\label{b.4-attribue-province}

    \begin{Verbatim}[commandchars=\\\{\}]
{\color{incolor}In [{\color{incolor}467}]:} \PY{n}{Techno}\PY{o}{.}\PY{n}{columns}
\end{Verbatim}

            \begin{Verbatim}[commandchars=\\\{\}]
{\color{outcolor}Out[{\color{outcolor}467}]:} Index([u'ID', u'SCHOOLSTATUS', u'SCHOOL\_RIGHT', u'OPTION\_RIGHT', u'FAC',
                 u'SCHOOLPROVINCE', u'GENDER', u'DIPPERC', u'AGE', u'CGPA',
                 u'DistinctionRatio', u'EchecRatio', u'NACADYEAR',
                 u'Pass1stSessionRatio'],
                dtype='object')
\end{Verbatim}
        
    \begin{Verbatim}[commandchars=\\\{\}]
{\color{incolor}In [{\color{incolor}566}]:} \PY{n}{moore\PYZus{}lm} \PY{o}{=} \PY{n}{ols}\PY{p}{(}\PY{l+s+s1}{\PYZsq{}}\PY{l+s+s1}{CGPA \PYZti{} C(SCHOOLPROVINCE)}\PY{l+s+s1}{\PYZsq{}}\PY{p}{,}\PY{n}{data}\PY{o}{=}\PY{n}{Economie}\PY{p}{)}\PY{o}{.}\PY{n}{fit}\PY{p}{(}\PY{p}{)}
          \PY{n}{aov\PYZus{}table} \PY{o}{=} \PY{n}{sm}\PY{o}{.}\PY{n}{stats}\PY{o}{.}\PY{n}{anova\PYZus{}lm}\PY{p}{(}\PY{n}{moore\PYZus{}lm}\PY{p}{,} \PY{n}{typ}\PY{o}{=}\PY{l+m+mi}{1}\PY{p}{)}
          \PY{n}{aov\PYZus{}table}
\end{Verbatim}

            \begin{Verbatim}[commandchars=\\\{\}]
{\color{outcolor}Out[{\color{outcolor}566}]:}                        df        sum\_sq    mean\_sq         F    PR(>F)
          C(SCHOOLPROVINCE)    10.0    666.899640  66.689964  1.704509  0.074467
          Residual           1538.0  60175.187048  39.125609       NaN       NaN
\end{Verbatim}
        
    PR =0.078 est superieur à 0.05 nous pouvons conclure que la moyenne est
la meme sur toutes les facultés

    \begin{Verbatim}[commandchars=\\\{\}]
{\color{incolor}In [{\color{incolor}568}]:} \PY{n}{EchoGroup}\PY{o}{=}\PY{n}{Economie}\PY{o}{.}\PY{n}{groupby}\PY{p}{(}\PY{l+s+s1}{\PYZsq{}}\PY{l+s+s1}{SCHOOLPROVINCE}\PY{l+s+s1}{\PYZsq{}}\PY{p}{)}\PY{o}{.}\PY{n}{mean}\PY{p}{(}\PY{p}{)}\PY{o}{.}\PY{n}{sort}\PY{p}{(}\PY{n}{axis}\PY{o}{=}\PY{l+m+mi}{0}\PY{p}{,}\PY{n}{columns}\PY{o}{=}\PY{l+s+s1}{\PYZsq{}}\PY{l+s+s1}{CGPA}\PY{l+s+s1}{\PYZsq{}}\PY{p}{,}\PY{n}{ascending}\PY{o}{=}\PY{n+nb+bp}{False}\PY{p}{)}
\end{Verbatim}

    \begin{Verbatim}[commandchars=\\\{\}]
{\color{incolor}In [{\color{incolor}569}]:} \PY{n}{EchoGroup}\PY{o}{.}\PY{n}{reset\PYZus{}index}\PY{p}{(}\PY{n}{inplace}\PY{o}{=}\PY{n+nb+bp}{True}
                               \PY{p}{)}
\end{Verbatim}

    \begin{Verbatim}[commandchars=\\\{\}]
{\color{incolor}In [{\color{incolor}570}]:} \PY{n}{EchoGroup}
\end{Verbatim}

            \begin{Verbatim}[commandchars=\\\{\}]
{\color{outcolor}Out[{\color{outcolor}570}]:}     SCHOOLPROVINCE            ID    DIPPERC        AGE       CGPA  NACADYEAR
          0        BAS CONGO   6223.000000  51.000000  27.000000  61.150002   2.000000
          1          inconnu   5873.000000  60.062957  28.900000  60.363333   1.900000
          2        NORD-KIVU   8417.178955  56.786587  24.122888  56.596275   1.924731
          3         BANDUNDU   8907.000000  53.500000  24.500000  56.016667   2.500000
          4         SUD-KIVU   8222.881250  57.131250  25.237500  55.962396   1.937500
          5         KINSHASA   8691.433333  57.933333  25.000000  55.115555   1.633333
          6          KATANGA   7409.250000  58.000000  25.750000  55.050000   2.000000
          7         EQUATEUR   6751.500000  56.500000  24.000000  55.049999   1.500000
          8        ORIENTALE   9735.791667  56.203188  24.250000  54.071528   1.708333
          9   KASAI ORIENTAL  10078.200000  54.000000  27.600000  52.825000   2.000000
          10         MANIEMA   9134.111111  55.111111  27.222222  52.250000   1.777778
\end{Verbatim}
        
    Nous remarquons que la moyenne de GPA est superieur Pour la province Bas
congo , ensuite vienne la province du Nord kivu, le kasai bandudu et la
province du Sud Kivu

    \begin{Verbatim}[commandchars=\\\{\}]
{\color{incolor}In [{\color{incolor}571}]:} \PY{n}{plt}\PY{o}{.}\PY{n}{figure}\PY{p}{(}\PY{n}{figsize}\PY{o}{=}\PY{p}{(}\PY{l+m+mi}{15}\PY{p}{,}\PY{l+m+mi}{6}\PY{p}{)}\PY{p}{)}
          \PY{n}{ax}\PY{o}{=}\PY{n}{sns}\PY{o}{.}\PY{n}{boxplot}\PY{p}{(}\PY{n}{x}\PY{o}{=}\PY{l+s+s2}{\PYZdq{}}\PY{l+s+s2}{SCHOOLPROVINCE}\PY{l+s+s2}{\PYZdq{}}\PY{p}{,} \PY{n}{y}\PY{o}{=}\PY{l+s+s2}{\PYZdq{}}\PY{l+s+s2}{CGPA}\PY{l+s+s2}{\PYZdq{}}\PY{p}{,} \PY{n}{data}\PY{o}{=}\PY{n}{Economie}\PY{p}{,}\PY{n}{order}\PY{o}{=}\PY{n}{EchoGroup}\PY{o}{.}\PY{n}{SCHOOLPROVINCE}\PY{p}{)}
          \PY{n}{plt}\PY{o}{.}\PY{n}{savefig}\PY{p}{(}\PY{l+s+s1}{\PYZsq{}}\PY{l+s+s1}{GENDER\PYZhy{}SCHOOLPROVINCE\PYZhy{}Tech.png}\PY{l+s+s1}{\PYZsq{}}\PY{p}{,}\PY{n}{dpi}\PY{o}{=}\PY{l+m+mi}{100}\PY{p}{)}
\end{Verbatim}

    \begin{center}
    \adjustimage{max size={0.9\linewidth}{0.9\paperheight}}{output_244_0.png}
    \end{center}
    { \hspace*{\fill} \\}
    
    On remarque que la province du Kasie orientale est la pire province en
faculté d'economie car 75\% des étudiants provenant de cette province on
moins de 57\% au CGPA ,ceux qui distinguent le plus proviennent de la
province du nord kivu,etc..

    \begin{Verbatim}[commandchars=\\\{\}]
{\color{incolor}In [{\color{incolor}573}]:} \PY{n+nb}{len}\PY{p}{(}\PY{n}{datasetFin}\PY{o}{.}\PY{n}{loc}\PY{p}{[}\PY{n}{datasetFin}\PY{o}{.}\PY{n}{SCHOOLPROVINCE}\PY{o}{==}\PY{l+s+s1}{\PYZsq{}}\PY{l+s+s1}{inconnu}\PY{l+s+s1}{\PYZsq{}}\PY{p}{]}\PY{p}{)}
\end{Verbatim}

            \begin{Verbatim}[commandchars=\\\{\}]
{\color{outcolor}Out[{\color{outcolor}573}]:} 24
\end{Verbatim}
        
    \paragraph{B.5 Stattus de L'école}\label{b.5-stattus-de-luxe9cole}

    \begin{Verbatim}[commandchars=\\\{\}]
{\color{incolor}In [{\color{incolor}574}]:} \PY{n}{moore\PYZus{}lm} \PY{o}{=} \PY{n}{ols}\PY{p}{(}\PY{l+s+s1}{\PYZsq{}}\PY{l+s+s1}{CGPA \PYZti{} C(SCHOOLSTATUS)}\PY{l+s+s1}{\PYZsq{}}\PY{p}{,}\PY{n}{data}\PY{o}{=}\PY{n}{Economie}\PY{p}{)}\PY{o}{.}\PY{n}{fit}\PY{p}{(}\PY{p}{)}
          \PY{n}{aov\PYZus{}table} \PY{o}{=} \PY{n}{sm}\PY{o}{.}\PY{n}{stats}\PY{o}{.}\PY{n}{anova\PYZus{}lm}\PY{p}{(}\PY{n}{moore\PYZus{}lm}\PY{p}{,} \PY{n}{typ}\PY{o}{=}\PY{l+m+mi}{1}\PY{p}{)}
          \PY{n}{aov\PYZus{}table}
\end{Verbatim}

            \begin{Verbatim}[commandchars=\\\{\}]
{\color{outcolor}Out[{\color{outcolor}574}]:}                      df        sum\_sq     mean\_sq          F        PR(>F)
          C(SCHOOLSTATUS)     7.0   2850.635647  407.233664  10.821372  2.218935e-13
          Residual         1541.0  57991.451042   37.632350        NaN           NaN
\end{Verbatim}
        
    Il n'ya aucun lien entre les moyennes de CGPA pour les differents status
de l"ecole

    \begin{Verbatim}[commandchars=\\\{\}]
{\color{incolor}In [{\color{incolor}575}]:} \PY{n}{Economie}\PY{o}{.}\PY{n}{groupby}\PY{p}{(}\PY{l+s+s1}{\PYZsq{}}\PY{l+s+s1}{SCHOOLSTATUS}\PY{l+s+s1}{\PYZsq{}}\PY{p}{)}\PY{o}{.}\PY{n}{mean}\PY{p}{(}\PY{p}{)}\PY{o}{.}\PY{n}{sort}\PY{p}{(}\PY{n}{axis}\PY{o}{=}\PY{l+m+mi}{0}\PY{p}{,}\PY{n}{columns}\PY{o}{=}\PY{l+s+s1}{\PYZsq{}}\PY{l+s+s1}{CGPA}\PY{l+s+s1}{\PYZsq{}}\PY{p}{,}\PY{n}{ascending}\PY{o}{=}\PY{n+nb+bp}{False}\PY{p}{)}
\end{Verbatim}

            \begin{Verbatim}[commandchars=\\\{\}]
{\color{outcolor}Out[{\color{outcolor}575}]:}                         ID    DIPPERC        AGE       CGPA  NACADYEAR
          SCHOOLSTATUS                                                          
          catholique     8518.012526  58.694425  23.741127  57.987961   2.002088
          protestant     8241.739669  56.034614  24.117769  56.668974   1.940083
          publique       8284.400922  56.414178  24.631336  55.865630   1.875576
          inconnu        7997.815789  56.346247  26.388158  55.165241   1.940789
          kimbanguiste  10421.500000  54.000000  23.000000  55.133333   2.000000
          musulman       8901.647059  55.110384  24.529412  54.968627   1.647059
          privé          8973.063158  55.336192  24.278947  54.054649   1.705263
          autodidacte    8521.125000  54.125000  25.000000  50.989583   1.625000
\end{Verbatim}
        
    Nous remarquons que les éetudiant venant des écoles catholiques
reussiseent mieux en faculté d'econome , ensuite les autodidacte
,ensuite ceux des ecoles protestatntes

    Voyons cela de plus pret avec box-plot

    \begin{Verbatim}[commandchars=\\\{\}]
{\color{incolor}In [{\color{incolor}577}]:} \PY{n}{plt}\PY{o}{.}\PY{n}{figure}\PY{p}{(}\PY{n}{figsize}\PY{o}{=}\PY{p}{(}\PY{l+m+mi}{12}\PY{p}{,}\PY{l+m+mi}{6}\PY{p}{)}\PY{p}{)}
          \PY{n}{ax}\PY{o}{=}\PY{n}{sns}\PY{o}{.}\PY{n}{boxplot}\PY{p}{(}\PY{n}{x}\PY{o}{=}\PY{l+s+s2}{\PYZdq{}}\PY{l+s+s2}{SCHOOLSTATUS}\PY{l+s+s2}{\PYZdq{}}\PY{p}{,} \PY{n}{y}\PY{o}{=}\PY{l+s+s2}{\PYZdq{}}\PY{l+s+s2}{CGPA}\PY{l+s+s2}{\PYZdq{}}\PY{p}{,} \PY{n}{data}\PY{o}{=}\PY{n}{Economie}\PY{p}{)}
          \PY{n}{plt}\PY{o}{.}\PY{n}{savefig}\PY{p}{(}\PY{l+s+s1}{\PYZsq{}}\PY{l+s+s1}{\PYZhy{}SCHOOLSTATUS\PYZhy{}ECO.png}\PY{l+s+s1}{\PYZsq{}}\PY{p}{,}\PY{n}{dpi}\PY{o}{=}\PY{l+m+mi}{100}\PY{p}{)}
\end{Verbatim}

    \begin{center}
    \adjustimage{max size={0.9\linewidth}{0.9\paperheight}}{output_253_0.png}
    \end{center}
    { \hspace*{\fill} \\}
    
    Nous remarquons que les celles étudiants qui distingunt c'est ceux
provennant des écoles protestants et catholiques

    \paragraph{B.Les Ecoles de
provenances}\label{b.les-ecoles-de-provenances}

    \begin{Verbatim}[commandchars=\\\{\}]
{\color{incolor}In [{\color{incolor}578}]:} \PY{n}{moore\PYZus{}lm} \PY{o}{=} \PY{n}{ols}\PY{p}{(}\PY{l+s+s1}{\PYZsq{}}\PY{l+s+s1}{CGPA \PYZti{} C(SCHOOL\PYZus{}RIGHT)}\PY{l+s+s1}{\PYZsq{}}\PY{p}{,}\PY{n}{data}\PY{o}{=}\PY{n}{Economie}\PY{p}{)}\PY{o}{.}\PY{n}{fit}\PY{p}{(}\PY{p}{)}
          \PY{n}{aov\PYZus{}table} \PY{o}{=} \PY{n}{sm}\PY{o}{.}\PY{n}{stats}\PY{o}{.}\PY{n}{anova\PYZus{}lm}\PY{p}{(}\PY{n}{moore\PYZus{}lm}\PY{p}{,} \PY{n}{typ}\PY{o}{=}\PY{l+m+mi}{1}\PY{p}{)}
          \PY{n}{aov\PYZus{}table}
\end{Verbatim}

            \begin{Verbatim}[commandchars=\\\{\}]
{\color{outcolor}Out[{\color{outcolor}578}]:}                      df        sum\_sq    mean\_sq         F        PR(>F)
          C(SCHOOL\_RIGHT)   260.0  20026.184839  77.023788  2.430588  1.884616e-24
          Residual         1288.0  40815.901850  31.689365       NaN           NaN
\end{Verbatim}
        
    Aucunne correlation n'existe entre l"ecole et le CGPA au vu de la valeur
de PR'

    \begin{Verbatim}[commandchars=\\\{\}]
{\color{incolor}In [{\color{incolor}579}]:} \PY{n}{SchoolGroup}\PY{o}{=}\PY{n}{Economie}\PY{o}{.}\PY{n}{groupby}\PY{p}{(}\PY{l+s+s1}{\PYZsq{}}\PY{l+s+s1}{SCHOOL\PYZus{}RIGHT}\PY{l+s+s1}{\PYZsq{}}\PY{p}{)}\PY{o}{.}\PY{n}{mean}\PY{p}{(}\PY{p}{)}\PY{o}{.}\PY{n}{sort}\PY{p}{(}\PY{n}{axis}\PY{o}{=}\PY{l+m+mi}{0}\PY{p}{,}\PY{n}{columns}\PY{o}{=}\PY{l+s+s1}{\PYZsq{}}\PY{l+s+s1}{CGPA}\PY{l+s+s1}{\PYZsq{}}\PY{p}{,}\PY{n}{ascending}\PY{o}{=}\PY{n+nb+bp}{False}\PY{p}{)}
\end{Verbatim}

    \begin{Verbatim}[commandchars=\\\{\}]
{\color{incolor}In [{\color{incolor}580}]:} \PY{n}{SchoolGroup}\PY{o}{.}\PY{n}{reset\PYZus{}index}\PY{p}{(}\PY{n}{inplace}\PY{o}{=}\PY{n+nb+bp}{True}\PY{p}{)}
\end{Verbatim}

    \begin{Verbatim}[commandchars=\\\{\}]
{\color{incolor}In [{\color{incolor}581}]:} \PY{n}{SchoolGroup}
\end{Verbatim}

            \begin{Verbatim}[commandchars=\\\{\}]
{\color{outcolor}Out[{\color{outcolor}581}]:}                     SCHOOL\_RIGHT        ID    DIPPERC        AGE       CGPA  \textbackslash{}
          0                        shaloom   7221.00  53.000000  23.000000  66.624999   
          1                 it kasabinyole   6948.00  50.000000  27.000000  65.300003   
          2                           mulo   8005.50  59.500000  24.500000  64.799999   
          3                         avenir   9374.00  58.000000  22.500000  64.100000   
          4                         loyola   7223.00  73.000000  24.000000  63.825000   
          5               action kusaidiya   4346.00  55.000000  32.000000  63.700001   
          6                   LYCEE MWANDU   3251.00  54.000000  30.000000  63.600001   
          7                       alfajiri   6521.50  60.250000  26.000000  63.566667   
          8                        kirumba   4554.00  51.000000  29.000000  63.200001   
          9                        ujasiri   5198.00  53.500000  27.000000  63.100001   
          10                      uaminifu   6380.00  63.000000  29.000000  62.349998   
          11                      kirikiri   6573.00  54.000000  25.000000  62.049999   
          12                    mama mwilu  10458.00  61.000000  26.000000  62.049999   
          13                         nduba  10320.00  69.000000  26.000000  61.900000   
          14              notre dame olame   7270.00  58.666667  24.666667  61.877777   
          15                    de l'unite   9098.00  52.000000  23.000000  61.833333   
          16        petit séminaire mugeri  12211.00  65.000000  20.000000  61.799999   
          17                        masisi   7946.00  57.000000  24.500000  61.699999   
          18                         amani   6828.75  59.750000  25.291667  61.666319   
          19                    itm kizito   7701.00  64.000000  39.000000  61.650000   
          20                      l mapema   4297.50  59.000000  26.500000  61.600000   
          21   groupe scol. ndbc de byumba   2958.00  56.876522  52.000000  61.599998   
          22               essi nyamirambo   7311.00  56.876522  26.000000  61.533333   
          23                      de bunia   9942.00  52.500000  24.500000  61.300001   
          24                 saint raphael   9786.00  68.000000  25.000000  61.300001   
          25                        mayele   7146.00  63.000000  25.000000  61.266666   
          26                   pain de vie   8397.50  62.000000  25.000000  61.233334   
          27                     veronique   7084.00  69.000000  24.000000  61.233334   
          28                gs de la salle   6223.00  51.000000  27.000000  61.150002   
          29                        katana   4562.00  53.000000  32.000000  61.099998   
          ..                           {\ldots}       {\ldots}        {\ldots}        {\ldots}        {\ldots}   
          231                      buhimba  11789.00  51.000000  22.000000  47.000000   
          232                       mapema   4365.00  57.000000  27.000000  47.000000   
          233                      namango  11549.00  56.000000  21.000000  46.500000   
          234               sainte famille  10038.00  57.000000  24.000000  46.099998   
          235                        macha  10522.00  54.000000  24.000000  46.000000   
          236                        kausa  10130.00  55.000000  22.000000  46.000000   
          237                   st etienne  10061.50  59.000000  22.000000  46.000000   
          238                       kiraku   9459.00  55.500000  23.000000  45.600000   
          239                      Mugunga   9309.50  53.500000  23.000000  45.500000   
          240                         luka   9197.00  56.000000  23.000000  45.000000   
          241          technique mapendano   9083.00  51.000000  22.000000  45.000000   
          242                       kiyabo   9481.50  60.500000  25.000000  44.350000   
          243            chemchem ya uzima  10755.00  61.000000  22.000000  44.200001   
          244    notre dame de la jeunesse   9788.00  52.000000  32.000000  44.000000   
          245                     kashenda   8621.00  51.000000  29.000000  44.000000   
          246                      babwise   9200.00  51.000000  24.000000  43.750000   
          247                       tshoka  12217.00  52.000000  26.000000  43.000000   
          248                       bagira   6297.00  55.000000  29.000000  43.000000   
          249                     dibasana  12184.00  52.000000  23.000000  43.000000   
          250                        alpha  11470.00  54.000000  21.000000  43.000000   
          251                      virunga   9076.00  57.000000  24.000000  43.000000   
          252      congolaise de bujumbura   9443.00  50.000000  24.000000  42.000000   
          253      groupe scolaire gilgali  11972.00  52.500000  20.500000  41.900000   
          254                      mandayi  10843.00  55.000000  22.000000  41.000000   
          255                       mandai  12042.00  52.000000  23.000000  41.000000   
          256     alliance ouest africaine   8739.00  61.000000  28.000000  41.000000   
          257                      burhiba  10578.00  50.000000  27.000000  40.000000   
          258            mont des oliviers  12063.00  50.000000  20.000000  40.000000   
          259               révérend samba   9153.00  55.000000  25.000000  40.000000   
          260                     ibanga 2  10204.00  57.000000  31.000000  36.099998   
          
               NACADYEAR  
          0     4.000000  
          1     1.000000  
          2     2.000000  
          3     3.000000  
          4     4.000000  
          5     1.000000  
          6     3.000000  
          7     2.250000  
          8     1.000000  
          9     1.500000  
          10    2.000000  
          11    2.000000  
          12    2.000000  
          13    2.000000  
          14    2.333333  
          15    3.000000  
          16    1.000000  
          17    1.000000  
          18    2.041667  
          19    4.000000  
          20    1.000000  
          21    1.000000  
          22    3.000000  
          23    1.000000  
          24    3.000000  
          25    3.000000  
          26    3.000000  
          27    3.000000  
          28    2.000000  
          29    1.000000  
          ..         {\ldots}  
          231   1.000000  
          232   1.000000  
          233   1.000000  
          234   1.000000  
          235   1.000000  
          236   1.000000  
          237   1.000000  
          238   1.000000  
          239   1.000000  
          240   1.000000  
          241   1.000000  
          242   1.000000  
          243   1.000000  
          244   1.000000  
          245   1.000000  
          246   2.000000  
          247   1.000000  
          248   1.000000  
          249   1.000000  
          250   1.000000  
          251   1.000000  
          252   1.000000  
          253   1.000000  
          254   1.000000  
          255   1.000000  
          256   1.000000  
          257   1.000000  
          258   1.000000  
          259   1.000000  
          260   1.000000  
          
          [261 rows x 6 columns]
\end{Verbatim}
        
    Nous allons Voir de plus pret pour les 10 ecoles les plus representé, le
5 éecoles avec un pourcentage elevé et 5 dernier

    \begin{Verbatim}[commandchars=\\\{\}]
{\color{incolor}In [{\color{incolor}586}]:} \PY{n}{plt}\PY{o}{.}\PY{n}{figure}\PY{p}{(}\PY{n}{figsize}\PY{o}{=}\PY{p}{(}\PY{l+m+mi}{30}\PY{p}{,}\PY{l+m+mi}{10}\PY{p}{)}\PY{p}{)}
          \PY{n}{ax}\PY{o}{=}\PY{n}{sns}\PY{o}{.}\PY{n}{boxplot}\PY{p}{(}\PY{n}{x}\PY{o}{=}\PY{l+s+s2}{\PYZdq{}}\PY{l+s+s2}{SCHOOL\PYZus{}RIGHT}\PY{l+s+s2}{\PYZdq{}}\PY{p}{,} \PY{n}{y}\PY{o}{=}\PY{l+s+s2}{\PYZdq{}}\PY{l+s+s2}{CGPA}\PY{l+s+s2}{\PYZdq{}}\PY{p}{,}  \PY{n}{data}\PY{o}{=}\PY{n}{Economie}\PY{o}{.}\PY{n}{loc}\PY{p}{[}\PY{n}{operator}\PY{o}{.}\PY{n}{or\PYZus{}}\PY{p}{(}\PY{n}{Economie}\PY{o}{.}\PY{n}{SCHOOL\PYZus{}RIGHT}\PY{o}{.}\PY{n}{isin}\PY{p}{(}\PY{n}{SchoolGroup}\PY{o}{.}\PY{n}{loc}\PY{p}{[} \PY{n+nb}{range}\PY{p}{(}\PY{l+m+mi}{0}\PY{p}{,}\PY{l+m+mi}{11}\PY{p}{)}\PY{p}{]}\PY{o}{.}\PY{n}{SCHOOL\PYZus{}RIGHT}\PY{p}{)} 
                       \PY{p}{,} \PY{n}{Economie}\PY{o}{.}\PY{n}{SCHOOL\PYZus{}RIGHT}\PY{o}{.}\PY{n}{isin}\PY{p}{(}\PY{n}{Economie}\PY{o}{.}\PY{n}{SCHOOL\PYZus{}RIGHT}\PY{o}{.}\PY{n}{value\PYZus{}counts}\PY{p}{(}\PY{p}{)}\PY{p}{[}\PY{p}{:}\PY{l+m+mi}{10}\PY{p}{]}\PY{o}{.}\PY{n}{index} \PY{p}{)}\PY{p}{)}\PY{p}{]}\PY{p}{)}
          \PY{n}{plt}\PY{o}{.}\PY{n}{savefig}\PY{p}{(}\PY{l+s+s1}{\PYZsq{}}\PY{l+s+s1}{SCHOOL\PYZus{}RIGHT\PYZhy{}CGPAECon.png}\PY{l+s+s1}{\PYZsq{}}\PY{p}{,}\PY{n}{dpi}\PY{o}{=}\PY{l+m+mi}{100}\PY{p}{)}
\end{Verbatim}

    \begin{center}
    \adjustimage{max size={0.9\linewidth}{0.9\paperheight}}{output_262_0.png}
    \end{center}
    { \hspace*{\fill} \\}
    
    On remarques sur la figure que les grandes écoles connus on une bonne
moyenne et un bon maximun de CGPA comme le lycée Amani, l'ITIG , l'ITFM!

    \paragraph{B.7 OPTION DU DIPLOME}\label{b.7-option-du-diplome}

    \begin{Verbatim}[commandchars=\\\{\}]
{\color{incolor}In [{\color{incolor}587}]:} \PY{n+nb}{len}\PY{p}{(}\PY{n}{Economie}\PY{o}{.}\PY{n}{OPTION\PYZus{}RIGHT}\PY{o}{.}\PY{n}{value\PYZus{}counts}\PY{p}{(}\PY{p}{)}\PY{p}{)}
\end{Verbatim}

            \begin{Verbatim}[commandchars=\\\{\}]
{\color{outcolor}Out[{\color{outcolor}587}]:} 21
\end{Verbatim}
        
    Nous pouvons constater que nous avons 21 options differents Voyons
comment evoluer la moyenne

    \begin{Verbatim}[commandchars=\\\{\}]
{\color{incolor}In [{\color{incolor}588}]:} \PY{n}{moore\PYZus{}lm} \PY{o}{=} \PY{n}{ols}\PY{p}{(}\PY{l+s+s1}{\PYZsq{}}\PY{l+s+s1}{CGPA \PYZti{} C(OPTION\PYZus{}RIGHT)}\PY{l+s+s1}{\PYZsq{}}\PY{p}{,}\PY{n}{data}\PY{o}{=}\PY{n}{Economie}\PY{p}{)}\PY{o}{.}\PY{n}{fit}\PY{p}{(}\PY{p}{)}
          \PY{n}{aov\PYZus{}table} \PY{o}{=} \PY{n}{sm}\PY{o}{.}\PY{n}{stats}\PY{o}{.}\PY{n}{anova\PYZus{}lm}\PY{p}{(}\PY{n}{moore\PYZus{}lm}\PY{p}{,} \PY{n}{typ}\PY{o}{=}\PY{l+m+mi}{1}\PY{p}{)}
          \PY{n}{aov\PYZus{}table}
\end{Verbatim}

            \begin{Verbatim}[commandchars=\\\{\}]
{\color{outcolor}Out[{\color{outcolor}588}]:}                      df        sum\_sq     mean\_sq         F        PR(>F)
          C(OPTION\_RIGHT)    20.0   3739.407704  186.970385  5.003106  3.534757e-12
          Residual         1528.0  57102.678985   37.370863       NaN           NaN
\end{Verbatim}
        
    La valeur est inferieur à 0.05 donc il n'yas pas de relation entre les
valeurs , certaines écoles on une bonne moyenne du GPA que d'autres donc
elle depend de l'ecole de provenance

    \begin{Verbatim}[commandchars=\\\{\}]
{\color{incolor}In [{\color{incolor}589}]:} \PY{n}{GroupOption}\PY{o}{=}\PY{n}{Economie}\PY{o}{.}\PY{n}{groupby}\PY{p}{(}\PY{l+s+s1}{\PYZsq{}}\PY{l+s+s1}{OPTION\PYZus{}RIGHT}\PY{l+s+s1}{\PYZsq{}}\PY{p}{)}\PY{o}{.}\PY{n}{mean}\PY{p}{(}\PY{p}{)}\PY{o}{.}\PY{n}{sort}\PY{p}{(}\PY{n}{axis}\PY{o}{=}\PY{l+m+mi}{0}\PY{p}{,}\PY{n}{columns}\PY{o}{=}\PY{l+s+s1}{\PYZsq{}}\PY{l+s+s1}{CGPA}\PY{l+s+s1}{\PYZsq{}}\PY{p}{,}\PY{n}{ascending}\PY{o}{=}\PY{n+nb+bp}{False}\PY{p}{)}\PY{o}{.}\PY{n}{reset\PYZus{}index}\PY{p}{(}\PY{p}{)}
\end{Verbatim}

    \begin{Verbatim}[commandchars=\\\{\}]
{\color{incolor}In [{\color{incolor}593}]:} \PY{n}{GroupOption}
\end{Verbatim}

            \begin{Verbatim}[commandchars=\\\{\}]
{\color{outcolor}Out[{\color{outcolor}593}]:}                 OPTION\_RIGHT           ID    DIPPERC        AGE       CGPA  \textbackslash{}
          0                   agrecole  9435.000000  68.000000  26.000000  63.450001   
          1                    inconnu  3389.333333  54.584348  29.333333  62.900000   
          2                       elec  7020.500000  53.000000  26.000000  62.533335   
          3                   economie   356.000000  60.000000  39.000000  59.900002   
          4                 bio-chimie  8033.025641  57.492798  24.352564  58.498932   
          5          hotesse d'acceuil  4566.000000  55.000000  28.000000  58.400002   
          6              math-physique  7214.666667  54.544679  26.619048  58.193056   
          7                       nutr  6793.333333  54.666667  24.000000  57.605555   
          8              coupe couture  8284.750000  62.250000  23.500000  57.520833   
          9                vétérinaire  8120.000000  50.000000  28.000000  57.400000   
          10       relations publiques  7765.000000  61.000000  25.000000  57.366666   
          11       commmerciale et adm  8432.245856  56.257776  24.240331  57.206733   
          12               latin philo  8151.000000  58.766234  24.311688  56.543506   
          13  commerciale informatique  9648.904762  55.825397  22.650794  56.361773   
          14               elec indust  7273.000000  53.000000  25.500000  55.950001   
          15                  batiment  3918.000000  51.000000  26.000000  55.500000   
          16                   sociale  8902.125000  57.394055  23.630435  54.583197   
          17                 pedagogie  8380.955556  57.770370  24.870370  54.169753   
          18               secretariat  8315.000000  55.000000  24.200000  53.034999   
          19                  mec gene  6721.500000  55.000000  27.500000  52.766666   
          20              construction  9477.333333  53.333333  23.666667  49.833334   
          
              NACADYEAR  
          0    2.000000  
          1    1.000000  
          2    2.000000  
          3    1.000000  
          4    2.019231  
          5    1.000000  
          6    2.214286  
          7    1.666667  
          8    2.250000  
          9    3.000000  
          10   3.000000  
          11   1.962707  
          12   1.844156  
          13   1.714286  
          14   2.000000  
          15   1.000000  
          16   1.902174  
          17   1.766667  
          18   2.200000  
          19   2.500000  
          20   2.000000  
\end{Verbatim}
        
    Voila comme on pourait s'y attendre les options techniques et
Scientifiques viennent en tete de liste et en bas de l'echel vient les
ption littereraires et commerciales

    \begin{Verbatim}[commandchars=\\\{\}]
{\color{incolor}In [{\color{incolor}592}]:} \PY{n}{plt}\PY{o}{.}\PY{n}{figure}\PY{p}{(}\PY{n}{figsize}\PY{o}{=}\PY{p}{(}\PY{l+m+mi}{35}\PY{p}{,}\PY{l+m+mi}{10}\PY{p}{)}\PY{p}{)}
          \PY{n}{ax}\PY{o}{=}\PY{n}{sns}\PY{o}{.}\PY{n}{boxplot}\PY{p}{(}\PY{n}{x}\PY{o}{=}\PY{l+s+s2}{\PYZdq{}}\PY{l+s+s2}{OPTION\PYZus{}RIGHT}\PY{l+s+s2}{\PYZdq{}}\PY{p}{,} \PY{n}{y}\PY{o}{=}\PY{l+s+s2}{\PYZdq{}}\PY{l+s+s2}{CGPA}\PY{l+s+s2}{\PYZdq{}}\PY{p}{,} \PY{n}{data}\PY{o}{=}\PY{n}{Economie}\PY{p}{,}\PY{n}{order}\PY{o}{=}\PY{n}{GroupOption}\PY{o}{.}\PY{n}{OPTION\PYZus{}RIGHT}\PY{p}{)}
          \PY{n}{plt}\PY{o}{.}\PY{n}{savefig}\PY{p}{(}\PY{l+s+s1}{\PYZsq{}}\PY{l+s+s1}{OPTION\PYZus{}RIGHT\PYZhy{}CGPAEco.png}\PY{l+s+s1}{\PYZsq{}}\PY{p}{,}\PY{n}{dpi}\PY{o}{=}\PY{l+m+mi}{100}\PY{p}{)}
\end{Verbatim}

    \begin{center}
    \adjustimage{max size={0.9\linewidth}{0.9\paperheight}}{output_272_0.png}
    \end{center}
    { \hspace*{\fill} \\}
    
    cela peut se remarquer aisement sur la figures que les sections
techniques sont ceux dont le étudiants reussisnet le mieux

    On constate que le pourcentage de CGPA varie differament pour chaque
option du diplome , les options peda, sociale ,vienne en dernier en ce
qui concerne le cGPA mean

    Nous remarquons que les celles étudiants qui distingunt c'est ceux
provennant des écoles protestants et catholiques, en queu de la liste se
trouve les autodidacte

    \subparagraph{Faculté de Santé et
Devellopement}\label{facultuxe9-de-santuxe9-et-devellopement}

    \begin{Verbatim}[commandchars=\\\{\}]
{\color{incolor}In [{\color{incolor}627}]:} \PY{n}{Sante}\PY{o}{=}\PY{n}{datasetFin}\PY{o}{.}\PY{n}{loc}\PY{p}{[}\PY{n}{datasetFin}\PY{o}{.}\PY{n}{FAC}\PY{o}{==}\PY{l+s+s1}{\PYZsq{}}\PY{l+s+s1}{FSDC}\PY{l+s+s1}{\PYZsq{}}\PY{p}{]}
\end{Verbatim}

    \begin{Verbatim}[commandchars=\\\{\}]
{\color{incolor}In [{\color{incolor}596}]:} \PY{n}{Economie}\PY{o}{.}\PY{n}{shape}
\end{Verbatim}

            \begin{Verbatim}[commandchars=\\\{\}]
{\color{outcolor}Out[{\color{outcolor}596}]:} (1549, 14)
\end{Verbatim}
        
    \begin{Verbatim}[commandchars=\\\{\}]
{\color{incolor}In [{\color{incolor}551}]:} \PY{n}{Techno}\PY{o}{.}\PY{n}{shape}
\end{Verbatim}

            \begin{Verbatim}[commandchars=\\\{\}]
{\color{outcolor}Out[{\color{outcolor}551}]:} (903, 14)
\end{Verbatim}
        
    \begin{Verbatim}[commandchars=\\\{\}]
{\color{incolor}In [{\color{incolor}628}]:} \PY{n}{Sante}\PY{o}{.}\PY{n}{shape}
\end{Verbatim}

            \begin{Verbatim}[commandchars=\\\{\}]
{\color{outcolor}Out[{\color{outcolor}628}]:} (758, 14)
\end{Verbatim}
        
    https://www.wellbeingatschool.org.nz/information-sheet/understanding-and-interpreting-box-plots

    Nous voici enfin au sein de la faculté d'économie ! Commencons par voir
la répartition des probabilitées de notre variable CGPA

    A.0 Distribution du CGPA

    \begin{Verbatim}[commandchars=\\\{\}]
{\color{incolor}In [{\color{incolor}629}]:} \PY{n}{plt}\PY{o}{.}\PY{n}{figure}\PY{p}{(}\PY{p}{)}
          \PY{n}{ax} \PY{o}{=} \PY{n}{sns}\PY{o}{.}\PY{n}{boxplot}\PY{p}{(}\PY{n}{x}\PY{o}{=}\PY{n}{Sante}\PY{p}{[}\PY{l+s+s1}{\PYZsq{}}\PY{l+s+s1}{CGPA}\PY{l+s+s1}{\PYZsq{}}\PY{p}{]}\PY{p}{)}
          \PY{n}{plt}\PY{o}{.}\PY{n}{savefig}\PY{p}{(}\PY{l+s+s1}{\PYZsq{}}\PY{l+s+s1}{CGPA\PYZhy{}ECO\PYZhy{}Sante.png}\PY{l+s+s1}{\PYZsq{}}\PY{p}{,}\PY{n}{dpi}\PY{o}{=}\PY{l+m+mi}{100}\PY{p}{)}
\end{Verbatim}

    \begin{center}
    \adjustimage{max size={0.9\linewidth}{0.9\paperheight}}{output_284_0.png}
    \end{center}
    { \hspace*{\fill} \\}
    
    \begin{Verbatim}[commandchars=\\\{\}]
{\color{incolor}In [{\color{incolor}630}]:} \PY{n}{plt}\PY{o}{.}\PY{n}{figure}\PY{p}{(}\PY{p}{)}
          \PY{n}{sns}\PY{o}{.}\PY{n}{distplot}\PY{p}{(}\PY{n}{Sante}\PY{p}{[}\PY{l+s+s1}{\PYZsq{}}\PY{l+s+s1}{CGPA}\PY{l+s+s1}{\PYZsq{}}\PY{p}{]}\PY{p}{,}\PY{n}{bins}\PY{o}{=}\PY{l+m+mi}{20}\PY{p}{,}\PY{n}{axlabel}\PY{o}{=}\PY{l+s+s1}{\PYZsq{}}\PY{l+s+s1}{CGPA}\PY{l+s+s1}{\PYZsq{}}\PY{p}{,}\PY{n}{kde}\PY{o}{=}\PY{l+m+mi}{1}\PY{p}{,}\PY{n}{norm\PYZus{}hist}\PY{o}{=}\PY{l+m+mi}{0}\PY{p}{)}
          \PY{n}{plt}\PY{o}{.}\PY{n}{savefig}\PY{p}{(}\PY{l+s+s1}{\PYZsq{}}\PY{l+s+s1}{CGPADistDroit.png}\PY{l+s+s1}{\PYZsq{}}\PY{p}{,}\PY{n}{dpi}\PY{o}{=}\PY{l+m+mi}{100}\PY{p}{)}
\end{Verbatim}

    \begin{center}
    \adjustimage{max size={0.9\linewidth}{0.9\paperheight}}{output_285_0.png}
    \end{center}
    { \hspace*{\fill} \\}
    
    Il suit une distribution presque normale d'écart type : 6.5

    \begin{Verbatim}[commandchars=\\\{\}]
{\color{incolor}In [{\color{incolor}631}]:} \PY{n}{np}\PY{o}{.}\PY{n}{std}\PY{p}{(}\PY{n}{Sante}\PY{o}{.}\PY{n}{CGPA}\PY{p}{)}
\end{Verbatim}

            \begin{Verbatim}[commandchars=\\\{\}]
{\color{outcolor}Out[{\color{outcolor}631}]:} 6.526651956504482
\end{Verbatim}
        
    \begin{Verbatim}[commandchars=\\\{\}]
{\color{incolor}In [{\color{incolor}602}]:} \PY{n}{np}\PY{o}{.}\PY{n}{mean}\PY{p}{(}\PY{n}{Droit}\PY{o}{.}\PY{n}{CGPA}\PY{p}{)}
\end{Verbatim}

            \begin{Verbatim}[commandchars=\\\{\}]
{\color{outcolor}Out[{\color{outcolor}602}]:} 58.022023761201453
\end{Verbatim}
        
    Commencons par les attribues numeriques et analysons la correlation
chercher comme sur le point précedant la correlation avec le CGPA

    \begin{Verbatim}[commandchars=\\\{\}]
{\color{incolor}In [{\color{incolor}632}]:} \PY{n}{Sante}\PY{o}{.}\PY{n}{corr}\PY{p}{(}\PY{p}{)}
\end{Verbatim}

            \begin{Verbatim}[commandchars=\\\{\}]
{\color{outcolor}Out[{\color{outcolor}632}]:}                  ID   DIPPERC       AGE      CGPA  NACADYEAR
          ID         1.000000  0.054474 -0.546904  0.067762  -0.009021
          DIPPERC    0.054474  1.000000 -0.094969  0.196664   0.162441
          AGE       -0.546904 -0.094969  1.000000 -0.062776  -0.111260
          CGPA       0.067762  0.196664 -0.062776  1.000000   0.329851
          NACADYEAR -0.009021  0.162441 -0.111260  0.329851   1.000000
\end{Verbatim}
        
    Nous pouvons aisement qu'il n'yas aucune correlation entre le le CGPA et
le pourcentage à l'exetat ni meme l'age des étudiants

    \paragraph{C.2 GENDER}\label{c.2-gender}

    \begin{Verbatim}[commandchars=\\\{\}]
{\color{incolor}In [{\color{incolor}633}]:} \PY{n}{moore\PYZus{}lm} \PY{o}{=} \PY{n}{ols}\PY{p}{(}\PY{l+s+s1}{\PYZsq{}}\PY{l+s+s1}{CGPA \PYZti{} C(GENDER)}\PY{l+s+s1}{\PYZsq{}}\PY{p}{,}\PY{n}{data}\PY{o}{=}\PY{n}{Sante}\PY{p}{)}\PY{o}{.}\PY{n}{fit}\PY{p}{(}\PY{p}{)}
          \PY{n}{aov\PYZus{}table} \PY{o}{=} \PY{n}{sm}\PY{o}{.}\PY{n}{stats}\PY{o}{.}\PY{n}{anova\PYZus{}lm}\PY{p}{(}\PY{n}{moore\PYZus{}lm}\PY{p}{,} \PY{n}{typ}\PY{o}{=}\PY{l+m+mi}{1}\PY{p}{)}
          \PY{n}{aov\PYZus{}table}
\end{Verbatim}

            \begin{Verbatim}[commandchars=\\\{\}]
{\color{outcolor}Out[{\color{outcolor}633}]:}               df        sum\_sq    mean\_sq         F    PR(>F)
          C(GENDER)    1.0     77.581150  77.581150  1.820844  0.177617
          Residual   756.0  32211.085657  42.607256       NaN       NaN
\end{Verbatim}
        
    Notre metrique nous montre que la moyenne de CGPA est la meme pour les
deux sexes ! jettons un coup d'oeil aux distribution avec des box-plot.

    \begin{Verbatim}[commandchars=\\\{\}]
{\color{incolor}In [{\color{incolor}635}]:} \PY{n}{plt}\PY{o}{.}\PY{n}{figure}\PY{p}{(}\PY{n}{figsize}\PY{o}{=}\PY{p}{(}\PY{l+m+mi}{12}\PY{p}{,}\PY{l+m+mi}{6}\PY{p}{)}\PY{p}{)}
          \PY{n}{ax}\PY{o}{=}\PY{n}{sns}\PY{o}{.}\PY{n}{boxplot}\PY{p}{(}\PY{n}{x}\PY{o}{=}\PY{l+s+s2}{\PYZdq{}}\PY{l+s+s2}{GENDER}\PY{l+s+s2}{\PYZdq{}}\PY{p}{,} \PY{n}{y}\PY{o}{=}\PY{l+s+s2}{\PYZdq{}}\PY{l+s+s2}{CGPA}\PY{l+s+s2}{\PYZdq{}}\PY{p}{,} \PY{n}{data}\PY{o}{=}\PY{n}{Sante}\PY{p}{)}
          \PY{n}{plt}\PY{o}{.}\PY{n}{savefig}\PY{p}{(}\PY{l+s+s1}{\PYZsq{}}\PY{l+s+s1}{GENDER\PYZhy{}CGPA\PYZhy{}Sante.png}\PY{l+s+s1}{\PYZsq{}}\PY{p}{,}\PY{n}{dpi}\PY{o}{=}\PY{l+m+mi}{100}\PY{p}{)}
\end{Verbatim}

    \begin{center}
    \adjustimage{max size={0.9\linewidth}{0.9\paperheight}}{output_295_0.png}
    \end{center}
    { \hspace*{\fill} \\}
    
    \begin{Verbatim}[commandchars=\\\{\}]
{\color{incolor}In [{\color{incolor}636}]:} \PY{n}{Sante}\PY{o}{.}\PY{n}{GENDER}\PY{o}{.}\PY{n}{value\PYZus{}counts}\PY{p}{(}\PY{p}{)}\PY{o}{/}\PY{n+nb}{len}\PY{p}{(}\PY{n}{Sante}\PY{p}{)}
\end{Verbatim}

            \begin{Verbatim}[commandchars=\\\{\}]
{\color{outcolor}Out[{\color{outcolor}636}]:} F    0.618734
          H    0.381266
          Name: GENDER, dtype: float64
\end{Verbatim}
        
    Nous remarquons que les 2 graphiques sont les memes , equilibre car la
facultée est constitué par 60\% des femmes et 40\% des hommes , c'est la
faculté la plus feminine

    \begin{Verbatim}[commandchars=\\\{\}]
{\color{incolor}In [{\color{incolor}637}]:} \PY{n}{Sante}\PY{o}{.}\PY{n}{groupby}\PY{p}{(}\PY{l+s+s1}{\PYZsq{}}\PY{l+s+s1}{GENDER}\PY{l+s+s1}{\PYZsq{}}\PY{p}{)}\PY{o}{.}\PY{n}{mean}\PY{p}{(}\PY{p}{)}
\end{Verbatim}

            \begin{Verbatim}[commandchars=\\\{\}]
{\color{outcolor}Out[{\color{outcolor}637}]:}                  ID    DIPPERC        AGE       CGPA  NACADYEAR
          GENDER                                                         
          F       8461.498934  55.481163  25.366738  59.168834   1.925373
          H       7762.536332  55.082618  26.961938  58.510150   2.055363
\end{Verbatim}
        
    Mais on peut remarquer que les moyenees restent la meme

    \paragraph{B.4 Attribue Province}\label{b.4-attribue-province}

    \begin{Verbatim}[commandchars=\\\{\}]
{\color{incolor}In [{\color{incolor}638}]:} \PY{n}{moore\PYZus{}lm} \PY{o}{=} \PY{n}{ols}\PY{p}{(}\PY{l+s+s1}{\PYZsq{}}\PY{l+s+s1}{CGPA \PYZti{} C(SCHOOLPROVINCE)}\PY{l+s+s1}{\PYZsq{}}\PY{p}{,}\PY{n}{data}\PY{o}{=}\PY{n}{Sante}\PY{p}{)}\PY{o}{.}\PY{n}{fit}\PY{p}{(}\PY{p}{)}
          \PY{n}{aov\PYZus{}table} \PY{o}{=} \PY{n}{sm}\PY{o}{.}\PY{n}{stats}\PY{o}{.}\PY{n}{anova\PYZus{}lm}\PY{p}{(}\PY{n}{moore\PYZus{}lm}\PY{p}{,} \PY{n}{typ}\PY{o}{=}\PY{l+m+mi}{1}\PY{p}{)}
          \PY{n}{aov\PYZus{}table}
\end{Verbatim}

            \begin{Verbatim}[commandchars=\\\{\}]
{\color{outcolor}Out[{\color{outcolor}638}]:}                       df        sum\_sq    mean\_sq         F    PR(>F)
          C(SCHOOLPROVINCE)   11.0    240.767767  21.887979  0.509501  0.897649
          Residual           746.0  32047.899040  42.959650       NaN       NaN
\end{Verbatim}
        
    PR =0.18 est superieur à 0.05 nous pouvons conclure que la moyenne est
la meme sur toutes les facultés

    \begin{Verbatim}[commandchars=\\\{\}]
{\color{incolor}In [{\color{incolor}467}]:} \PY{n}{Techno}\PY{o}{.}\PY{n}{columns}
\end{Verbatim}

            \begin{Verbatim}[commandchars=\\\{\}]
{\color{outcolor}Out[{\color{outcolor}467}]:} Index([u'ID', u'SCHOOLSTATUS', u'SCHOOL\_RIGHT', u'OPTION\_RIGHT', u'FAC',
                 u'SCHOOLPROVINCE', u'GENDER', u'DIPPERC', u'AGE', u'CGPA',
                 u'DistinctionRatio', u'EchecRatio', u'NACADYEAR',
                 u'Pass1stSessionRatio'],
                dtype='object')
\end{Verbatim}
        
    \begin{Verbatim}[commandchars=\\\{\}]
{\color{incolor}In [{\color{incolor}639}]:} \PY{n}{SanteGroup}\PY{o}{=}\PY{n}{Sante}\PY{o}{.}\PY{n}{groupby}\PY{p}{(}\PY{l+s+s1}{\PYZsq{}}\PY{l+s+s1}{SCHOOLPROVINCE}\PY{l+s+s1}{\PYZsq{}}\PY{p}{)}\PY{o}{.}\PY{n}{mean}\PY{p}{(}\PY{p}{)}\PY{o}{.}\PY{n}{sort}\PY{p}{(}\PY{n}{axis}\PY{o}{=}\PY{l+m+mi}{0}\PY{p}{,}\PY{n}{columns}\PY{o}{=}\PY{l+s+s1}{\PYZsq{}}\PY{l+s+s1}{CGPA}\PY{l+s+s1}{\PYZsq{}}\PY{p}{,}\PY{n}{ascending}\PY{o}{=}\PY{n+nb+bp}{False}\PY{p}{)}
\end{Verbatim}

    \begin{Verbatim}[commandchars=\\\{\}]
{\color{incolor}In [{\color{incolor}640}]:} \PY{n}{SanteGroup}\PY{o}{.}\PY{n}{reset\PYZus{}index}\PY{p}{(}\PY{n}{inplace}\PY{o}{=}\PY{n+nb+bp}{True}
                               \PY{p}{)}
\end{Verbatim}

    \begin{Verbatim}[commandchars=\\\{\}]
{\color{incolor}In [{\color{incolor}641}]:} \PY{n}{SanteGroup}
\end{Verbatim}

            \begin{Verbatim}[commandchars=\\\{\}]
{\color{outcolor}Out[{\color{outcolor}641}]:}       SCHOOLPROVINCE           ID    DIPPERC        AGE       CGPA  NACADYEAR
          0            KATANGA  8142.250000  63.000000  24.250000  61.591666   3.500000
          1             KIGALI  5698.500000  53.438261  28.500000  61.550000   1.500000
          2          ORIENTALE  8885.043478  54.240435  26.173913  60.647826   2.086957
          3           KINSHASA  7973.200000  55.066667  27.333333  59.762222   2.133333
          4            MANIEMA  8701.500000  53.500000  27.000000  59.700001   2.000000
          5              OUEST  7394.000000  56.876522  29.000000  59.400002   1.000000
          6           BANDUNDU  8105.833333  52.666667  30.000000  59.038889   2.500000
          7          NORD-KIVU  8292.347315  55.198629  25.687919  58.919827   1.951342
          8   KASAI OCCIDENTAL  4781.000000  51.000000  30.000000  58.900000   2.000000
          9     KASAI ORIENTAL  5677.000000  60.000000  30.500000  58.750001   1.500000
          10          SUD-KIVU  7689.475248  56.257426  26.940594  58.344224   2.029703
          11           inconnu  6016.666667  56.292174  28.000000  53.666667   1.000000
\end{Verbatim}
        
    Pour conclure cocernant la moyenne on tien compte du nombre des individu
d'ou un boxplot est importtant

    \begin{Verbatim}[commandchars=\\\{\}]
{\color{incolor}In [{\color{incolor}642}]:} \PY{n}{plt}\PY{o}{.}\PY{n}{figure}\PY{p}{(}\PY{n}{figsize}\PY{o}{=}\PY{p}{(}\PY{l+m+mi}{15}\PY{p}{,}\PY{l+m+mi}{6}\PY{p}{)}\PY{p}{)}
          \PY{n}{ax}\PY{o}{=}\PY{n}{sns}\PY{o}{.}\PY{n}{boxplot}\PY{p}{(}\PY{n}{x}\PY{o}{=}\PY{l+s+s2}{\PYZdq{}}\PY{l+s+s2}{SCHOOLPROVINCE}\PY{l+s+s2}{\PYZdq{}}\PY{p}{,} \PY{n}{y}\PY{o}{=}\PY{l+s+s2}{\PYZdq{}}\PY{l+s+s2}{CGPA}\PY{l+s+s2}{\PYZdq{}}\PY{p}{,} \PY{n}{data}\PY{o}{=}\PY{n}{Sante}\PY{p}{,}\PY{n}{order}\PY{o}{=}\PY{n}{SanteGroup}\PY{o}{.}\PY{n}{SCHOOLPROVINCE}\PY{p}{)}
          \PY{n}{plt}\PY{o}{.}\PY{n}{savefig}\PY{p}{(}\PY{l+s+s1}{\PYZsq{}}\PY{l+s+s1}{GENDER\PYZhy{}SCHOOLPROVINCE\PYZhy{}Sante.png}\PY{l+s+s1}{\PYZsq{}}\PY{p}{,}\PY{n}{dpi}\PY{o}{=}\PY{l+m+mi}{100}\PY{p}{)}
\end{Verbatim}

    \begin{center}
    \adjustimage{max size={0.9\linewidth}{0.9\paperheight}}{output_308_0.png}
    \end{center}
    { \hspace*{\fill} \\}
    
    La tendance est inversé les étudiant de la province du sud et nord kivu
echouent plus en faculté de sante, les distinctions provienent de la
province oriantale et du sud kivu.

    \paragraph{B.5 Stattus de L'école}\label{b.5-stattus-de-luxe9cole}

    \begin{Verbatim}[commandchars=\\\{\}]
{\color{incolor}In [{\color{incolor}643}]:} \PY{n}{moore\PYZus{}lm} \PY{o}{=} \PY{n}{ols}\PY{p}{(}\PY{l+s+s1}{\PYZsq{}}\PY{l+s+s1}{CGPA \PYZti{} C(SCHOOLSTATUS)}\PY{l+s+s1}{\PYZsq{}}\PY{p}{,}\PY{n}{data}\PY{o}{=}\PY{n}{Sante}\PY{p}{)}\PY{o}{.}\PY{n}{fit}\PY{p}{(}\PY{p}{)}
          \PY{n}{aov\PYZus{}table} \PY{o}{=} \PY{n}{sm}\PY{o}{.}\PY{n}{stats}\PY{o}{.}\PY{n}{anova\PYZus{}lm}\PY{p}{(}\PY{n}{moore\PYZus{}lm}\PY{p}{,} \PY{n}{typ}\PY{o}{=}\PY{l+m+mi}{1}\PY{p}{)}
          \PY{n}{aov\PYZus{}table}
\end{Verbatim}

            \begin{Verbatim}[commandchars=\\\{\}]
{\color{outcolor}Out[{\color{outcolor}643}]:}                     df        sum\_sq     mean\_sq         F    PR(>F)
          C(SCHOOLSTATUS)    6.0   1413.669221  235.611537  5.730989  0.000008
          Residual         751.0  30874.997586   41.111848       NaN       NaN
\end{Verbatim}
        
    Il n'ya aucun lien entre les moyennes de CGPA pour les differents status
de l"ecole

    \begin{Verbatim}[commandchars=\\\{\}]
{\color{incolor}In [{\color{incolor}644}]:} \PY{n}{Sante}\PY{o}{.}\PY{n}{groupby}\PY{p}{(}\PY{l+s+s1}{\PYZsq{}}\PY{l+s+s1}{SCHOOLSTATUS}\PY{l+s+s1}{\PYZsq{}}\PY{p}{)}\PY{o}{.}\PY{n}{mean}\PY{p}{(}\PY{p}{)}\PY{o}{.}\PY{n}{sort}\PY{p}{(}\PY{n}{axis}\PY{o}{=}\PY{l+m+mi}{0}\PY{p}{,}\PY{n}{columns}\PY{o}{=}\PY{l+s+s1}{\PYZsq{}}\PY{l+s+s1}{CGPA}\PY{l+s+s1}{\PYZsq{}}\PY{p}{,}\PY{n}{ascending}\PY{o}{=}\PY{n+nb+bp}{False}\PY{p}{)}
\end{Verbatim}

            \begin{Verbatim}[commandchars=\\\{\}]
{\color{outcolor}Out[{\color{outcolor}644}]:}                        ID    DIPPERC        AGE       CGPA  NACADYEAR
          SCHOOLSTATUS                                                         
          catholique    8367.889610  57.522120  25.012987  61.016017   2.142857
          protestant    7915.912821  54.721810  26.107692  59.493077   2.000000
          inconnu       7828.561644  54.749196  27.643836  58.156792   1.972603
          publique      8195.568627  55.130719  25.686275  58.039107   1.836601
          privé         8951.095745  54.127660  25.276596  57.616046   1.882979
          autodidacte   8677.111111  59.888889  25.000000  55.796296   2.444444
          musulman      9024.142857  50.714286  25.571429  53.292858   1.285714
\end{Verbatim}
        
    Nous avons les écoles catholiques et musulman en tete de liste , mais
les écoles musulmanes , privé , et autodidacte viennent en bas de
l'echelle

    Voyons cela de plus pret avec box-plot

    \begin{Verbatim}[commandchars=\\\{\}]
{\color{incolor}In [{\color{incolor}645}]:} \PY{n}{plt}\PY{o}{.}\PY{n}{figure}\PY{p}{(}\PY{n}{figsize}\PY{o}{=}\PY{p}{(}\PY{l+m+mi}{12}\PY{p}{,}\PY{l+m+mi}{6}\PY{p}{)}\PY{p}{)}
          \PY{n}{ax}\PY{o}{=}\PY{n}{sns}\PY{o}{.}\PY{n}{boxplot}\PY{p}{(}\PY{n}{x}\PY{o}{=}\PY{l+s+s2}{\PYZdq{}}\PY{l+s+s2}{SCHOOLSTATUS}\PY{l+s+s2}{\PYZdq{}}\PY{p}{,} \PY{n}{y}\PY{o}{=}\PY{l+s+s2}{\PYZdq{}}\PY{l+s+s2}{CGPA}\PY{l+s+s2}{\PYZdq{}}\PY{p}{,} \PY{n}{data}\PY{o}{=}\PY{n}{Sante}\PY{p}{)}
          \PY{n}{plt}\PY{o}{.}\PY{n}{savefig}\PY{p}{(}\PY{l+s+s1}{\PYZsq{}}\PY{l+s+s1}{SCHOOLSTATUS\PYZhy{}ECO\PYZhy{}Droit.png}\PY{l+s+s1}{\PYZsq{}}\PY{p}{,}\PY{n}{dpi}\PY{o}{=}\PY{l+m+mi}{100}\PY{p}{)}
\end{Verbatim}

    \begin{center}
    \adjustimage{max size={0.9\linewidth}{0.9\paperheight}}{output_316_0.png}
    \end{center}
    { \hspace*{\fill} \\}
    
    Nous remarquons que les celles étudiants qui distingunt c'est ceux
provennant des écoles protestants et catholiques, en queu de la liste se
trouve les autodidacte

    \paragraph{B.Les Ecoles de
provenances}\label{b.les-ecoles-de-provenances}

    \begin{Verbatim}[commandchars=\\\{\}]
{\color{incolor}In [{\color{incolor}646}]:} \PY{n}{moore\PYZus{}lm} \PY{o}{=} \PY{n}{ols}\PY{p}{(}\PY{l+s+s1}{\PYZsq{}}\PY{l+s+s1}{CGPA \PYZti{} C(SCHOOL\PYZus{}RIGHT)}\PY{l+s+s1}{\PYZsq{}}\PY{p}{,}\PY{n}{data}\PY{o}{=}\PY{n}{Sante}\PY{p}{)}\PY{o}{.}\PY{n}{fit}\PY{p}{(}\PY{p}{)}
          \PY{n}{aov\PYZus{}table} \PY{o}{=} \PY{n}{sm}\PY{o}{.}\PY{n}{stats}\PY{o}{.}\PY{n}{anova\PYZus{}lm}\PY{p}{(}\PY{n}{moore\PYZus{}lm}\PY{p}{,} \PY{n}{typ}\PY{o}{=}\PY{l+m+mi}{1}\PY{p}{)}
          \PY{n}{aov\PYZus{}table}
\end{Verbatim}

            \begin{Verbatim}[commandchars=\\\{\}]
{\color{outcolor}Out[{\color{outcolor}646}]:}                     df        sum\_sq    mean\_sq         F   PR(>F)
          C(SCHOOL\_RIGHT)  237.0  12362.421092  52.162114  1.361235  0.00223
          Residual         520.0  19926.245715  38.319703       NaN      NaN
\end{Verbatim}
        
    Aucunne correlation n'existe entre l"ecole et le CGPA au vu de la valeur
de PR'

    \begin{Verbatim}[commandchars=\\\{\}]
{\color{incolor}In [{\color{incolor}647}]:} \PY{n}{SchoolGroup}\PY{o}{=}\PY{n}{Sante}\PY{o}{.}\PY{n}{groupby}\PY{p}{(}\PY{l+s+s1}{\PYZsq{}}\PY{l+s+s1}{SCHOOL\PYZus{}RIGHT}\PY{l+s+s1}{\PYZsq{}}\PY{p}{)}\PY{o}{.}\PY{n}{mean}\PY{p}{(}\PY{p}{)}\PY{o}{.}\PY{n}{sort}\PY{p}{(}\PY{n}{axis}\PY{o}{=}\PY{l+m+mi}{0}\PY{p}{,}\PY{n}{columns}\PY{o}{=}\PY{l+s+s1}{\PYZsq{}}\PY{l+s+s1}{CGPA}\PY{l+s+s1}{\PYZsq{}}\PY{p}{,}\PY{n}{ascending}\PY{o}{=}\PY{n+nb+bp}{False}\PY{p}{)}
\end{Verbatim}

    \begin{Verbatim}[commandchars=\\\{\}]
{\color{incolor}In [{\color{incolor}648}]:} \PY{n}{SchoolGroup}\PY{o}{.}\PY{n}{reset\PYZus{}index}\PY{p}{(}\PY{n}{inplace}\PY{o}{=}\PY{n+nb+bp}{True}\PY{p}{)}
\end{Verbatim}

    \begin{Verbatim}[commandchars=\\\{\}]
{\color{incolor}In [{\color{incolor}649}]:} \PY{n}{SchoolGroup}
\end{Verbatim}

            \begin{Verbatim}[commandchars=\\\{\}]
{\color{outcolor}Out[{\color{outcolor}649}]:}                      SCHOOL\_RIGHT            ID    DIPPERC        AGE  \textbackslash{}
          0                     ITM DE BOGA   8498.000000  71.000000  45.000000   
          1                             itm   6989.000000  56.876522  59.000000   
          2                          malula   9846.000000  62.000000  20.000000   
          3                           bibwe   9451.000000  51.000000  49.000000   
          4              maendeleo de oicha  10154.000000  56.000000  32.000000   
          5                         kirumba   6987.000000  55.000000  39.000000   
          6                    edap isp bkv   9827.000000  55.000000  23.000000   
          7                          kasali  10053.000000  57.000000  25.000000   
          8                         i katwa   6674.000000  53.000000  28.000000   
          9                    mgr kataliko   8158.500000  56.500000  31.000000   
          10            itm tulizeni/kyondo   5182.000000  64.000000  56.000000   
          11                          weza2  10076.000000  60.000000  23.000000   
          12                          amani   7175.500000  58.800000  27.400000   
          13                          kaoze   7327.000000  76.000000  23.000000   
          14                      de kalamu   4808.000000  57.000000  47.000000   
          15           communautaire du lac  11831.000000  64.000000  21.000000   
          16                    bemba gombo   9875.000000  59.750000  21.500000   
          17        complexe scolaire kyabo  10343.000000  52.000000  21.000000   
          18                         katana   6802.000000  54.000000  41.500000   
          19                         masisi   4873.000000  55.000000  25.000000   
          20                        majengo   6586.714286  55.000000  26.857143   
          21                     mama sarah  10532.000000  54.000000  23.000000   
          22                         cirezi   7427.500000  56.500000  25.000000   
          23                      la vision   7424.000000  57.000000  25.000000   
          24                           luka   6430.000000  68.000000  24.000000   
          25                         mwanga   8506.000000  61.000000  23.285714   
          26                      nyakasaza   6322.500000  52.500000  26.500000   
          27               malikia wa bingu   9852.800000  57.600000  24.000000   
          28                           1uto   6746.000000  50.000000  32.000000   
          29                         isingo   8670.000000  62.000000  21.000000   
          ..                            {\ldots}           {\ldots}        {\ldots}        {\ldots}   
          208                        ibanda   8810.000000  54.800000  29.400000   
          209                        lukuga  10347.500000  57.000000  22.500000   
          210                         kimua   9634.000000  56.000000  25.666667   
          211                     bweremana  11126.500000  52.000000  24.000000   
          212                          wima   5669.500000  58.500000  33.500000   
          213                  tumaini letu   4245.000000  54.000000  29.000000   
          214                     gs kigali   7436.500000  55.500000  26.000000   
          215                     kiribunye   8378.000000  62.000000  24.000000   
          216                       kavanda   8647.000000  51.000000  33.000000   
          217              action kusaidiya   9175.000000  55.000000  21.000000   
          218                     nyamukola   6072.000000  58.500000  28.000000   
          219              mikeno islamique   9568.000000  50.666667  24.000000   
          220                     kitumaini   1963.000000  58.000000  41.000000   
          221                        taraja   8768.000000  50.000000  28.000000   
          222                       lumiere   3693.500000  56.500000  32.000000   
          223                       molende   8632.000000  52.000000  25.000000   
          224                      bugarula   6980.000000  50.000000  35.000000   
          225                        kiraku  10664.000000  58.500000  23.500000   
          226                idap isp bunia   6440.000000  58.000000  28.000000   
          227                        mapema   6978.000000  50.000000  34.000000   
          228                     de bagira   4710.000000  50.000000  30.000000   
          229                       MUGUNGA   4733.000000  55.000000  28.000000   
          230                      le germe   7416.000000  50.000000  26.000000   
          231                        kasika   4980.000000  55.000000  31.000000   
          232  complexe scolaire de l'unité  10650.000000  51.000000  22.000000   
          233                         mboga   9819.000000  53.000000  20.000000   
          234                          azma   7411.000000  62.000000  30.000000   
          235                        hodari   9180.000000  57.000000  24.000000   
          236                     nyantende   9041.000000  54.000000  22.000000   
          237                        mandai  10845.000000  54.000000  24.000000   
          
                    CGPA  NACADYEAR  
          0    72.433334   3.000000  
          1    71.300003   1.000000  
          2    68.549999   2.000000  
          3    67.333333   3.000000  
          4    67.200001   2.000000  
          5    67.199997   1.000000  
          6    67.049999   2.000000  
          7    65.800003   2.000000  
          8    65.599998   2.000000  
          9    65.416668   3.000000  
          10   65.400002   1.000000  
          11   65.299999   2.000000  
          12   65.250834   2.300000  
          13   65.166665   3.000000  
          14   65.099998   1.000000  
          15   64.900002   1.000000  
          16   64.862499   2.250000  
          17   64.850002   2.000000  
          18   64.749999   1.500000  
          19   64.350000   2.000000  
          20   64.292857   1.857143  
          21   64.200001   2.000000  
          22   64.166666   3.000000  
          23   64.133333   3.000000  
          24   64.066668   3.000000  
          25   63.939881   2.428571  
          26   63.849998   1.000000  
          27   63.713333   1.400000  
          28   63.700001   1.000000  
          29   63.699999   2.000000  
          ..         {\ldots}        {\ldots}  
          208  52.100000   1.200000  
          209  52.075001   1.500000  
          210  51.955555   1.666667  
          211  51.724999   1.500000  
          212  51.316667   2.000000  
          213  51.299999   2.000000  
          214  50.750000   1.250000  
          215  50.600000   2.000000  
          216  50.166667   3.000000  
          217  50.000000   2.000000  
          218  49.600000   2.000000  
          219  49.533333   1.333333  
          220  49.000000   1.000000  
          221  48.100000   2.000000  
          222  48.049999   1.000000  
          223  48.000000   2.000000  
          224  48.000000   1.000000  
          225  47.450001   1.500000  
          226  47.000000   1.000000  
          227  45.000000   1.000000  
          228  44.500000   2.000000  
          229  44.000000   1.000000  
          230  44.000000   1.000000  
          231  44.000000   1.000000  
          232  44.000000   1.000000  
          233  42.000000   1.000000  
          234  41.000000   1.000000  
          235  41.000000   1.000000  
          236  41.000000   1.000000  
          237  40.000000   1.000000  
          
          [238 rows x 6 columns]
\end{Verbatim}
        
    Nous allons Voir de plus pret pour les 10 ecoles les plus representé, le
5 éecoles avec un pourcentage elevé et 5 dernier

    \begin{Verbatim}[commandchars=\\\{\}]
{\color{incolor}In [{\color{incolor}650}]:} \PY{n}{plt}\PY{o}{.}\PY{n}{figure}\PY{p}{(}\PY{n}{figsize}\PY{o}{=}\PY{p}{(}\PY{l+m+mi}{30}\PY{p}{,}\PY{l+m+mi}{10}\PY{p}{)}\PY{p}{)}
          \PY{n}{ax}\PY{o}{=}\PY{n}{sns}\PY{o}{.}\PY{n}{boxplot}\PY{p}{(}\PY{n}{x}\PY{o}{=}\PY{l+s+s2}{\PYZdq{}}\PY{l+s+s2}{SCHOOL\PYZus{}RIGHT}\PY{l+s+s2}{\PYZdq{}}\PY{p}{,} \PY{n}{y}\PY{o}{=}\PY{l+s+s2}{\PYZdq{}}\PY{l+s+s2}{CGPA}\PY{l+s+s2}{\PYZdq{}}\PY{p}{,}  \PY{n}{data}\PY{o}{=}\PY{n}{Sante}\PY{o}{.}\PY{n}{loc}\PY{p}{[}\PY{n}{operator}\PY{o}{.}\PY{n}{or\PYZus{}}\PY{p}{(}\PY{n}{Sante}\PY{o}{.}\PY{n}{SCHOOL\PYZus{}RIGHT}\PY{o}{.}\PY{n}{isin}\PY{p}{(}\PY{n}{SchoolGroup}\PY{o}{.}\PY{n}{loc}\PY{p}{[} \PY{n+nb}{range}\PY{p}{(}\PY{l+m+mi}{0}\PY{p}{,}\PY{l+m+mi}{11}\PY{p}{)}\PY{p}{]}\PY{o}{.}\PY{n}{SCHOOL\PYZus{}RIGHT}\PY{p}{)} 
                       \PY{p}{,} \PY{n}{Sante}\PY{o}{.}\PY{n}{SCHOOL\PYZus{}RIGHT}\PY{o}{.}\PY{n}{isin}\PY{p}{(}\PY{n}{Sante}\PY{o}{.}\PY{n}{SCHOOL\PYZus{}RIGHT}\PY{o}{.}\PY{n}{value\PYZus{}counts}\PY{p}{(}\PY{p}{)}\PY{p}{[}\PY{p}{:}\PY{l+m+mi}{10}\PY{p}{]}\PY{o}{.}\PY{n}{index} \PY{p}{)}\PY{p}{)}\PY{p}{]}\PY{p}{)}
          \PY{n}{plt}\PY{o}{.}\PY{n}{savefig}\PY{p}{(}\PY{l+s+s1}{\PYZsq{}}\PY{l+s+s1}{SCHOOL\PYZus{}RIGHT\PYZhy{}CGPASante.png}\PY{l+s+s1}{\PYZsq{}}\PY{p}{,}\PY{n}{dpi}\PY{o}{=}\PY{l+m+mi}{100}\PY{p}{)}
\end{Verbatim}

    \begin{center}
    \adjustimage{max size={0.9\linewidth}{0.9\paperheight}}{output_325_0.png}
    \end{center}
    { \hspace*{\fill} \\}
    
    Les étudiant provenant des écoles suivantes echouent rarementen faculté
de sante : Maendeleo, sainte ursule , bakandja

    \paragraph{B.7 OPTION DU DIPLOME}\label{b.7-option-du-diplome}

    \begin{Verbatim}[commandchars=\\\{\}]
{\color{incolor}In [{\color{incolor}652}]:} \PY{n+nb}{len}\PY{p}{(}\PY{n}{Sante}\PY{o}{.}\PY{n}{OPTION\PYZus{}RIGHT}\PY{o}{.}\PY{n}{value\PYZus{}counts}\PY{p}{(}\PY{p}{)}\PY{p}{)}
\end{Verbatim}

            \begin{Verbatim}[commandchars=\\\{\}]
{\color{outcolor}Out[{\color{outcolor}652}]:} 15
\end{Verbatim}
        
    Nous pouvons constater que nous avons 17 options differents Voyons
comment evoluer la moyenne

    \begin{Verbatim}[commandchars=\\\{\}]
{\color{incolor}In [{\color{incolor}653}]:} \PY{n}{moore\PYZus{}lm} \PY{o}{=} \PY{n}{ols}\PY{p}{(}\PY{l+s+s1}{\PYZsq{}}\PY{l+s+s1}{CGPA \PYZti{} C(OPTION\PYZus{}RIGHT)}\PY{l+s+s1}{\PYZsq{}}\PY{p}{,}\PY{n}{data}\PY{o}{=}\PY{n}{Sante}\PY{p}{)}\PY{o}{.}\PY{n}{fit}\PY{p}{(}\PY{p}{)}
          \PY{n}{aov\PYZus{}table} \PY{o}{=} \PY{n}{sm}\PY{o}{.}\PY{n}{stats}\PY{o}{.}\PY{n}{anova\PYZus{}lm}\PY{p}{(}\PY{n}{moore\PYZus{}lm}\PY{p}{,} \PY{n}{typ}\PY{o}{=}\PY{l+m+mi}{1}\PY{p}{)}
          \PY{n}{aov\PYZus{}table}
\end{Verbatim}

            \begin{Verbatim}[commandchars=\\\{\}]
{\color{outcolor}Out[{\color{outcolor}653}]:}                     df        sum\_sq     mean\_sq         F    PR(>F)
          C(OPTION\_RIGHT)   14.0   2256.496238  161.178303  3.987573  0.000001
          Residual         743.0  30032.170569   40.420149       NaN       NaN
\end{Verbatim}
        
    La valeur est inferieur à 0.05 donc il n'yas pas de relation entre les
valeurs , certaines écoles on une bonne moyenne du GPA que d'autres donc
elle depend de l'ecole de provenance

    \begin{Verbatim}[commandchars=\\\{\}]
{\color{incolor}In [{\color{incolor}654}]:} \PY{n}{GroupOption}\PY{o}{=}\PY{n}{Sante}\PY{o}{.}\PY{n}{groupby}\PY{p}{(}\PY{l+s+s1}{\PYZsq{}}\PY{l+s+s1}{OPTION\PYZus{}RIGHT}\PY{l+s+s1}{\PYZsq{}}\PY{p}{)}\PY{o}{.}\PY{n}{mean}\PY{p}{(}\PY{p}{)}\PY{o}{.}\PY{n}{sort}\PY{p}{(}\PY{n}{axis}\PY{o}{=}\PY{l+m+mi}{0}\PY{p}{,}\PY{n}{columns}\PY{o}{=}\PY{l+s+s1}{\PYZsq{}}\PY{l+s+s1}{CGPA}\PY{l+s+s1}{\PYZsq{}}\PY{p}{,}\PY{n}{ascending}\PY{o}{=}\PY{n+nb+bp}{False}\PY{p}{)}\PY{o}{.}\PY{n}{reset\PYZus{}index}\PY{p}{(}\PY{p}{)}
\end{Verbatim}

    \begin{Verbatim}[commandchars=\\\{\}]
{\color{incolor}In [{\color{incolor}655}]:} \PY{n}{GroupOption}
\end{Verbatim}

            \begin{Verbatim}[commandchars=\\\{\}]
{\color{outcolor}Out[{\color{outcolor}655}]:}                 OPTION\_RIGHT            ID    DIPPERC        AGE       CGPA  \textbackslash{}
          0                vétérinaire   9157.600000  57.200000  33.800000  64.756667   
          1   commerciale informatique   9240.000000  50.000000  23.000000  63.166668   
          2          hotesse d'acceuil   9850.000000  50.000000  22.000000  62.600000   
          3                    inconnu   5693.400000  57.150609  39.800000  61.933335   
          4                 bio-chimie   7584.410256  54.622876  26.487179  61.615741   
          5                latin philo   7856.148936  56.923543  25.680851  61.501596   
          6              math-physique   7090.000000  54.625138  28.727273  60.657576   
          7                       diet   4769.500000  60.000000  26.500000  60.299999   
          8                       nutr   8030.702128  57.191489  24.148936  59.773227   
          9        commmerciale et adm   8258.451613  55.354839  24.774194  59.697043   
          10                 pedagogie   8314.579310  55.450873  26.417241  57.940632   
          11                   sociale   8611.016043  54.593583  24.716578  57.563102   
          12             coupe couture   8057.100000  55.000000  28.300000  56.921666   
          13              hospitalière   9575.333333  60.958841  37.000000  54.699999   
          14                  agrecole  10087.000000  54.000000  21.000000  52.600000   
          
              NACADYEAR  
          0    2.600000  
          1    3.000000  
          2    2.000000  
          3    1.400000  
          4    2.085470  
          5    2.085106  
          6    1.818182  
          7    1.000000  
          8    2.085106  
          9    1.806452  
          10   1.996552  
          11   1.893048  
          12   1.400000  
          13   1.666667  
          14   2.000000  
\end{Verbatim}
        
    Voila comme on pourait s'y attendre l'option latin philo est en tete de
kiste les options commerciale , pedagogie , sociale , math-physique ,mec
sont en bas de l'echele

    \begin{Verbatim}[commandchars=\\\{\}]
{\color{incolor}In [{\color{incolor}656}]:} \PY{n}{plt}\PY{o}{.}\PY{n}{figure}\PY{p}{(}\PY{n}{figsize}\PY{o}{=}\PY{p}{(}\PY{l+m+mi}{35}\PY{p}{,}\PY{l+m+mi}{10}\PY{p}{)}\PY{p}{)}
          \PY{n}{ax}\PY{o}{=}\PY{n}{sns}\PY{o}{.}\PY{n}{boxplot}\PY{p}{(}\PY{n}{x}\PY{o}{=}\PY{l+s+s2}{\PYZdq{}}\PY{l+s+s2}{OPTION\PYZus{}RIGHT}\PY{l+s+s2}{\PYZdq{}}\PY{p}{,} \PY{n}{y}\PY{o}{=}\PY{l+s+s2}{\PYZdq{}}\PY{l+s+s2}{CGPA}\PY{l+s+s2}{\PYZdq{}}\PY{p}{,} \PY{n}{data}\PY{o}{=}\PY{n}{Sante}\PY{p}{,}\PY{n}{order}\PY{o}{=}\PY{n}{GroupOption}\PY{o}{.}\PY{n}{OPTION\PYZus{}RIGHT}\PY{p}{)}
          \PY{n}{plt}\PY{o}{.}\PY{n}{savefig}\PY{p}{(}\PY{l+s+s1}{\PYZsq{}}\PY{l+s+s1}{OPTION\PYZus{}RIGHT\PYZhy{}CGPADroit.png}\PY{l+s+s1}{\PYZsq{}}\PY{p}{,}\PY{n}{dpi}\PY{o}{=}\PY{l+m+mi}{100}\PY{p}{)}
\end{Verbatim}

    \begin{center}
    \adjustimage{max size={0.9\linewidth}{0.9\paperheight}}{output_335_0.png}
    \end{center}
    { \hspace*{\fill} \\}
    
    Au sein de la faculté de sante certaines option comme : biochimie,
latin-phil,nutrition, o on une meme moyenne

    \subparagraph{Faculté de Psycologie}\label{facultuxe9-de-psycologie}

    \begin{Verbatim}[commandchars=\\\{\}]
{\color{incolor}In [{\color{incolor}657}]:} \PY{n}{Psyco}\PY{o}{=}\PY{n}{datasetFin}\PY{o}{.}\PY{n}{loc}\PY{p}{[}\PY{n}{datasetFin}\PY{o}{.}\PY{n}{FAC}\PY{o}{==}\PY{l+s+s1}{\PYZsq{}}\PY{l+s+s1}{FPSE}\PY{l+s+s1}{\PYZsq{}}\PY{p}{]}
\end{Verbatim}

    On constate que le pourcentage de CGPA varie differament pour chaque
option du diplome , les options peda, sociale ,vienne en dernier en ce
qui concerne le cGPA mean

    \begin{Verbatim}[commandchars=\\\{\}]
{\color{incolor}In [{\color{incolor}658}]:} \PY{n}{Psyco}\PY{o}{.}\PY{n}{shape}
\end{Verbatim}

            \begin{Verbatim}[commandchars=\\\{\}]
{\color{outcolor}Out[{\color{outcolor}658}]:} (227, 14)
\end{Verbatim}
        
    https://www.wellbeingatschool.org.nz/information-sheet/understanding-and-interpreting-box-plots

    A.0 Distribution du CGPA

    \begin{Verbatim}[commandchars=\\\{\}]
{\color{incolor}In [{\color{incolor}659}]:} \PY{n}{plt}\PY{o}{.}\PY{n}{figure}\PY{p}{(}\PY{p}{)}
          \PY{n}{ax} \PY{o}{=} \PY{n}{sns}\PY{o}{.}\PY{n}{boxplot}\PY{p}{(}\PY{n}{x}\PY{o}{=}\PY{n}{Psyco}\PY{p}{[}\PY{l+s+s1}{\PYZsq{}}\PY{l+s+s1}{CGPA}\PY{l+s+s1}{\PYZsq{}}\PY{p}{]}\PY{p}{)}
          \PY{n}{plt}\PY{o}{.}\PY{n}{savefig}\PY{p}{(}\PY{l+s+s1}{\PYZsq{}}\PY{l+s+s1}{CGPA\PYZhy{}ECO\PYZhy{}Psyco.png}\PY{l+s+s1}{\PYZsq{}}\PY{p}{,}\PY{n}{dpi}\PY{o}{=}\PY{l+m+mi}{100}\PY{p}{)}
\end{Verbatim}

    \begin{center}
    \adjustimage{max size={0.9\linewidth}{0.9\paperheight}}{output_343_0.png}
    \end{center}
    { \hspace*{\fill} \\}
    
    \begin{Verbatim}[commandchars=\\\{\}]
{\color{incolor}In [{\color{incolor}661}]:} \PY{n}{plt}\PY{o}{.}\PY{n}{figure}\PY{p}{(}\PY{p}{)}
          \PY{n}{sns}\PY{o}{.}\PY{n}{distplot}\PY{p}{(}\PY{n}{Psyco}\PY{p}{[}\PY{l+s+s1}{\PYZsq{}}\PY{l+s+s1}{CGPA}\PY{l+s+s1}{\PYZsq{}}\PY{p}{]}\PY{p}{,}\PY{n}{bins}\PY{o}{=}\PY{l+m+mi}{20}\PY{p}{,}\PY{n}{axlabel}\PY{o}{=}\PY{l+s+s1}{\PYZsq{}}\PY{l+s+s1}{CGPA}\PY{l+s+s1}{\PYZsq{}}\PY{p}{,}\PY{n}{kde}\PY{o}{=}\PY{l+m+mi}{1}\PY{p}{,}\PY{n}{norm\PYZus{}hist}\PY{o}{=}\PY{l+m+mi}{0}\PY{p}{)}
          \PY{n}{plt}\PY{o}{.}\PY{n}{savefig}\PY{p}{(}\PY{l+s+s1}{\PYZsq{}}\PY{l+s+s1}{CGPADistPsycho.png}\PY{l+s+s1}{\PYZsq{}}\PY{p}{,}\PY{n}{dpi}\PY{o}{=}\PY{l+m+mi}{100}\PY{p}{)}
\end{Verbatim}

    \begin{center}
    \adjustimage{max size={0.9\linewidth}{0.9\paperheight}}{output_344_0.png}
    \end{center}
    { \hspace*{\fill} \\}
    
    Il suit une distribution presque normale d'écart type : 7,68

    \begin{Verbatim}[commandchars=\\\{\}]
{\color{incolor}In [{\color{incolor}662}]:} \PY{n}{np}\PY{o}{.}\PY{n}{std}\PY{p}{(}\PY{n}{Psyco}\PY{o}{.}\PY{n}{CGPA}\PY{p}{)}
\end{Verbatim}

            \begin{Verbatim}[commandchars=\\\{\}]
{\color{outcolor}Out[{\color{outcolor}662}]:} 7.6876434337029549
\end{Verbatim}
        
    \begin{Verbatim}[commandchars=\\\{\}]
{\color{incolor}In [{\color{incolor}602}]:} \PY{n}{np}\PY{o}{.}\PY{n}{mean}\PY{p}{(}\PY{n}{Droit}\PY{o}{.}\PY{n}{CGPA}\PY{p}{)}
\end{Verbatim}

            \begin{Verbatim}[commandchars=\\\{\}]
{\color{outcolor}Out[{\color{outcolor}602}]:} 58.022023761201453
\end{Verbatim}
        
    Commencons par les attribues numeriques et analysons la correlation
chercher comme sur le point précedant la correlation avec le CGPA

    \begin{Verbatim}[commandchars=\\\{\}]
{\color{incolor}In [{\color{incolor}663}]:} \PY{n}{Psyco}\PY{o}{.}\PY{n}{corr}\PY{p}{(}\PY{p}{)}
\end{Verbatim}

            \begin{Verbatim}[commandchars=\\\{\}]
{\color{outcolor}Out[{\color{outcolor}663}]:}                  ID   DIPPERC       AGE      CGPA  NACADYEAR
          ID         1.000000  0.111114 -0.607498 -0.152894  -0.079377
          DIPPERC    0.111114  1.000000 -0.233634  0.327076   0.291769
          AGE       -0.607498 -0.233634  1.000000  0.095474  -0.036995
          CGPA      -0.152894  0.327076  0.095474  1.000000   0.434137
          NACADYEAR -0.079377  0.291769 -0.036995  0.434137   1.000000
\end{Verbatim}
        
    Nous pouvons aisement qu'il n'yas aucune correlation entre le le CGPA et
le pourcentage à l'exetat ni meme l'age des étudiants

    \paragraph{C.2 GENDER}\label{c.2-gender}

    \begin{Verbatim}[commandchars=\\\{\}]
{\color{incolor}In [{\color{incolor}664}]:} \PY{n}{moore\PYZus{}lm} \PY{o}{=} \PY{n}{ols}\PY{p}{(}\PY{l+s+s1}{\PYZsq{}}\PY{l+s+s1}{CGPA \PYZti{} C(GENDER)}\PY{l+s+s1}{\PYZsq{}}\PY{p}{,}\PY{n}{data}\PY{o}{=}\PY{n}{Psyco}\PY{p}{)}\PY{o}{.}\PY{n}{fit}\PY{p}{(}\PY{p}{)}
          \PY{n}{aov\PYZus{}table} \PY{o}{=} \PY{n}{sm}\PY{o}{.}\PY{n}{stats}\PY{o}{.}\PY{n}{anova\PYZus{}lm}\PY{p}{(}\PY{n}{moore\PYZus{}lm}\PY{p}{,} \PY{n}{typ}\PY{o}{=}\PY{l+m+mi}{1}\PY{p}{)}
          \PY{n}{aov\PYZus{}table}
\end{Verbatim}

            \begin{Verbatim}[commandchars=\\\{\}]
{\color{outcolor}Out[{\color{outcolor}664}]:}               df        sum\_sq    mean\_sq         F    PR(>F)
          C(GENDER)    1.0     16.096698  16.096698  0.270289  0.603649
          Residual   225.0  13399.571877  59.553653       NaN       NaN
\end{Verbatim}
        
    Notre metrique nous montre que la moyenne de CGPA est la meme pour les
deux sexes ! jettons un coup d'oeil aux distribution avec des box-plot.

    \begin{Verbatim}[commandchars=\\\{\}]
{\color{incolor}In [{\color{incolor}665}]:} \PY{n}{plt}\PY{o}{.}\PY{n}{figure}\PY{p}{(}\PY{n}{figsize}\PY{o}{=}\PY{p}{(}\PY{l+m+mi}{12}\PY{p}{,}\PY{l+m+mi}{6}\PY{p}{)}\PY{p}{)}
          \PY{n}{ax}\PY{o}{=}\PY{n}{sns}\PY{o}{.}\PY{n}{boxplot}\PY{p}{(}\PY{n}{x}\PY{o}{=}\PY{l+s+s2}{\PYZdq{}}\PY{l+s+s2}{GENDER}\PY{l+s+s2}{\PYZdq{}}\PY{p}{,} \PY{n}{y}\PY{o}{=}\PY{l+s+s2}{\PYZdq{}}\PY{l+s+s2}{CGPA}\PY{l+s+s2}{\PYZdq{}}\PY{p}{,} \PY{n}{data}\PY{o}{=}\PY{n}{Psyco}\PY{p}{)}
          \PY{n}{plt}\PY{o}{.}\PY{n}{savefig}\PY{p}{(}\PY{l+s+s1}{\PYZsq{}}\PY{l+s+s1}{GENDER\PYZhy{}CGPA\PYZhy{}Eco.png}\PY{l+s+s1}{\PYZsq{}}\PY{p}{,}\PY{n}{dpi}\PY{o}{=}\PY{l+m+mi}{100}\PY{p}{)}
\end{Verbatim}

    \begin{center}
    \adjustimage{max size={0.9\linewidth}{0.9\paperheight}}{output_354_0.png}
    \end{center}
    { \hspace*{\fill} \\}
    
    \begin{Verbatim}[commandchars=\\\{\}]
{\color{incolor}In [{\color{incolor}666}]:} \PY{n}{Psyco}\PY{o}{.}\PY{n}{GENDER}\PY{o}{.}\PY{n}{value\PYZus{}counts}\PY{p}{(}\PY{p}{)}\PY{o}{/}\PY{n+nb}{len}\PY{p}{(}\PY{n}{Psyco}\PY{p}{)}
\end{Verbatim}

            \begin{Verbatim}[commandchars=\\\{\}]
{\color{outcolor}Out[{\color{outcolor}666}]:} F    0.524229
          H    0.475771
          Name: GENDER, dtype: float64
\end{Verbatim}
        
    la faculté de psycho est aussi feminine avec 52\% de femmes , et 48 des
hommes

    \begin{Verbatim}[commandchars=\\\{\}]
{\color{incolor}In [{\color{incolor}667}]:} \PY{n}{Psyco}\PY{o}{.}\PY{n}{groupby}\PY{p}{(}\PY{l+s+s1}{\PYZsq{}}\PY{l+s+s1}{GENDER}\PY{l+s+s1}{\PYZsq{}}\PY{p}{)}\PY{o}{.}\PY{n}{mean}\PY{p}{(}\PY{p}{)}
\end{Verbatim}

            \begin{Verbatim}[commandchars=\\\{\}]
{\color{outcolor}Out[{\color{outcolor}667}]:}                  ID    DIPPERC        AGE       CGPA  NACADYEAR
          GENDER                                                         
          F       8550.411765  56.377114  28.067227  59.812605   2.100840
          H       8270.953704  56.009259  28.398148  59.279398   1.916667
\end{Verbatim}
        
    Mais on peut remarquer que les moyenees restent la meme

    \paragraph{B.4 Attribue Province}\label{b.4-attribue-province}

    \begin{Verbatim}[commandchars=\\\{\}]
{\color{incolor}In [{\color{incolor}668}]:} \PY{n}{moore\PYZus{}lm} \PY{o}{=} \PY{n}{ols}\PY{p}{(}\PY{l+s+s1}{\PYZsq{}}\PY{l+s+s1}{CGPA \PYZti{} C(SCHOOLPROVINCE)}\PY{l+s+s1}{\PYZsq{}}\PY{p}{,}\PY{n}{data}\PY{o}{=}\PY{n}{Psyco}\PY{p}{)}\PY{o}{.}\PY{n}{fit}\PY{p}{(}\PY{p}{)}
          \PY{n}{aov\PYZus{}table} \PY{o}{=} \PY{n}{sm}\PY{o}{.}\PY{n}{stats}\PY{o}{.}\PY{n}{anova\PYZus{}lm}\PY{p}{(}\PY{n}{moore\PYZus{}lm}\PY{p}{,} \PY{n}{typ}\PY{o}{=}\PY{l+m+mi}{1}\PY{p}{)}
          \PY{n}{aov\PYZus{}table}
\end{Verbatim}

            \begin{Verbatim}[commandchars=\\\{\}]
{\color{outcolor}Out[{\color{outcolor}668}]:}                       df        sum\_sq    mean\_sq         F    PR(>F)
          C(SCHOOLPROVINCE)    7.0    335.677395  47.953914  0.802899  0.585665
          Residual           219.0  13079.991180  59.725987       NaN       NaN
\end{Verbatim}
        
    PR =0.58 est superieur à 0.05 nous pouvons conclure que la moyenne est
la meme sur toutes les facultés

    \begin{Verbatim}[commandchars=\\\{\}]
{\color{incolor}In [{\color{incolor}669}]:} \PY{n}{PsychoGroup}\PY{o}{=}\PY{n}{Psyco}\PY{o}{.}\PY{n}{groupby}\PY{p}{(}\PY{l+s+s1}{\PYZsq{}}\PY{l+s+s1}{SCHOOLPROVINCE}\PY{l+s+s1}{\PYZsq{}}\PY{p}{)}\PY{o}{.}\PY{n}{mean}\PY{p}{(}\PY{p}{)}\PY{o}{.}\PY{n}{sort}\PY{p}{(}\PY{n}{axis}\PY{o}{=}\PY{l+m+mi}{0}\PY{p}{,}\PY{n}{columns}\PY{o}{=}\PY{l+s+s1}{\PYZsq{}}\PY{l+s+s1}{CGPA}\PY{l+s+s1}{\PYZsq{}}\PY{p}{,}\PY{n}{ascending}\PY{o}{=}\PY{n+nb+bp}{False}\PY{p}{)}
\end{Verbatim}

    \begin{Verbatim}[commandchars=\\\{\}]
{\color{incolor}In [{\color{incolor}670}]:} \PY{n}{PsychoGroup}\PY{o}{.}\PY{n}{reset\PYZus{}index}\PY{p}{(}\PY{n}{inplace}\PY{o}{=}\PY{n+nb+bp}{True}
                               \PY{p}{)}
\end{Verbatim}

    \begin{Verbatim}[commandchars=\\\{\}]
{\color{incolor}In [{\color{incolor}671}]:} \PY{n}{PsychoGroup}
\end{Verbatim}

            \begin{Verbatim}[commandchars=\\\{\}]
{\color{outcolor}Out[{\color{outcolor}671}]:}      SCHOOLPROVINCE            ID    DIPPERC        AGE       CGPA  NACADYEAR
          0         ORIENTALE   6709.000000  52.000000  38.000000  68.800001   3.000000
          1  KASAI OCCIDENTAL  10342.000000  53.000000  36.000000  62.600000   2.000000
          2          SUD-KIVU   7982.117647  56.176471  29.960784  60.612908   2.137255
          3         NORD-KIVU   8500.976048  56.238782  27.520958  59.325948   1.988024
          4          KINSHASA  12037.000000  61.500000  26.500000  59.150000   1.000000
          5           inconnu   2523.000000  50.000000  38.000000  59.000000   1.000000
          6          BANDUNDU  12037.000000  61.000000  25.000000  55.700001   1.000000
          7           MANIEMA   9442.000000  54.000000  31.000000  52.261111   2.333333
\end{Verbatim}
        
    Pour conclure cocernant la moyenne on tien compte du nombre des individu
d'ou un boxplot est importtant

    \begin{Verbatim}[commandchars=\\\{\}]
{\color{incolor}In [{\color{incolor}672}]:} \PY{n}{plt}\PY{o}{.}\PY{n}{figure}\PY{p}{(}\PY{n}{figsize}\PY{o}{=}\PY{p}{(}\PY{l+m+mi}{15}\PY{p}{,}\PY{l+m+mi}{6}\PY{p}{)}\PY{p}{)}
          \PY{n}{ax}\PY{o}{=}\PY{n}{sns}\PY{o}{.}\PY{n}{boxplot}\PY{p}{(}\PY{n}{x}\PY{o}{=}\PY{l+s+s2}{\PYZdq{}}\PY{l+s+s2}{SCHOOLPROVINCE}\PY{l+s+s2}{\PYZdq{}}\PY{p}{,} \PY{n}{y}\PY{o}{=}\PY{l+s+s2}{\PYZdq{}}\PY{l+s+s2}{CGPA}\PY{l+s+s2}{\PYZdq{}}\PY{p}{,} \PY{n}{data}\PY{o}{=}\PY{n}{Psyco}\PY{p}{,}\PY{n}{order}\PY{o}{=}\PY{n}{PsychoGroup}\PY{o}{.}\PY{n}{SCHOOLPROVINCE}\PY{p}{)}
          \PY{n}{plt}\PY{o}{.}\PY{n}{savefig}\PY{p}{(}\PY{l+s+s1}{\PYZsq{}}\PY{l+s+s1}{GENDER\PYZhy{}SCHOOLPROVINCE\PYZhy{}Droit.png}\PY{l+s+s1}{\PYZsq{}}\PY{p}{,}\PY{n}{dpi}\PY{o}{=}\PY{l+m+mi}{100}\PY{p}{)}
\end{Verbatim}

    \begin{center}
    \adjustimage{max size={0.9\linewidth}{0.9\paperheight}}{output_366_0.png}
    \end{center}
    { \hspace*{\fill} \\}
    
    Nous remarquons que le moyenne du CGPA est la meme pour les provinces
suivantes : Nord-Kivu, Sud-Kivu, Kinshasa , Orientale ,Maniema, les
distinctions proviennent beaucoup plus des provincces du nord-kivu, sud
kivu .

    \paragraph{B.5 Stattus de L'école}\label{b.5-stattus-de-luxe9cole}

    \begin{Verbatim}[commandchars=\\\{\}]
{\color{incolor}In [{\color{incolor}673}]:} \PY{n}{moore\PYZus{}lm} \PY{o}{=} \PY{n}{ols}\PY{p}{(}\PY{l+s+s1}{\PYZsq{}}\PY{l+s+s1}{CGPA \PYZti{} C(SCHOOLSTATUS)}\PY{l+s+s1}{\PYZsq{}}\PY{p}{,}\PY{n}{data}\PY{o}{=}\PY{n}{Psyco}\PY{p}{)}\PY{o}{.}\PY{n}{fit}\PY{p}{(}\PY{p}{)}
          \PY{n}{aov\PYZus{}table} \PY{o}{=} \PY{n}{sm}\PY{o}{.}\PY{n}{stats}\PY{o}{.}\PY{n}{anova\PYZus{}lm}\PY{p}{(}\PY{n}{moore\PYZus{}lm}\PY{p}{,} \PY{n}{typ}\PY{o}{=}\PY{l+m+mi}{1}\PY{p}{)}
          \PY{n}{aov\PYZus{}table}
\end{Verbatim}

            \begin{Verbatim}[commandchars=\\\{\}]
{\color{outcolor}Out[{\color{outcolor}673}]:}                     df        sum\_sq     mean\_sq         F    PR(>F)
          C(SCHOOLSTATUS)    6.0    900.916938  150.152823  2.639575  0.017156
          Residual         220.0  12514.751637   56.885235       NaN       NaN
\end{Verbatim}
        
    la moyenne est presque la meme pour toutes les categories

    \begin{Verbatim}[commandchars=\\\{\}]
{\color{incolor}In [{\color{incolor}674}]:} \PY{n}{Psyco}\PY{o}{.}\PY{n}{groupby}\PY{p}{(}\PY{l+s+s1}{\PYZsq{}}\PY{l+s+s1}{SCHOOLSTATUS}\PY{l+s+s1}{\PYZsq{}}\PY{p}{)}\PY{o}{.}\PY{n}{mean}\PY{p}{(}\PY{p}{)}\PY{o}{.}\PY{n}{sort}\PY{p}{(}\PY{n}{axis}\PY{o}{=}\PY{l+m+mi}{0}\PY{p}{,}\PY{n}{columns}\PY{o}{=}\PY{l+s+s1}{\PYZsq{}}\PY{l+s+s1}{CGPA}\PY{l+s+s1}{\PYZsq{}}\PY{p}{,}\PY{n}{ascending}\PY{o}{=}\PY{n+nb+bp}{False}\PY{p}{)}
\end{Verbatim}

            \begin{Verbatim}[commandchars=\\\{\}]
{\color{outcolor}Out[{\color{outcolor}674}]:}                         ID    DIPPERC        AGE       CGPA  NACADYEAR
          SCHOOLSTATUS                                                          
          catholique     9047.420000  59.460000  25.040000  62.376167   2.420000
          inconnu        7796.695652  55.606011  31.065217  59.707790   2.195652
          protestant     8059.927711  55.277108  28.891566  59.187651   1.939759
          publique       8605.392857  55.107143  28.214286  58.550000   1.642857
          privé          9650.125000  54.687500  25.875000  55.908333   1.500000
          autodidacte    8311.333333  55.666667  34.000000  53.333333   1.000000
          musulman      10482.000000  54.000000  22.000000  48.000000   1.000000
\end{Verbatim}
        
    Les écoles catholiques viennent en tet ensuite vien t les écoles
protestatnt et publiques

    Voyons cela de plus pret avec box-plot

    \begin{Verbatim}[commandchars=\\\{\}]
{\color{incolor}In [{\color{incolor}675}]:} \PY{n}{plt}\PY{o}{.}\PY{n}{figure}\PY{p}{(}\PY{n}{figsize}\PY{o}{=}\PY{p}{(}\PY{l+m+mi}{12}\PY{p}{,}\PY{l+m+mi}{6}\PY{p}{)}\PY{p}{)}
          \PY{n}{ax}\PY{o}{=}\PY{n}{sns}\PY{o}{.}\PY{n}{boxplot}\PY{p}{(}\PY{n}{x}\PY{o}{=}\PY{l+s+s2}{\PYZdq{}}\PY{l+s+s2}{SCHOOLSTATUS}\PY{l+s+s2}{\PYZdq{}}\PY{p}{,} \PY{n}{y}\PY{o}{=}\PY{l+s+s2}{\PYZdq{}}\PY{l+s+s2}{CGPA}\PY{l+s+s2}{\PYZdq{}}\PY{p}{,} \PY{n}{data}\PY{o}{=}\PY{n}{Psyco}\PY{p}{)}
          \PY{n}{plt}\PY{o}{.}\PY{n}{savefig}\PY{p}{(}\PY{l+s+s1}{\PYZsq{}}\PY{l+s+s1}{SCHOOLSTATUS\PYZhy{}ECO\PYZhy{}Psyco.png}\PY{l+s+s1}{\PYZsq{}}\PY{p}{,}\PY{n}{dpi}\PY{o}{=}\PY{l+m+mi}{100}\PY{p}{)}
\end{Verbatim}

    \begin{center}
    \adjustimage{max size={0.9\linewidth}{0.9\paperheight}}{output_374_0.png}
    \end{center}
    { \hspace*{\fill} \\}
    
    les étudiants issue des écoles privées catholiques, et protesstantes
obtienent des bonnes notes au sein de la faculté de psycho

    \paragraph{B.Les Ecoles de
provenances}\label{b.les-ecoles-de-provenances}

    \begin{Verbatim}[commandchars=\\\{\}]
{\color{incolor}In [{\color{incolor}676}]:} \PY{n}{moore\PYZus{}lm} \PY{o}{=} \PY{n}{ols}\PY{p}{(}\PY{l+s+s1}{\PYZsq{}}\PY{l+s+s1}{CGPA \PYZti{} C(SCHOOL\PYZus{}RIGHT)}\PY{l+s+s1}{\PYZsq{}}\PY{p}{,}\PY{n}{data}\PY{o}{=}\PY{n}{Psyco}\PY{p}{)}\PY{o}{.}\PY{n}{fit}\PY{p}{(}\PY{p}{)}
          \PY{n}{aov\PYZus{}table} \PY{o}{=} \PY{n}{sm}\PY{o}{.}\PY{n}{stats}\PY{o}{.}\PY{n}{anova\PYZus{}lm}\PY{p}{(}\PY{n}{moore\PYZus{}lm}\PY{p}{,} \PY{n}{typ}\PY{o}{=}\PY{l+m+mi}{1}\PY{p}{)}
          \PY{n}{aov\PYZus{}table}
\end{Verbatim}

            \begin{Verbatim}[commandchars=\\\{\}]
{\color{outcolor}Out[{\color{outcolor}676}]:}                     df       sum\_sq    mean\_sq         F    PR(>F)
          C(SCHOOL\_RIGHT)  107.0  8727.908041  81.569234  2.070656  0.000061
          Residual         119.0  4687.760534  39.392946       NaN       NaN
\end{Verbatim}
        
    Aucunne correlation n'existe entre l"ecole et le CGPA au vu de la valeur
de PR'

    \begin{Verbatim}[commandchars=\\\{\}]
{\color{incolor}In [{\color{incolor}677}]:} \PY{n}{SchoolGroup}\PY{o}{=}\PY{n}{Psyco}\PY{o}{.}\PY{n}{groupby}\PY{p}{(}\PY{l+s+s1}{\PYZsq{}}\PY{l+s+s1}{SCHOOL\PYZus{}RIGHT}\PY{l+s+s1}{\PYZsq{}}\PY{p}{)}\PY{o}{.}\PY{n}{mean}\PY{p}{(}\PY{p}{)}\PY{o}{.}\PY{n}{sort}\PY{p}{(}\PY{n}{axis}\PY{o}{=}\PY{l+m+mi}{0}\PY{p}{,}\PY{n}{columns}\PY{o}{=}\PY{l+s+s1}{\PYZsq{}}\PY{l+s+s1}{CGPA}\PY{l+s+s1}{\PYZsq{}}\PY{p}{,}\PY{n}{ascending}\PY{o}{=}\PY{n+nb+bp}{False}\PY{p}{)}
\end{Verbatim}

    \begin{Verbatim}[commandchars=\\\{\}]
{\color{incolor}In [{\color{incolor}678}]:} \PY{n}{SchoolGroup}\PY{o}{.}\PY{n}{reset\PYZus{}index}\PY{p}{(}\PY{n}{inplace}\PY{o}{=}\PY{n+nb+bp}{True}\PY{p}{)}
\end{Verbatim}

    \begin{Verbatim}[commandchars=\\\{\}]
{\color{incolor}In [{\color{incolor}679}]:} \PY{n}{SchoolGroup}
\end{Verbatim}

            \begin{Verbatim}[commandchars=\\\{\}]
{\color{outcolor}Out[{\color{outcolor}679}]:}            SCHOOL\_RIGHT            ID    DIPPERC        AGE       CGPA  \textbackslash{}
          0                  wima   8594.000000  66.000000  24.000000  73.725000   
          1                mawato  10701.000000  59.000000  23.000000  71.400002   
          2                 lwiro   9226.000000  57.000000  26.000000  70.750000   
          3      malikia wa bingu   7040.000000  51.000000  41.000000  70.075001   
          4          kanyabayonga   5662.000000  52.000000  25.000000  69.333333   
          5    gs asteria urafiki   8578.000000  65.000000  29.000000  68.950001   
          6           INST LWANGA   3298.000000  61.000000  33.000000  68.949997   
          7              bsangani   7850.000000  55.500000  24.000000  68.924999   
          8                amkeni  10049.000000  51.000000  35.000000  68.849998   
          9                 amani   9881.000000  66.000000  23.500000  68.837501   
          10             kirikiri   6709.000000  52.000000  38.000000  68.800001   
          11               mwanga   9245.000000  63.000000  23.000000  68.683334   
          12               bikuba  10396.000000  51.000000  22.000000  67.950001   
          13              kambali   9109.000000  64.000000  25.000000  67.633334   
          14           de l'unite  12137.000000  61.000000  26.000000  67.599998   
          15            de bagira   5094.000000  56.000000  53.000000  66.699997   
          16               kasali   5079.000000  51.000000  48.000000  66.500000   
          17               Lwanga   9293.000000  58.666667  22.666667  66.491667   
          18              buramba   7034.000000  61.000000  25.000000  66.050001   
          19             kyatenga   5089.000000  57.000000  42.000000  65.900002   
          20             muungano   9721.000000  50.000000  35.000000  65.600002   
          21               cirezi   8117.500000  54.500000  43.500000  65.350000   
          22             bakandja   9249.375000  57.250000  24.125000  64.995833   
          23               kasika   7606.000000  58.666667  24.666667  64.625000   
          24             metanoia   8536.250000  59.437500  24.812500  64.377604   
          25              de goma   5035.000000  50.000000  29.000000  64.099998   
          26               bikuka   5020.000000  51.000000  31.000000  64.000000   
          27           kitsombiro   8323.000000  59.666667  24.333333  63.950001   
          28               asseco  12160.000000  58.000000  25.000000  63.799999   
          29             kishanga   8285.500000  52.500000  44.500000  63.549999   
          ..                  {\ldots}           {\ldots}        {\ldots}        {\ldots}        {\ldots}   
          78              visogho  10168.500000  52.000000  27.500000  55.874999   
          79               mukaba  12037.000000  61.000000  25.000000  55.700001   
          80               faraja   8319.000000  52.800000  28.600000  55.230000   
          81                 luka   5969.500000  57.000000  28.500000  54.749999   
          82    tisiesi/karisimbi  11788.000000  62.000000  27.000000  54.700001   
          83             chemchem   8138.000000  57.000000  30.500000  54.425000   
          84               ndosho  10593.500000  55.000000  25.500000  54.049999   
          85             kilimani  10039.000000  55.000000  22.000000  53.750000   
          86              tujenge   9701.000000  50.000000  43.000000  53.533333   
          87          autodidacte   8311.333333  55.666667  34.000000  53.333333   
          88            maranatha   9475.800000  52.400000  23.800000  53.000001   
          89               totoro  11264.250000  51.750000  23.500000  52.362500   
          90                moria   7123.000000  52.000000  26.000000  51.000000   
          91             kyeshero   7641.666667  51.333333  29.333333  50.694444   
          92     action kusaidiya   9478.000000  54.000000  23.000000  50.150000   
          93             tuungane   9836.000000  50.000000  24.000000  49.549999   
          94            maendeleo   6783.600000  52.400000  29.000000  49.420000   
          95              tisiesi   7027.000000  51.000000  26.000000  48.000000   
          96         tumaini letu   9035.000000  57.000000  26.000000  48.000000   
          97     mikeno islamique  10482.000000  54.000000  22.000000  48.000000   
          98              bimenya   8602.000000  52.000000  29.000000  46.000000   
          99           mululusake  11347.000000  55.000000  22.000000  46.000000   
          100              lukeba   8928.000000  64.000000  24.000000  45.000000   
          101               kando  11967.000000  50.000000  21.000000  44.000000   
          102             kashozi   9378.000000  54.000000  25.000000  44.000000   
          103         bukinanyana   5192.000000  52.000000  36.000000  43.000000   
          104           kalangala  10789.000000  51.000000  22.000000  43.000000   
          105              salamu  10440.000000  54.000000  31.000000  42.000000   
          106            ufamandu   9329.000000  50.000000  43.000000  42.000000   
          107                mehe  10373.000000  57.000000  26.000000  40.000000   
          
               NACADYEAR  
          0     4.000000  
          1     2.000000  
          2     2.000000  
          3     4.000000  
          4     3.000000  
          5     4.000000  
          6     2.000000  
          7     2.000000  
          8     2.000000  
          9     2.500000  
          10    3.000000  
          11    2.800000  
          12    2.000000  
          13    3.000000  
          14    1.000000  
          15    1.000000  
          16    1.000000  
          17    3.333333  
          18    4.000000  
          19    1.000000  
          20    2.000000  
          21    2.000000  
          22    2.125000  
          23    2.333333  
          24    2.062500  
          25    1.000000  
          26    1.000000  
          27    2.000000  
          28    1.000000  
          29    3.000000  
          ..         {\ldots}  
          78    1.500000  
          79    1.000000  
          80    1.200000  
          81    1.500000  
          82    1.000000  
          83    2.000000  
          84    1.500000  
          85    2.000000  
          86    3.000000  
          87    1.000000  
          88    1.400000  
          89    1.250000  
          90    1.000000  
          91    2.000000  
          92    2.000000  
          93    2.000000  
          94    1.800000  
          95    1.000000  
          96    1.000000  
          97    1.000000  
          98    1.000000  
          99    1.000000  
          100   1.000000  
          101   1.000000  
          102   1.000000  
          103   1.000000  
          104   1.000000  
          105   1.000000  
          106   1.000000  
          107   1.000000  
          
          [108 rows x 6 columns]
\end{Verbatim}
        
    Nous allons Voir de plus pret pour les 10 ecoles les plus representé, le
5 éecoles avec un pourcentage elevé et 5 dernier

    \begin{Verbatim}[commandchars=\\\{\}]
{\color{incolor}In [{\color{incolor}680}]:} \PY{n}{plt}\PY{o}{.}\PY{n}{figure}\PY{p}{(}\PY{n}{figsize}\PY{o}{=}\PY{p}{(}\PY{l+m+mi}{30}\PY{p}{,}\PY{l+m+mi}{10}\PY{p}{)}\PY{p}{)}
          \PY{n}{ax}\PY{o}{=}\PY{n}{sns}\PY{o}{.}\PY{n}{boxplot}\PY{p}{(}\PY{n}{x}\PY{o}{=}\PY{l+s+s2}{\PYZdq{}}\PY{l+s+s2}{SCHOOL\PYZus{}RIGHT}\PY{l+s+s2}{\PYZdq{}}\PY{p}{,} \PY{n}{y}\PY{o}{=}\PY{l+s+s2}{\PYZdq{}}\PY{l+s+s2}{CGPA}\PY{l+s+s2}{\PYZdq{}}\PY{p}{,}  \PY{n}{data}\PY{o}{=}\PY{n}{Psyco}\PY{o}{.}\PY{n}{loc}\PY{p}{[}\PY{n}{operator}\PY{o}{.}\PY{n}{or\PYZus{}}\PY{p}{(}\PY{n}{Psyco}\PY{o}{.}\PY{n}{SCHOOL\PYZus{}RIGHT}\PY{o}{.}\PY{n}{isin}\PY{p}{(}\PY{n}{SchoolGroup}\PY{o}{.}\PY{n}{loc}\PY{p}{[} \PY{n+nb}{range}\PY{p}{(}\PY{l+m+mi}{0}\PY{p}{,}\PY{l+m+mi}{11}\PY{p}{)}\PY{p}{]}\PY{o}{.}\PY{n}{SCHOOL\PYZus{}RIGHT}\PY{p}{)} 
                       \PY{p}{,} \PY{n}{Psyco}\PY{o}{.}\PY{n}{SCHOOL\PYZus{}RIGHT}\PY{o}{.}\PY{n}{isin}\PY{p}{(}\PY{n}{Psyco}\PY{o}{.}\PY{n}{SCHOOL\PYZus{}RIGHT}\PY{o}{.}\PY{n}{value\PYZus{}counts}\PY{p}{(}\PY{p}{)}\PY{p}{[}\PY{p}{:}\PY{l+m+mi}{10}\PY{p}{]}\PY{o}{.}\PY{n}{index} \PY{p}{)}\PY{p}{)}\PY{p}{]}\PY{p}{)}
          \PY{n}{plt}\PY{o}{.}\PY{n}{savefig}\PY{p}{(}\PY{l+s+s1}{\PYZsq{}}\PY{l+s+s1}{SCHOOL\PYZus{}RIGHT\PYZhy{}CGPAPsyc.png}\PY{l+s+s1}{\PYZsq{}}\PY{p}{,}\PY{n}{dpi}\PY{o}{=}\PY{l+m+mi}{100}\PY{p}{)}
\end{Verbatim}

    \begin{center}
    \adjustimage{max size={0.9\linewidth}{0.9\paperheight}}{output_383_0.png}
    \end{center}
    { \hspace*{\fill} \\}
    
    Les éecoles comme : bakandja , amani , meta anuarite , ont des bonne
notes au sein de cette faculté

    \paragraph{B.7 OPTION DU DIPLOME}\label{b.7-option-du-diplome}

    \begin{Verbatim}[commandchars=\\\{\}]
{\color{incolor}In [{\color{incolor}684}]:} \PY{n+nb}{len}\PY{p}{(}\PY{n}{Psyco}\PY{o}{.}\PY{n}{OPTION\PYZus{}RIGHT}\PY{o}{.}\PY{n}{value\PYZus{}counts}\PY{p}{(}\PY{p}{)}\PY{p}{)}
\end{Verbatim}

            \begin{Verbatim}[commandchars=\\\{\}]
{\color{outcolor}Out[{\color{outcolor}684}]:} 7
\end{Verbatim}
        
    Nous pouvons constater que nous avons 7 options differents Voyons
comment evoluer la moyenne

    \begin{Verbatim}[commandchars=\\\{\}]
{\color{incolor}In [{\color{incolor}685}]:} \PY{n}{moore\PYZus{}lm} \PY{o}{=} \PY{n}{ols}\PY{p}{(}\PY{l+s+s1}{\PYZsq{}}\PY{l+s+s1}{CGPA \PYZti{} C(OPTION\PYZus{}RIGHT)}\PY{l+s+s1}{\PYZsq{}}\PY{p}{,}\PY{n}{data}\PY{o}{=}\PY{n}{Psyco}\PY{p}{)}\PY{o}{.}\PY{n}{fit}\PY{p}{(}\PY{p}{)}
          \PY{n}{aov\PYZus{}table} \PY{o}{=} \PY{n}{sm}\PY{o}{.}\PY{n}{stats}\PY{o}{.}\PY{n}{anova\PYZus{}lm}\PY{p}{(}\PY{n}{moore\PYZus{}lm}\PY{p}{,} \PY{n}{typ}\PY{o}{=}\PY{l+m+mi}{1}\PY{p}{)}
          \PY{n}{aov\PYZus{}table}
\end{Verbatim}

            \begin{Verbatim}[commandchars=\\\{\}]
{\color{outcolor}Out[{\color{outcolor}685}]:}                     df        sum\_sq     mean\_sq         F    PR(>F)
          C(OPTION\_RIGHT)    6.0   1487.677327  247.946221  4.573123  0.000214
          Residual         220.0  11927.991248   54.218142       NaN       NaN
\end{Verbatim}
        
    La valeur est inferieur à 0.05 donc il n'yas pas de relation entre les
valeurs , certaines écoles on une bonne moyenne du GPA que d'autres donc
elle depend de l'ecole de provenance

    \begin{Verbatim}[commandchars=\\\{\}]
{\color{incolor}In [{\color{incolor}686}]:} \PY{n}{GroupOption}\PY{o}{=}\PY{n}{Psyco}\PY{o}{.}\PY{n}{groupby}\PY{p}{(}\PY{l+s+s1}{\PYZsq{}}\PY{l+s+s1}{OPTION\PYZus{}RIGHT}\PY{l+s+s1}{\PYZsq{}}\PY{p}{)}\PY{o}{.}\PY{n}{mean}\PY{p}{(}\PY{p}{)}\PY{o}{.}\PY{n}{sort}\PY{p}{(}\PY{n}{axis}\PY{o}{=}\PY{l+m+mi}{0}\PY{p}{,}\PY{n}{columns}\PY{o}{=}\PY{l+s+s1}{\PYZsq{}}\PY{l+s+s1}{CGPA}\PY{l+s+s1}{\PYZsq{}}\PY{p}{,}\PY{n}{ascending}\PY{o}{=}\PY{n+nb+bp}{False}\PY{p}{)}\PY{o}{.}\PY{n}{reset\PYZus{}index}\PY{p}{(}\PY{p}{)}
\end{Verbatim}

    \begin{Verbatim}[commandchars=\\\{\}]
{\color{incolor}In [{\color{incolor}687}]:} \PY{n}{GroupOption}
\end{Verbatim}

            \begin{Verbatim}[commandchars=\\\{\}]
{\color{outcolor}Out[{\color{outcolor}687}]:}           OPTION\_RIGHT            ID    DIPPERC        AGE       CGPA  \textbackslash{}
          0           bio-chimie  10142.000000  62.500000  22.750000  69.660418   
          1        math-physique   7364.000000  53.500000  30.000000  64.724999   
          2          latin philo   9708.166667  59.166667  24.416667  64.441667   
          3            pedagogie   8168.439153  56.184532  29.031746  59.440123   
          4        coupe couture   8242.000000  54.000000  29.000000  57.191666   
          5              sociale   9751.937500  53.750000  23.437500  56.557292   
          6  commmerciale et adm  11309.000000  52.000000  21.500000  42.500000   
          
             NACADYEAR  
          0   2.250000  
          1   1.500000  
          2   2.416667  
          3   2.005291  
          4   2.500000  
          5   1.875000  
          6   1.000000  
\end{Verbatim}
        
    contrairement aux idées reussis ceux qui reussisent plus en faculté de
psycho sont les étudiant issues des section sientifique ensuite vient
'option latin philo et en 4eme position l'option pédagaogique

    \begin{Verbatim}[commandchars=\\\{\}]
{\color{incolor}In [{\color{incolor}688}]:} \PY{n}{plt}\PY{o}{.}\PY{n}{figure}\PY{p}{(}\PY{n}{figsize}\PY{o}{=}\PY{p}{(}\PY{l+m+mi}{12}\PY{p}{,}\PY{l+m+mi}{6}\PY{p}{)}\PY{p}{)}
          \PY{n}{ax}\PY{o}{=}\PY{n}{sns}\PY{o}{.}\PY{n}{boxplot}\PY{p}{(}\PY{n}{x}\PY{o}{=}\PY{l+s+s2}{\PYZdq{}}\PY{l+s+s2}{OPTION\PYZus{}RIGHT}\PY{l+s+s2}{\PYZdq{}}\PY{p}{,} \PY{n}{y}\PY{o}{=}\PY{l+s+s2}{\PYZdq{}}\PY{l+s+s2}{CGPA}\PY{l+s+s2}{\PYZdq{}}\PY{p}{,} \PY{n}{data}\PY{o}{=}\PY{n}{Psyco}\PY{p}{,}\PY{n}{order}\PY{o}{=}\PY{n}{GroupOption}\PY{o}{.}\PY{n}{OPTION\PYZus{}RIGHT}\PY{p}{)}
          \PY{n}{plt}\PY{o}{.}\PY{n}{savefig}\PY{p}{(}\PY{l+s+s1}{\PYZsq{}}\PY{l+s+s1}{OPTION\PYZus{}RIGHT\PYZhy{}CGPADroit.png}\PY{l+s+s1}{\PYZsq{}}\PY{p}{,}\PY{n}{dpi}\PY{o}{=}\PY{l+m+mi}{100}\PY{p}{)}
\end{Verbatim}

    \begin{center}
    \adjustimage{max size={0.9\linewidth}{0.9\paperheight}}{output_393_0.png}
    \end{center}
    { \hspace*{\fill} \\}
    
    On peut conclure que les étudiant issues de l'option commerciale et
admin echouent plus au sein de cette faculté/

    \paragraph{Faculté De Theologie}\label{facultuxe9-de-theologie}

    \begin{Verbatim}[commandchars=\\\{\}]
{\color{incolor}In [{\color{incolor}689}]:} \PY{n}{Teologie}\PY{o}{=}\PY{n}{datasetFin}\PY{o}{.}\PY{n}{loc}\PY{p}{[}\PY{n}{datasetFin}\PY{o}{.}\PY{n}{FAC}\PY{o}{==}\PY{l+s+s1}{\PYZsq{}}\PY{l+s+s1}{FT}\PY{l+s+s1}{\PYZsq{}}\PY{p}{]}
\end{Verbatim}

    \begin{Verbatim}[commandchars=\\\{\}]
{\color{incolor}In [{\color{incolor}690}]:} \PY{n}{Teologie}\PY{o}{.}\PY{n}{shape}
\end{Verbatim}

            \begin{Verbatim}[commandchars=\\\{\}]
{\color{outcolor}Out[{\color{outcolor}690}]:} (140, 14)
\end{Verbatim}
        
    Nous voici enfin au sein de la faculté de theologie ! Commencons par
voir la répartition des probabilitées de notre variable CGPA

    https://www.wellbeingatschool.org.nz/information-sheet/understanding-and-interpreting-box-plots

    A.0 Distribution du CGPA

    \begin{Verbatim}[commandchars=\\\{\}]
{\color{incolor}In [{\color{incolor}691}]:} \PY{n}{plt}\PY{o}{.}\PY{n}{figure}\PY{p}{(}\PY{p}{)}
          \PY{n}{ax} \PY{o}{=} \PY{n}{sns}\PY{o}{.}\PY{n}{boxplot}\PY{p}{(}\PY{n}{x}\PY{o}{=}\PY{n}{Teologie}\PY{p}{[}\PY{l+s+s1}{\PYZsq{}}\PY{l+s+s1}{CGPA}\PY{l+s+s1}{\PYZsq{}}\PY{p}{]}\PY{p}{)}
          \PY{n}{plt}\PY{o}{.}\PY{n}{savefig}\PY{p}{(}\PY{l+s+s1}{\PYZsq{}}\PY{l+s+s1}{CGPA\PYZhy{}ECO.png}\PY{l+s+s1}{\PYZsq{}}\PY{p}{,}\PY{n}{dpi}\PY{o}{=}\PY{l+m+mi}{100}\PY{p}{)}
\end{Verbatim}

    \begin{center}
    \adjustimage{max size={0.9\linewidth}{0.9\paperheight}}{output_401_0.png}
    \end{center}
    { \hspace*{\fill} \\}
    
    \begin{Verbatim}[commandchars=\\\{\}]
{\color{incolor}In [{\color{incolor}692}]:} \PY{n}{plt}\PY{o}{.}\PY{n}{figure}\PY{p}{(}\PY{p}{)}
          \PY{n}{sns}\PY{o}{.}\PY{n}{distplot}\PY{p}{(}\PY{n}{Teologie}\PY{p}{[}\PY{l+s+s1}{\PYZsq{}}\PY{l+s+s1}{CGPA}\PY{l+s+s1}{\PYZsq{}}\PY{p}{]}\PY{p}{,}\PY{n}{bins}\PY{o}{=}\PY{l+m+mi}{20}\PY{p}{,}\PY{n}{axlabel}\PY{o}{=}\PY{l+s+s1}{\PYZsq{}}\PY{l+s+s1}{CGPA}\PY{l+s+s1}{\PYZsq{}}\PY{p}{,}\PY{n}{kde}\PY{o}{=}\PY{l+m+mi}{1}\PY{p}{,}\PY{n}{norm\PYZus{}hist}\PY{o}{=}\PY{l+m+mi}{0}\PY{p}{)}
          \PY{n}{plt}\PY{o}{.}\PY{n}{savefig}\PY{p}{(}\PY{l+s+s1}{\PYZsq{}}\PY{l+s+s1}{CGPADistteo.png}\PY{l+s+s1}{\PYZsq{}}\PY{p}{,}\PY{n}{dpi}\PY{o}{=}\PY{l+m+mi}{100}\PY{p}{)}
\end{Verbatim}

    \begin{center}
    \adjustimage{max size={0.9\linewidth}{0.9\paperheight}}{output_402_0.png}
    \end{center}
    { \hspace*{\fill} \\}
    
    Il suit une distribution presque normale d'écart type : 6.23

    \begin{Verbatim}[commandchars=\\\{\}]
{\color{incolor}In [{\color{incolor}693}]:} \PY{n}{np}\PY{o}{.}\PY{n}{std}\PY{p}{(}\PY{n}{Teologie}\PY{o}{.}\PY{n}{CGPA}\PY{p}{)}
\end{Verbatim}

            \begin{Verbatim}[commandchars=\\\{\}]
{\color{outcolor}Out[{\color{outcolor}693}]:} 6.2067599618743134
\end{Verbatim}
        
    Commencons par les attribues numeriques et analysons la correlation
chercher comme sur le point précedant la correlation avec le CGPA

    \begin{Verbatim}[commandchars=\\\{\}]
{\color{incolor}In [{\color{incolor}694}]:} \PY{n}{Teologie}\PY{o}{.}\PY{n}{corr}\PY{p}{(}\PY{p}{)}
\end{Verbatim}

            \begin{Verbatim}[commandchars=\\\{\}]
{\color{outcolor}Out[{\color{outcolor}694}]:}                  ID   DIPPERC       AGE      CGPA  NACADYEAR
          ID         1.000000  0.068864 -0.536884 -0.306211  -0.071491
          DIPPERC    0.068864  1.000000 -0.176726  0.141674   0.186177
          AGE       -0.536884 -0.176726  1.000000  0.148100  -0.089653
          CGPA      -0.306211  0.141674  0.148100  1.000000   0.282592
          NACADYEAR -0.071491  0.186177 -0.089653  0.282592   1.000000
\end{Verbatim}
        
    Nous pouvons aisement qu'il n'yas aucune correlation entre le le CGPA et
le pourcentage à l'exetat ni meme l'age des étudiants

    \paragraph{C.2 GENDER}\label{c.2-gender}

    \begin{Verbatim}[commandchars=\\\{\}]
{\color{incolor}In [{\color{incolor}695}]:} \PY{n}{moore\PYZus{}lm} \PY{o}{=} \PY{n}{ols}\PY{p}{(}\PY{l+s+s1}{\PYZsq{}}\PY{l+s+s1}{CGPA \PYZti{} C(GENDER)}\PY{l+s+s1}{\PYZsq{}}\PY{p}{,}\PY{n}{data}\PY{o}{=}\PY{n}{Teologie}\PY{p}{)}\PY{o}{.}\PY{n}{fit}\PY{p}{(}\PY{p}{)}
          \PY{n}{aov\PYZus{}table} \PY{o}{=} \PY{n}{sm}\PY{o}{.}\PY{n}{stats}\PY{o}{.}\PY{n}{anova\PYZus{}lm}\PY{p}{(}\PY{n}{moore\PYZus{}lm}\PY{p}{,} \PY{n}{typ}\PY{o}{=}\PY{l+m+mi}{1}\PY{p}{)}
          \PY{n}{aov\PYZus{}table}
\end{Verbatim}

            \begin{Verbatim}[commandchars=\\\{\}]
{\color{outcolor}Out[{\color{outcolor}695}]:}               df       sum\_sq    mean\_sq         F    PR(>F)
          C(GENDER)    1.0     4.122392   4.122392  0.105561  0.745748
          Residual   138.0  5389.219299  39.052314       NaN       NaN
\end{Verbatim}
        
    Notre metrique nous montre que la moyenne de CGPA est la meme pour les
deux sexes ! jettons un coup d'oeil aux distribution avec des box-plot.

    \begin{Verbatim}[commandchars=\\\{\}]
{\color{incolor}In [{\color{incolor}696}]:} \PY{n}{plt}\PY{o}{.}\PY{n}{figure}\PY{p}{(}\PY{n}{figsize}\PY{o}{=}\PY{p}{(}\PY{l+m+mi}{12}\PY{p}{,}\PY{l+m+mi}{6}\PY{p}{)}\PY{p}{)}
          \PY{n}{ax}\PY{o}{=}\PY{n}{sns}\PY{o}{.}\PY{n}{boxplot}\PY{p}{(}\PY{n}{x}\PY{o}{=}\PY{l+s+s2}{\PYZdq{}}\PY{l+s+s2}{GENDER}\PY{l+s+s2}{\PYZdq{}}\PY{p}{,} \PY{n}{y}\PY{o}{=}\PY{l+s+s2}{\PYZdq{}}\PY{l+s+s2}{CGPA}\PY{l+s+s2}{\PYZdq{}}\PY{p}{,} \PY{n}{data}\PY{o}{=}\PY{n}{Teologie}\PY{p}{)}
          \PY{n}{plt}\PY{o}{.}\PY{n}{savefig}\PY{p}{(}\PY{l+s+s1}{\PYZsq{}}\PY{l+s+s1}{GENDER\PYZhy{}CGPA\PYZhy{}Teologie.png}\PY{l+s+s1}{\PYZsq{}}\PY{p}{,}\PY{n}{dpi}\PY{o}{=}\PY{l+m+mi}{100}\PY{p}{)}
\end{Verbatim}

    \begin{center}
    \adjustimage{max size={0.9\linewidth}{0.9\paperheight}}{output_411_0.png}
    \end{center}
    { \hspace*{\fill} \\}
    
    \begin{Verbatim}[commandchars=\\\{\}]
{\color{incolor}In [{\color{incolor}697}]:} \PY{n}{Teologie}\PY{o}{.}\PY{n}{GENDER}\PY{o}{.}\PY{n}{value\PYZus{}counts}\PY{p}{(}\PY{p}{)}\PY{o}{/}\PY{n+nb}{len}\PY{p}{(}\PY{n}{Teologie}\PY{p}{)}
\end{Verbatim}

            \begin{Verbatim}[commandchars=\\\{\}]
{\color{outcolor}Out[{\color{outcolor}697}]:} H    0.821429
          F    0.178571
          Name: GENDER, dtype: float64
\end{Verbatim}
        
    il ya un desiquilibre car le femmes c'est uniquemen 20\% des étudiant
ausein de la faculté

    \begin{Verbatim}[commandchars=\\\{\}]
{\color{incolor}In [{\color{incolor}698}]:} \PY{n}{Teologie}\PY{o}{.}\PY{n}{groupby}\PY{p}{(}\PY{l+s+s1}{\PYZsq{}}\PY{l+s+s1}{GENDER}\PY{l+s+s1}{\PYZsq{}}\PY{p}{)}\PY{o}{.}\PY{n}{mean}\PY{p}{(}\PY{p}{)}
\end{Verbatim}

            \begin{Verbatim}[commandchars=\\\{\}]
{\color{outcolor}Out[{\color{outcolor}698}]:}                 ID    DIPPERC   AGE       CGPA  NACADYEAR
          GENDER                                                   
          F       7894.28000  53.080000  28.8  61.631667   2.000000
          H       8175.46087  53.973913  32.0  62.079710   2.104348
\end{Verbatim}
        
    Mais on peut remarquer que les moyenees restent la meme

    \paragraph{B.4 Attribue Province}\label{b.4-attribue-province}

    \begin{Verbatim}[commandchars=\\\{\}]
{\color{incolor}In [{\color{incolor}467}]:} \PY{n}{Techno}\PY{o}{.}\PY{n}{columns}
\end{Verbatim}

            \begin{Verbatim}[commandchars=\\\{\}]
{\color{outcolor}Out[{\color{outcolor}467}]:} Index([u'ID', u'SCHOOLSTATUS', u'SCHOOL\_RIGHT', u'OPTION\_RIGHT', u'FAC',
                 u'SCHOOLPROVINCE', u'GENDER', u'DIPPERC', u'AGE', u'CGPA',
                 u'DistinctionRatio', u'EchecRatio', u'NACADYEAR',
                 u'Pass1stSessionRatio'],
                dtype='object')
\end{Verbatim}
        
    \begin{Verbatim}[commandchars=\\\{\}]
{\color{incolor}In [{\color{incolor}699}]:} \PY{n}{moore\PYZus{}lm} \PY{o}{=} \PY{n}{ols}\PY{p}{(}\PY{l+s+s1}{\PYZsq{}}\PY{l+s+s1}{CGPA \PYZti{} C(SCHOOLPROVINCE)}\PY{l+s+s1}{\PYZsq{}}\PY{p}{,}\PY{n}{data}\PY{o}{=}\PY{n}{Teologie}\PY{p}{)}\PY{o}{.}\PY{n}{fit}\PY{p}{(}\PY{p}{)}
          \PY{n}{aov\PYZus{}table} \PY{o}{=} \PY{n}{sm}\PY{o}{.}\PY{n}{stats}\PY{o}{.}\PY{n}{anova\PYZus{}lm}\PY{p}{(}\PY{n}{moore\PYZus{}lm}\PY{p}{,} \PY{n}{typ}\PY{o}{=}\PY{l+m+mi}{1}\PY{p}{)}
          \PY{n}{aov\PYZus{}table}
\end{Verbatim}

            \begin{Verbatim}[commandchars=\\\{\}]
{\color{outcolor}Out[{\color{outcolor}699}]:}                       df       sum\_sq    mean\_sq         F    PR(>F)
          C(SCHOOLPROVINCE)    5.0   462.025874  92.405175  2.510951  0.033001
          Residual           134.0  4931.315817  36.800864       NaN       NaN
\end{Verbatim}
        
    PR =0.078 est superieur à 0.05 nous pouvons conclure que la moyenne est
la meme sur toutes les facultés

    \begin{Verbatim}[commandchars=\\\{\}]
{\color{incolor}In [{\color{incolor}700}]:} \PY{n}{TeolGroup}\PY{o}{=}\PY{n}{Teologie}\PY{o}{.}\PY{n}{groupby}\PY{p}{(}\PY{l+s+s1}{\PYZsq{}}\PY{l+s+s1}{SCHOOLPROVINCE}\PY{l+s+s1}{\PYZsq{}}\PY{p}{)}\PY{o}{.}\PY{n}{mean}\PY{p}{(}\PY{p}{)}\PY{o}{.}\PY{n}{sort}\PY{p}{(}\PY{n}{axis}\PY{o}{=}\PY{l+m+mi}{0}\PY{p}{,}\PY{n}{columns}\PY{o}{=}\PY{l+s+s1}{\PYZsq{}}\PY{l+s+s1}{CGPA}\PY{l+s+s1}{\PYZsq{}}\PY{p}{,}\PY{n}{ascending}\PY{o}{=}\PY{n+nb+bp}{False}\PY{p}{)}
\end{Verbatim}

    \begin{Verbatim}[commandchars=\\\{\}]
{\color{incolor}In [{\color{incolor}701}]:} \PY{n}{TeolGroup}\PY{o}{.}\PY{n}{reset\PYZus{}index}\PY{p}{(}\PY{n}{inplace}\PY{o}{=}\PY{n+nb+bp}{True}
                               \PY{p}{)}
\end{Verbatim}

    \begin{Verbatim}[commandchars=\\\{\}]
{\color{incolor}In [{\color{incolor}702}]:} \PY{n}{TeolGroup}
\end{Verbatim}

            \begin{Verbatim}[commandchars=\\\{\}]
{\color{outcolor}Out[{\color{outcolor}702}]:}   SCHOOLPROVINCE            ID    DIPPERC        AGE       CGPA  NACADYEAR
          0      ORIENTALE   9313.666667  57.000000  48.333333  62.633333   2.000000
          1      NORD-KIVU   8148.727273  53.504132  30.809917  62.231061   2.090909
          2       SUD-KIVU   7340.307692  55.538462  33.461538  61.689744   2.000000
          3       KINSHASA   8576.000000  65.000000  25.000000  59.933333   3.000000
          4     NYARUKENGE   7508.000000  51.000000  31.000000  59.200001   3.000000
          5        MANIEMA  12090.000000  51.000000  36.000000  41.000000   1.000000
\end{Verbatim}
        
    Nous remarquons que la moyenne de GPA est superieur Pour la province
orientale , ensuite vienne la province du Nord kivu et la province du
Sud Kivu

    \begin{Verbatim}[commandchars=\\\{\}]
{\color{incolor}In [{\color{incolor}704}]:} \PY{n}{plt}\PY{o}{.}\PY{n}{figure}\PY{p}{(}\PY{n}{figsize}\PY{o}{=}\PY{p}{(}\PY{l+m+mi}{12}\PY{p}{,}\PY{l+m+mi}{6}\PY{p}{)}\PY{p}{)}
          \PY{n}{ax}\PY{o}{=}\PY{n}{sns}\PY{o}{.}\PY{n}{boxplot}\PY{p}{(}\PY{n}{x}\PY{o}{=}\PY{l+s+s2}{\PYZdq{}}\PY{l+s+s2}{SCHOOLPROVINCE}\PY{l+s+s2}{\PYZdq{}}\PY{p}{,} \PY{n}{y}\PY{o}{=}\PY{l+s+s2}{\PYZdq{}}\PY{l+s+s2}{CGPA}\PY{l+s+s2}{\PYZdq{}}\PY{p}{,} \PY{n}{data}\PY{o}{=}\PY{n}{Teologie}\PY{p}{,}\PY{n}{order}\PY{o}{=}\PY{n}{TeolGroup}\PY{o}{.}\PY{n}{SCHOOLPROVINCE}\PY{p}{)}
          \PY{n}{plt}\PY{o}{.}\PY{n}{savefig}\PY{p}{(}\PY{l+s+s1}{\PYZsq{}}\PY{l+s+s1}{GENDER\PYZhy{}SCHOOLPROVINCE\PYZhy{}TeO.png}\PY{l+s+s1}{\PYZsq{}}\PY{p}{,}\PY{n}{dpi}\PY{o}{=}\PY{l+m+mi}{100}\PY{p}{)}
\end{Verbatim}

    \begin{center}
    \adjustimage{max size={0.9\linewidth}{0.9\paperheight}}{output_424_0.png}
    \end{center}
    { \hspace*{\fill} \\}
    
    \begin{Verbatim}[commandchars=\\\{\}]
{\color{incolor}In [{\color{incolor} }]:} 
\end{Verbatim}

    \paragraph{B.5 Stattus de L'école}\label{b.5-stattus-de-luxe9cole}

    \begin{Verbatim}[commandchars=\\\{\}]
{\color{incolor}In [{\color{incolor}706}]:} \PY{n}{moore\PYZus{}lm} \PY{o}{=} \PY{n}{ols}\PY{p}{(}\PY{l+s+s1}{\PYZsq{}}\PY{l+s+s1}{CGPA \PYZti{} C(SCHOOLSTATUS)}\PY{l+s+s1}{\PYZsq{}}\PY{p}{,}\PY{n}{data}\PY{o}{=}\PY{n}{Teologie}\PY{p}{)}\PY{o}{.}\PY{n}{fit}\PY{p}{(}\PY{p}{)}
          \PY{n}{aov\PYZus{}table} \PY{o}{=} \PY{n}{sm}\PY{o}{.}\PY{n}{stats}\PY{o}{.}\PY{n}{anova\PYZus{}lm}\PY{p}{(}\PY{n}{moore\PYZus{}lm}\PY{p}{,} \PY{n}{typ}\PY{o}{=}\PY{l+m+mi}{1}\PY{p}{)}
          \PY{n}{aov\PYZus{}table}
\end{Verbatim}

            \begin{Verbatim}[commandchars=\\\{\}]
{\color{outcolor}Out[{\color{outcolor}706}]:}                     df       sum\_sq    mean\_sq         F    PR(>F)
          C(SCHOOLSTATUS)    5.0   113.697185  22.739437  0.577138  0.717429
          Residual         134.0  5279.644506  39.400332       NaN       NaN
\end{Verbatim}
        
    les moyennes sont le meme

    \begin{Verbatim}[commandchars=\\\{\}]
{\color{incolor}In [{\color{incolor}707}]:} \PY{n}{Teologie}\PY{o}{.}\PY{n}{groupby}\PY{p}{(}\PY{l+s+s1}{\PYZsq{}}\PY{l+s+s1}{SCHOOLSTATUS}\PY{l+s+s1}{\PYZsq{}}\PY{p}{)}\PY{o}{.}\PY{n}{mean}\PY{p}{(}\PY{p}{)}\PY{o}{.}\PY{n}{sort}\PY{p}{(}\PY{n}{axis}\PY{o}{=}\PY{l+m+mi}{0}\PY{p}{,}\PY{n}{columns}\PY{o}{=}\PY{l+s+s1}{\PYZsq{}}\PY{l+s+s1}{CGPA}\PY{l+s+s1}{\PYZsq{}}\PY{p}{,}\PY{n}{ascending}\PY{o}{=}\PY{n+nb+bp}{False}\PY{p}{)}
\end{Verbatim}

            \begin{Verbatim}[commandchars=\\\{\}]
{\color{outcolor}Out[{\color{outcolor}707}]:}                         ID    DIPPERC        AGE       CGPA  NACADYEAR
          SCHOOLSTATUS                                                          
          publique       7832.884615  53.884615  33.038462  63.456731   1.961538
          privé          9352.571429  55.714286  34.142857  63.289286   2.285714
          catholique    10692.000000  51.333333  23.666667  63.133334   2.000000
          protestant     8548.438596  54.245614  29.245614  61.890790   2.070175
          inconnu        7520.434783  53.195652  33.021739  61.082246   2.173913
          autodidacte    3135.000000  50.000000  45.000000  60.099998   1.000000
\end{Verbatim}
        
    contrirement aux idéées reussi ceux provennat des écoles publiques ,
privées et protestant sont les mieux representé

    Voyons cela de plus pret avec box-plot

    \begin{Verbatim}[commandchars=\\\{\}]
{\color{incolor}In [{\color{incolor}708}]:} \PY{n}{plt}\PY{o}{.}\PY{n}{figure}\PY{p}{(}\PY{n}{figsize}\PY{o}{=}\PY{p}{(}\PY{l+m+mi}{12}\PY{p}{,}\PY{l+m+mi}{6}\PY{p}{)}\PY{p}{)}
          \PY{n}{ax}\PY{o}{=}\PY{n}{sns}\PY{o}{.}\PY{n}{boxplot}\PY{p}{(}\PY{n}{x}\PY{o}{=}\PY{l+s+s2}{\PYZdq{}}\PY{l+s+s2}{SCHOOLSTATUS}\PY{l+s+s2}{\PYZdq{}}\PY{p}{,} \PY{n}{y}\PY{o}{=}\PY{l+s+s2}{\PYZdq{}}\PY{l+s+s2}{CGPA}\PY{l+s+s2}{\PYZdq{}}\PY{p}{,} \PY{n}{data}\PY{o}{=}\PY{n}{Teologie}\PY{p}{)}
          \PY{n}{plt}\PY{o}{.}\PY{n}{savefig}\PY{p}{(}\PY{l+s+s1}{\PYZsq{}}\PY{l+s+s1}{\PYZhy{}SCHOOLSTATUS\PYZhy{}ECO.png}\PY{l+s+s1}{\PYZsq{}}\PY{p}{,}\PY{n}{dpi}\PY{o}{=}\PY{l+m+mi}{100}\PY{p}{)}
\end{Verbatim}

    \begin{center}
    \adjustimage{max size={0.9\linewidth}{0.9\paperheight}}{output_432_0.png}
    \end{center}
    { \hspace*{\fill} \\}
    
    que ceux qui echouent sont ceux provenant des écoles publiques et
protestants

    \paragraph{B.Les Ecoles de
provenances}\label{b.les-ecoles-de-provenances}

    \begin{Verbatim}[commandchars=\\\{\}]
{\color{incolor}In [{\color{incolor}709}]:} \PY{n}{moore\PYZus{}lm} \PY{o}{=} \PY{n}{ols}\PY{p}{(}\PY{l+s+s1}{\PYZsq{}}\PY{l+s+s1}{CGPA \PYZti{} C(SCHOOL\PYZus{}RIGHT)}\PY{l+s+s1}{\PYZsq{}}\PY{p}{,}\PY{n}{data}\PY{o}{=}\PY{n}{Teologie}\PY{p}{)}\PY{o}{.}\PY{n}{fit}\PY{p}{(}\PY{p}{)}
          \PY{n}{aov\PYZus{}table} \PY{o}{=} \PY{n}{sm}\PY{o}{.}\PY{n}{stats}\PY{o}{.}\PY{n}{anova\PYZus{}lm}\PY{p}{(}\PY{n}{moore\PYZus{}lm}\PY{p}{,} \PY{n}{typ}\PY{o}{=}\PY{l+m+mi}{1}\PY{p}{)}
          \PY{n}{aov\PYZus{}table}
\end{Verbatim}

            \begin{Verbatim}[commandchars=\\\{\}]
{\color{outcolor}Out[{\color{outcolor}709}]:}                    df       sum\_sq    mean\_sq         F    PR(>F)
          C(SCHOOL\_RIGHT)  89.0  3765.774777  42.312076  1.299857  0.156736
          Residual         50.0  1627.566915  32.551338       NaN       NaN
\end{Verbatim}
        
    Aucunne correlation n'existe entre l"ecole et le CGPA au vu de la valeur
de PR'

    \begin{Verbatim}[commandchars=\\\{\}]
{\color{incolor}In [{\color{incolor}710}]:} \PY{n}{SchoolGroup}\PY{o}{=}\PY{n}{Teologie}\PY{o}{.}\PY{n}{groupby}\PY{p}{(}\PY{l+s+s1}{\PYZsq{}}\PY{l+s+s1}{SCHOOL\PYZus{}RIGHT}\PY{l+s+s1}{\PYZsq{}}\PY{p}{)}\PY{o}{.}\PY{n}{mean}\PY{p}{(}\PY{p}{)}\PY{o}{.}\PY{n}{sort}\PY{p}{(}\PY{n}{axis}\PY{o}{=}\PY{l+m+mi}{0}\PY{p}{,}\PY{n}{columns}\PY{o}{=}\PY{l+s+s1}{\PYZsq{}}\PY{l+s+s1}{CGPA}\PY{l+s+s1}{\PYZsq{}}\PY{p}{,}\PY{n}{ascending}\PY{o}{=}\PY{n+nb+bp}{False}\PY{p}{)}
\end{Verbatim}

    \begin{Verbatim}[commandchars=\\\{\}]
{\color{incolor}In [{\color{incolor}711}]:} \PY{n}{SchoolGroup}\PY{o}{.}\PY{n}{reset\PYZus{}index}\PY{p}{(}\PY{n}{inplace}\PY{o}{=}\PY{n+nb+bp}{True}\PY{p}{)}
\end{Verbatim}

    \begin{Verbatim}[commandchars=\\\{\}]
{\color{incolor}In [{\color{incolor}712}]:} \PY{n}{SchoolGroup}
\end{Verbatim}

            \begin{Verbatim}[commandchars=\\\{\}]
{\color{outcolor}Out[{\color{outcolor}712}]:}          SCHOOL\_RIGHT       ID    DIPPERC   AGE       CGPA  NACADYEAR
          0            anuarite  11784.0  52.000000  19.0  73.400002   1.000000
          1              vuhika   5758.0  52.000000  27.0  72.399998   2.000000
          2             ndahura   5178.0  50.000000  43.0  71.533333   3.000000
          3              boyulu   8490.0  51.000000  46.0  71.399998   2.000000
          4           mikeno nc   4149.0  62.000000  39.0  71.300003   1.000000
          5               uhuru   8676.0  58.000000  24.0  70.775002   4.000000
          6            metanoia   4642.0  58.000000  29.0  69.450001   2.000000
          7             vikanzu   5746.0  57.000000  39.0  69.266668   3.000000
          8           mgr guido   5747.0  51.000000  39.0  68.733332   3.000000
          9   Institut de KATWA  11528.0  55.000000  26.0  67.900002   1.000000
          10             muhini   7047.0  51.000000  33.0  67.800003   1.000000
          11           kyatenga   6756.0  55.000000  37.0  67.600001   3.000000
          12            bulumbi   5268.5  56.000000  28.5  67.525001   2.500000
          13            majengo   6965.0  54.666667  29.0  67.508334   3.000000
          14          maendeleo   5217.0  52.000000  51.0  66.599998   2.000000
          15            chidasa   6613.5  58.000000  26.5  66.316667   2.500000
          16              bashu   3647.0  52.000000  42.0  65.800003   1.000000
          17           kasalala  12086.0  50.000000  26.0  65.800003   1.000000
          18             mukaba   7280.0  58.500000  37.0  65.475001   2.500000
          19             mabula   4643.0  53.000000  28.0  65.400002   2.000000
          20              kasuo   4637.0  50.000000  39.0  65.399998   2.000000
          21              ngoma   8457.8  53.800000  34.6  64.953333   2.600000
          22            rwabika   4057.0  52.000000  45.0  64.800003   1.000000
          23             bulera   7635.0  53.000000  31.6  64.693333   3.200000
          24           kashenda   4709.5  50.500000  28.5  64.600000   1.000000
          25           NDEREMBI   7049.0  54.000000  27.0  64.533332   3.000000
          26          kantundwe  10739.0  50.000000  28.0  64.450001   2.000000
          27               auto   4923.0  51.000000  49.5  64.425001   2.000000
          28            butembo  11729.0  60.000000  25.0  64.400002   1.000000
          29            kalungu   7670.2  54.600000  29.0  64.280000   2.800000
          ..                {\ldots}      {\ldots}        {\ldots}   {\ldots}        {\ldots}        {\ldots}
          60             bokolo   8576.0  65.000000  25.0  59.933333   3.000000
          61         mululusake   8692.0  52.000000  48.0  59.933333   3.000000
          62             salama   8607.0  53.000000  27.0  59.733334   3.000000
          63            buramba   6886.0  58.000000  30.0  59.500000   2.000000
          64             kasika   7508.0  51.000000  31.0  59.200001   3.000000
          65            lukweti  10903.0  54.000000  29.0  59.000000   1.000000
          66      aigle de dieu  10404.5  53.000000  26.5  58.950001   1.500000
          67             amkeni   7727.0  53.000000  31.5  58.774999   2.000000
          68             masapi   9273.0  51.000000  30.0  58.650002   2.000000
          69            kirumba  10839.0  51.000000  28.0  58.600000   2.000000
          70           kishanga   9614.5  56.000000  29.0  58.497222   2.166667
          71             esengo   8915.0  51.000000  30.0  58.333335   3.000000
          72        st augustin   9502.0  50.000000  31.0  58.000001   3.000000
          73             jikaze   9359.0  54.000000  26.0  57.733334   3.000000
          74           kirikiri   9955.0  50.666667  31.0  56.750001   1.333333
          75        pain de vie   9588.0  51.000000  29.0  55.400000   3.000000
          76               isea   8926.0  62.000000  43.0  55.200001   2.000000
          77          gs kigali   9534.0  51.000000  22.0  54.799999   1.000000
          78             kiruli   9147.0  50.000000  25.0  54.500000   1.000000
          79              amina  12041.0  56.000000  21.0  54.099998   1.000000
          80            kitunda   9058.0  60.000000  28.0  53.700001   1.000000
          81      imaki/kirumba   9536.0  52.000000  34.0  53.400000   3.000000
          82        bukinanyana   3378.0  52.500000  33.0  52.566667   2.000000
          83          bweremana   9691.0  54.000000  27.0  51.799999   2.000000
          84             bikuka   9245.0  59.000000  26.0  51.599998   1.000000
          85               zako  10393.0  50.000000  23.0  50.100000   2.000000
          86         de kirumba  12237.0  50.000000  29.0  49.400002   1.000000
          87            kigonza  12230.0  51.000000  24.0  49.000000   1.000000
          88             totoro   9346.0  52.000000  30.0  43.000000   1.000000
          89               aero  12090.0  51.000000  36.0  41.000000   1.000000
          
          [90 rows x 6 columns]
\end{Verbatim}
        
    Nous allons Voir de plus pret pour les 10 ecoles les plus representé, le
5 éecoles avec un pourcentage elevé et 5 dernier

    \begin{Verbatim}[commandchars=\\\{\}]
{\color{incolor}In [{\color{incolor}714}]:} \PY{n}{plt}\PY{o}{.}\PY{n}{figure}\PY{p}{(}\PY{n}{figsize}\PY{o}{=}\PY{p}{(}\PY{l+m+mi}{30}\PY{p}{,}\PY{l+m+mi}{10}\PY{p}{)}\PY{p}{)}
          \PY{n}{ax}\PY{o}{=}\PY{n}{sns}\PY{o}{.}\PY{n}{boxplot}\PY{p}{(}\PY{n}{x}\PY{o}{=}\PY{l+s+s2}{\PYZdq{}}\PY{l+s+s2}{SCHOOL\PYZus{}RIGHT}\PY{l+s+s2}{\PYZdq{}}\PY{p}{,} \PY{n}{y}\PY{o}{=}\PY{l+s+s2}{\PYZdq{}}\PY{l+s+s2}{CGPA}\PY{l+s+s2}{\PYZdq{}}\PY{p}{,}  \PY{n}{data}\PY{o}{=}\PY{n}{Teologie}\PY{o}{.}\PY{n}{loc}\PY{p}{[}\PY{n}{operator}\PY{o}{.}\PY{n}{or\PYZus{}}\PY{p}{(}\PY{n}{Teologie}\PY{o}{.}\PY{n}{SCHOOL\PYZus{}RIGHT}\PY{o}{.}\PY{n}{isin}\PY{p}{(}\PY{n}{SchoolGroup}\PY{o}{.}\PY{n}{loc}\PY{p}{[} \PY{n+nb}{range}\PY{p}{(}\PY{l+m+mi}{0}\PY{p}{,}\PY{l+m+mi}{8}\PY{p}{)} \PY{o}{+}\PY{n+nb}{range}\PY{p}{(}\PY{l+m+mi}{83}\PY{p}{,}\PY{l+m+mi}{90}\PY{p}{)}\PY{p}{]}\PY{o}{.}\PY{n}{SCHOOL\PYZus{}RIGHT}\PY{p}{)} 
                       \PY{p}{,}\PY{n}{Teologie}\PY{o}{.}\PY{n}{SCHOOL\PYZus{}RIGHT}\PY{o}{.}\PY{n}{isin}\PY{p}{(}\PY{n}{Teologie}\PY{o}{.}\PY{n}{SCHOOL\PYZus{}RIGHT}\PY{o}{.}\PY{n}{value\PYZus{}counts}\PY{p}{(}\PY{p}{)}\PY{p}{[}\PY{p}{:}\PY{l+m+mi}{8}\PY{p}{]}\PY{o}{.}\PY{n}{index} \PY{p}{)}\PY{p}{)}\PY{p}{]}\PY{p}{)}
          \PY{n}{plt}\PY{o}{.}\PY{n}{savefig}\PY{p}{(}\PY{l+s+s1}{\PYZsq{}}\PY{l+s+s1}{SCHOOL\PYZus{}RIGHT\PYZhy{}CGPATeo.png}\PY{l+s+s1}{\PYZsq{}}\PY{p}{,}\PY{n}{dpi}\PY{o}{=}\PY{l+m+mi}{100}\PY{p}{)}
\end{Verbatim}

    \begin{center}
    \adjustimage{max size={0.9\linewidth}{0.9\paperheight}}{output_441_0.png}
    \end{center}
    { \hspace*{\fill} \\}
    
    certain ecoles éechoulent lementablement d'autres reussisent'

    \paragraph{B.7 OPTION DU DIPLOME}\label{b.7-option-du-diplome}

    \begin{Verbatim}[commandchars=\\\{\}]
{\color{incolor}In [{\color{incolor}715}]:} \PY{n+nb}{len}\PY{p}{(}\PY{n}{Teologie}\PY{o}{.}\PY{n}{OPTION\PYZus{}RIGHT}\PY{o}{.}\PY{n}{value\PYZus{}counts}\PY{p}{(}\PY{p}{)}\PY{p}{)}
\end{Verbatim}

            \begin{Verbatim}[commandchars=\\\{\}]
{\color{outcolor}Out[{\color{outcolor}715}]:} 11
\end{Verbatim}
        
    Nous pouvons constater que nous avons 11 options differents Voyons
comment evoluer la moyenne

    \begin{Verbatim}[commandchars=\\\{\}]
{\color{incolor}In [{\color{incolor}716}]:} \PY{n}{moore\PYZus{}lm} \PY{o}{=} \PY{n}{ols}\PY{p}{(}\PY{l+s+s1}{\PYZsq{}}\PY{l+s+s1}{CGPA \PYZti{} C(OPTION\PYZus{}RIGHT)}\PY{l+s+s1}{\PYZsq{}}\PY{p}{,}\PY{n}{data}\PY{o}{=}\PY{n}{Teologie}\PY{p}{)}\PY{o}{.}\PY{n}{fit}\PY{p}{(}\PY{p}{)}
          \PY{n}{aov\PYZus{}table} \PY{o}{=} \PY{n}{sm}\PY{o}{.}\PY{n}{stats}\PY{o}{.}\PY{n}{anova\PYZus{}lm}\PY{p}{(}\PY{n}{moore\PYZus{}lm}\PY{p}{,} \PY{n}{typ}\PY{o}{=}\PY{l+m+mi}{1}\PY{p}{)}
          \PY{n}{aov\PYZus{}table}
\end{Verbatim}

            \begin{Verbatim}[commandchars=\\\{\}]
{\color{outcolor}Out[{\color{outcolor}716}]:}                     df       sum\_sq    mean\_sq         F    PR(>F)
          C(OPTION\_RIGHT)   10.0   428.945960  42.894596  1.114618  0.356164
          Residual         129.0  4964.395731  38.483688       NaN       NaN
\end{Verbatim}
        
    il existe un lient entrre les moyenne de l'option de porvenance

    \begin{Verbatim}[commandchars=\\\{\}]
{\color{incolor}In [{\color{incolor}717}]:} \PY{n}{GroupOption}\PY{o}{=}\PY{n}{Teologie}\PY{o}{.}\PY{n}{groupby}\PY{p}{(}\PY{l+s+s1}{\PYZsq{}}\PY{l+s+s1}{OPTION\PYZus{}RIGHT}\PY{l+s+s1}{\PYZsq{}}\PY{p}{)}\PY{o}{.}\PY{n}{mean}\PY{p}{(}\PY{p}{)}\PY{o}{.}\PY{n}{sort}\PY{p}{(}\PY{n}{axis}\PY{o}{=}\PY{l+m+mi}{0}\PY{p}{,}\PY{n}{columns}\PY{o}{=}\PY{l+s+s1}{\PYZsq{}}\PY{l+s+s1}{CGPA}\PY{l+s+s1}{\PYZsq{}}\PY{p}{,}\PY{n}{ascending}\PY{o}{=}\PY{n+nb+bp}{False}\PY{p}{)}\PY{o}{.}\PY{n}{reset\PYZus{}index}\PY{p}{(}\PY{p}{)}
\end{Verbatim}

    \begin{Verbatim}[commandchars=\\\{\}]
{\color{incolor}In [{\color{incolor}718}]:} \PY{n}{GroupOption}
\end{Verbatim}

            \begin{Verbatim}[commandchars=\\\{\}]
{\color{outcolor}Out[{\color{outcolor}718}]:}            OPTION\_RIGHT            ID    DIPPERC        AGE       CGPA  \textbackslash{}
          0         math-physique   6066.000000  50.750000  34.500000  66.164583   
          1            bio-chimie   6189.833333  52.666667  37.666667  65.586112   
          2           latin philo   9706.714286  55.142857  25.857143  64.836905   
          3   commmerciale et adm   8399.000000  56.300000  29.300000  64.560834   
          4          construction  10361.500000  50.500000  26.500000  63.633335   
          5           vétérinaire  11568.000000  61.000000  26.000000  63.299999   
          6               sociale   8135.000000  53.461538  30.230769  62.775001   
          7             pedagogie   8051.677778  53.744444  32.022222  61.126944   
          8     hotesse d'acceuil   9062.333333  50.666667  27.000000  60.533333   
          9                  nutr   8461.666667  52.666667  28.000000  59.150001   
          10            agronomie   8926.000000  62.000000  43.000000  55.200001   
          
              NACADYEAR  
          0    3.250000  
          1    2.166667  
          2    2.285714  
          3    1.800000  
          4    2.000000  
          5    1.000000  
          6    1.615385  
          7    2.133333  
          8    2.000000  
          9    2.000000  
          10   2.000000  
\end{Verbatim}
        
    On peut aisement remarquer que le CGPA es plus elevé selon les sections

    \begin{Verbatim}[commandchars=\\\{\}]
{\color{incolor}In [{\color{incolor}720}]:} \PY{n}{plt}\PY{o}{.}\PY{n}{figure}\PY{p}{(}\PY{n}{figsize}\PY{o}{=}\PY{p}{(}\PY{l+m+mi}{12}\PY{p}{,}\PY{l+m+mi}{6}\PY{p}{)}\PY{p}{)}
          \PY{n}{ax}\PY{o}{=}\PY{n}{sns}\PY{o}{.}\PY{n}{boxplot}\PY{p}{(}\PY{n}{x}\PY{o}{=}\PY{l+s+s2}{\PYZdq{}}\PY{l+s+s2}{OPTION\PYZus{}RIGHT}\PY{l+s+s2}{\PYZdq{}}\PY{p}{,} \PY{n}{y}\PY{o}{=}\PY{l+s+s2}{\PYZdq{}}\PY{l+s+s2}{CGPA}\PY{l+s+s2}{\PYZdq{}}\PY{p}{,} \PY{n}{data}\PY{o}{=}\PY{n}{Teologie}\PY{p}{,}\PY{n}{order}\PY{o}{=}\PY{n}{GroupOption}\PY{o}{.}\PY{n}{OPTION\PYZus{}RIGHT}\PY{p}{)}
          \PY{n}{plt}\PY{o}{.}\PY{n}{savefig}\PY{p}{(}\PY{l+s+s1}{\PYZsq{}}\PY{l+s+s1}{OPTION\PYZus{}RIGHT\PYZhy{}CGPATeolo.png}\PY{l+s+s1}{\PYZsq{}}\PY{p}{,}\PY{n}{dpi}\PY{o}{=}\PY{l+m+mi}{100}\PY{p}{)}
\end{Verbatim}

    \begin{center}
    \adjustimage{max size={0.9\linewidth}{0.9\paperheight}}{output_451_0.png}
    \end{center}
    { \hspace*{\fill} \\}
    
    On conclu que l'echecs n'est ne sont que des cas rare en teologie

    \subparagraph{Faculté de Droit}\label{facultuxe9-de-droit}

    \begin{Verbatim}[commandchars=\\\{\}]
{\color{incolor}In [{\color{incolor}597}]:} \PY{n}{Droit}\PY{o}{=}\PY{n}{datasetFin}\PY{o}{.}\PY{n}{loc}\PY{p}{[}\PY{n}{datasetFin}\PY{o}{.}\PY{n}{FAC}\PY{o}{==}\PY{l+s+s1}{\PYZsq{}}\PY{l+s+s1}{FD}\PY{l+s+s1}{\PYZsq{}}\PY{p}{]}
\end{Verbatim}

    On constate que le pourcentage de CGPA varie differament pour chaque
option du diplome , les options peda, sociale ,vienne en dernier en ce
qui concerne le cGPA mean

    \begin{Verbatim}[commandchars=\\\{\}]
{\color{incolor}In [{\color{incolor}598}]:} \PY{n}{Droit}\PY{o}{.}\PY{n}{shape}
\end{Verbatim}

            \begin{Verbatim}[commandchars=\\\{\}]
{\color{outcolor}Out[{\color{outcolor}598}]:} (896, 14)
\end{Verbatim}
        
    https://www.wellbeingatschool.org.nz/information-sheet/understanding-and-interpreting-box-plots

    A.0 Distribution du CGPA

    \begin{Verbatim}[commandchars=\\\{\}]
{\color{incolor}In [{\color{incolor}599}]:} \PY{n}{plt}\PY{o}{.}\PY{n}{figure}\PY{p}{(}\PY{p}{)}
          \PY{n}{ax} \PY{o}{=} \PY{n}{sns}\PY{o}{.}\PY{n}{boxplot}\PY{p}{(}\PY{n}{x}\PY{o}{=}\PY{n}{Droit}\PY{p}{[}\PY{l+s+s1}{\PYZsq{}}\PY{l+s+s1}{CGPA}\PY{l+s+s1}{\PYZsq{}}\PY{p}{]}\PY{p}{)}
          \PY{n}{plt}\PY{o}{.}\PY{n}{savefig}\PY{p}{(}\PY{l+s+s1}{\PYZsq{}}\PY{l+s+s1}{CGPA\PYZhy{}ECO\PYZhy{}Droit.png}\PY{l+s+s1}{\PYZsq{}}\PY{p}{,}\PY{n}{dpi}\PY{o}{=}\PY{l+m+mi}{100}\PY{p}{)}
\end{Verbatim}

    \begin{center}
    \adjustimage{max size={0.9\linewidth}{0.9\paperheight}}{output_459_0.png}
    \end{center}
    { \hspace*{\fill} \\}
    
    \begin{Verbatim}[commandchars=\\\{\}]
{\color{incolor}In [{\color{incolor}600}]:} \PY{n}{plt}\PY{o}{.}\PY{n}{figure}\PY{p}{(}\PY{p}{)}
          \PY{n}{sns}\PY{o}{.}\PY{n}{distplot}\PY{p}{(}\PY{n}{Droit}\PY{p}{[}\PY{l+s+s1}{\PYZsq{}}\PY{l+s+s1}{CGPA}\PY{l+s+s1}{\PYZsq{}}\PY{p}{]}\PY{p}{,}\PY{n}{bins}\PY{o}{=}\PY{l+m+mi}{20}\PY{p}{,}\PY{n}{axlabel}\PY{o}{=}\PY{l+s+s1}{\PYZsq{}}\PY{l+s+s1}{CGPA}\PY{l+s+s1}{\PYZsq{}}\PY{p}{,}\PY{n}{kde}\PY{o}{=}\PY{l+m+mi}{1}\PY{p}{,}\PY{n}{norm\PYZus{}hist}\PY{o}{=}\PY{l+m+mi}{0}\PY{p}{)}
          \PY{n}{plt}\PY{o}{.}\PY{n}{savefig}\PY{p}{(}\PY{l+s+s1}{\PYZsq{}}\PY{l+s+s1}{CGPADistDroit.png}\PY{l+s+s1}{\PYZsq{}}\PY{p}{,}\PY{n}{dpi}\PY{o}{=}\PY{l+m+mi}{100}\PY{p}{)}
\end{Verbatim}

    \begin{center}
    \adjustimage{max size={0.9\linewidth}{0.9\paperheight}}{output_460_0.png}
    \end{center}
    { \hspace*{\fill} \\}
    
    Il suit une distribution presque normale d'écart type : 7

    \begin{Verbatim}[commandchars=\\\{\}]
{\color{incolor}In [{\color{incolor}601}]:} \PY{n}{np}\PY{o}{.}\PY{n}{std}\PY{p}{(}\PY{n}{Droit}\PY{o}{.}\PY{n}{CGPA}\PY{p}{)}
\end{Verbatim}

            \begin{Verbatim}[commandchars=\\\{\}]
{\color{outcolor}Out[{\color{outcolor}601}]:} 7.0136400827197347
\end{Verbatim}
        
    \begin{Verbatim}[commandchars=\\\{\}]
{\color{incolor}In [{\color{incolor}602}]:} \PY{n}{np}\PY{o}{.}\PY{n}{mean}\PY{p}{(}\PY{n}{Droit}\PY{o}{.}\PY{n}{CGPA}\PY{p}{)}
\end{Verbatim}

            \begin{Verbatim}[commandchars=\\\{\}]
{\color{outcolor}Out[{\color{outcolor}602}]:} 58.022023761201453
\end{Verbatim}
        
    Commencons par les attribues numeriques et analysons la correlation
chercher comme sur le point précedant la correlation avec le CGPA

    \begin{Verbatim}[commandchars=\\\{\}]
{\color{incolor}In [{\color{incolor}603}]:} \PY{n}{Droit}\PY{o}{.}\PY{n}{corr}\PY{p}{(}\PY{p}{)}
\end{Verbatim}

            \begin{Verbatim}[commandchars=\\\{\}]
{\color{outcolor}Out[{\color{outcolor}603}]:}                  ID   DIPPERC       AGE      CGPA  NACADYEAR
          ID         1.000000  0.041124 -0.610896 -0.303752  -0.212584
          DIPPERC    0.041124  1.000000 -0.157011  0.220699   0.155471
          AGE       -0.610896 -0.157011  1.000000  0.114151   0.043887
          CGPA      -0.303752  0.220699  0.114151  1.000000   0.342643
          NACADYEAR -0.212584  0.155471  0.043887  0.342643   1.000000
\end{Verbatim}
        
    Nous pouvons aisement qu'il n'yas aucune correlation entre le le CGPA et
le pourcentage à l'exetat ni meme l'age des étudiants

    \paragraph{C.2 GENDER}\label{c.2-gender}

    \begin{Verbatim}[commandchars=\\\{\}]
{\color{incolor}In [{\color{incolor}604}]:} \PY{n}{moore\PYZus{}lm} \PY{o}{=} \PY{n}{ols}\PY{p}{(}\PY{l+s+s1}{\PYZsq{}}\PY{l+s+s1}{CGPA \PYZti{} C(GENDER)}\PY{l+s+s1}{\PYZsq{}}\PY{p}{,}\PY{n}{data}\PY{o}{=}\PY{n}{Droit}\PY{p}{)}\PY{o}{.}\PY{n}{fit}\PY{p}{(}\PY{p}{)}
          \PY{n}{aov\PYZus{}table} \PY{o}{=} \PY{n}{sm}\PY{o}{.}\PY{n}{stats}\PY{o}{.}\PY{n}{anova\PYZus{}lm}\PY{p}{(}\PY{n}{moore\PYZus{}lm}\PY{p}{,} \PY{n}{typ}\PY{o}{=}\PY{l+m+mi}{1}\PY{p}{)}
          \PY{n}{aov\PYZus{}table}
\end{Verbatim}

            \begin{Verbatim}[commandchars=\\\{\}]
{\color{outcolor}Out[{\color{outcolor}604}]:}               df        sum\_sq    mean\_sq         F    PR(>F)
          C(GENDER)    1.0      6.556803   6.556803  0.133015  0.715413
          Residual   894.0  44068.711097  49.293860       NaN       NaN
\end{Verbatim}
        
    Notre metrique nous montre que la moyenne de CGPA est la meme pour les
deux sexes ! jettons un coup d'oeil aux distribution avec des box-plot.

    \begin{Verbatim}[commandchars=\\\{\}]
{\color{incolor}In [{\color{incolor}605}]:} \PY{n}{plt}\PY{o}{.}\PY{n}{figure}\PY{p}{(}\PY{n}{figsize}\PY{o}{=}\PY{p}{(}\PY{l+m+mi}{12}\PY{p}{,}\PY{l+m+mi}{6}\PY{p}{)}\PY{p}{)}
          \PY{n}{ax}\PY{o}{=}\PY{n}{sns}\PY{o}{.}\PY{n}{boxplot}\PY{p}{(}\PY{n}{x}\PY{o}{=}\PY{l+s+s2}{\PYZdq{}}\PY{l+s+s2}{GENDER}\PY{l+s+s2}{\PYZdq{}}\PY{p}{,} \PY{n}{y}\PY{o}{=}\PY{l+s+s2}{\PYZdq{}}\PY{l+s+s2}{CGPA}\PY{l+s+s2}{\PYZdq{}}\PY{p}{,} \PY{n}{data}\PY{o}{=}\PY{n}{Droit}\PY{p}{)}
          \PY{n}{plt}\PY{o}{.}\PY{n}{savefig}\PY{p}{(}\PY{l+s+s1}{\PYZsq{}}\PY{l+s+s1}{GENDER\PYZhy{}CGPA\PYZhy{}Eco.png}\PY{l+s+s1}{\PYZsq{}}\PY{p}{,}\PY{n}{dpi}\PY{o}{=}\PY{l+m+mi}{100}\PY{p}{)}
\end{Verbatim}

    \begin{center}
    \adjustimage{max size={0.9\linewidth}{0.9\paperheight}}{output_470_0.png}
    \end{center}
    { \hspace*{\fill} \\}
    
    \begin{Verbatim}[commandchars=\\\{\}]
{\color{incolor}In [{\color{incolor}606}]:} \PY{n}{Droit}\PY{o}{.}\PY{n}{GENDER}\PY{o}{.}\PY{n}{value\PYZus{}counts}\PY{p}{(}\PY{p}{)}\PY{o}{/}\PY{n+nb}{len}\PY{p}{(}\PY{n}{Droit}\PY{p}{)}
\end{Verbatim}

            \begin{Verbatim}[commandchars=\\\{\}]
{\color{outcolor}Out[{\color{outcolor}606}]:} H    0.602679
          F    0.397321
          Name: GENDER, dtype: float64
\end{Verbatim}
        
    Nous remarquons que les 2 graphiques sont les memes , equilibre car la
facultée est constitué par 60\% des hommes et 40\% des femmes

    \begin{Verbatim}[commandchars=\\\{\}]
{\color{incolor}In [{\color{incolor}608}]:} \PY{n}{Droit}\PY{o}{.}\PY{n}{groupby}\PY{p}{(}\PY{l+s+s1}{\PYZsq{}}\PY{l+s+s1}{GENDER}\PY{l+s+s1}{\PYZsq{}}\PY{p}{)}\PY{o}{.}\PY{n}{mean}\PY{p}{(}\PY{p}{)}
\end{Verbatim}

            \begin{Verbatim}[commandchars=\\\{\}]
{\color{outcolor}Out[{\color{outcolor}608}]:}                  ID    DIPPERC        AGE       CGPA  NACADYEAR
          GENDER                                                         
          F       9039.446629  56.353239  23.766854  57.916667   1.966292
          H       8704.253704  56.018290  25.433333  58.091481   1.944444
\end{Verbatim}
        
    Mais on peut remarquer que les moyenees restent la meme

    \paragraph{B.4 Attribue Province}\label{b.4-attribue-province}

    \begin{Verbatim}[commandchars=\\\{\}]
{\color{incolor}In [{\color{incolor}609}]:} \PY{n}{moore\PYZus{}lm} \PY{o}{=} \PY{n}{ols}\PY{p}{(}\PY{l+s+s1}{\PYZsq{}}\PY{l+s+s1}{CGPA \PYZti{} C(SCHOOLPROVINCE)}\PY{l+s+s1}{\PYZsq{}}\PY{p}{,}\PY{n}{data}\PY{o}{=}\PY{n}{Droit}\PY{p}{)}\PY{o}{.}\PY{n}{fit}\PY{p}{(}\PY{p}{)}
          \PY{n}{aov\PYZus{}table} \PY{o}{=} \PY{n}{sm}\PY{o}{.}\PY{n}{stats}\PY{o}{.}\PY{n}{anova\PYZus{}lm}\PY{p}{(}\PY{n}{moore\PYZus{}lm}\PY{p}{,} \PY{n}{typ}\PY{o}{=}\PY{l+m+mi}{1}\PY{p}{)}
          \PY{n}{aov\PYZus{}table}
\end{Verbatim}

            \begin{Verbatim}[commandchars=\\\{\}]
{\color{outcolor}Out[{\color{outcolor}609}]:}                       df       sum\_sq    mean\_sq        F    PR(>F)
          C(SCHOOLPROVINCE)   10.0    673.67125  67.367125  1.37368  0.187473
          Residual           885.0  43401.59665  49.041352      NaN       NaN
\end{Verbatim}
        
    PR =0.18 est superieur à 0.05 nous pouvons conclure que la moyenne est
la meme sur toutes les facultés

    \begin{Verbatim}[commandchars=\\\{\}]
{\color{incolor}In [{\color{incolor}610}]:} \PY{n}{DroitGroup}\PY{o}{=}\PY{n}{Droit}\PY{o}{.}\PY{n}{groupby}\PY{p}{(}\PY{l+s+s1}{\PYZsq{}}\PY{l+s+s1}{SCHOOLPROVINCE}\PY{l+s+s1}{\PYZsq{}}\PY{p}{)}\PY{o}{.}\PY{n}{mean}\PY{p}{(}\PY{p}{)}\PY{o}{.}\PY{n}{sort}\PY{p}{(}\PY{n}{axis}\PY{o}{=}\PY{l+m+mi}{0}\PY{p}{,}\PY{n}{columns}\PY{o}{=}\PY{l+s+s1}{\PYZsq{}}\PY{l+s+s1}{CGPA}\PY{l+s+s1}{\PYZsq{}}\PY{p}{,}\PY{n}{ascending}\PY{o}{=}\PY{n+nb+bp}{False}\PY{p}{)}
\end{Verbatim}

    \begin{Verbatim}[commandchars=\\\{\}]
{\color{incolor}In [{\color{incolor}611}]:} \PY{n}{DroitGroup}\PY{o}{.}\PY{n}{reset\PYZus{}index}\PY{p}{(}\PY{n}{inplace}\PY{o}{=}\PY{n+nb+bp}{True}
                               \PY{p}{)}
\end{Verbatim}

    \begin{Verbatim}[commandchars=\\\{\}]
{\color{incolor}In [{\color{incolor}612}]:} \PY{n}{DroitGroup}
\end{Verbatim}

            \begin{Verbatim}[commandchars=\\\{\}]
{\color{outcolor}Out[{\color{outcolor}612}]:}     SCHOOLPROVINCE            ID    DIPPERC        AGE       CGPA  NACADYEAR
          0         EQUATEUR   4659.000000  59.000000  27.000000  63.849998   2.000000
          1          inconnu   6546.600000  53.775304  26.800000  63.820001   1.800000
          2        BAS CONGO   9194.000000  67.000000  21.000000  60.633334   3.000000
          3        NORD-KIVU   8891.984709  56.271794  24.313456  58.141055   1.963303
          4         SUD-KIVU   8735.592179  55.955307  25.916201  58.058147   1.955307
          5         KINSHASA   8367.750000  56.416667  24.333333  57.284722   1.833333
          6        ORIENTALE   8562.875000  55.812500  25.937500  57.107812   1.875000
          7          MANIEMA   9738.562500  54.250000  26.187500  55.998958   1.875000
          8   KASAI ORIENTAL   7375.000000  52.000000  28.000000  52.950000   2.000000
          9          KATANGA   7724.375000  54.375000  30.000000  52.883333   1.625000
          10        BANDUNDU  11472.000000  61.000000  25.500000  50.700000   1.500000
\end{Verbatim}
        
    Pour conclure cocernant la moyenne on tien compte du nombre des individu
d'ou un boxplot est importtant

    \begin{Verbatim}[commandchars=\\\{\}]
{\color{incolor}In [{\color{incolor}613}]:} \PY{n}{plt}\PY{o}{.}\PY{n}{figure}\PY{p}{(}\PY{n}{figsize}\PY{o}{=}\PY{p}{(}\PY{l+m+mi}{15}\PY{p}{,}\PY{l+m+mi}{6}\PY{p}{)}\PY{p}{)}
          \PY{n}{ax}\PY{o}{=}\PY{n}{sns}\PY{o}{.}\PY{n}{boxplot}\PY{p}{(}\PY{n}{x}\PY{o}{=}\PY{l+s+s2}{\PYZdq{}}\PY{l+s+s2}{SCHOOLPROVINCE}\PY{l+s+s2}{\PYZdq{}}\PY{p}{,} \PY{n}{y}\PY{o}{=}\PY{l+s+s2}{\PYZdq{}}\PY{l+s+s2}{CGPA}\PY{l+s+s2}{\PYZdq{}}\PY{p}{,} \PY{n}{data}\PY{o}{=}\PY{n}{Droit}\PY{p}{,}\PY{n}{order}\PY{o}{=}\PY{n}{DroitGroup}\PY{o}{.}\PY{n}{SCHOOLPROVINCE}\PY{p}{)}
          \PY{n}{plt}\PY{o}{.}\PY{n}{savefig}\PY{p}{(}\PY{l+s+s1}{\PYZsq{}}\PY{l+s+s1}{GENDER\PYZhy{}SCHOOLPROVINCE\PYZhy{}Droit.png}\PY{l+s+s1}{\PYZsq{}}\PY{p}{,}\PY{n}{dpi}\PY{o}{=}\PY{l+m+mi}{100}\PY{p}{)}
\end{Verbatim}

    \begin{center}
    \adjustimage{max size={0.9\linewidth}{0.9\paperheight}}{output_482_0.png}
    \end{center}
    { \hspace*{\fill} \\}
    
    Nous remarquons que le moyenne du CGPA est la meme pour les provinces
suivantes : Nord-Kivu, Sud-Kivu, Kinshase , Orientale ,Maniema, les
distinctions proviennent beaucoup plus des provincces du nord-kivu, sud
kivu et kinshasa.

    \paragraph{B.5 Stattus de L'école}\label{b.5-stattus-de-luxe9cole}

    \begin{Verbatim}[commandchars=\\\{\}]
{\color{incolor}In [{\color{incolor}614}]:} \PY{n}{moore\PYZus{}lm} \PY{o}{=} \PY{n}{ols}\PY{p}{(}\PY{l+s+s1}{\PYZsq{}}\PY{l+s+s1}{CGPA \PYZti{} C(SCHOOLSTATUS)}\PY{l+s+s1}{\PYZsq{}}\PY{p}{,}\PY{n}{data}\PY{o}{=}\PY{n}{Droit}\PY{p}{)}\PY{o}{.}\PY{n}{fit}\PY{p}{(}\PY{p}{)}
          \PY{n}{aov\PYZus{}table} \PY{o}{=} \PY{n}{sm}\PY{o}{.}\PY{n}{stats}\PY{o}{.}\PY{n}{anova\PYZus{}lm}\PY{p}{(}\PY{n}{moore\PYZus{}lm}\PY{p}{,} \PY{n}{typ}\PY{o}{=}\PY{l+m+mi}{1}\PY{p}{)}
          \PY{n}{aov\PYZus{}table}
\end{Verbatim}

            \begin{Verbatim}[commandchars=\\\{\}]
{\color{outcolor}Out[{\color{outcolor}614}]:}                     df        sum\_sq     mean\_sq         F        PR(>F)
          C(SCHOOLSTATUS)    7.0   2425.484297  346.497757  7.387554  1.203365e-08
          Residual         888.0  41649.783603   46.902909       NaN           NaN
\end{Verbatim}
        
    Il n'ya aucun lien entre les moyennes de CGPA pour les differents status
de l"ecole

    \begin{Verbatim}[commandchars=\\\{\}]
{\color{incolor}In [{\color{incolor}615}]:} \PY{n}{Droit}\PY{o}{.}\PY{n}{groupby}\PY{p}{(}\PY{l+s+s1}{\PYZsq{}}\PY{l+s+s1}{SCHOOLSTATUS}\PY{l+s+s1}{\PYZsq{}}\PY{p}{)}\PY{o}{.}\PY{n}{mean}\PY{p}{(}\PY{p}{)}\PY{o}{.}\PY{n}{sort}\PY{p}{(}\PY{n}{axis}\PY{o}{=}\PY{l+m+mi}{0}\PY{p}{,}\PY{n}{columns}\PY{o}{=}\PY{l+s+s1}{\PYZsq{}}\PY{l+s+s1}{CGPA}\PY{l+s+s1}{\PYZsq{}}\PY{p}{,}\PY{n}{ascending}\PY{o}{=}\PY{n+nb+bp}{False}\PY{p}{)}
\end{Verbatim}

            \begin{Verbatim}[commandchars=\\\{\}]
{\color{outcolor}Out[{\color{outcolor}615}]:}                        ID    DIPPERC        AGE       CGPA  NACADYEAR
          SCHOOLSTATUS                                                         
          catholique    8957.460674  57.505618  24.365169  59.695786   1.932584
          protestant    8703.625954  55.991895  24.282443  59.365522   2.087786
          kimbanguiste  9094.000000  58.000000  27.500000  57.699999   1.000000
          inconnu       7941.496815  56.278682  26.477707  57.519055   2.121019
          publique      9119.818182  55.895105  25.041958  57.066900   1.881119
          musulman      9339.000000  50.900000  23.600000  56.260000   1.800000
          privé         9649.330769  54.976923  23.730769  55.325833   1.653846
          autodidacte   9044.428571  57.500000  27.285714  53.335714   1.571429
\end{Verbatim}
        
    Les écoles catholiques et protestant son coude à coude pour la moyenne
du pourcentage

    Voyons cela de plus pret avec box-plot

    \begin{Verbatim}[commandchars=\\\{\}]
{\color{incolor}In [{\color{incolor}616}]:} \PY{n}{plt}\PY{o}{.}\PY{n}{figure}\PY{p}{(}\PY{n}{figsize}\PY{o}{=}\PY{p}{(}\PY{l+m+mi}{12}\PY{p}{,}\PY{l+m+mi}{6}\PY{p}{)}\PY{p}{)}
          \PY{n}{ax}\PY{o}{=}\PY{n}{sns}\PY{o}{.}\PY{n}{boxplot}\PY{p}{(}\PY{n}{x}\PY{o}{=}\PY{l+s+s2}{\PYZdq{}}\PY{l+s+s2}{SCHOOLSTATUS}\PY{l+s+s2}{\PYZdq{}}\PY{p}{,} \PY{n}{y}\PY{o}{=}\PY{l+s+s2}{\PYZdq{}}\PY{l+s+s2}{CGPA}\PY{l+s+s2}{\PYZdq{}}\PY{p}{,} \PY{n}{data}\PY{o}{=}\PY{n}{Droit}\PY{p}{)}
          \PY{n}{plt}\PY{o}{.}\PY{n}{savefig}\PY{p}{(}\PY{l+s+s1}{\PYZsq{}}\PY{l+s+s1}{SCHOOLSTATUS\PYZhy{}ECO\PYZhy{}Droit.png}\PY{l+s+s1}{\PYZsq{}}\PY{p}{,}\PY{n}{dpi}\PY{o}{=}\PY{l+m+mi}{100}\PY{p}{)}
\end{Verbatim}

    \begin{center}
    \adjustimage{max size={0.9\linewidth}{0.9\paperheight}}{output_490_0.png}
    \end{center}
    { \hspace*{\fill} \\}
    
    Nous remarquons que les celles étudiants qui distingunt c'est ceux
provennant des écoles protestants et catholiques, en queu de la liste se
trouve les autodidacte

    \paragraph{B.Les Ecoles de
provenances}\label{b.les-ecoles-de-provenances}

    \begin{Verbatim}[commandchars=\\\{\}]
{\color{incolor}In [{\color{incolor}617}]:} \PY{n}{moore\PYZus{}lm} \PY{o}{=} \PY{n}{ols}\PY{p}{(}\PY{l+s+s1}{\PYZsq{}}\PY{l+s+s1}{CGPA \PYZti{} C(SCHOOL\PYZus{}RIGHT)}\PY{l+s+s1}{\PYZsq{}}\PY{p}{,}\PY{n}{data}\PY{o}{=}\PY{n}{Droit}\PY{p}{)}\PY{o}{.}\PY{n}{fit}\PY{p}{(}\PY{p}{)}
          \PY{n}{aov\PYZus{}table} \PY{o}{=} \PY{n}{sm}\PY{o}{.}\PY{n}{stats}\PY{o}{.}\PY{n}{anova\PYZus{}lm}\PY{p}{(}\PY{n}{moore\PYZus{}lm}\PY{p}{,} \PY{n}{typ}\PY{o}{=}\PY{l+m+mi}{1}\PY{p}{)}
          \PY{n}{aov\PYZus{}table}
\end{Verbatim}

            \begin{Verbatim}[commandchars=\\\{\}]
{\color{outcolor}Out[{\color{outcolor}617}]:}                     df        sum\_sq    mean\_sq         F        PR(>F)
          C(SCHOOL\_RIGHT)  278.0  20021.578509  72.020067  1.847383  2.591052e-10
          Residual         617.0  24053.689392  38.984910       NaN           NaN
\end{Verbatim}
        
    Aucunne correlation n'existe entre l"ecole et le CGPA au vu de la valeur
de PR'

    \begin{Verbatim}[commandchars=\\\{\}]
{\color{incolor}In [{\color{incolor}618}]:} \PY{n}{SchoolGroup}\PY{o}{=}\PY{n}{Droit}\PY{o}{.}\PY{n}{groupby}\PY{p}{(}\PY{l+s+s1}{\PYZsq{}}\PY{l+s+s1}{SCHOOL\PYZus{}RIGHT}\PY{l+s+s1}{\PYZsq{}}\PY{p}{)}\PY{o}{.}\PY{n}{mean}\PY{p}{(}\PY{p}{)}\PY{o}{.}\PY{n}{sort}\PY{p}{(}\PY{n}{axis}\PY{o}{=}\PY{l+m+mi}{0}\PY{p}{,}\PY{n}{columns}\PY{o}{=}\PY{l+s+s1}{\PYZsq{}}\PY{l+s+s1}{CGPA}\PY{l+s+s1}{\PYZsq{}}\PY{p}{,}\PY{n}{ascending}\PY{o}{=}\PY{n+nb+bp}{False}\PY{p}{)}
\end{Verbatim}

    \begin{Verbatim}[commandchars=\\\{\}]
{\color{incolor}In [{\color{incolor}619}]:} \PY{n}{SchoolGroup}\PY{o}{.}\PY{n}{reset\PYZus{}index}\PY{p}{(}\PY{n}{inplace}\PY{o}{=}\PY{n+nb+bp}{True}\PY{p}{)}
\end{Verbatim}

    \begin{Verbatim}[commandchars=\\\{\}]
{\color{incolor}In [{\color{incolor}620}]:} \PY{n}{SchoolGroup}
\end{Verbatim}

            \begin{Verbatim}[commandchars=\\\{\}]
{\color{outcolor}Out[{\color{outcolor}620}]:}           SCHOOL\_RIGHT            ID    DIPPERC        AGE       CGPA  \textbackslash{}
          0                 muhe   8630.000000  56.000000  26.000000  76.450001   
          1             L MWANDU   4716.000000  51.000000  28.000000  70.250000   
          2             MWANGAZA   6592.000000  59.000000  26.000000  70.150002   
          3         LYCEE MWANDU   3776.000000  55.000000  27.000000  70.099998   
          4         edap isp bkv   8700.000000  50.000000  23.000000  69.700001   
          5                belge   4997.500000  53.438261  29.500000  69.650000   
          6               vuhika   5720.000000  59.000000  24.000000  69.000003   
          7            ziwa kivu   8905.000000  61.000000  27.000000  68.500000   
          8            nyantende   8949.000000  58.000000  23.000000  68.500000   
          9          imani panzi   3954.000000  53.000000  30.000000  68.300003   
          10            I DE BKV   3577.000000  56.000000  29.000000  68.000000   
          11        epsk/fomulac   4274.000000  50.000000  30.000000  67.800003   
          12              katana   8252.000000  63.000000  31.000000  67.099998   
          13              visolo   5767.000000  58.000000  27.000000  66.099998   
          14              fazili   8778.000000  53.000000  29.000000  65.800003   
          15            lisalisi   4392.000000  51.000000  31.000000  65.400002   
          16            bunyanga   7719.000000  63.000000  25.000000  65.375001   
          17             kambali   7615.000000  60.500000  24.750000  65.349998   
          18             shaloom  11420.000000  64.000000  20.000000  65.199997   
          19               amina  10450.000000  58.000000  22.000000  65.099998   
          20              mululu   7886.333333  53.000000  23.333333  65.066667   
          21      idap isp bunia   8681.000000  60.000000  26.000000  65.066666   
          22              Mwandu   7826.000000  62.000000  24.500000  64.875000   
          23     zawadi ya raisi   8266.750000  55.250000  27.000000  64.825000   
          24           bethsaida  12084.000000  63.000000  20.000000  64.699997   
          25               amani   5654.500000  61.500000  25.000000  64.679168   
          26              ujumbe  11348.000000  56.000000  23.000000  64.599998   
          27              kasali   7431.000000  56.000000  27.000000  64.441667   
          28          mululusake   5918.000000  52.000000  31.000000  64.400002   
          29      it kasabinyole   6242.000000  56.000000  25.000000  64.400000   
          ..                 {\ldots}           {\ldots}        {\ldots}        {\ldots}        {\ldots}   
          249   virunga/quartier  11696.000000  53.000000  22.500000  47.299999   
          250               jiwe  11957.500000  51.000000  26.000000  47.200001   
          251   lumiere du monde  11797.000000  51.000000  21.000000  47.000000   
          252        la félicité   9255.000000  62.000000  25.000000  47.000000   
          253             laudjo  10473.000000  50.000000  27.000000  46.000000   
          254             nsindo  10595.000000  54.000000  26.000000  46.000000   
          255             amkeni   9072.000000  51.000000  24.000000  46.000000   
          256             kiyabo  11408.000000  51.000000  24.000000  45.500000   
          257              ceago   8843.000000  63.000000  27.000000  44.000000   
          258    INSTITUT BWINZI   7125.000000  70.000000  27.000000  44.000000   
          259                fdf  12360.000000  60.000000  29.000000  44.000000   
          260             cirezi   9110.000000  60.000000  22.000000  44.000000   
          261              mfuki   7809.000000  52.000000  26.000000  44.000000   
          262             kabila   8162.000000  51.000000  29.000000  43.799999   
          263   source du savoir  10777.000000  59.000000  23.000000  43.000000   
          264           kishanga   9933.000000  63.000000  27.000000  43.000000   
          265  edap isp walikale   9427.000000  56.000000  29.000000  42.099998   
          266           kirikiri  10567.000000  51.000000  22.000000  42.000000   
          267    Institut BWANGA  11796.000000  55.000000  20.000000  42.000000   
          268             kiwele   1295.000000  51.000000  42.000000  42.000000   
          269    de kingasani ii  11852.000000  64.000000  27.000000  42.000000   
          270        neema kwetu  12170.000000  58.000000  21.000000  42.000000   
          271         c,s, umoja    429.000000  56.876522  44.000000  41.000000   
          272             lusaka  11884.000000  52.000000  20.000000  41.000000   
          273   instituti ndosho  10389.000000  51.000000  23.000000  41.000000   
          274      gs la promise  10427.000000  50.000000  22.000000  40.000000   
          275           lumbishi  11982.000000  52.000000  25.000000  40.000000   
          276                  z  10728.000000  53.000000  21.000000  40.000000   
          277         masanikilo  11669.000000  57.000000  26.000000  40.000000   
          278             wamaza   9580.000000  52.000000  31.000000  36.400002   
          
               NACADYEAR  
          0     2.000000  
          1     2.000000  
          2     2.000000  
          3     1.000000  
          4     4.000000  
          5     1.000000  
          6     3.000000  
          7     2.000000  
          8     3.000000  
          9     1.000000  
          10    1.000000  
          11    1.000000  
          12    1.000000  
          13    2.000000  
          14    1.000000  
          15    2.000000  
          16    4.000000  
          17    1.750000  
          18    1.000000  
          19    2.000000  
          20    3.000000  
          21    2.500000  
          22    3.500000  
          23    2.750000  
          24    1.000000  
          25    3.500000  
          26    1.000000  
          27    2.666667  
          28    1.000000  
          29    2.000000  
          ..         {\ldots}  
          249   1.000000  
          250   1.000000  
          251   1.000000  
          252   1.000000  
          253   1.000000  
          254   1.000000  
          255   1.000000  
          256   1.000000  
          257   1.000000  
          258   1.000000  
          259   1.000000  
          260   1.000000  
          261   2.000000  
          262   2.000000  
          263   1.000000  
          264   1.000000  
          265   1.000000  
          266   1.000000  
          267   1.000000  
          268   1.000000  
          269   1.000000  
          270   1.000000  
          271   1.000000  
          272   1.000000  
          273   1.000000  
          274   1.000000  
          275   1.000000  
          276   1.000000  
          277   1.000000  
          278   1.000000  
          
          [279 rows x 6 columns]
\end{Verbatim}
        
    Nous allons Voir de plus pret pour les 10 ecoles les plus representé, le
5 éecoles avec un pourcentage elevé et 5 dernier

    \begin{Verbatim}[commandchars=\\\{\}]
{\color{incolor}In [{\color{incolor}621}]:} \PY{n}{plt}\PY{o}{.}\PY{n}{figure}\PY{p}{(}\PY{n}{figsize}\PY{o}{=}\PY{p}{(}\PY{l+m+mi}{30}\PY{p}{,}\PY{l+m+mi}{10}\PY{p}{)}\PY{p}{)}
          \PY{n}{ax}\PY{o}{=}\PY{n}{sns}\PY{o}{.}\PY{n}{boxplot}\PY{p}{(}\PY{n}{x}\PY{o}{=}\PY{l+s+s2}{\PYZdq{}}\PY{l+s+s2}{SCHOOL\PYZus{}RIGHT}\PY{l+s+s2}{\PYZdq{}}\PY{p}{,} \PY{n}{y}\PY{o}{=}\PY{l+s+s2}{\PYZdq{}}\PY{l+s+s2}{CGPA}\PY{l+s+s2}{\PYZdq{}}\PY{p}{,}  \PY{n}{data}\PY{o}{=}\PY{n}{Droit}\PY{o}{.}\PY{n}{loc}\PY{p}{[}\PY{n}{operator}\PY{o}{.}\PY{n}{or\PYZus{}}\PY{p}{(}\PY{n}{Droit}\PY{o}{.}\PY{n}{SCHOOL\PYZus{}RIGHT}\PY{o}{.}\PY{n}{isin}\PY{p}{(}\PY{n}{SchoolGroup}\PY{o}{.}\PY{n}{loc}\PY{p}{[} \PY{n+nb}{range}\PY{p}{(}\PY{l+m+mi}{0}\PY{p}{,}\PY{l+m+mi}{11}\PY{p}{)}\PY{p}{]}\PY{o}{.}\PY{n}{SCHOOL\PYZus{}RIGHT}\PY{p}{)} 
                       \PY{p}{,} \PY{n}{Droit}\PY{o}{.}\PY{n}{SCHOOL\PYZus{}RIGHT}\PY{o}{.}\PY{n}{isin}\PY{p}{(}\PY{n}{Droit}\PY{o}{.}\PY{n}{SCHOOL\PYZus{}RIGHT}\PY{o}{.}\PY{n}{value\PYZus{}counts}\PY{p}{(}\PY{p}{)}\PY{p}{[}\PY{p}{:}\PY{l+m+mi}{10}\PY{p}{]}\PY{o}{.}\PY{n}{index} \PY{p}{)}\PY{p}{)}\PY{p}{]}\PY{p}{)}
          \PY{n}{plt}\PY{o}{.}\PY{n}{savefig}\PY{p}{(}\PY{l+s+s1}{\PYZsq{}}\PY{l+s+s1}{SCHOOL\PYZus{}RIGHT\PYZhy{}CGPADroit.png}\PY{l+s+s1}{\PYZsq{}}\PY{p}{,}\PY{n}{dpi}\PY{o}{=}\PY{l+m+mi}{100}\PY{p}{)}
\end{Verbatim}

    \begin{center}
    \adjustimage{max size={0.9\linewidth}{0.9\paperheight}}{output_499_0.png}
    \end{center}
    { \hspace*{\fill} \\}
    
    \paragraph{B.7 OPTION DU DIPLOME}\label{b.7-option-du-diplome}

    \begin{Verbatim}[commandchars=\\\{\}]
{\color{incolor}In [{\color{incolor}622}]:} \PY{n+nb}{len}\PY{p}{(}\PY{n}{Droit}\PY{o}{.}\PY{n}{OPTION\PYZus{}RIGHT}\PY{o}{.}\PY{n}{value\PYZus{}counts}\PY{p}{(}\PY{p}{)}\PY{p}{)}
\end{Verbatim}

            \begin{Verbatim}[commandchars=\\\{\}]
{\color{outcolor}Out[{\color{outcolor}622}]:} 17
\end{Verbatim}
        
    Nous pouvons constater que nous avons 17 options differents Voyons
comment evoluer la moyenne

    \begin{Verbatim}[commandchars=\\\{\}]
{\color{incolor}In [{\color{incolor}623}]:} \PY{n}{moore\PYZus{}lm} \PY{o}{=} \PY{n}{ols}\PY{p}{(}\PY{l+s+s1}{\PYZsq{}}\PY{l+s+s1}{CGPA \PYZti{} C(OPTION\PYZus{}RIGHT)}\PY{l+s+s1}{\PYZsq{}}\PY{p}{,}\PY{n}{data}\PY{o}{=}\PY{n}{Droit}\PY{p}{)}\PY{o}{.}\PY{n}{fit}\PY{p}{(}\PY{p}{)}
          \PY{n}{aov\PYZus{}table} \PY{o}{=} \PY{n}{sm}\PY{o}{.}\PY{n}{stats}\PY{o}{.}\PY{n}{anova\PYZus{}lm}\PY{p}{(}\PY{n}{moore\PYZus{}lm}\PY{p}{,} \PY{n}{typ}\PY{o}{=}\PY{l+m+mi}{1}\PY{p}{)}
          \PY{n}{aov\PYZus{}table}
\end{Verbatim}

            \begin{Verbatim}[commandchars=\\\{\}]
{\color{outcolor}Out[{\color{outcolor}623}]:}                     df        sum\_sq     mean\_sq         F        PR(>F)
          C(OPTION\_RIGHT)   16.0   3643.285363  227.705335  4.950363  7.251535e-10
          Residual         879.0  40431.982537   45.997705       NaN           NaN
\end{Verbatim}
        
    La valeur est inferieur à 0.05 donc il n'yas pas de relation entre les
valeurs , certaines écoles on une bonne moyenne du GPA que d'autres donc
elle depend de l'ecole de provenance

    \begin{Verbatim}[commandchars=\\\{\}]
{\color{incolor}In [{\color{incolor}624}]:} \PY{n}{GroupOption}\PY{o}{=}\PY{n}{Droit}\PY{o}{.}\PY{n}{groupby}\PY{p}{(}\PY{l+s+s1}{\PYZsq{}}\PY{l+s+s1}{OPTION\PYZus{}RIGHT}\PY{l+s+s1}{\PYZsq{}}\PY{p}{)}\PY{o}{.}\PY{n}{mean}\PY{p}{(}\PY{p}{)}\PY{o}{.}\PY{n}{sort}\PY{p}{(}\PY{n}{axis}\PY{o}{=}\PY{l+m+mi}{0}\PY{p}{,}\PY{n}{columns}\PY{o}{=}\PY{l+s+s1}{\PYZsq{}}\PY{l+s+s1}{CGPA}\PY{l+s+s1}{\PYZsq{}}\PY{p}{,}\PY{n}{ascending}\PY{o}{=}\PY{n+nb+bp}{False}\PY{p}{)}\PY{o}{.}\PY{n}{reset\PYZus{}index}\PY{p}{(}\PY{p}{)}
\end{Verbatim}

    \begin{Verbatim}[commandchars=\\\{\}]
{\color{incolor}In [{\color{incolor}625}]:} \PY{n}{GroupOption}
\end{Verbatim}

            \begin{Verbatim}[commandchars=\\\{\}]
{\color{outcolor}Out[{\color{outcolor}625}]:}                 OPTION\_RIGHT            ID    DIPPERC        AGE       CGPA  \textbackslash{}
          0                   agrecole   8905.000000  61.000000  27.000000  68.500000   
          1      electronique générale   8782.000000  52.000000  28.000000  64.024999   
          2                    inconnu   5019.500000  59.188261  30.750000  62.237500   
          3          hotesse d'acceuil  11449.000000  52.000000  21.000000  60.599998   
          4                latin philo   8642.457565  57.221402  24.254613  60.580350   
          5              coupe couture   8169.000000  54.000000  25.000000  59.588889   
          6        commmerciale et adm   8452.080645  54.288331  24.467742  59.008871   
          7                   economie  10066.000000  55.000000  24.000000  58.700001   
          8                elec indust  10773.000000  56.000000  30.000000  58.600000   
          9                 bio-chimie   8302.535714  55.321429  24.428571  58.247321   
          10  commerciale informatique  10525.750000  55.166667  22.500000  57.531250   
          11                 pedagogie   8823.003058  56.272171  25.654434  56.648930   
          12                   sociale   9390.964497  55.260355  23.852071  56.323866   
          13             math-physique   8174.700000  54.000000  27.200000  56.023333   
          14                  mec gene   9266.000000  52.000000  25.666667  53.433334   
          15                      nutr   4765.000000  57.000000  25.000000  45.000000   
          16                       fdf  12360.000000  60.000000  29.000000  44.000000   
          
              NACADYEAR  
          0    2.000000  
          1    4.000000  
          2    2.250000  
          3    1.000000  
          4    2.103321  
          5    2.333333  
          6    1.854839  
          7    2.000000  
          8    2.000000  
          9    2.107143  
          10   2.000000  
          11   1.877676  
          12   1.840237  
          13   2.000000  
          14   2.666667  
          15   1.000000  
          16   1.000000  
\end{Verbatim}
        
    Voila comme on pourait s'y attendre l'option latin philo est en tete de
kiste les options commerciale , pedagogie , sociale , math-physique ,mec
sont en bas de l'echele

    \begin{Verbatim}[commandchars=\\\{\}]
{\color{incolor}In [{\color{incolor}626}]:} \PY{n}{plt}\PY{o}{.}\PY{n}{figure}\PY{p}{(}\PY{n}{figsize}\PY{o}{=}\PY{p}{(}\PY{l+m+mi}{35}\PY{p}{,}\PY{l+m+mi}{10}\PY{p}{)}\PY{p}{)}
          \PY{n}{ax}\PY{o}{=}\PY{n}{sns}\PY{o}{.}\PY{n}{boxplot}\PY{p}{(}\PY{n}{x}\PY{o}{=}\PY{l+s+s2}{\PYZdq{}}\PY{l+s+s2}{OPTION\PYZus{}RIGHT}\PY{l+s+s2}{\PYZdq{}}\PY{p}{,} \PY{n}{y}\PY{o}{=}\PY{l+s+s2}{\PYZdq{}}\PY{l+s+s2}{CGPA}\PY{l+s+s2}{\PYZdq{}}\PY{p}{,} \PY{n}{data}\PY{o}{=}\PY{n}{Droit}\PY{p}{,}\PY{n}{order}\PY{o}{=}\PY{n}{GroupOption}\PY{o}{.}\PY{n}{OPTION\PYZus{}RIGHT}\PY{p}{)}
          \PY{n}{plt}\PY{o}{.}\PY{n}{savefig}\PY{p}{(}\PY{l+s+s1}{\PYZsq{}}\PY{l+s+s1}{OPTION\PYZus{}RIGHT\PYZhy{}CGPADroit.png}\PY{l+s+s1}{\PYZsq{}}\PY{p}{,}\PY{n}{dpi}\PY{o}{=}\PY{l+m+mi}{100}\PY{p}{)}
\end{Verbatim}

    \begin{center}
    \adjustimage{max size={0.9\linewidth}{0.9\paperheight}}{output_508_0.png}
    \end{center}
    { \hspace*{\fill} \\}
    
    Nous voici a la fin de notre anayse bi-varie 2 eme iterartion !

    \begin{Verbatim}[commandchars=\\\{\}]
{\color{incolor}In [{\color{incolor}722}]:} \PY{n}{datasetFin}\PY{o}{.}\PY{n}{to\PYZus{}csv}\PY{p}{(}\PY{l+s+s1}{\PYZsq{}}\PY{l+s+s1}{DatasetFinalV1.csv}\PY{l+s+s1}{\PYZsq{}}\PY{p}{)}
          \PY{n}{Techno}\PY{o}{.}\PY{n}{to\PYZus{}csv}\PY{p}{(}\PY{l+s+s2}{\PYZdq{}}\PY{l+s+s2}{DatasetTechno.csv}\PY{l+s+s2}{\PYZdq{}}\PY{p}{)}
          \PY{n}{Medecine}\PY{o}{.}\PY{n}{to\PYZus{}csv}\PY{p}{(}\PY{l+s+s1}{\PYZsq{}}\PY{l+s+s1}{DatasetMedecine.csv}\PY{l+s+s1}{\PYZsq{}}\PY{p}{)}
          \PY{n}{Economie}\PY{o}{.}\PY{n}{to\PYZus{}csv}\PY{p}{(}\PY{l+s+s1}{\PYZsq{}}\PY{l+s+s1}{DatasetEconomie.csv}\PY{l+s+s1}{\PYZsq{}}\PY{p}{)}
          \PY{n}{Droit}\PY{o}{.}\PY{n}{to\PYZus{}csv}\PY{p}{(}\PY{l+s+s1}{\PYZsq{}}\PY{l+s+s1}{DatasetDroit.csv}\PY{l+s+s1}{\PYZsq{}}\PY{p}{)}
          \PY{n}{Sante}\PY{o}{.}\PY{n}{to\PYZus{}csv}\PY{p}{(}\PY{l+s+s1}{\PYZsq{}}\PY{l+s+s1}{DatsetSante.csv}\PY{l+s+s1}{\PYZsq{}}\PY{p}{)}
          \PY{n}{Psyco}\PY{o}{.}\PY{n}{to\PYZus{}csv}\PY{p}{(}\PY{l+s+s1}{\PYZsq{}}\PY{l+s+s1}{DatasetPsycho.csv}\PY{l+s+s1}{\PYZsq{}}\PY{p}{)}
          \PY{n}{Teologie}\PY{o}{.}\PY{n}{to\PYZus{}csv}\PY{p}{(}\PY{l+s+s1}{\PYZsq{}}\PY{l+s+s1}{DatasetTeo.csv}\PY{l+s+s1}{\PYZsq{}}\PY{p}{)}
\end{Verbatim}

    Avant de conclure cette phase nous allons essayer de voir comment se
comporte nos 3 autres variables de sorties face au CGPA

    \begin{Verbatim}[commandchars=\\\{\}]
{\color{incolor}In [{\color{incolor}4}]:} \PY{n}{datasetFin}\PY{o}{=}\PY{n}{pd}\PY{o}{.}\PY{n}{read\PYZus{}csv}\PY{p}{(}\PY{l+s+s1}{\PYZsq{}}\PY{l+s+s1}{dataset/DatasetFinalV1.csv}\PY{l+s+s1}{\PYZsq{}}\PY{p}{,}\PY{n}{index\PYZus{}col}\PY{o}{=}\PY{l+s+s2}{\PYZdq{}}\PY{l+s+s2}{Unnamed: 0}\PY{l+s+s2}{\PYZdq{}}\PY{p}{)}
\end{Verbatim}

    \begin{Verbatim}[commandchars=\\\{\}]
{\color{incolor}In [{\color{incolor}5}]:} \PY{n}{datasetFin}\PY{o}{.}\PY{n}{head}\PY{p}{(}\PY{l+m+mi}{3}\PY{p}{)}
\end{Verbatim}

            \begin{Verbatim}[commandchars=\\\{\}]
{\color{outcolor}Out[{\color{outcolor}5}]:}      ID SCHOOLSTATUS SCHOOL\_RIGHT         OPTION\_RIGHT   FAC SCHOOLPROVINCE  \textbackslash{}
        0  3895   protestant       zanner  commmerciale et adm  FSEG      NORD-KIVU   
        1  4048   protestant       zanner  commmerciale et adm  FSEG      NORD-KIVU   
        2  4217   protestant       zanner  commmerciale et adm  FSEG      NORD-KIVU   
        
          GENDER  DIPPERC  AGE       CGPA DistinctionRatio EchecRatio  NACADYEAR  \textbackslash{}
        0      H     52.0   31  59.400002                G          G        1.0   
        1      H     53.0   30  40.000000                G          A        1.0   
        2      H     54.0   28  61.299999                G          G        1.0   
        
          Pass1stSessionRatio  
        0                   G  
        1                   G  
        2                   A  
\end{Verbatim}
        
    Analysons les correlations existants entre le ratio des distinction et
le CGPA

    \paragraph{CGPA-PassageEnPremierre
Sesion}\label{cgpa-passageenpremierre-sesion}

    \begin{Verbatim}[commandchars=\\\{\}]
{\color{incolor}In [{\color{incolor}17}]:} \PY{n}{datasetFin}\PY{o}{.}\PY{n}{columns}
\end{Verbatim}

            \begin{Verbatim}[commandchars=\\\{\}]
{\color{outcolor}Out[{\color{outcolor}17}]:} Index([u'ID', u'SCHOOLSTATUS', u'SCHOOL\_RIGHT', u'OPTION\_RIGHT', u'FAC',
                u'SCHOOLPROVINCE', u'GENDER', u'DIPPERC', u'AGE', u'CGPA',
                u'DistinctionRatio', u'EchecRatio', u'NACADYEAR',
                u'Pass1stSessionRatio'],
               dtype='object')
\end{Verbatim}
        
    \begin{Verbatim}[commandchars=\\\{\}]
{\color{incolor}In [{\color{incolor}18}]:} \PY{n}{moore\PYZus{}lm} \PY{o}{=} \PY{n}{ols}\PY{p}{(}\PY{l+s+s1}{\PYZsq{}}\PY{l+s+s1}{CGPA \PYZti{} C(Pass1stSessionRatio)}\PY{l+s+s1}{\PYZsq{}}\PY{p}{,}\PY{n}{data}\PY{o}{=}\PY{n}{datasetFin}\PY{p}{)}\PY{o}{.}\PY{n}{fit}\PY{p}{(}\PY{p}{)}
         \PY{n}{aov\PYZus{}table} \PY{o}{=} \PY{n}{sm}\PY{o}{.}\PY{n}{stats}\PY{o}{.}\PY{n}{anova\PYZus{}lm}\PY{p}{(}\PY{n}{moore\PYZus{}lm}\PY{p}{,} \PY{n}{typ}\PY{o}{=}\PY{l+m+mi}{1}\PY{p}{)}
         \PY{n}{aov\PYZus{}table}
\end{Verbatim}

            \begin{Verbatim}[commandchars=\\\{\}]
{\color{outcolor}Out[{\color{outcolor}18}]:}                             df         sum\_sq       mean\_sq           F  \textbackslash{}
         C(Pass1stSessionRatio)     6.0   72133.769233  12022.294872  322.197877   
         Residual                4708.0  175671.437790     37.313390         NaN   
         
                                 PR(>F)  
         C(Pass1stSessionRatio)     0.0  
         Residual                   NaN  
\end{Verbatim}
        
    \begin{Verbatim}[commandchars=\\\{\}]
{\color{incolor}In [{\color{incolor}19}]:} \PY{n}{datasetFin}\PY{o}{.}\PY{n}{Pass1stSessionRatio}\PY{o}{.}\PY{n}{value\PYZus{}counts}\PY{p}{(}\PY{p}{)}
\end{Verbatim}

            \begin{Verbatim}[commandchars=\\\{\}]
{\color{outcolor}Out[{\color{outcolor}19}]:} G    3224
         A     646
         D     338
         E     268
         C     157
         F      42
         B      40
         Name: Pass1stSessionRatio, dtype: int64
\end{Verbatim}
        
    \begin{Verbatim}[commandchars=\\\{\}]
{\color{incolor}In [{\color{incolor}21}]:} \PY{n}{plt}\PY{o}{.}\PY{n}{figure}\PY{p}{(}\PY{n}{figsize}\PY{o}{=}\PY{p}{(}\PY{l+m+mi}{12}\PY{p}{,}\PY{l+m+mi}{6}\PY{p}{)}\PY{p}{)}
         \PY{n}{ax}\PY{o}{=}\PY{n}{sns}\PY{o}{.}\PY{n}{boxplot}\PY{p}{(}\PY{n}{x}\PY{o}{=}\PY{l+s+s2}{\PYZdq{}}\PY{l+s+s2}{Pass1stSessionRatio}\PY{l+s+s2}{\PYZdq{}}\PY{p}{,} \PY{n}{y}\PY{o}{=}\PY{l+s+s2}{\PYZdq{}}\PY{l+s+s2}{CGPA}\PY{l+s+s2}{\PYZdq{}}\PY{p}{,} \PY{n}{data}\PY{o}{=}\PY{n}{datasetFin}\PY{p}{,}\PY{n}{order}\PY{o}{=}\PY{p}{[}\PY{l+s+s1}{\PYZsq{}}\PY{l+s+s1}{A}\PY{l+s+s1}{\PYZsq{}}\PY{p}{,}\PY{l+s+s1}{\PYZsq{}}\PY{l+s+s1}{B}\PY{l+s+s1}{\PYZsq{}}\PY{p}{,}\PY{l+s+s1}{\PYZsq{}}\PY{l+s+s1}{C}\PY{l+s+s1}{\PYZsq{}}\PY{p}{,}\PY{l+s+s1}{\PYZsq{}}\PY{l+s+s1}{D}\PY{l+s+s1}{\PYZsq{}}\PY{p}{,}\PY{l+s+s1}{\PYZsq{}}\PY{l+s+s1}{E}\PY{l+s+s1}{\PYZsq{}}\PY{p}{,}\PY{l+s+s1}{\PYZsq{}}\PY{l+s+s1}{F}\PY{l+s+s1}{\PYZsq{}}\PY{p}{,}\PY{l+s+s1}{\PYZsq{}}\PY{l+s+s1}{G}\PY{l+s+s1}{\PYZsq{}}\PY{p}{]}\PY{p}{)}
         \PY{n}{plt}\PY{o}{.}\PY{n}{savefig}\PY{p}{(}\PY{l+s+s1}{\PYZsq{}}\PY{l+s+s1}{pic/CGPA\PYZhy{}Disti.png}\PY{l+s+s1}{\PYZsq{}}\PY{p}{,}\PY{n}{dpi}\PY{o}{=}\PY{l+m+mi}{100}\PY{p}{)}
\end{Verbatim}

    \begin{center}
    \adjustimage{max size={0.9\linewidth}{0.9\paperheight}}{output_519_0.png}
    \end{center}
    { \hspace*{\fill} \\}
    
    Regradons commen se repartissent les moyenne au sein de chaque categorie

    \begin{Verbatim}[commandchars=\\\{\}]
{\color{incolor}In [{\color{incolor}22}]:} \PY{n}{Pass1stSessionRatiogroup}\PY{o}{=}\PY{n}{datasetFin}\PY{o}{.}\PY{n}{groupby}\PY{p}{(}\PY{l+s+s1}{\PYZsq{}}\PY{l+s+s1}{Pass1stSessionRatio}\PY{l+s+s1}{\PYZsq{}}\PY{p}{)}\PY{o}{.}\PY{n}{mean}\PY{p}{(}\PY{p}{)}\PY{o}{.}\PY{n}{sort\PYZus{}values}\PY{p}{(}\PY{n}{axis}\PY{o}{=}\PY{l+m+mi}{0}\PY{p}{,}\PY{n}{by}\PY{o}{=}\PY{l+s+s1}{\PYZsq{}}\PY{l+s+s1}{CGPA}\PY{l+s+s1}{\PYZsq{}}\PY{p}{,}\PY{n}{ascending}\PY{o}{=}\PY{n+nb+bp}{False}\PY{p}{)}\PY{o}{.}\PY{n}{reset\PYZus{}index}\PY{p}{(}\PY{p}{)}
\end{Verbatim}

    \begin{Verbatim}[commandchars=\\\{\}]
{\color{incolor}In [{\color{incolor}23}]:} \PY{n}{Pass1stSessionRatiogroup}
\end{Verbatim}

            \begin{Verbatim}[commandchars=\\\{\}]
{\color{outcolor}Out[{\color{outcolor}23}]:}   Pass1stSessionRatio           ID    DIPPERC        AGE       CGPA  NACADYEAR
         0                   B  8027.100000  60.925000  25.375000  64.480625   4.000000
         1                   A  8324.917957  59.308833  25.113003  64.399471   1.682663
         2                   C  8191.713376  59.101124  24.726115  62.569002   3.000000
         3                   D  8599.124260  58.430491  24.428994  62.198299   2.331361
         4                   E  8084.843284  57.944625  24.895522  60.812065   3.000000
         5                   F  7713.833333  59.639917  24.928571  60.181548   4.000000
         6                   G  8913.153226  55.946388  24.664702  54.724152   1.720223
\end{Verbatim}
        
    Avec un valeur de Pr de 0 \textless{}0.05 on conclu que la moyene ne
peut jamais etre la meme groupes de cette valeur de satisfaction e
premierre session par consequent nous pouvons grouper les classes entre
elles mais juste apres verification du box plot

    \begin{Verbatim}[commandchars=\\\{\}]
{\color{incolor}In [{\color{incolor} }]:} 
\end{Verbatim}

    \paragraph{CGPA-EchecRatio}\label{cgpa-echecratio}

    \begin{Verbatim}[commandchars=\\\{\}]
{\color{incolor}In [{\color{incolor}17}]:} \PY{n}{datasetFin}\PY{o}{.}\PY{n}{columns}
\end{Verbatim}

            \begin{Verbatim}[commandchars=\\\{\}]
{\color{outcolor}Out[{\color{outcolor}17}]:} Index([u'ID', u'SCHOOLSTATUS', u'SCHOOL\_RIGHT', u'OPTION\_RIGHT', u'FAC',
                u'SCHOOLPROVINCE', u'GENDER', u'DIPPERC', u'AGE', u'CGPA',
                u'DistinctionRatio', u'EchecRatio', u'NACADYEAR',
                u'Pass1stSessionRatio'],
               dtype='object')
\end{Verbatim}
        
    \begin{Verbatim}[commandchars=\\\{\}]
{\color{incolor}In [{\color{incolor}36}]:} \PY{n}{moore\PYZus{}lm} \PY{o}{=} \PY{n}{ols}\PY{p}{(}\PY{l+s+s1}{\PYZsq{}}\PY{l+s+s1}{CGPA \PYZti{} C(EchecRatio)}\PY{l+s+s1}{\PYZsq{}}\PY{p}{,}\PY{n}{data}\PY{o}{=}\PY{n}{datasetFin}\PY{p}{)}\PY{o}{.}\PY{n}{fit}\PY{p}{(}\PY{p}{)}
         \PY{n}{aov\PYZus{}table} \PY{o}{=} \PY{n}{sm}\PY{o}{.}\PY{n}{stats}\PY{o}{.}\PY{n}{anova\PYZus{}lm}\PY{p}{(}\PY{n}{moore\PYZus{}lm}\PY{p}{,} \PY{n}{typ}\PY{o}{=}\PY{l+m+mi}{1}\PY{p}{)}
         \PY{n}{aov\PYZus{}table}
\end{Verbatim}

            \begin{Verbatim}[commandchars=\\\{\}]
{\color{outcolor}Out[{\color{outcolor}36}]:}                    df         sum\_sq       mean\_sq            F  PR(>F)
         C(EchecRatio)     5.0  173017.739123  34603.547825  2178.815666     0.0
         Residual       4709.0   74787.467899     15.881815          NaN     NaN
\end{Verbatim}
        
    \begin{Verbatim}[commandchars=\\\{\}]
{\color{incolor}In [{\color{incolor}37}]:} \PY{n}{datasetFin}\PY{o}{.}\PY{n}{EchecRatio}\PY{o}{.}\PY{n}{value\PYZus{}counts}\PY{p}{(}\PY{p}{)}
\end{Verbatim}

            \begin{Verbatim}[commandchars=\\\{\}]
{\color{outcolor}Out[{\color{outcolor}37}]:} G    3289
         A     916
         D     255
         E     185
         C      56
         F      14
         Name: EchecRatio, dtype: int64
\end{Verbatim}
        
    \begin{Verbatim}[commandchars=\\\{\}]
{\color{incolor}In [{\color{incolor}38}]:} \PY{n}{plt}\PY{o}{.}\PY{n}{figure}\PY{p}{(}\PY{n}{figsize}\PY{o}{=}\PY{p}{(}\PY{l+m+mi}{12}\PY{p}{,}\PY{l+m+mi}{6}\PY{p}{)}\PY{p}{)}
         \PY{n}{ax}\PY{o}{=}\PY{n}{sns}\PY{o}{.}\PY{n}{boxplot}\PY{p}{(}\PY{n}{x}\PY{o}{=}\PY{l+s+s2}{\PYZdq{}}\PY{l+s+s2}{EchecRatio}\PY{l+s+s2}{\PYZdq{}}\PY{p}{,} \PY{n}{y}\PY{o}{=}\PY{l+s+s2}{\PYZdq{}}\PY{l+s+s2}{CGPA}\PY{l+s+s2}{\PYZdq{}}\PY{p}{,} \PY{n}{data}\PY{o}{=}\PY{n}{datasetFin}\PY{p}{,}\PY{n}{order}\PY{o}{=}\PY{p}{[}\PY{l+s+s1}{\PYZsq{}}\PY{l+s+s1}{A}\PY{l+s+s1}{\PYZsq{}}\PY{p}{,}\PY{l+s+s1}{\PYZsq{}}\PY{l+s+s1}{B}\PY{l+s+s1}{\PYZsq{}}\PY{p}{,}\PY{l+s+s1}{\PYZsq{}}\PY{l+s+s1}{C}\PY{l+s+s1}{\PYZsq{}}\PY{p}{,}\PY{l+s+s1}{\PYZsq{}}\PY{l+s+s1}{D}\PY{l+s+s1}{\PYZsq{}}\PY{p}{,}\PY{l+s+s1}{\PYZsq{}}\PY{l+s+s1}{E}\PY{l+s+s1}{\PYZsq{}}\PY{p}{,}\PY{l+s+s1}{\PYZsq{}}\PY{l+s+s1}{F}\PY{l+s+s1}{\PYZsq{}}\PY{p}{,}\PY{l+s+s1}{\PYZsq{}}\PY{l+s+s1}{G}\PY{l+s+s1}{\PYZsq{}}\PY{p}{]}\PY{p}{)}
         \PY{n}{plt}\PY{o}{.}\PY{n}{savefig}\PY{p}{(}\PY{l+s+s1}{\PYZsq{}}\PY{l+s+s1}{pic/CGPA\PYZhy{}Echec.png}\PY{l+s+s1}{\PYZsq{}}\PY{p}{,}\PY{n}{dpi}\PY{o}{=}\PY{l+m+mi}{100}\PY{p}{)}
\end{Verbatim}

    \begin{center}
    \adjustimage{max size={0.9\linewidth}{0.9\paperheight}}{output_529_0.png}
    \end{center}
    { \hspace*{\fill} \\}
    
    Regradons commen se repartissent les moyenne au sein de chaque categorie

    \begin{Verbatim}[commandchars=\\\{\}]
{\color{incolor}In [{\color{incolor}40}]:} \PY{n}{EchecRatiogroup}\PY{o}{=}\PY{n}{datasetFin}\PY{o}{.}\PY{n}{groupby}\PY{p}{(}\PY{l+s+s1}{\PYZsq{}}\PY{l+s+s1}{EchecRatio}\PY{l+s+s1}{\PYZsq{}}\PY{p}{)}\PY{o}{.}\PY{n}{mean}\PY{p}{(}\PY{p}{)}\PY{o}{.}\PY{n}{sort\PYZus{}values}\PY{p}{(}\PY{n}{axis}\PY{o}{=}\PY{l+m+mi}{0}\PY{p}{,}\PY{n}{by}\PY{o}{=}\PY{l+s+s1}{\PYZsq{}}\PY{l+s+s1}{CGPA}\PY{l+s+s1}{\PYZsq{}}\PY{p}{,}\PY{n}{ascending}\PY{o}{=}\PY{n+nb+bp}{False}\PY{p}{)}\PY{o}{.}\PY{n}{reset\PYZus{}index}\PY{p}{(}\PY{p}{)}
\end{Verbatim}

    \begin{Verbatim}[commandchars=\\\{\}]
{\color{incolor}In [{\color{incolor}41}]:} \PY{n}{EchecRatiogroup}
\end{Verbatim}

            \begin{Verbatim}[commandchars=\\\{\}]
{\color{outcolor}Out[{\color{outcolor}41}]:}   EchecRatio           ID    DIPPERC        AGE       CGPA  NACADYEAR
         0          G  8605.802067  57.390488  24.761630  61.051781   2.041654
         1          F  7559.571429  60.428571  24.428571  58.769643   4.000000
         2          E  7768.600000  56.313514  25.637838  55.420541   3.000000
         3          D  8742.254902  55.943645  24.533333  52.944118   2.039216
         4          C  7754.517857  56.973323  26.214286  51.186905   3.000000
         5          A  9396.586245  55.356582  24.415939  45.896798   1.102620
\end{Verbatim}
        
    Avec une PR de 0 on conclu fortement que la moyenne ne peut pas etre la
meme au sein de chaque groupe !

    \subsubsection{CGPA-Distinction Ratio}\label{cgpa-distinction-ratio}

    \begin{Verbatim}[commandchars=\\\{\}]
{\color{incolor}In [{\color{incolor}8}]:} \PY{n}{moore\PYZus{}lm} \PY{o}{=} \PY{n}{ols}\PY{p}{(}\PY{l+s+s1}{\PYZsq{}}\PY{l+s+s1}{CGPA \PYZti{} C(DistinctionRatio)}\PY{l+s+s1}{\PYZsq{}}\PY{p}{,}\PY{n}{data}\PY{o}{=}\PY{n}{datasetFin}\PY{p}{)}\PY{o}{.}\PY{n}{fit}\PY{p}{(}\PY{p}{)}
        \PY{n}{aov\PYZus{}table} \PY{o}{=} \PY{n}{sm}\PY{o}{.}\PY{n}{stats}\PY{o}{.}\PY{n}{anova\PYZus{}lm}\PY{p}{(}\PY{n}{moore\PYZus{}lm}\PY{p}{,} \PY{n}{typ}\PY{o}{=}\PY{l+m+mi}{1}\PY{p}{)}
        \PY{n}{aov\PYZus{}table}
\end{Verbatim}

    \begin{Verbatim}[commandchars=\\\{\}]
/Users/espyMur/Desktop/Memory-WorkingDir/memoryVenv/lib/python2.7/site-packages/scipy/stats/\_distn\_infrastructure.py:879: RuntimeWarning: invalid value encountered in greater
  return (self.a < x) \& (x < self.b)
/Users/espyMur/Desktop/Memory-WorkingDir/memoryVenv/lib/python2.7/site-packages/scipy/stats/\_distn\_infrastructure.py:879: RuntimeWarning: invalid value encountered in less
  return (self.a < x) \& (x < self.b)
/Users/espyMur/Desktop/Memory-WorkingDir/memoryVenv/lib/python2.7/site-packages/scipy/stats/\_distn\_infrastructure.py:1818: RuntimeWarning: invalid value encountered in less\_equal
  cond2 = cond0 \& (x <= self.a)

    \end{Verbatim}

            \begin{Verbatim}[commandchars=\\\{\}]
{\color{outcolor}Out[{\color{outcolor}8}]:}                          df         sum\_sq      mean\_sq           F  \textbackslash{}
        C(DistinctionRatio)     6.0   41087.520237  6847.920039  155.961534   
        Residual             4708.0  206717.686786    43.907750         NaN   
        
                                    PR(>F)  
        C(DistinctionRatio)  3.525333e-181  
        Residual                       NaN  
\end{Verbatim}
        
    \begin{Verbatim}[commandchars=\\\{\}]
{\color{incolor}In [{\color{incolor}9}]:} \PY{n}{datasetFin}\PY{o}{.}\PY{n}{DistinctionRatio}\PY{o}{.}\PY{n}{value\PYZus{}counts}\PY{p}{(}\PY{p}{)}
\end{Verbatim}

            \begin{Verbatim}[commandchars=\\\{\}]
{\color{outcolor}Out[{\color{outcolor}9}]:} G    4467
        A      81
        D      76
        E      43
        C      22
        F      21
        B       5
        Name: DistinctionRatio, dtype: int64
\end{Verbatim}
        
    \begin{Verbatim}[commandchars=\\\{\}]
{\color{incolor}In [{\color{incolor}13}]:} \PY{n}{plt}\PY{o}{.}\PY{n}{figure}\PY{p}{(}\PY{n}{figsize}\PY{o}{=}\PY{p}{(}\PY{l+m+mi}{12}\PY{p}{,}\PY{l+m+mi}{6}\PY{p}{)}\PY{p}{)}
         \PY{n}{ax}\PY{o}{=}\PY{n}{sns}\PY{o}{.}\PY{n}{boxplot}\PY{p}{(}\PY{n}{x}\PY{o}{=}\PY{l+s+s2}{\PYZdq{}}\PY{l+s+s2}{DistinctionRatio}\PY{l+s+s2}{\PYZdq{}}\PY{p}{,} \PY{n}{y}\PY{o}{=}\PY{l+s+s2}{\PYZdq{}}\PY{l+s+s2}{CGPA}\PY{l+s+s2}{\PYZdq{}}\PY{p}{,} \PY{n}{data}\PY{o}{=}\PY{n}{datasetFin}\PY{p}{,}\PY{n}{order}\PY{o}{=}\PY{p}{[}\PY{l+s+s1}{\PYZsq{}}\PY{l+s+s1}{A}\PY{l+s+s1}{\PYZsq{}}\PY{p}{,}\PY{l+s+s1}{\PYZsq{}}\PY{l+s+s1}{B}\PY{l+s+s1}{\PYZsq{}}\PY{p}{,}\PY{l+s+s1}{\PYZsq{}}\PY{l+s+s1}{C}\PY{l+s+s1}{\PYZsq{}}\PY{p}{,}\PY{l+s+s1}{\PYZsq{}}\PY{l+s+s1}{D}\PY{l+s+s1}{\PYZsq{}}\PY{p}{,}\PY{l+s+s1}{\PYZsq{}}\PY{l+s+s1}{E}\PY{l+s+s1}{\PYZsq{}}\PY{p}{,}\PY{l+s+s1}{\PYZsq{}}\PY{l+s+s1}{F}\PY{l+s+s1}{\PYZsq{}}\PY{p}{,}\PY{l+s+s1}{\PYZsq{}}\PY{l+s+s1}{G}\PY{l+s+s1}{\PYZsq{}}\PY{p}{]}\PY{p}{)}
         \PY{n}{plt}\PY{o}{.}\PY{n}{savefig}\PY{p}{(}\PY{l+s+s1}{\PYZsq{}}\PY{l+s+s1}{pic/CGPA\PYZhy{}Disti.png}\PY{l+s+s1}{\PYZsq{}}\PY{p}{,}\PY{n}{dpi}\PY{o}{=}\PY{l+m+mi}{100}\PY{p}{)}
\end{Verbatim}

    \begin{center}
    \adjustimage{max size={0.9\linewidth}{0.9\paperheight}}{output_537_0.png}
    \end{center}
    { \hspace*{\fill} \\}
    
    Regradons commen se repartissent les moyenne au sein de chaque categorie

    \begin{Verbatim}[commandchars=\\\{\}]
{\color{incolor}In [{\color{incolor}15}]:} \PY{n}{DistinctionOption}\PY{o}{=}\PY{n}{datasetFin}\PY{o}{.}\PY{n}{groupby}\PY{p}{(}\PY{l+s+s1}{\PYZsq{}}\PY{l+s+s1}{DistinctionRatio}\PY{l+s+s1}{\PYZsq{}}\PY{p}{)}\PY{o}{.}\PY{n}{mean}\PY{p}{(}\PY{p}{)}\PY{o}{.}\PY{n}{sort\PYZus{}values}\PY{p}{(}\PY{n}{axis}\PY{o}{=}\PY{l+m+mi}{0}\PY{p}{,}\PY{n}{by}\PY{o}{=}\PY{l+s+s1}{\PYZsq{}}\PY{l+s+s1}{CGPA}\PY{l+s+s1}{\PYZsq{}}\PY{p}{,}\PY{n}{ascending}\PY{o}{=}\PY{n+nb+bp}{False}\PY{p}{)}\PY{o}{.}\PY{n}{reset\PYZus{}index}\PY{p}{(}\PY{p}{)}
\end{Verbatim}

    \begin{Verbatim}[commandchars=\\\{\}]
{\color{incolor}In [{\color{incolor}16}]:} \PY{n}{DistinctionOption}
\end{Verbatim}

            \begin{Verbatim}[commandchars=\\\{\}]
{\color{outcolor}Out[{\color{outcolor}16}]:}   DistinctionRatio           ID    DIPPERC        AGE       CGPA  NACADYEAR
         0                A  7843.283951  63.910531  25.925926  73.066769   1.851852
         1                B  8214.400000  61.400000  26.200000  71.425000   4.000000
         2                C  7979.500000  61.636364  26.818182  69.574242   3.000000
         3                D  7520.631579  59.513158  26.000000  68.486513   2.315789
         4                E  7810.860465  59.627907  24.232558  66.616279   3.000000
         5                F  7902.523810  60.803644  24.190476  65.686905   4.000000
         6                G  8773.896127  56.633188  24.685024  56.643944   1.880233
\end{Verbatim}
        
    Nous concluons que ceux qui on distinguer à 100\% on une moyenne de 73
et on ne peut pas dire qu'il ya une correlation notable entre ces
attribues

    Enfin voyons comment evolue cette caracteristique au sein de chaque
faculté

    \paragraph{CGPA-FAC}\label{cgpa-fac}

    \begin{Verbatim}[commandchars=\\\{\}]
{\color{incolor}In [{\color{incolor}17}]:} \PY{n}{datasetFin}\PY{o}{.}\PY{n}{columns}
\end{Verbatim}

            \begin{Verbatim}[commandchars=\\\{\}]
{\color{outcolor}Out[{\color{outcolor}17}]:} Index([u'ID', u'SCHOOLSTATUS', u'SCHOOL\_RIGHT', u'OPTION\_RIGHT', u'FAC',
                u'SCHOOLPROVINCE', u'GENDER', u'DIPPERC', u'AGE', u'CGPA',
                u'DistinctionRatio', u'EchecRatio', u'NACADYEAR',
                u'Pass1stSessionRatio'],
               dtype='object')
\end{Verbatim}
        
    \begin{Verbatim}[commandchars=\\\{\}]
{\color{incolor}In [{\color{incolor}42}]:} \PY{n}{moore\PYZus{}lm} \PY{o}{=} \PY{n}{ols}\PY{p}{(}\PY{l+s+s1}{\PYZsq{}}\PY{l+s+s1}{CGPA \PYZti{} C(FAC)}\PY{l+s+s1}{\PYZsq{}}\PY{p}{,}\PY{n}{data}\PY{o}{=}\PY{n}{datasetFin}\PY{p}{)}\PY{o}{.}\PY{n}{fit}\PY{p}{(}\PY{p}{)}
         \PY{n}{aov\PYZus{}table} \PY{o}{=} \PY{n}{sm}\PY{o}{.}\PY{n}{stats}\PY{o}{.}\PY{n}{anova\PYZus{}lm}\PY{p}{(}\PY{n}{moore\PYZus{}lm}\PY{p}{,} \PY{n}{typ}\PY{o}{=}\PY{l+m+mi}{1}\PY{p}{)}
         \PY{n}{aov\PYZus{}table}
\end{Verbatim}

            \begin{Verbatim}[commandchars=\\\{\}]
{\color{outcolor}Out[{\color{outcolor}42}]:}               df         sum\_sq      mean\_sq          F        PR(>F)
         C(FAC)       6.0   11054.669798  1842.444966  36.638696  1.244254e-43
         Residual  4708.0  236750.537224    50.286860        NaN           NaN
\end{Verbatim}
        
    \begin{Verbatim}[commandchars=\\\{\}]
{\color{incolor}In [{\color{incolor}43}]:} \PY{n}{datasetFin}\PY{o}{.}\PY{n}{FAC}\PY{o}{.}\PY{n}{value\PYZus{}counts}\PY{p}{(}\PY{p}{)}\PY{o}{*}\PY{l+m+mi}{100}\PY{o}{/}\PY{n+nb}{len}\PY{p}{(}\PY{n}{datasetFin}\PY{p}{)}
\end{Verbatim}

            \begin{Verbatim}[commandchars=\\\{\}]
{\color{outcolor}Out[{\color{outcolor}43}]:} FSEG    32.852598
         FSTA    19.151644
         FD      19.003181
         FSDC    16.076352
         FM       5.132556
         FPSE     4.814422
         FT       2.969247
         Name: FAC, dtype: float64
\end{Verbatim}
        
    Avec ces metriques nous pouvons voir la repartition de nos echantillons
au sein de chaque faculté

    \begin{Verbatim}[commandchars=\\\{\}]
{\color{incolor}In [{\color{incolor}44}]:} \PY{n}{plt}\PY{o}{.}\PY{n}{figure}\PY{p}{(}\PY{n}{figsize}\PY{o}{=}\PY{p}{(}\PY{l+m+mi}{12}\PY{p}{,}\PY{l+m+mi}{6}\PY{p}{)}\PY{p}{)}
         \PY{n}{ax}\PY{o}{=}\PY{n}{sns}\PY{o}{.}\PY{n}{boxplot}\PY{p}{(}\PY{n}{x}\PY{o}{=}\PY{l+s+s2}{\PYZdq{}}\PY{l+s+s2}{FAC}\PY{l+s+s2}{\PYZdq{}}\PY{p}{,} \PY{n}{y}\PY{o}{=}\PY{l+s+s2}{\PYZdq{}}\PY{l+s+s2}{CGPA}\PY{l+s+s2}{\PYZdq{}}\PY{p}{,} \PY{n}{data}\PY{o}{=}\PY{n}{datasetFin}\PY{p}{)}
         \PY{n}{plt}\PY{o}{.}\PY{n}{savefig}\PY{p}{(}\PY{l+s+s1}{\PYZsq{}}\PY{l+s+s1}{pic/CGPA\PYZhy{}FAC.png}\PY{l+s+s1}{\PYZsq{}}\PY{p}{,}\PY{n}{dpi}\PY{o}{=}\PY{l+m+mi}{100}\PY{p}{)}
\end{Verbatim}

    \begin{center}
    \adjustimage{max size={0.9\linewidth}{0.9\paperheight}}{output_548_0.png}
    \end{center}
    { \hspace*{\fill} \\}
    
    Regradons commen se repartissent les moyenne au sein de chaque categorie

    \begin{Verbatim}[commandchars=\\\{\}]
{\color{incolor}In [{\color{incolor}45}]:} \PY{n}{FACgroup}\PY{o}{=}\PY{n}{datasetFin}\PY{o}{.}\PY{n}{groupby}\PY{p}{(}\PY{l+s+s1}{\PYZsq{}}\PY{l+s+s1}{FAC}\PY{l+s+s1}{\PYZsq{}}\PY{p}{)}\PY{o}{.}\PY{n}{mean}\PY{p}{(}\PY{p}{)}\PY{o}{.}\PY{n}{sort\PYZus{}values}\PY{p}{(}\PY{n}{axis}\PY{o}{=}\PY{l+m+mi}{0}\PY{p}{,}\PY{n}{by}\PY{o}{=}\PY{l+s+s1}{\PYZsq{}}\PY{l+s+s1}{CGPA}\PY{l+s+s1}{\PYZsq{}}\PY{p}{,}\PY{n}{ascending}\PY{o}{=}\PY{n+nb+bp}{False}\PY{p}{)}\PY{o}{.}\PY{n}{reset\PYZus{}index}\PY{p}{(}\PY{p}{)}
\end{Verbatim}

    \begin{Verbatim}[commandchars=\\\{\}]
{\color{incolor}In [{\color{incolor}47}]:} \PY{n}{FACgroup}
\end{Verbatim}

            \begin{Verbatim}[commandchars=\\\{\}]
{\color{outcolor}Out[{\color{outcolor}47}]:}     FAC            ID    DIPPERC        AGE       CGPA  NACADYEAR
         0    FT   8125.250000  53.814286  31.428571  61.999703   2.085714
         1  FPSE   8417.453744  56.202099  28.224670  59.558921   2.013216
         2  FSDC   8195.007916  55.329211  25.974934  58.917700   1.974934
         3    FD   8837.433036  56.151372  24.771205  58.022024   1.953125
         4    FM  10887.719008  59.434420  21.487603  57.675207   1.471074
         5  FSEG   8410.416398  56.832564  24.323434  56.446100   1.916720
         6  FSTA   9166.437431  58.941594  23.307863  55.419684   1.885936
\end{Verbatim}
        
    La moyenne de GPA n'est pas la meme au sein de chaque faculté nous
pouvons remqrque comment il se repatie sur la figure

    \begin{Verbatim}[commandchars=\\\{\}]
{\color{incolor}In [{\color{incolor} }]:} 
\end{Verbatim}


    % Add a bibliography block to the postdoc
    
    
    
    \end{document}
